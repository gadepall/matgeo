\begin{enumerate}[label=\thesubsection.\arabic*.,ref=\thesubsection.\theenumi]
    \item The line $x+3y = 0$ is a diameter of the circle $ x^{2} + y^{2} - 6x +2y = 0$.
    \hfill{(1989 )}
     \item The centre of the circle inscribed in square formed by the lines $x^2-8x+12=0$ and $y^2-14y+45=0$,  is
         \hfill$\brak{2003}$
         \begin{multicols}{4}
\begin{enumerate}
         \item $\brak{4, 7}$
         \item $\brak{7, 4}$
         \item $\brak{9, 4}$
         \item $\brak{4, 9}$
     \end{enumerate}
     \end{multicols}
    \item If a circle is passing through the point $\brak{a, b}$ and it is cutting the circle $x^{2}+y^{2}=k^{2}$ orthogonally,  then the equation of the locus of its centre is 
    \hfill {(1988)}
    \begin{multicols}{2}
\begin{enumerate}
    	\item $2ax + 2by - (a^{2}+b^{2}+k^{2}) = 0$
    	\item $2ax + 2by - (a^{2}-b^{2}+k^{2}) = 0$
    	\item $x^{2} + y^{2}-3ax-4by+ (a^{2}+b^{2}-k^{2}) = 0$
    	\item $x^{2} + y^{2}-2ax-3by+ (a^{2}-b^{2}-k^{2}) = 0$
    \end{enumerate}
\end{multicols}
%
\item A circle $S$ passes through the point \brak{0, 1} and is orthogonal to the circle $(x-1)^2+y^2=16$ and $x^2+y^2=1$. Then
%
\hfill {\brak{ 2014}}
\begin{multicols}{2}
\begin{enumerate}
\item Radius of $S$ is $8$
\item Radius of $S$ is $7$
\item Centre of $S$ is \brak{-7, 1}
\item Centre of $S$ is \brak{-8, 1}
\end{enumerate}
\end{multicols}
\item Let $\Vec{S}$ be the focus of the parabola $y^2=8x$ and let $PQ$ be the common chord of the circle $x^2+y^2-2x-4y=0$ and the given parabola. The area of the triangle $PQS$ is

\hfill(2012)
\item Let the curve $C$ be the mirror image of the parabola $y^2=4x$ with respect to the line $x+y+4=0$. If $\Vec{A}$ and $\Vec{B}$ are the points of the intersection of $C$ with the line $y=-5$, then the distance between $\Vec{A}$ and $\Vec{B}$ is \rule{1cm}{0.1pt}.
\hfill(2015)
      \item If a chord, which is not tangent, of the parabola $y^2=16x$ has equation $2x+y=p$, and midpoint $(h,k)$, then which of the following is (are) possible value(s) of $p$, $h$ and $k$? 
	      \hfill(2017)
	     % 
\begin{multicols}{2}
	       \begin{enumerate}
		      \item $p=-2,h=2,k=-4$
		      \item $p=-1,h=1,k=3$
		      \item $p=2,h=3,k=-4$
		      \item $p=5,h=4,k=-3$
	       \end{enumerate}
\end{multicols}
\item If one end of a focal chord of the parabola, $y^2=16x$ is at $\brak{1,4}$, then the length of this focal chord is 
     \hfill{(2019)}
\begin{multicols}{4}
	\begin{enumerate}
    		\item $25$
    		\item $22$
    		\item $24$
    		\item $20$
	\end{enumerate}
\end{multicols}    

\item If chord $PQ$ subtends an angle $\theta$  at the vertex of $y^2=4ax$
\hfill(2013)

\begin{multicols}{4}
\begin{enumerate}
    \item $\frac{2}{3}\sqrt{7}$
    
    \item $\frac{-2}{3}\sqrt{7}$
    
    \item $\frac{2}{3}\sqrt{5}$
    
    \item $\frac{-2}{3}\sqrt{5}$
\end{enumerate}
\end{multicols}
\item Let $a$,$r$,$s$,$t$ be nonzero real numbers. Let $\Vec{P}$$(at^2,2as)$,$\Vec{Q}$,$\Vec{R}$$(as^2,2as)$ be distinct points on the parabola $y^2=4ax$. Suppose that $PQ$ is the focal chord and lines $QR$ and $PK$ are parallel,where $K$ is the point $(2a,0)$.
 The value of $r$ is 
\hfill(2014)
\begin{multicols}{4}
\begin{enumerate}
    \item $\frac{-1}{t}$ 
    \item $\frac{t^2+1}{t}$
    \item $\frac{1}{t}$
    \item $\frac{t^2-1}{t}$
\end{enumerate}
\end{multicols}
\item Let $\vec{F_1}$($x_1$,0) and  $\vec{F_2}$($x_2$,0) for $x_1<0$ and $x_2>0$, be the focii of the ellipse $\frac{x^2}{9}+\frac{y^2}{8} =1$. Suppose a parabola having vertex at the origin and focus at $\vec{F_2}$ intersects the ellipse at point $\Vec{M}$ in the first quadrant and the point $\Vec{N}$ in the first quadrant.
The orthocentre of the triangle $F_1MN$ is
         \hfill( 2016)
\begin{multicols}{4}
\begin{enumerate}
	\item $\brak{\frac{-9}{10},0}$   
	\item $\brak{\frac{2}{3},0}$     
	\item $\brak{\frac{9}{10},0}$ 
     \item $\brak{\frac{2}{3},\sqrt{6}}$ 
    \end{enumerate}
\end{multicols}
		
      \item Through the vertex $\vec{O}$ of parabola $y^2=4x$, chords $OP$ and $OQ$ are drawn at right angles to one another. Show that for all positions of $\vec{P}$, $PQ$ cuts the axis of the parabola at a fixed point. Also find the locus of the middle point of $PQ$. 

		\hfill(1994)
		
      \item Show that the locus of a point that divides a chord of slope $2$ of the parabola $y^2=4x$ internally in the ratio $1:2$ is a parabola. Find the vertex of this parabola. 

	      \hfill(1995)
\item If the circle $C_1:x^{2}+y^{2}=16$ intersects another circle $C_2$ of radius 5 in such a manner that the common chord is of maximum length and has a slope equal to $\frac{3}{4}$,  then the coordinates of the centre of $C_2$ are
\rule{1cm}{0.01pt}.
%
	\hfill\brak{1988}
\item If the two circles $(x-1)^2+(y-3)^2=r^2$ and $x^2+y^2-8x+2y+8=0$ intersect in two distinct points,  then \hfill(2003)
\begin{multicols}{4}
\begin{enumerate}
\item$r>2$
\item$2<r<8$
\item$r<2$
\item$r=2$
\end{enumerate}
\end{multicols}
\item If the circles $x^2+y^2+2ax+cy+a=0$ and $x^2+y^2-3ax+dy-1=0$ intersect in two distinct points $\vec{P}$ and $\vec{Q},$ then the line $5x+by-a=0$ passes through $\vec{P}$ and $\vec{Q}$ for
\hfill{(2005)}
\begin{multicols}{2}
\begin{enumerate}
\item exactly one value of $a$
\item no value of $a$
\item infinitely many values of $a$
\item exactly two values of $a$
\end{enumerate}
\end{multicols}
\item If a circle passes through the point \brak{a, b} and cuts the circle $x^2+y^2=p^2$ orthogonally,  then the equation of the locus of its centre is 
\hfill{(2005)}
\begin{multicols}{2}
\begin{enumerate}
\item $x^2+y^2-3ax-4by+\brak{a^2+b^2-p^2}=0$
\item $2ax+2by-\brak{a^2-b^2+p^2}=0$
\item $x^2+y^2-2ax-3by+\brak{a^2-b^2-p^2}=0$
\item $2ax+2by-\brak{a^2+b^2+p^2}=0$
\end{enumerate}
\end{multicols}
\item If $\vec{P}$ and $\vec{Q}$ are the points of intersection of the circles $x^2+y^2+3x+7y+2p-5=0$ and $x^2+y^2+2x+2y-p^2=0$ then there is a circle passing through $\vec{P},  \vec{Q}$ and \brak{1, 1} for:
\hfill{(2009)}
\begin{multicols}{2}
\begin{enumerate}
\item all except one value of $p$
\item all except two values of $p$
\item exactly one value of $p$
\item all value of $p$
\end{enumerate}
\end{multicols}
    \item Two circles $x^{2} + y^{2} = 6$ and $x^{2} + y^{2}-6x +8=0$ are given. Then the equation of the circle through their points of intersection and the point $\brak{1, 1}$ is \hfill {(1980)}
    \begin{multicols}{2}
\begin{enumerate}
    	\item $x^{2}+y^{2}-6x+4=0$ 
    	\item $x^{2}+y^{2}-3x+1=0$
    	\item $x^{2}+y^{2}-4y+2=0$
    	\item none of these
    \end{enumerate}
\end{multicols}
    \item The equation of circle passing through $\brak{1, 1}$ and points of intersection of the circles $x^{2}+y^{2}+13x-3y=0$ and $2x^{2}+2y^{2}+4x-7y-25=0$ is
    \hfill {(1983)}
    \begin{multicols}{2}
\begin{enumerate}
    	\item $4x^{2}+4y^{2}-30x-10y-25=0$
    	\item $4x^{2}+4y^{2}+30x-13y-25=0$
    	\item $4x^{2}+4y^{2}-17x-10y+25=0$
    	\item none of these
    \end{enumerate}
\end{multicols}
    \item If the two circles $(x-1)^{2} + (y-3)^{2} = r^{2}$ and $x^{2}+y^{2}-8x+2y+8=0$ intersect in two distinct points,  then \hfill {(1989)} 
    \begin{multicols}{4}
\begin{enumerate}
    	\item $2<r<8$
    	\item $r<2$
    	\item $r=2$
    	\item $r>2$
    \end{enumerate}
\end{multicols}
    \item The circles $x^{2}-10x+16=0$ and $x^{2}+y^{2}=r^{2}$ intersect each other in the two distinct points if
    \hfill {(1994)}
    \begin{multicols}{4}
\begin{enumerate}
    	\item $r<2$
    	\item $r>8$
    	\item $2<r<8$
    	\item $2\leq r\leq8$
    \end{enumerate}
\end{multicols}
     \item If the circles $x^2+y^2+2x+2ky+k=0$ intersect orthogonally, then $k$ is
        \hfill$\brak{2000}$
    \begin{multicols}{4}
\begin{enumerate}
        \item 2 or $-\frac{3}{2}$
        \item -2 or $-\frac{3}{2}$
        \item 2 or $\frac{3}{2}$
        \item $-$2 or $\frac{3}{2}$
    \end{enumerate}
    \end{multicols}
             \item A line $y=mx+1$ intersects the circle $(x-3)^2+(y+2)^2=25$ at the points $\vec{P}$ and $\vec{Q}$. If the mid point of the line segment $PQ$ has $X$ coordinate $-\frac{3}{5}$,  then which one of the following options is correct?
                 \hfill$\brak{ 2019}$
                 \begin{multicols}{4}
\begin{enumerate}
                 \item $2\le m<4$
                 \item $-3\le m<-1$
                 \item $4\le m<6$
                 \item $6\le m<8$
             \end{enumerate}
\end{multicols}
\item If a circle passes through the point \brak{a, b} and cuts the circle $x^2+y^2=4$ orthogonally,  then the locus of its centre is
\hfill{(2004)}
\begin{multicols}{2}
\begin{enumerate}
\item $2ax-2by-\brak{a^2+b^2+4}=0$
\item $2ax+2by-\brak{a^2+b^2+4}=0$
\item $2ax-2by+\brak{a^2+b^2+4}=0$
\item $2ax+2by+\brak{a^2+b^2+4}=4$
\end{enumerate}
\end{multicols}
\item Intercept on the line $y=x$ by the circle $x^2+y^2-2x=0$ is $AB$. Equation of the circle on $AB$ as a diameter is 
\hfill{(2004)}
\begin{multicols}{2}
\begin{enumerate}
\item $x^2+y^2+x-y=0$
\item $x^2+y^2-x+y=0$
\item $x^2+y^2+x+y=0$
\item $x^2+y^2-x-y=0$
\end{enumerate}
\end{multicols}
\item If the pair of lines $ax^2+2(a+b)xy+by^2=0$ lie along diameters of a circle and divide the circle into four sectors such that the area of one of the sectors is thrice the area of another sector then
\hfill{(2005)}
\begin{multicols}{2}
\begin{enumerate}
\item $3a^2-10ab+3b^2=0$
\item $3a^2-2ab+3b^2=0$
\item $3a^2+10ab+3b^2=0$
\item $3a^2+2ab+3b^2=0$
\end{enumerate}
\end{multicols}
\item The differential equation of the family of circles with fixed radius 5 units and centre on the line $y=2$ is
\begin{multicols}{2}
\begin{enumerate}
\item $\brak{x-2}y^{\prime 2}=25-\brak{y-2}^2$
\item $\brak{y-2}y^{\prime 2}=25-\brak{y-2}^2$
\item $\brak{y-2}^2y^{\prime 2}=25-\brak{y-2}^2$
\item $\brak{x-2}^2y^{\prime 2}=25-(y-2)^2$
\end{enumerate}
\end{multicols}
	\item Let $\vec{A}$ and $\vec{B}$ be two distinct points on the parabola $y^2=4x$. If
      	      the axis of the parabola touches a circle of radius $r$ having
		$AB$ as diameter, then the slope of the line joining $\vec{A}$ and $\vec{B}$
	     can be 
		\hfill(2010)
		
		 \begin{multicols}{4}
\begin{enumerate}
			\item$\frac{-1}{r}$
			\item$\frac{1}{r}$
			\item$\frac{2}{r}$
			\item$\frac{2}{r}$
	        \end{enumerate}
\end{multicols}
	\item Let $ E_1$  and $ E_2$ be two elipses whose centres are at the orgin.
              The major axes of $E_1$ and $ E_2$ lie along the x-axis and the
              y-axis,respectively. Let S be the circle $x^2+(y-1)^2=2$. The
		straight line $x+y=3$ touches the curves S, $E_1$ and $E_2$ at $\vec{P}$, $\vec{Q}$
		and $\vec{R}$ respectively. Suppose that $PQ=PR$=$\frac{2\sqrt{2}}{3}$. If $e_1$ and
              $e_2$ are the eccentricities of $E_1$ and $E_2$, respectively, then the 
              correct expression(s) is (are) 
	        
		\hfill(2015)
		
		 \begin{multicols}{4}
\begin{enumerate}
			\item $e_1^2+e_2^2=\frac{43}{40}$
			\item $e_1e_2=\frac{\sqrt{7}}{2\sqrt{10}}$
			\item $\abs{ e_1^2-e_2^2}=\frac{5}{8}$
			\item $e_1e_2=\frac{\sqrt{3}}{4}$ 
		 \end{enumerate}
\end{multicols}
\item Consider a circle with its centre lying on focus of the parabola $y^2=2px$ such that it touches the directrix of the parabola. Then a point of intersection of the circle and the parabola is
        \hfill\brak{1995S}
\begin{multicols}{2}
\begin{enumerate}
    \item $\brak{\frac{p}{2},p}$ or $\brak{\frac{p}{2},-p}$
    \item $\brak{\frac{p}{2},-\frac{p}{2}}$
    \item $\brak{-\frac{p}{2},p}$
    \item $\brak{-\frac{p}{2},-\frac{p}{2}}$
\end{enumerate}
\end{multicols}
    \item The centre of the circle passing through the point $\brak{0,1}$ and touching the curve $y = x^{2}$ at $\brak{2, 4}$ is
    \hfill {(1983 )}
    	\begin{multicols}{4}
\begin{enumerate}
    		\item $\brak{\frac{-16}{5}, \frac{27}{10}}$
    		\item $\brak{\frac{-16}{7}, \frac{53}{10}}$
    		\item $\brak{\frac{-16}{5}, \frac{53}{10}}$
    		\item none of these
    	\end{enumerate}
    \end{multicols}
\item The points of intersection of the line $4x-3y-10=0$ and the circle $x^{2}+y^{2}-2x+4y-20=0$ are    \rule{1cm}{0.01pt}.
	\hfill\brak{1983}
\item The equation of the line passing through the points of intersection of the circles\\ $3x^{2}+3y^{2}-2x+12y-9=0$ and $x^{2}+y^{2}+6x+2y-15=0$ is
\rule{1cm}{0.01pt}.
	\hfill\brak{1986}
\item An ellipse intersects the hyperbola $2x^2-2y^2=1$ orthogonally. The eccentricity of the ellipse is reciprocal of that of the hyperbola. If the axes of the ellipse are along the coordinate axes, then \hfill(2009)
		\begin{multicols}{2}
\begin{enumerate}
			\item equation of the ellipse is $x^2+2y^2=2$
			\item the foci of ellipse are $\brak{\pm1,0}$
			\item equation of the ellipse is $x^2+2y^2=4$
			\item the foci of ellipse are $\brak{\pm\sqrt{2},0}$
		\end{enumerate}
\end{multicols}

\item If the circle $x^2+y^2=a^2$ intersects the hyperbola $xy=c^2$ in four points $\vec{P}\brak{x_1, y_1}$, $\vec{Q}\brak{x_2, y_2}$, $\vec{R}\brak{x_3, y_3}$, $\vec{S}\brak{x_4, y_4}$, then
%
    \hfill$\brak{1998}$
\begin{multicols}{2}
\begin{enumerate}
    \item $x_1+x_2+x_3+x_4=0$
    \item $y_1+y_2+y_3+y_4=0$
    \item $x_1x_2x_3x_4=c^4$
    \item $y_1y_2y_3y_4=c^4$
%
\end{enumerate}
\end{multicols}
\end{enumerate}
