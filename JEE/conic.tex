	\begin{enumerate}

\item STATEMENT-1: The curve $y=\frac{-x^2}{2}+x+1$ is symmetric with respect to the line $x=1$.because
STATEMENT-2: A Parabola is symmetric about its axis.
\hfill(2007-3 marks)
%\begin{enumerate}
    \item Statement-1 is True,Statement-2 is True;Statement-2 is a correct explanation for Statement-1\item  Statement-1 is True,Statement-2 is True;Statement-2 is NOT a correct explanation for Statement-1\item Statement-1 is True,Statement-2 is False\item Statement-1 is False,Statement-2 is True
     \item The point of intersection of the tangents at the ends of the latus rectum of the parabola $y^2=4x$ is \ldots.
    \hfill\brak{1994-2-Marks}
    
    \item An ellipse has eccentricity $\frac{1}{2}$ and one focus at the point $p\brak{\frac{1}{2},1}$.Its one directrix is common tangent,nearer to the point $P$, to the circle $x^2+y^2=1$ and the hyperbola $x^2-y^2=1$.The equation of the ellipse, in the standard form, is\dots.
    \hfill\brak{1992-2 Marks} 
\item The line $2x+y=1$ is the tangent to the hyperbola $\frac{x^2}{a^2}-\frac{y^2}{b^2}=1$. If this line passes through the point of intersection of the nearest directrix and the X-axis, then the eccentricity of the hyperbola is
\hfill(2010)
\item Consider the parabola $y^2=8x$ . Let $\Delta_1$ be the area of the triangle formed by the end points of its latus rectum and the point $\Vec{P}$$(\frac{1}{2},2)$ on the parabola and $\Delta_2$ be the area of the triangle formed by drawing tangents at $\Vec{P}$ and at the end points of the latus rectum.Then $\frac{\Delta_1}{\Delta_2}$ is 
\hfill(2011)
\item Let $\Vec{S}$ be the focus of the parabola $y^2=8x$ and let $PQ$ be the common chord of the circle $x^2+y^2-2x-4y=0$ and the given parabola. The area of the triangle $PQS$ is
\hfill(2012)
\item A Vertical line passing through point $(h,0)$ intersects the ellipse   at the points  $\Vec{P}$ and $\Vec{Q}$ . Let the tangents to the ellipse at $\Vec{P}$ and $\Vec{Q}$ meet at the points $\Vec{R}$.If $\Delta(h)$= area of the triangle $PQR$, $\Delta_1$= ma
then 
\hfill(JEE Adv.2013)
\begin{enumerate}
    \item $g(x)$ is continuous but not differentiable at a
    \item $g(x)$ is differentiable on R
    \item $g(x)$ is continuous but not differentiable at b
    \item $g(x)$ is continuous and differentiable either(a) or (b) but not both 
    \end{enumerate}
\item If the normal of the parabola $y^2=4x$ drawn at the end points of its latusrectum are the tangents of th circle $(x-3)^2+(y+2)^2=r^2$, then the value of $r^2$ is
\hfill(JEE Adv.2015)
\item Let the curve C be the mirror image of the parabola $y^2=4x$ with respect to the line $x+y+4=0$.If $\Vec{A}$ and $\Vec{B}$ are the points of the intersection of C with the line $y=-5$,then the distance between $\Vec{A}$and $\Vec{B}$ is
\hfill(JEE Adv.2015)
\item Suppose that the focii of the ellipse $\frac{x^2}{9}+\frac{y^2}{5}=1$ are $(f_1,0)$ and ($f_2$,0) where $f_1>0$ and $f_1<0$.Let $P_1$ and $P_2$ be two parabolas with a common vertex at $(0,0)$ and with foci at ($f_1$,0) and (2$f_2$,0),respectively. Let $T_1$ be a tangent to $P_1$ which passes through (2$f_2$,0) and $T_2$ be a tangent to $P_2$ which passes through ($f_1$,0).If $m_1$ is the slope of $T_1$ and $m_2$ is the slope of $T_2$,then the value of
\hfill(JEE Adv. 2015)
\item A parabola has the origin as its focus and $x=2$ as the directrix. Then the vertex of the parabola is at\hfill(2009)
\begin{enumerate}
    \item $\brak{0,2}$
    \item $\brak{1,0}$
    \item $\brak{0,1}$
    \item $\brak{2,0}$
\end{enumerate}
\item The ellipse $x^2+y^2 = 4$ is inscribed in a rectangle aligned with the coordinate axes, which in turn is inscribed in another ellipse that passes through the point$\brak{4,0}$. Then the equation of the ellipse is:\hfill(2009)
\begin{enumerate}
    \item $x^2+12y^2+16$
    \item $4x^2+48y^2=48$
    \item $4x^2+64y^2=48$
    \item $x^2+16y^2=16$
\end{enumerate}
\item If two tangents drawn from a point $P$ to the parabola $y^2=4x$ are at right angles, then the locus of $P$ is\hfill(2010)
\begin{enumerate}
    \item $2x+1=0$
    \item $x=-1$
    \item $2x-1=0$
    \item $x=1$
\end{enumerate}
\item Equation of the ellipse whose axes are the coordinates and which passes through the point $\brak{-3,1}$ and has eccentricity $\sqrt{\frac{2}{5}}$ is\hfill(2011)
\begin{enumerate}
    \item $5x^2+3y^2-48=0$
    \item $3x^2+5y^2-15=0$
    \item $5x^2+3y^2-32=0$
    \item $3x^2+5y^2-32=0$
\end{enumerate}
\item Statement -1 : An equation of a common tangent to the  parabola $y^2=16\sqrt{3}x$ and the ellipse $2x^2+y^2=4$ is $y=2x+2\sqrt{3}$\\
Statement -2 : If the line $y=mx+\frac{4\sqrt{3}}{m}$ $\brak{m \neq 0}$is a common tangent to the parabola $y^2=16\sqrt{3}x$ and the ellipse $2x^2+y^2=4$, then $m$ satisfies $m^4+2m^2=24$\hfill(2012)
\begin{enumerate}
    \item Statement-1 is false, statement-2 is true.
    \item Statement-1 is true,  statement-2 is true; statement-2 is correct explanation for statement-1.
    \item Statement-1 is true, statement-2 is true; statement-2 is not a correct explanation for statement-1.
    \item Statement-1 is true, statement-2 is false.
\end{enumerate}
\item An ellipse is drawn by taking a diameter of the circle $(x-1)^2+y^2=1$ as its semi minor axis and a diameter of the circle $x^2+(y-2)^2=4$ is semi-major axis. If the centre of the ellipse is at the origin and its axes are the coordinate axes, then the equation of the ellipse is: 

\hfill(2012)
\begin{enumerate}
    \item $4x^2+y^2=4$ 
    \item $x^2+4y^2=8$
    \item $4x^2+y^2=8$
    \item $x^2+4y^2=1$
\end{enumerate}
\item The equation of the circle passing through the foci of the ellipse $\frac{x^2}{16}+\frac{y^2}{9}=1$, and having centre at $\brak{0,3}$ is

\hfill(JEE M 2013)
\begin{enumerate}
    \item $x^2+y^2-6y-7=0$
    \item $x^2+y^2-6y+7=0$
    \item $x^2+y^2-6y-5=0$
    \item $x^2+y^2-6y+5=0$
\end{enumerate}
\item {Given: A circle, $2x^2+2y^2=5$ and a parabola, $y^2=4\sqrt{5}x$.\\
Statement-1: An equation of a common tangent to these curves is $y=x+\sqrt{5}$.\\
Statement -2:If the line, $y=mx+\frac{\sqrt{5}}{m}$ $\brak{m \neq 0}$ is their common tangent, then $m$ satisfies $m^4-3m^2+2=0$.}

\hfill(JEE M 2013)
\begin{enumerate}
    \item Statement-1 is true; Statement-2 is true; Statement-2 is a correct explanation for Statement-1.
    \item Statement-1 is true; Statement-2 is true; Statement-2 is not a correct explanation for Statement-1.
    \item Statement-1 is true; Statement-2 is false.
    \item Statement-1 is false; Statement-2 is true.
\end{enumerate}
\item The locus of the foot of perpendicular drawn from the centre of the ellipse $x^2+3y^2=6$ on an tangent to it is

\hfill(JEE M 2014)
\begin{enumerate}
    \item $(x^2+y^2)^2=6x^2+2y^2$
    \item $(x^2+y^2)^2=6x^2-2y^2$
    \item $(x^2-y^2)^2=6x^2+2y^2$
    \item $(x^2-y^2)^2=6x^2-2y^2$
\end{enumerate}
\item The slope of the line touching both the parabolas $y^2=4x$ and $x^2=-32y$ is

\hfill(JEE M 2014)
\begin{enumerate}
    \item $\frac{1}{8}$
    \item $\frac{2}{3}$
    \item $\frac{1}{2}$
    \item $\frac{3}{2}$
\end{enumerate}
\item Let $O$ be the vertex and $Q$ be any point on the parabola, $x^2=8y$. If the point $P$ divides the line segment $OQ$ internally in the ratio \brak{1:3}, then the locus of $P$ is:

\hfill(JEE M 2015)
\begin{enumerate}
    \item $y^2=2x$
    \item $x^2=2y$
    \item $x^2=y$
    \item $y^2=x$
\end{enumerate}
\item The normal to the curve, $x^2+2xy-3y^2=0$, at $(0,1)$\hfill(JEE M 2015)
\begin{enumerate}
    \item meets the curve again in the third quadrant.
    \item meets the curve again in the fourth quadrant.
    \item doesn't meet the curve again.
    \item meets the curve again in the second quadrant.
\end{enumerate}
\item The area (in sq. units) of the quadrilateral formed by the tangents at the end points of the latera recta to the ellipse $\frac{x^2}{9}+\frac{y^2}{5}=1$ is :\hfill(JEE M 2015)
\begin{enumerate}
    \item $\frac{27}{2}$
    \item $27$
    \item $\frac{27}{4}$
    \item $18$
\end{enumerate}
\item Let $P$ be the point on the parabola, $y^2=8x$ which is a minimum distance from the centre $C$ of the circle, passing through $C$ and having its centre at $P$ is

\hfill(JEE M 2016)
\begin{enumerate}
    \item $x^2+y^2-\frac{x}{4}+2y-24=0$ 
    \item $x^2+y^2-4x+9y+18=0$ 
    \item $x^2+y^2-4x+8y+12=0$
    \item $x^2+y^2-x+4y-12=0$
\end{enumerate}
\item The eccentricity of the hyperbola whose length of the latus rectum is equal to $8$ and the length of its conjugate axis is equal to half of the distance between its foci, is : 
\hfill(JEE M 2016)
\begin{enumerate}
    \item $\frac{2}{\sqrt{3}}$
    \item $\sqrt{3}$
    \item $\frac{4}{3}$
    \item $\frac{4}{\sqrt{3}}$
\end{enumerate}
	\item   Match the following: $\vec{(3,0)}$ is the pt. from which three normals are drawn to the parabola $y^2 = 4x$ which meet the parabola in the points P, Q and R. Then \hfill{(2006 - 6M)}
\begin{multicols}{2}
\textbf{Column I}
\begin{enumerate}
    \item Area of $\Delta$POR 
    \item Radius of circumcircle of $\Delta$PQR
    \item Centroid of $\Delta$POR 
    \item  Circumcentre of $\Delta$PQR 
\end{enumerate} 

\textbf{Column II}
\begin{enumerate}
    \item 2
    \item $\frac{5}{2}$
    \item $\vec{(\frac{5}{2},0)}$
    \item $\vec{(\frac{2}{3},0)}$
\end{enumerate}
\end{multicols}

\item Match the statements in Column I with the properties in Column II and indicate your answer by darkening the appropriate bubbles in the 4 x4 matrix given in the ORS  \hfill{(2007 -6 marks)}

\begin{multicols}{2}
\textbf{Column I}
\begin{enumerate}
    \item  Two intersecting circles
    \item Twomutually external circles
    \item Two circles, one strictly inside the other 
    \item  Two branches of a hyperbola  
\end{enumerate} 

\textbf{Column II}
\begin{enumerate}
    \item have a common tangent 
    \item havea common normal 
    \item do not have a common tangent
    \item do not have a common normal
\end{enumerate}
\end{multicols}


\item Match the conics in column 1with the statements/expressions in column 2\hfill{(2009)}

\begin{multicols}{2}
\textbf{Column I}
\begin{enumerate}
    \item  Circle
    \item  Parabola
    \item  Ellipse 
    \item  Hyperbola  
\end{enumerate} 
\columnbreak
\textbf{Column II}
\begin{enumerate}
	\item The locus of the point $\vec{(h,k)}$ for which the lines $hx + ky =1$ touches the circle $x^2 + y^2 = 4$
    \item Points z in the complex plane satisfying $|z + 2|- |z - 2|= \pm 3$
    \item Points of the conic have parametric representations $x=\sqrt{3}
(\frac{1 - t^2}{1 + t^2})$ ' $ y = \frac{2t}{1 + t^2}$

    \item The eccentricity of the conic lies in the interval $1 \leq x < \infty$

\end{enumerate}
\end{multicols}

\item line L : $y = mx + 3$ meets y- axis at E$\vec{(0,3)}$ and the arc of the arc of the parabola $y^2 = 16x$,$ 0\leq y \leq6$ at the point F$\vec{(x_o,y_o)}$. The tangent to the parabola at F$\vec{(x_o,y_o)}$ intersects the y-axis at G$\vec\vec{(0,y_1)}$. the slope m of the line L is chosen such that the area of the triangle EFG has a local maximum.\hfill{(JEE Adv. 2013)} \\
\textbf{Match List 1 with List 2 and select the correct answer using the code given below the list :}

\begin{minipage}[t]{0.45\textwidth}
\textbf{List 1}
\begin{enumerate}
    \item m=
    \item Maximum area of $\Delta EFG$
    \item $y_o=$
    \item $y_1=$
\end{enumerate}
\end{minipage}%
\hfill
\begin{minipage}[t]{0.45\textwidth}
\textbf{List 2}
\begin{enumerate}
    \item $\frac{1}{2}$
    \item 4
    \item 2
    \item 1
\end{enumerate}
\end{minipage} 


\begin{table}[h!]
\centering
\begin{tabular}{|c|c|c|c|c|}
\hline
S no & P & Q & R & s \\
A &4 & 1 & 2 & 3 \\
B & 3 & 4 & 1 & 2 \\
C & 1 & 3& 2 & 4 \\
D & 1 & 3 & 4 & 2 \\
\hline
\end{tabular}

\end{table}

\textbf{(Qs. 5-7) : By appropriately matching the information given in the three columns of the following table. Column 1,2, and 3 contain conics and points of contact, respectively}

\begin{table}[h!]
\centering
\begin{tabular}{|c|c|c|c|}
\hline
Sno & Column 1       & Column 2 & Column 3 \\
\hline
A & $x^2 + y^2 = a^2$ & $my = m^2x + a$ & $(\frac{a}{m^2})$, $\frac{2a}{m}$ \\
B & $x^2 + a^2y^2 = a^2  $      & $y = mx +a\sqrt{m^2 + 1}$  & $(\frac{-ma}{\sqrt{m^2 + 1}},\frac{a}{\sqrt{m^2 + 1}})   $   \\
C & $y^2=4ax $  & $y = mx + \sqrt{a^2m^2 - 1}$ & ($\frac{-a^2m}{\sqrt{a^2m^2 + 1}}$, $\frac{1}{\sqrt{a^2m^2 + 1}}$)    \\
D & $x^2 - a^2y^2 = a^2$     & $y = mx + \sqrt{a^2m^2 - 1 }$  & $(\frac{-a^2m}{\sqrt{a^2m^2 - 1}}, \frac{-1}{\sqrt{a^2m^2 - 1}})$      \\

\hline
\end{tabular}

\end{table}

\item For $a =\sqrt{2}$, if a tangent is drawn to a suitable conic (Column 1) at the point of contact $\vec{(-1, 1)}$, then which of the following
options is the only correct combination for obtaining its equation?\hfill{(2009)}

\begin{enumerate}[label=(\alph*)]
    \item $(I)(i)(P)$
    \item $(I)(ii)(Q)$
    \item $(II)(ii)(Q)$
    \item $(III)(i)(P)$
\end{enumerate}

\item If a tangent to a suitable conic (column 1) is found to be y = x + 8 and its point of contact is $\vec{(8, 16)}$, then which of the following
options is the only correct combination?\hfill {(JEE Adv. 2018)}

\begin{enumerate}[label=(\alph*)]
    \item $(I)(ii)(Q)$
    \item $(II)(iv)(R)$
    \item $(III)(i)(P)$
    \item $(III)(ii)(Q)$
\end{enumerate}

\item  The tangent to a suitable conic(column 1) at $(\sqrt{3},\frac{1}{2})$ is found to be $\sqrt{3}x + 2y = 4$, then which of the following options is the only correct option ? 

\begin{enumerate}[label=(\alph*)]
    \item $(IV)(iii)(S)$
    \item $(IV)(iv)(S)$
    \item $(II)(iii)(R)$
    \item $(II)(iv)(R)$
\end{enumerate}

 \item Let H : $\frac{x^2}{a^2}-\frac{y^2}{b^2}= 1$, where $a>b>0$, be a hyperbola in xy-plane whose conjugate axis LM subtends an angle of $60^0$ at one of its vertices N. let the area of the triangle LMN be 4$\sqrt{3}$.

\begin{multicols}{2}
\textbf{List 1}
\begin{enumerate}
    \item The length of the conjugate axis of H is
    \item The eccentricity of H is
    \item The distance between the foci of H is
    \item The length of the latus rectum of H is
\end{enumerate}
\columnbreak


\textbf{List 2}
\begin{enumerate}
    \item 8
    \item $\frac{4}{\sqrt{3}}$
    \item $\frac{2}{\sqrt{3}}$
    \item 4

\end{enumerate}
\end{multicols}
\textbf{The correct option from (A,B,C,D) is :}

\begin{table}[h]
\centering

\begin{tabular}{|c|c|c|c|c|}
\hline
Sno & P & Q & R & S \\ \hline
A & 4 & 2 & 1 & 3 \\ \hline
B & 4 & 3 & 1 & 2 \\ \hline
C & 4 & 1 & 3 & 2 \\ \hline
D & 3 & 4 & 2 & 1 \\ \hline
\end{tabular}
\end{table}
    \item The number of the values of $c$ such that the straight line $y=4x+c$ touches the curves $(x^2/4)+y^2=1$ is \hfill(1998 - 2 Marks)\\
	\begin{enumerate}
			\begin{multicols}{2}
	\item $0$
	\item $1$
		\columnbreak
	\item $2$
	\item infinite
			\end{multicols}
	\end{enumerate}

\item If $\vec{P}=\brak{x,y}$, $\vec{F}_{1}=\brak{3,0}$, $\vec{F}_{2}=\brak{-3,0}$ and $16x^2+25y^2=400$, then $\vec{PF}_{1}+\vec{PF}_{2}$ equals \hfill(1998-2 Marks)\\
	\begin{enumerate}
			\begin{multicols}{2}
		\item $8$
		\item $6$
			\columnbreak
		\item $10$
		\item $12$
			\end{multicols}
	\end{enumerate}

\item On the ellipse $4x^2+9y^2=1$, the points at which the tangents are parallel to the line $8x=9y$ are \hfill(1999-3 Marks)\\
	\begin{enumerate}
			\begin{multicols}{2}
		\item $\brak{\frac{2}{5},\frac{1}{5}}$
		\item $\brak{-\frac{2}{5},\frac{1}{5}}$
			\columnbreak
		\item $\brak{-\frac{2}{5},-\frac{1}{5}}$
		\item $\brak{\frac{2}{5},-\frac{1}{5}}$
			\end{multicols}
	\end{enumerate}

\item The equations of the common tangents to the parabola $y=x^2$ and $y=-\brak{x-2}^2$ is/are \hfill(2006-5M,-1)\\
	\begin{enumerate}
			\begin{multicols}{2}
		\item $y=4\brak{x-1}$
		\item $y=0$
			\columnbreak
		\item $y=-4\brak{x-1}$
		\item $y=-30x-50$
			\end{multicols}
	\end{enumerate}

\item Let a hyperbola passes through the focus of the ellipse $\frac{x^2}{25}+\frac{y^2}{16}=1$. The transverse and conjugate axes of this hyperbola coincide with the major and minor axes of the given ellipse, also the product of eccentricities of given ellipse and hyperbola is $1$, then \hfill (2006-5M,-1)\\
	\begin{enumerate}
		\item the equation of hyperbola is $\frac{x^2}{9}-\frac{y^2}{16}=1$
		\item the equation of hyperbola is $\frac{x^2}{9}-\frac{y^2}{25}=1$
		\item focus of hyperbola is $\brak{5,0}$
		\item vertex of hyperbola is $\brak{5\sqrt{3},0}$
	\end{enumerate}

\item Let $\vec{P}\brak{x_{1}, y_{1}}$ and $\vec{Q}\brak{x_{2},y_{2}}$, $y_{1}<0,y_{2}<0$, be the end points of the latus rectum of the ellipse $x^2+4y^2=4$. The equations of parabolas with latus rectum $\vec{PQ}$ are \hfill(2008)\\
	\begin{enumerate}
			\begin{multicols}{2}
		\item $x^2+2\sqrt{3}y=3+\sqrt{3}$
		\item $x^2-2\sqrt{3}y=3+\sqrt{3}$
			\columnbreak
		\item $x^2+2\sqrt{3}y=3-\sqrt{3}$
		\item $x^2-2\sqrt{3}y=3-\sqrt{3}$
			\end{multicols}
	\end{enumerate}

\item In a traingle $\vec{ABC}$ with fixed base $\vec{BC}$, the vertex $\vec{A}$ moves such that
	$$\cos{B}+\cos{C}=4\sin^2{\frac{A}{2}}$$.
		If $a,b$ and $c$ denote the lengths of the sides of the traingle opposite to the angles $A,B$ and $C$, respectively, then \hfill(2009)\\
		\begin{enumerate}
			\item $b+c=4a$
			\item $b+c=2a$
			\item locus of the point $\vec{A}$ is an ellipse
			\item locus of the point $\vec{A}$ is a pair of straight lines
		\end{enumerate}

	\item The tangent $\vec{PT}$ and the normal $\vec{PN}$ to the parabola $y^2=4ax$ at a point $\vec{T}$ and $\vec{N}$, respectively. The locus of the centroid of the traingle $\vec{PTN}$ is a parabola whose \hfill(2009)\\
		\begin{enumerate}
				\begin{multicols}{2}
			\item vertex is $\brak{\frac{2a}{3},0}$
			\item directrix is $x=0$
				\columnbreak
			\item latus rectum is $\frac{2a}{3}$
			\item focus is $\brak{a,0}$
				\end{multicols}
		\end{enumerate}

\item An ellipse intersects the hyperbola $2x^2-2y^2=1$ orthogonally. The eccentricity of the ellipse is reciprocal of that of the hyperbola. If the axes of the ellipse are along the coordinate axes, then \hfill(2009)\\
		\begin{enumerate}
			\item equation of the ellipse is $x^2+2y^2=2$
			\item the foci of ellipse are $\brak{\pm1,0}$
			\item equation of the ellipse is $x^2+2y^2=4$
			\item the foci of ellipse are $\brak{\pm\sqrt{2},0}$
		\end{enumerate}
	\item Let $\vec{A}$ and $\vec{B}$ be two distinct points on the parabola $y^2=4x$. If
      	      the axis of the parabola touches a circle of radius r having
		$AB$ as diameter, then the slope of the line joining $\vec{A}$ and $\vec{B}$
	     can be 
		\hfill(2010)
		
		 \begin{enumerate}
			\item$\frac{-1}{r}$
			\item$\frac{1}{r}$
			\item$\frac{2}{r}$
			\item$\frac{2}{r}$
	        \end{enumerate}
	\item Let the eccentricity of the hyperbola $\frac{x^2}{a^2}-\frac{y^2}{b^2}=1$ be reciprocal to that of the elipse $x^2+4y^2=4$. If the hyperbola
	passes through a focus of the elipse, then 
		\hfill(2011)
		
		 \begin{enumerate}
			\item the equation of the hyperbola is $\frac{x^2}{3}-\frac{y^2}{2}=1$
			\item a focus of the hypebola is $(2,0)$
			\item the eccentricity of the hyperbola is $\sqrt{\frac{5}{3}}$
			\item the equation of the hyperbola is $x^2-3y^2=3$
		 \end{enumerate}
	\item Let L be a normal to the parabola $y^2=4x$. If L passes through the point $(9,6)$, then the L is given by 
		\hfill(2010)
		
		 \begin{enumerate}
			\item $y-x+3=0$
			\item $y+3x-33=0$
			\item $y+x-15=0$
			\item $y-2x+12=0$
		 \end{enumerate}
	\item Tangents are drawn to the hyperbola $\frac{x^2}{9}-\frac{y^2}{4}=1$, parallel to the straight line $2x-y=1$. The points of contact of the tangents to the hyperbola
		are  
		\hfill(2012)
		
		 \begin{enumerate}
			\item $\brak{\frac{9}{2\sqrt{2}},\frac{1}{\sqrt{2}}}$ 
			\item $ \brak{\frac{-9}{2\sqrt{2}},\frac{-1}{\sqrt{2}}}$
			\item $\brak{3\sqrt{3},-2\sqrt{2}}$
			\item $ \brak{-3\sqrt{3},2\sqrt{2}}$
		
		 \end{enumerate}

	\item Let $\vec{P}$ and $\vec{Q}$ be distinct points on the parabola $y^2=2x$ such 
		that a circle with $PQ$ as diameter passes through the vertex
		$\vec{O}$ of the parabola. If $\vec{P}$ lies in the first quadrant and the area
		of the triangle  \(\Delta \)$OPQ$ is 3$\sqrt{2}$, then which of the following is
		(are) the coordinates of $\vec{P}$?  
		\hfill(JEE ADV.2015)
		
		 \begin{enumerate}
			\item $ \brak{4,2\sqrt{2}}$
			\item $\brak{9,3\sqrt{2}}$
			\item $ \brak{\frac{1}{4},\frac{1}{\sqrt{2}}}$
			\item  $ \brak{1,\sqrt{2}}$
		 \end{enumerate}
	\item Let $ E_1$  and $ E_2$ be two elipses whose centres are at the orgin.
              The major axes of $E_1$ and $ E_2$ lie along the x-axis and the
              y-axis,respectively. Let S be the circle $x^2+(y-1)^2=2$. The
		straight line $x+y=3$ touches the curves S, $E_1$ and $E_2$ at $\vec{P}$, $\vec{Q}$
		and $\vec{R}$ respectively. Suppose that $PQ=PR$=$\frac{2\sqrt{2}}{3}$. If $e_1$ and
              $e_2$ are the eccentricities of $E_1$ and $E_2$, respectively, then the 
              correct expression(s) is (are) 
	        
		\hfill(JEE ADV.2015)
		
		 \begin{enumerate}
			\item $e_1^2+e_2^2=\frac{43}{40}$
			\item $e_1e_2=\frac{\sqrt{7}}{2\sqrt{10}}$
			\item $\abs{ e_1^2-e_2^2}=\frac{5}{8}$
			\item $e_1e_2=\frac{\sqrt{3}}{4}$ 
		 \end{enumerate}
	\item Consider the hyperbola H:$x^2-y^2=1$ and a circle S with 
		centre $\vec{N}(x_2,0)$. Suppose that H and S touch each other at a 
	      point $\vec{P}(x_1,y_1)$ with $x_1>0$ and $y_1>0$. The common tangent to H and S at $\vec{P}$ intersects the x-axis at point $\vec{M}$. If $(l,m)$ is the centroid of the triangle $PMN$, then correct expressions(s) is(are)
	      
	      \hfill(JEE ADV.2015)
	      
	       \begin{enumerate}
		      \item $\frac{dl}{dx_1}=1-\frac{1}{3x^2}$ for $x_1>1$ 
		      \item $\frac{dm}{dx_1}=\frac{x_1}{3\sqrt{x_1^2-1}}$ for $x_1>1$ 
		      \item $\frac{dl}{dx_1}=1+\frac{1}{3x^2}$ for $x_1>1$
		      \item $\frac{dm}{dy_1}=\frac{1}{3}$ for $y_1>0$ 
	       \end{enumerate}
      \item The circle $C_1$:$x^2+y^2=3$, with centre at $\vec{O}$, intersects the parabola $x^2=2y$ at the point $\vec{P}$ in the first quadrant. Let the tangent to the circle $C_1$, at $\vec{P}$ touches other two circles $C_2$ and $C_3$ at $\vec{R_2}$ and $\vec{R_3}$, respectively. Suppose $C_2$ and $C_3$ have equal radii $2\sqrt{3}$ and the centres $\vec{Q_2}$ and $\vec{Q_3}$,respectively. If $\vec{Q_2}$ and $\vec{Q_3}$ lie on the y-axis, then 

	      \hfill(JEE ADV.2016)
	      
	       \begin{enumerate}
		      \item $Q_2Q_3=12$
		      \item $R_2R_3=4\sqrt{6}$
		      \item area of the triangle $OR_2R_3$ is $6\sqrt{2}$
		      \item area of the triangle $PQ_2Q_3$ is $4\sqrt{2}$
	       \end{enumerate}
      \item Let $\vec{P}$ be the point on the parabola $y^2=4x$ which is at the shortest distance from the center S of the circle $x^2+y^2-4x-16y+64=0$. Let $\vec{Q}$ be the point on the circle
	      dividing the line segment $SP$ internally. Then 
	      \hfill(JEE ADV.2016)
	      
	       \begin{enumerate}
		      \item $SP=2\sqrt{5}$
		      \item $SQ:QP=(\sqrt{5}+1):2$
		      \item the x-intercept of the normal to the parabola at $\vec{P}$ is $6$
		      \item the slope of the tangent to the circle at $\vec{Q}$ is $\frac{1}{2}$
		      
	       \end{enumerate}
      \item If $2x-y+1=0$ is a tangent to the hyperbola $\frac{x^2}{a^2}-\frac{y^2}{16}=1$ then which of the following cannot be sides of a right angled triangle? 
	      \hfill(JEE ADV.2017)
	      
	       \begin{enumerate}
		      \item $a,4,1$
		      \item $a,4,2$
		      \item $2a,8,1$
		      \item $2a,4,1$
	       \end{enumerate}
      \item If a chord, which is not tangent, of the parabola $y^2=16x$ has equation $2x+y=p$, and midpoint $(h,k)$, then which of the following is(are) possible value(s) of $p$, $h$ and $k$? 
	      \hfill(JEE ADV.2017)
	      
	       \begin{enumerate}
		      \item $p=-2,h=2,k=-4$
		      \item $p=-1,h=1,k=3$
		      \item $p=2,h=3,k=-4$
		      \item $p=5,h=4,k=-3$


	       \end{enumerate}
      \item Consider two straight lines, each of which is tangent to both the circle $x^2+y^2=\frac{1}{2}$
	      and the parabola $y^2=4x$. Let these lines intersect at the point $\vec{Q}$. Consider the elipse whose center is at orgin $\vec{O}(0,0)$ and whose semi-major axis is $OQ$.
	      If the length of the minor axis of the elipse is $\sqrt{2}$, then which of the following statement(s) is(are) TRUE? 
	      \hfill(JEE ADV.2018)
	      
	       \begin{enumerate}
		      \item For the elipse, the eccentricity is $\frac{1}{\sqrt{2}}$ and the length of the latus rectum is $1$

		      \item For the elipse, the eccentricity is $\frac{1}{2}$ and the length of the latus rectum is$\frac{1}{2}$
		      \item The area of the region bounded by the elipse between the lines $x=\frac{1}{\sqrt{2}}$ and $x=1$ is $\frac{1}{4\sqrt{2}}(\pi-2)$
		      \item The area of the region bounded by the elipse between the line $x=\frac{1}{\sqrt{2}}$ and $x=1$ is $\frac{1}{16}(\pi-2)$
	       \end{enumerate}
    \item The equation $\frac{x^2}{1-r}-\frac{y^2}{1+r}=1,r > 1$
represents:
          \hfill  \brak{1981-2 Marks}
\begin{enumerate}
    \item an ellipse    \item      b)   a hyperbola
   \item a circle     \item   d) none of there 
\end{enumerate}
\item Each of the four inequalities give below defines a region in $xy$ plane.One of these four regions does not have the following property.For any two points   $\brak{\frac{x_1+x_2}{2},\frac{y_1+y_2}{2}}$   is also in region.The inequality defining this region is:
         \hfill \brak{1981-2 Marks}
\begin{enumerate}
    \item $x^2+2y^2\le1$
    \item Max $\abs{x},\abs{y}$ $\le1$
    \item $x^2-y^2\le1$
    \item $y^2-x\le0$
\end{enumerate}
\item The equation $2x^2+3y^2-8x-18y+35=k$ represents:
        \hfill \brak{1994}
\begin{enumerate}
    \item no locus if $k\textless0$
    \item an ellipse if $k\textless0$
    \item a point if $k=0$
    \item a hyperbola if $k\textgreater0$ 
\end{enumerate}
\item Let $E$ be the ellipse $\frac{x^2}{9}+\frac{y^2}{4}=1$ and$C$ be the circle $x^2+y^2=9$.Let $P$ and $Q$ be the points $\brak{1,2}$ and $\brak{2,1}$ respectively.Then: 
        \hfill \brak{1994}
\begin{enumerate}
    \item $Q$ lies inside $C$ but outside $E$
    \item $Q$ lies outside both $C$ and $E$
    \item $P$ lies inside both $C$ and $E$
    \item $p$ lies inside $C$ but outside $E$ 
\end{enumerate}
\item Consider a circle with its centre lying on focus of the parabola $y^2=2px$ such that it touches the directrix of the parabola. Then a point of intersection of the circle and the parabola is
        \hfill\brak{1995S}
\begin{enumerate}
    \item $\brak{\frac{p}{2},p}$ or $\brak{\frac{p}{2},-p}$
    \item $\brak{\frac{p}{2},-\frac{p}{2}}$
    \item $\brak{-\frac{p}{2},p}$
    \item $\brak{-\frac{p}{2},-\frac{p}{2}}$
\end{enumerate}
\item The radius of the circle passing through the foci of the ellipse $\frac{x^2}{16}+\frac{y^2}{9}=1$. and having its centre at $\brak{0,3}$ is:
       \hfill \brak{1995S}
\begin{enumerate}
    \item $4$
    \item $3$
    \item $\sqrt{\frac{1}{2}}$
    \item $\frac{7}{2}$
\end{enumerate}
\item Let $P\brak{a\sec\theta,b\tan\theta}$ and $Q\brak{a\sec\phi,b\tan \phi}$,where $\theta+\phi=\pi/2$, be two points on the hyperbola $\frac{x^2}{a^2}-\frac{y^2}{b^2}=1$.If $\brak{h,k}$ is the point 0f intersection of the normals at $P$ and $Q$, then $K$ equal to 
      \hfill \brak{1999-2 Marks}
\begin{enumerate}
    \item $\frac{a^2+b^2}{a}$
    \item $-\brak{\frac{a^2+b^2}{a}}$
    \item $\frac{a^2+b^2}{b}$
    \item $-\brak{\frac{a^2+b^2}{b}}$
\end{enumerate}
\item If $x=9$ is the chord of contact of the hyperbola $x^2-y^2=9$,then the equation of the corresponding pair of tangents is:
    \hfill \brak{1999-2 Marks}
\begin{enumerate}
    \item $9x^2-8y^2+18x-9=0$
    \item $9x^2-8y^2-18x+9=0$
    \item $9x^2-8y^2-18x-9=0$
    \item $9x^2-8y^2+18x+9=0$
\end{enumerate}
\item The curve described parametrically by $x=t^2+t+1$,$y=t^2-t+1$ represents
     \hfill\brak{1999-2 Marks}
\begin{enumerate}
    \item a pair of straight lines
    \item an ellipse
    \item a parabola
    \item a hyperbola
\end{enumerate}
\item If $x+y=k$ is normal $y^2=12x$,then $K$ is
     \hfill\brak{2000s}
\begin{enumerate}
    \item $3$
    \item $9$
    \item $-9$
    \item $-3$
\end{enumerate}
\item If the line $x-1=0$ is the directrix of parabola $y^2-kx+8=0$,than one of the values of $K$ is
      \hfill\brak{2000S}
\begin{enumerate}
    \item $\frac{1}{8}$
    \item $8$
    \item $4$
    \item $\frac{1}{4}$ 
\end{enumerate}
\item The equation of the common tangent touching the circle $\brak{x-3}^2-kx+8=0$ and the parabola $y^2=4x$ above the $x$-axis is 
      \hfill\brak{2000s}
\begin{enumerate}
    \item $\sqrt{3}y=3x+1$
    \item $\sqrt{3}y=-\brak{x+3}$
    \item $\sqrt{3}y=x+3$
    \item $\sqrt{3}y=-\brak{3x+1}$
\end{enumerate}
    \item The equation of the directrix of the parabola $y^2+4y+4x+2=0$ is 
     \hfill \brak{2001S}
\begin{enumerate}
    \item $x=-1$
    \item $x=1$
    \item $x=-\frac{3}{2}$
     \item $x=\frac{3}{2}$
\end{enumerate}
\item If a $>$ 2b $>$ 0 then the positive value of m for which       $y=mx-b\sqrt{1+m^{2}} $ is a common tangent to $x^{2} + y^{2} = b^{2} $and  $(x-a)^{2} + y^{2} = b^{2}$ is   \hfill {(2002S)}
\begin{multicols}{2}
\begin{enumerate}
    \item $\frac{2b}{\sqrt{a^{2}-4b^{2}}}$\\\\
    \item $\frac{2b}{a-2b}$
    \item $\frac{\sqrt{a^{2}-4b^{2}}}{2b}$\\\\
    \item $\frac{b}{a-2b}$
\end{enumerate} 
\end{multicols}
\item The locus of the mid-point of the line segment joining the focus to a moving point on the parabola $y^{2} = 4ax$ is another parabola with directrix \hfill{(2002S)}
\begin{multicols}{4}
 \begin{enumerate}
    \item x=-a
    \item x=-a/2
    \item x=a
    \item x=a/2
 \end{enumerate}
\end{multicols}
\item The equation of the common tangent to the curves $y^{2}=8x$ and $xy=-1$ is \hfill{(2002S)}
\begin{multicols}{2}
\begin{enumerate}
    \item $3y=9x+2$\\\\
    \item $y=2x+1$
    \item $2y=x+8$\\\\
    \item $y=x+2$
\end{enumerate}
\end{multicols}
\item The area of the quadrilateral formed by the tangents at the end points of the latus rectum to the ellipse $\frac{x^{2}}{9}+\frac{y^{2}}{5}=1$, is \hfill{(2003S)}
\begin{multicols}{2}
 \begin{enumerate}
    \item 27/4 sq.units\\\\
    \item 9 sq.units
    \item 27/2 sq.units\\\\
    \item 27 sq.units
 \end{enumerate}
\end{multicols}
\item The focal chord to $y^{2}=16x$ is tangent to $(x-6)^{2}+y^{2}=2$,then the possible values of the slope of this chord,are \hfill{(2003S)}
\begin{multicols}{2}
\begin{enumerate}
    \item ${-1,1}$\\\\
    \item ${-2,2}$
    \item ${-2,-1/2}$\\\\
    \item ${2,-1/2}$
\end{enumerate}
\end{multicols}{2}
\item For hyperbola $\frac{x^{2}}{\cos^{2}\alpha}-\frac{y^{2}}{\sin^{2}\alpha}=1$ which of the following remains constant with change in '$\alpha$'

\hfill{(2003S)}
\begin{multicols}{2}
\begin{enumerate}
    \item abscissae of vertices\\\\
    \item abscissae of foci
    \item eccentricity\\\\
    \item directrix
\end{enumerate}
\end{multicols}{2}
\item If tangents are drawn to ellipse $x^{2}+2y^{2}=2$,then the locus of the mid-point of the intercept made by the tangents between the coordinate axes is 

\hfill{(2004S)}
\begin{multicols}{2}
\begin{enumerate}
    \item $\frac{1}{2x^{2}}+\frac{1}{4y^{2}}$ \\\\
    \item $\frac{1}{4x^{2}}+\frac{1}{2x^{2}}$ 
    \item $\frac{x^{2}}{2}+\frac{y^{2}}{4}=1$ \\\\
    \item $\frac{x^{2}}{4}+\frac{y^{2}}{2}=1$ 
\end{enumerate}
\end{multicols}
\item The angle between the tangents drawn from the point ${(1,4)}$ to the parabola $y^{2}=4x$ is 

\hfill{(2004S)}
\begin{multicols}{4}
\begin{enumerate}
    \item $\pi/6$ 
    \item $\pi/4$ 
    \item $\pi/3$
    \item $\pi/2$
\end{enumerate}
\end{multicols}
\item If the line $2x+\sqrt{6}y=2$ touches the hyperbola $x^{2}-2y^{2}=4$,then the point of contact is \hfill{(2004S)}
\begin{multicols}{2}
\begin{enumerate}
    \item ${(-2,\sqrt{6})}$\\\\
    \item ${(-5,2\sqrt{6})}$
    \item ${(\frac{1}{2},\frac{1}{\sqrt{6}})}$\\\\
    \item ${(4,-\sqrt{6})}$
\end{enumerate}
\end{multicols}
\item The minimum area of the triangle formed by the tangent to the $\frac{x^{2}}{a^{2}}+\frac{y^{2}}{b^{2}}=1$ \& coordinate axes is \hfill{(2005S)}
\begin{multicols}{2}
\begin{enumerate}
    \item ab sq. units\\\\
    \item $\frac{a^{2}+b^{2}}{2}$ sq. units
    \item $\frac{(a+b)^{2}}{2}$ sq. units\\\\
    \item $\frac{a^{2}+ab+b^{2}}{3}$ sq. units
\end{enumerate}
\end{multicols}
\item Tangent to the curve $y=x^{2}+6$ at a point ${(1,7)}$ touches the circle $x^{2}+y^{2}+16x+12y+c=0$ at a point Q.Then the coordinates of Q are \hfill{(2005S)}
\begin{multicols}{2}
\begin{enumerate}
    \item${(-6,-11)}$\\\\
    \item${(-9,-13)}$
    \item${(-10,-15)}$\\\\
    \item${(-6,-7)}$
\end{enumerate}
\end{multicols}
\item The axis of the parabola is along the line y=x and the distance of its vertex and focus from  origin are $\sqrt2$ and $2\sqrt2$  respectively.If the vertex and focus both lie in the first quadrant,then the equation of the parabola is \hfill{(2006-3M,-1)}
\begin{multicols}{2}
\begin{enumerate}
    \item $(x+y)^{2}=(x-y-2)$\\\\
    \item $(x-y)^{2}=(x+y-2)$
    \item $(x-y)^{2}=4(x+y-2)$\\\\
    \item $(x-y)^{2}=8(x+y-2)$
\end{enumerate}
\end{multicols}
\item A hyperbola, having the transverse axis of length $2\sin\theta$, is confocal with the ellipse $3x^{2}+4y^{2}=12$. Then its equation is \hfill{(2007-3 marks)}
\begin{enumerate}
    \item ${x^{2}\cosec^{2}\theta}-{y^{2}\sec^{2}\theta=1}$\\
    \item $x^{2}\sec^{2}\theta-y^{2}\cosec^{2}\theta=1$\\
    \item $x^{2}\sin^{2}\theta-y^{2}\cos^{2}\theta=1$\\
    \item $x^{2}\cos^{2}\theta-y^{2}\sin^{2}\theta=1$\\
\end{enumerate}
\item Let a and b be non-zero real numbers.Then,the equation $(ax^{2}+by^{2}+c)(x^{2}-5xy+6y^{2}=0)$ represents \hfill{(2008)}
\begin{enumerate}
    \item four straight lines,when c=0 and a,b are of the same sign.\\
    \item two straight lines and a circle,when a=b,and c is of sign opposite to that of a\\
    \item two straight lines and a hyperbola,when a and b are of the same sign and c is of opposite to that of a\\
    \item a circle and a ellipse,when a and b are of the same sign and c is of sign opposite to that of a\\
\end{enumerate}
\item Consider a branch of the hyperbola\\$x^{2}-2y{2}-2\sqrt{2}x-4\sqrt{2}y-6=0$\\with vertex at a point A.Let B be one of the end points of its latus rectum. If C is the focus of the hyperbola nearest to the point A,then the area of the triangle ABC is \hfill{(2008)}
\begin{multicols}{4}
\begin{enumerate}
    \item $1-\sqrt{\frac{2}{3}}$
    \item $\sqrt{\frac{3}{2}}-1$
    \item $1+\sqrt{\frac{2}{3}}$
    \item $\sqrt{\frac{3}{2}}+1$
\end{enumerate}
\end{multicols}
\item The line passing through the extremity $\vec{A}$ of the major axis and extremity $\vec{B}$ of the minor axis of the ellipse\\$x^{2}+9y^{2}=9$\\meets its auxillary circle at the point $\vec{M}$. Then the area of the triangle with the vertices at $\vec{A}$,$\vec{M}$ and the origin $\vec{O}$ is \hfill{(2009)}
\begin{multicols}{4}
\begin{enumerate}
    \item $\frac{31}{10}$
    \item $\frac{29}{10}$
    \item $\frac{21}{10}$
    \item $\frac{27}{10}$
\end{enumerate}
\end{multicols}
    \item The normal at a point $\vec{P}$ on the ellipse $x^2 +4y^2=16$ meets the $x$-axis at $\vec{Q}$. If $\vec{M}$ is the mid point of the line segment $\vec{PQ}$, then the locus of $\vec{M}$ interests the latusrectums of the given ellipse at the points
	
	\hfill (2009)
		\begin{enumerate}
				\begin{multicols}{2}
			\item $\brak{\pm\frac{3\sqrt{5}}{2},\pm\frac{2}{7}}$
			\item $\brak{\pm\frac{3\sqrt{5}}{2},\pm\sqrt{\frac{19}{4}}}$
				\columnbreak
			\item $\brak{\pm2\sqrt{3},\pm\frac{1}{7}}$
			\item $\brak{\pm2\sqrt{3},\pm\frac{4\sqrt{3}}{7}}$
				\end{multicols}
		\end{enumerate}
		
\item The locus of the orthocentre of the traingle formed by the lines
		
			$$\brak{1+p}x-py+p\brak{1+p}=0,$$
			$$\brak{1+q}x-qy+q\brak{1+q}=0,$$
		and $y=0$, where $p \neq q$, is
		\hfill(2009)
\begin{enumerate}
		\begin{multicols}{2}
	\item a hyperbola
	\item a parabola
		\columnbreak
	\item an ellipse
	\item a straight line
		\end{multicols}   
\end{enumerate}

\item Let $\vec{P}\brak{6,3}$ be a point on the hyperbola $\frac{x^2}{a^2}-\frac{y^2}{b^2}=1$. If the normal at the point $\vec{P}$ intersects the $x$-axis at $\brak{9,0}$, then the eccentricity of the hyperbola is 
	\hfill (2011)\\
		\begin{enumerate}
				\begin{multicols}{2}
			\item$\sqrt{\frac{5}{2}}$
			\item$\sqrt{\frac{3}{2}}$
				\columnbreak
			\item$\sqrt{2}$
			\item$\sqrt{3}$
				\end{multicols}
		\end{enumerate}

	\item Let $\brak{x,y}$ be any point on the parabola $y^2=4x$. Let $\vec{P}$ be the point that divides the line segment from $\brak{0,0}$ to $\brak{x,y}$ in the ratio $1:3$. Then the locus of $\vec{P}$ is  \hfill(2011)\\
		\begin{enumerate}
				\begin{multicols}{2}
			\item $x^2=y$
			\item $y^2=2x$
				\columnbreak
			\item $y^2=x$
			\item $x^2=2y$
				\end{multicols}
		\end{enumerate}

	\item The ellipse $E_{1}$:$\frac{x^2}{9}+\frac{y^2}{4}=1$ is inscribed in a rectangle $\vec{R}$ whose sides are parallel to the coordinate axes. Another ellipse $E_{2}$ passing through the point $\brak{0,4}$ circumscribes the rectangle $\vec{R}$.The eccentricity of the ellipse $E_{2}$ is \hfill(2012)\\

		\begin{enumerate}
				\begin{multicols}{2}
			\item $\frac{\sqrt{2}}{2}$
			\item $\frac{\sqrt{3}}{2}$
				\columnbreak
			\item $\frac{1}{2}$
			\item $\frac{3}{4}$
				\end{multicols}|
		\end{enumerate}

	\item The common tangents to the circie $x^2+y^2=2$ and the parabola $y^2=8x$ touch the circle at the points $\vec{P}$, $\vec{Q}$ and the parabola at the points $\vec{R}$, $\vec{S}$.Then the area of the quadrilateral $\vec{PQRS}$ is \hfill(JEE Adv. 2014)\\
		\begin{enumerate}
				\begin{multicols}{2}
			\item $3$
			\item $6$
				\columnbreak
			\item $9$
			\item $15$
				\end{multicols}
		\end{enumerate}
\item A hyperbola passes through point $\vec{P}\brak{\sqrt2,\sqrt2}$  and  has  foci  at $\brak{\pm2,0}$. Then  the  tangent  to  this  hyperbola at $\vec{P}$ also passes through the point :
      \hfill{(JEE M 2017)} 
	\begin{enumerate}
    		\item  $\brak{-\sqrt2,-\sqrt3}$
    		\item  $\brak{3\sqrt2,2\sqrt3}$
    		\item  $\brak{2\sqrt2,3\sqrt3}$
    		\item  $\brak{\sqrt3,\sqrt2}$
	\end{enumerate}
\item  The radius of a circle, having minimum area, which touches the curve $y=4-x^2$ and the lines, $y=\abs{x}$ is : 
   \hfill{(JEE M 2018)}
	\begin{enumerate}
     		\item $4\brak{\sqrt2+1}$
     		\item $2\brak{\sqrt2+1}$
     		\item $2\brak{\sqrt2-1}$
     		\item $4\brak{\sqrt2-1}$
	\end{enumerate}
\item Tangents are drawn to the hyperbola $4x^2-y^2=36$ at the points $\vec{P}$ and $\vec{Q}$. If  these tangents intersect  at the point $\vec{T}\brak{0,3}$ then the area (in sq.units) of $\Delta$ PTQ is:
     \hfill{(JEE M 2018)}
	\begin{enumerate}
     		\item $54\sqrt3$
     		\item $60\sqrt3$
     		\item $36\sqrt3$ 
     		\item $45\sqrt5$
	\end{enumerate}
\item Tangent and normal are drawn at $\vec{P}\brak{16,16}$ on the parabola $y^2=16x$,
which is intersect the axis of the parabola at $\vec{A}$ and $\vec{B}$, respectively. If $\vec{C}$ is the centre of the circle through the points $\vec{P}$, $\vec{A}$ and $\vec{B}$ and $\angle$ CPB=$\theta$, then the value of $\tan{\theta}$ is :
     \hfill{(JEE M 2018)}
	\begin{enumerate}
    		\item $2$
    		\item $3$
    		\item $\frac{4}{3}$
    		\item $\frac{1}{2}$
	\end{enumerate}
\item  Two sets $A$ and $B$ are as under:
$A=\cbrak{\brak{a,b}\in\mathbb{R}\times\mathbb{R}:\abs{a-5}<1\text{ and} \abs{b-5}<1}$
$B=\cbrak{\brak{a,b}\in\mathbb{R}\times\mathbb{R}:4(a-6)^2\text{+}9(b-5)^2\leq36}$
    \hfill{(JEE M 2018)}
	\begin{enumerate}
		\item $A\subset B$ 
		\item $A\cap B$
		\item neither $A\subset B$ nor $B\subset A$
		\item $B\subset A$
	\end{enumerate}
\item If the tangent at $\brak{1,7}$ to the curve $x^2=y-6$ touches the circle $x^2+y^2+16x+12y+c=0$ then the value of c is :
       \hfill{(JEEM 2018)}
	\begin{enumerate}
    		\item $185$
    		\item $85$
    		\item $95$
    		\item $195$
	\end{enumerate} 
\item Axis of a parabola lies along X-axis. If its vertex and focus are at a distance $2$ and $4$ respectively from origin, on the positive X-axis then which of the following points does not lie on it? 
     \hfill{(JEE M 2018)} 
	\begin{enumerate}
    		\item $\brak{5,2\sqrt6}$
    		\item $\brak{8,6}$
    		\item $\brak{6,4\sqrt2}$
    		\item $\brak{4,-4}$
	\end{enumerate}
\item Let $0<\theta<\pi/2$. If the eccentricty of the hyperbola $\frac{x^2}{\cos^2{\theta}} - \frac{y^2}{\sin^2{\theta}} = 1$ is greater than $2$, then the length of its latus rectum lies in the interval:
          \hfill{(JEE M 2019-9 Jan(M)}
	\begin{enumerate}
    		\item $\brak{5,\infty}$
    		\item $\lbrak{\frac{3}{2}},\rsbrak{3}$
    		\item $\lbrak{2},\rsbrak{3}$ 
    		\item $\lbrak{1},\rsbrak{\frac{3}{2}}$
	\end{enumerate} 
\item Equation of a common tangent to the circle $x^2+y^2-6x=0$ and the parabola $y^2=4x$, is:
     \hfill{( JEE M 2019-9 Jan(M))}
	\begin{enumerate}
    		\item $2\sqrt{3}y=12x+1$ 
    		\item $\sqrt{3}y=x+3$
    		\item $2\sqrt{3}y=-x-12$ 
    		\item $\sqrt{3}y=3x+1$
	\end{enumerate}   
\item If the line $y=mx+7\sqrt{3}$ is normal to the hyperbola $\frac{x^2}{24}$-$\frac{y^2}{18}$ then a value of m is: 
     \hfill{(JEEM 2019-9 April(M))}
	\begin{enumerate}
    		\item $\frac{\sqrt{5}}{2}$ 
    		\item $\frac{\sqrt{15}}{2}$
    		\item $\frac{2}{\sqrt5}$
    		\item $\frac{3}{\sqrt5}$
	\end{enumerate}
\item If one end of a focal chord of the parabola, $y^2=16x$ is at $\brak{1,4}$,then the length of this focal chord is :
     \hfill{( JEE M 2019-9 Jan(M))}
	\begin{enumerate}
    		\item $25$
    		\item $22$
    		\item $24$
    		\item $20$
	\end{enumerate}    
\item  The equation of the locus of the point whose distances from the point $\Vec{P}$ and the line $AB$ are equal, is
\end{enumerate}

\begin{enumerate}
     \item $9x^2+y^2-6xy-54x-62y+241=0$
     \item $x^2+9y^2+6xy-54x-62y-241=0$
     \item $9x^2+9y^2-6xy-54x-62y-241=0$
     \item $x^2+y^2-2xy+27x+31y-120=0$
\end{enumerate}
\subsection*{Passage 4} 
\begin{enumerate}
\item[] Let $PQ$ be a focal chord of the parabola $y^2=4ax$.The tangents to the parabola at $\Vec{P}$ and $\Vec{Q}$ meet at a point lying on the line $y=2x+a$,$a>0$
\item Length of the chord $PQ$ is

\hfill(JEE Adv.2013)        
\begin{enumerate}
    \item $7a$
    \item $5a$
    \item $2a$
    \item $3a$
\end{enumerate}

\item If chord $PQ$ subtends an angle $\theta$  at the vertex of $y^2=4ax$
\hfill(JEE Adv.2013)

\begin{enumerate}
    \item $\frac{2}{3}\sqrt{7}$
    
    \item $\frac{-2}{3}\sqrt{7}$
    
    \item $\frac{2}{3}\sqrt{5}$
    
    \item $\frac{-2}{3}\sqrt{5}$
\end{enumerate}
\end{enumerate}
\subsection*{Passage 5}
\begin{enumerate}
\item[] Let $a$,$r$,$s$,$t$ be nonzero real numbers. Lets $\Vec{P}$$(at^2,2as)$,$\Vec{Q}$,$\Vec{R}$$(as^2,2as)$ be distinct points on the parabola $y^2=4ax$.suppose that PQ is the focal chord and lines $QR$ and $PK$ are parallel,where $K$ is the point $(2a,0)$
\end{enumerate}
\begin{enumerate}
\item The value of $r$ is 
\hfill(JEE Adv.2014)
\begin{enumerate}
    \item $\frac{-1}{t}$ 
    \item $\frac{t^2+1}{t}$
    \item $\frac{1}{t}$
    \item $\frac{t^2-1}{t}$
\end{enumerate}
\item If $st=1$, then the tangent at $\Vec{P}$ and the normal at $\Vec{M}$ to the
parabola meet at a point whose ordinate is 
\begin{enumerate}
    \item $\frac{a(t^2+1)^2}{t^3}$
    \item $\frac{a(t^2+1)}{2t^3}$
    \item $\frac{1}{t}$
    \item $\frac{t^2-1}{t}$
    \end{enumerate}
\end{enumerate}
\subsection*{Passage 6}
\begin{enumerate}
\item[] Let $\vec{F_1}$($x_1$,0) and  $\vec{F_2}$($x_2$,0) for $x_1<0$ and $x_2>0$, be the focii of the ellipse $\frac{x^2}{9}+\frac{y^2}{8} =1$. Suppose a parabola having vertex at the origin and focus at $\vec{F_2}$ intersects the ellipse at point $\Vec{M}$ in the first quadrant and the point $\Vec{N}$ in the first quadrant.
\item The orthocentre of the triangle $F_1$MN is

         \hfill(JEE Adv. 2016)
\begin{enumerate}
     \item $(\frac{-9}{10},0)$   
     \item $(\frac{2}{3},0)$     
     \item $($9/10$,0)$ 
    \item $(\frac{2}{3},\sqrt{6})$ 
    \end{enumerate}
\item If the tangents to the ellipse at $\Vec{M}$ and $\Vec{N}$ meet at $\Vec{R}$ and the normal to the parabola at $\Vec{M}$ meets the X-Axis at $\Vec{Q}$, the the ratio of area of the triangle $MQR$ to the area of the quadrilateral M$F_1$N$F_2$ is
\hfill(JEE Adv.2016)
\begin{enumerate}
    \item $3:4$
    \item $4:5$
    \item $5:8$
    \item $2:3$
\end{enumerate}
    \item Suppose that the normals drawn at three different points on the parabola $y^2=4x$ pass through the point$(h,k)$. Show that $h>2$. 
		\hfill(1981-4 Marks)
		
	\item $\vec{A}$ is a point on the parabola $y^2=4ax$. The normal at $\vec{A}$ cuts the parabola again at point $\vec{B}$. If $AB$ subtends a right angle at the vertex of the parabola. Find the slope of $AB$.  
		\hfill(1982-4 Marks)
		
	\item Three normals are drawn from the point $(c,0)$ to the curve $y^2=x$. Show that $c$ must be greater than $\frac{1}{2}$. One normal is always the x-axis. Find $c$ for which the other two normals are perpendicular to each other. 
	      \hfill(1991-4 Marks)
		
      \item Through the vertex $\vec{O}$ of parabola $y^2=4x$, chords $OP$ and $OQ$ are drawn at right angles to one another. Show that for all positions of $\vec{P}$, $PQ$ cuts the axis of the parabola at a fixed point. Also find the locus of the middle point of $PQ$. 

		\hfill(1994-4 Marks)
		
      \item Show that the locus of a point that divides a chord of slope $2$ of the parabola $y^2=4x$ internally in the ratio $1:2$ is a parabola. Find the vertex of this parabola. 

	      \hfill(1995-5 Marks)
\item Let '$d$' be the perpendicular distance from the centre of the ellipse $\frac{x^2}{a^2}+\frac{y^2}{b^2}=1$ to the tangent drawn at a point $\vec{P}$ on the ellipse. If $\vec{F_1}$ and $\vec{F_2}$ are the two $foci$ of the ellipse, then show that $\brak{PF_1-PF_2}^2=4a^2\brak{1-\frac{b^2}{d^2}}$. \hfill\brak{1995- 5 marks}

\item Points $\vec{A}$, $\vec{B}$ and $\vec{C}$ lie on a parabola $y^2=4ax$. The tangents to the parabola at $\vec{A}$, $\vec{B}$ and $\vec{C}$ taken in pairs, intersect at points $\vec{P}$, $\vec{Q}$ and $\vec{R}$. Determine the ratios of the areas of triangles $ABC$ and $PQR$. \hfill\brak{1996- 3 marks}

\item From a point $\vec{A}$ common tangents are drawn to the circle $x^2+y^2=\frac{a^2}{2}$ and the parabola $y^2=4ax$. Find the area of the quadrilateral formed by the common tangents, the chord of contact of the circle, and the chord of contact of the parabola. \hfill\brak{1996- 2 marks}

\item A tangent to the ellipse $x^2+4y^2=4$ meets the ellipse $x^2+2y^2=6$ at $\vec{P}$ and $\vec{Q}$. Prove that the tangents at $\vec{P}$ and $\vec{Q}$ of the ellipse $x^2+2y^2=6$ are at right angles. \hfill\brak{1997- 5 marks}

\item The angle between a pair of tangents drawn from a point $\vec{P}$ to the parabola $y^2=4ax$ is 45\degree. Show that the locus of the point $\vec{P}$ is a hyperbola. \hfill\brak{1998- 8 marks}

\item Consider the family of circles $x^2+y^2=r^2$, $2<r<5$. If in the first quadrant, the common tangent to a circle of this family and the ellipse $4x^2+25y^2=100$ meets the coordinate axes at $\vec{A}$ and $\vec{B}$, then find the equation of the locus of the midpoint of $AB$. \hfill\brak{1999- 10 marks}

\item Find the coordinates of all the points $\vec{P}$ on the ellipse $\frac{x^2}{a^2}+\frac{y^2}{b^2}$=1, for which the area of the triangle $PON$ is maximum, where $\vec{O}$ denotes the origin and $\vec{N}$, the foot of the perpendicular from $\vec{O}$ to the tangent at $\vec{P}$. \hfill\brak{1999- 10 marks}

\item Let $ABC$ be an equilateral triangle inscribed in the circle $x^2+y^2=a^2$. Suppose perpendiculars from $\vec{A}$, $\vec{B}$, $\vec{C}$ to the major axis of the ellipse $\frac{x^2}{a^2}+\frac{y^2}{b^2}$=1, $(a>b)$ meet the ellipse respectively at $\vec{P}$, $\vec{Q}$, $\vec{R}$ such that $\vec{P}$, $\vec{Q}$, $\vec{R}$ lie on the same side of the major axis as $\vec{A}$, $\vec{B}$, $\vec{C}$ respectively. Prove that the normals to the ellipse drawn at the points $\vec{P}$, $\vec{Q}$, and $\vec{R}$ are concurrent. \hfill\brak{2000- 7 marks}

\item Let $C_1$ and $C_2$ be respectively, the parabolas $x^2=y-1$ and $y^2=x-1$. Let $\vec{P}$ be any point on $C_1$ and $\vec{Q}$ be any point on $C_2$. Let $P_1$ and $Q_1$ be the reflections of $\vec{P}$ and $\vec{Q}$ respectively with respect to the line $y=x$. Prove that $P_1$ lies on $C_2$, $Q_1$ lies on $C_1$, and $PQ \geq \text{min}({PP_1, QQ_1})$. Hence or otherwise determine points $P_0$ and $Q_0$ on the parabolas $C_1$ and $C_2$ respectively such that $P_0Q_0 \leq PQ$ for all pairs of points $(\vec{P},\vec{Q})$ with $\vec{P}$ on $C_1$ and $\vec{Q}$ on $C_2$. \hfill\brak{2000- 10 marks}

\item Let $\vec{P}$ be a point on the ellipse $\frac{x^2}{a^2}+\frac{y^2}{b^2}=1$, $0<b<a$. Let the line parallel to the y-axis passing through $\vec{P}$ meet the circle $x^2+y^2=a^2$ at the point $\vec{Q}$ such that $\vec{P}$ and $\vec{Q}$ are on the same side of the x-axis. For two positive real numbers $r$ and $s$, find the locus of the point $\vec{R}$ on $PQ$ such that $PR
= r$ as $\vec{P}$ varies over the ellipse. \hfill\brak{2001- 4 marks}

\item Prove that, in an ellipse, the perpendicular from a focus upon any tangent and the line joining the center of the ellipse to the point of contact meet on the corresponding directrix. \hfill\brak{2002- 5 marks}

\item Normals are drawn from the point $\vec{P}$ with slopes $m_1, m_2, m_3$ to the parabola $y^2=4x$. If the locus of $\vec{P}$ with $m_1m_2=\alpha$ is a part of the parabola itself, then find $\alpha$. \hfill\brak{2003- 4 marks}

\item A tangent is drawn to the parabola $y^2-2y-4x+5=0$ at a point $P$ which cuts the directrix at the point $\vec{Q}$. A point $\vec{R}$ is such that it divides $QP$ externally in the ratio 1:2. Find the locus of the point $\vec{R}$. \hfill\brak{2004 - 4 marks}

\item Tangents are drawn from any point on the hyperbola $\frac{x^2}{9}-\frac{y^2}{4}=1$ to the circle $x^2+y^2=9$. Find the locus of the midpoint of the chord of contact. \hfill\brak{2005 - 4 marks}

\item Find the equation of the common tangent in the 1st quadrant to the circle $x^2+y^2=16$ and the ellipse $\frac{x^2}{25}+\frac{y^2}{4}$=1. Also, find the length of the intercept of the tangent between the coordinate axes. \hfill\brak{2005 - 4 marks} 

\item If the circle $x^2+y^2=a^2$ intersects the hyperbola $xy=c^2$ in four points $\vec{P}\brak{x_1, y_1}$, $Q\brak{x_2, y_2}$, $R\brak{x_3, y_3}$, $S\brak{x_4, y_4}$, then
%
    \hfill$\brak{1998}$
\begin{multicols}{2}
\begin{enumerate}
    \item $x_1+x_2+x_3+x_4=0$
    \item $y_1+y_2+y_3+y_4=0$
    \item $x_1x_2x_3x_4=c^4$
    \item $y_1y_2y_3y_4=c^4$
%
\end{enumerate}
\end{multicols}
    \item The centre of the circle passing through the point $\brak{0.1}$ and touching the curve y = $x^{2}$ at $\brak{2, 4}$ is
    \hfill {(1983 )}
    	\begin{multicols}{2}
\begin{enumerate}
    		\item $\brak{\frac{-16}{5}, \frac{27}{10}}$
    		\item $\brak{\frac{-16}{7}, \frac{53}{10}}$
    		\item $\brak{\frac{-16}{5}, \frac{53}{10}}$
    		\item none of these
    	\end{enumerate}
    \end{multicols}
\item The points of intersection of the line $4x-3y-10=0$ and the circle $x^{2}+y^{2}-2x+4y-20=0$ are    \rule{1cm}{0.01pt}.
	\hfill\brak{1983}
\item The equation of the line passing through the points of intersection of the circles\\ $3x^{2}+3y^{2}-2x+12y-9=0$ and $x^{2}+y^{2}+6x+2y-15=0$ is
\rule{1cm}{0.01pt}.
	\hfill\brak{1986}
\item If the circle $C_1:x^{2}+y^{2}=16$ intersects another circle $C_2$ of radius 5 in such a manner that the common chord is of maximum length and has a slope equal to $\frac{3}{4}$,  then the coordinates of the centre of $C_2$ are
\rule{1cm}{0.01pt}.
%
	\hfill\brak{1988}
\item Let the point $\vec{B}$ be the reflection of the point $\vec{A}\brak{2, 3}$ with respect to the line $8x-6y-23=0$. Let $T_A$ and $T_B$ be circles of radii $2$ and $1$ with centres $\vec{A}$ and $\vec{B}$ respectively. Let T be a common tangent to the circles $T_A$ and $T_B$ such that both the circles are on the same side of T. If $\vec{C}$ is the point of intersection of T and the line passing through $\vec{A}$ and $\vec{B}$, then the length of the line segment $AC$ is  \rule{1cm}{0.01pt}.\hfill( 2019)
\item If the two circles $(x-1)^2+(y-3)^2=r^2$ and $x^2+y^2-8x+2y+8=0$ intersect in two distinct points,  then \hfill(2003)
\begin{multicols}{2}
\begin{enumerate}
\item$r>2$
\item$2<r<8$
\item$r<2$
\item$r=2$
\end{enumerate}
\end{multicols}
\item If the circles $x^2+y^2+2ax+cy+a=0$ and $x^2+y^2-3ax+dy-1=0$ intersect in two distinct points $\vec{P}$ and $\vec{Q},$ then the line $5x+by-a=0$ passes through $\vec{P}$ and $\vec{Q}$ for
\hfill{(2005)}
\begin{multicols}{2}
\begin{enumerate}
\item exactly one value of $a$
\item no value of $a$
\item infinitely many values of $a$
\item exactly two values of $a$
\end{enumerate}
\end{multicols}
\item A circle touches the $X$ axis and also touches the circle with centre at \brak{0, 3} and radius 2. The locus of the centre of the circle is
\hfill{(2005)}
\begin{multicols}{2}
\begin{enumerate}
\item an ellipse
\item a circle 
\item a hyperbola
\item a parabola
\end{enumerate}
\end{multicols}
\item If a circle passes through the point \brak{a, b} and cuts the circle $x^2+y^2=p^2$ orthogonally,  then the equation of the locus of its centre is 
\hfill{(2005)}
\begin{multicols}{2}
\begin{enumerate}
\item $x^2+y^2-3ax-4by+\brak{a^2+b^2-p^2}=0$
\item $2ax+2by-\brak{a^2-b^2+p^2}=0$
\item $x^2+y^2-2ax-3by+\brak{a^2-b^2-p^2}=0$
\item $2ax+2by-\brak{a^2+b^2+p^2}=0$
\end{enumerate}
\end{multicols}
\item If $\vec{P}$ and $\vec{Q}$ are the points of intersection of the circles $x^2+y^2+3x+7y+2p-5=0$ and $x^2+y^2+2x+2y-p^2=0$ then there is a circle passing through $\vec{P},  \vec{Q}$ and \brak{1, 1} for:
\hfill{(2009)}
\begin{multicols}{2}
\begin{enumerate}
\item all except one value of $p$
\item all except two values of $p$
\item exactly one value of $p$
\item all value of $p$
\end{enumerate}
\end{multicols}
    \item Let the circles $C_{1}$ : $x^2+y^2=9$ and $C_{2}$ : $\brak{x-3}^2+\brak{y-4}^2 = 16$,  intersect at the points $\vec{X}$ and $\vec{Y}$. Suppose that another circle $C_{3}$ : $\brak{x-h}^2+\brak{y-k}^2=r^2$ satisfies the following conditions
%
%
% \begin{multicols}{2}
\begin{enumerate}[label=(\roman*)]
 \item Centre of $C_{3}$ is collinear with the centres of $C_{1}$ and $C_{2}$
 \item $C_{1}$ and $C{2}$ both lie inside $C_{3}$,  and
%
 \item $C_{3}$ touches $C_{1}$ at $\vec{M}$ and $C_{2}$ at  $\vec{N}$ 
\end{enumerate}
%\end{multicols}
Let the line through $\vec{X}$ and $\vec{Y}$ intersect $C_{3}$ at Z and W,  and let a common tangent of $C_{1}$ and $C_{3}$ be a tangent to the parabola $x^2=8\alpha y.$\\ %
Match the expressions given in the List-1 with those in List -2 below
%
\begin{multicols}{2}
\textbf{Column 1}
\begin{multicols}{2}
\begin{enumerate}[label=(\Alph*)]           
\item $2h+k$                                  
\item $\frac{\text{Length of ZW}}{\text{length of XY}}$     
\item $\frac{\text{Area of triangle MZN}}{\text{Area of triangle ZMW}}$                     
\end{enumerate}
\end{multicols}
\columnbreak
 \textbf{Column 2}
 \begin{multicols}{2}
\begin{enumerate}[label=(\alph*),  start=16]
 \item 6
 \item $\sqrt{6}$
 \item $\frac{5}{4}$                           
 \item $\frac{21}{5}$                          
 \item $2\sqrt{6}$                             
 \item $\frac{10}{3}$                         
 \end{enumerate}
\end{multicols}
\end{multicols}
%
\item Let $RS$ be the diameter of the Circle $x^{2} + y^{2} = 1$,  where $\vec{S}$ is the point \brak{1, 0}. Let $\vec{P}$ be a variable point \brak{\text{other than R and S}} on the circle and tangents to the circle at $\vec{S}$ and $\vec{P}$ meet at the point $\vec{Q}$. The normal to the circle at $\vec{P}$ intersects a line drawn through $\vec{Q}$ parallel to $RS$ at point $\vec{E}$. Then the locus of $\vec{E}$ passes through the point\brak{\text{s}}
%
\hfill {\brak{ 2016}}
\begin{multicols}{4}
\begin{enumerate}
	\item \brak{\frac{1}{3},  \frac{1}{\sqrt{3}}}
	\item \brak{\frac{1}{4},  \frac{1}{2}}
	\item \brak{\frac{1}{3},  -\frac{1}{\sqrt{3}}}
	\item \brak{\frac{1}{4},  -\frac{1}{2}}
\end{enumerate}
\end{multicols}
\item Let $T$ be a line passing through the points $\vec{P}$\brak{-2, 7} and $\vec{Q}$\brak{2, -5}. Let $F_1$ be the set of all pairs of circles \brak{S_1, S_2} such that $T$ is tangent to $S_1$ at $\vec{P}$ and tangent to $S_2$ at $\vec{Q}$,  and also such that $S_1$ and $S_2$ touch each other at a point,  say $\vec{M}$. Let $E_1$ be the set representing the locus of $\vec{M}$ as the pair \brak{S_1, S_2} varies in $F_1$. Let the set of all straight line segments joining a pair of distinct points of $E_1$ and passing through the point $\vec{R}$\brak{1, 1} be $F_2$. Then which of the following statements is (are) TRUE?
%
\hfill{\brak{ 2018}}
\begin{multicols}{2}
\begin{enumerate}
\item The point \brak{-2, 7} lies on $E_1$
\item The point \brak{\frac{4}{5},  \frac{7}{5}} does \textbf{NOT} lie on $E_1$
\item The point \brak{\frac{1}{3}, 1} lies on $E_1$
\item The point \brak{0,  \frac{3}{2}} does not lie on $E_1$
\end{enumerate}
\end{multicols}
    \item Two circles $x^{2} + y^{2} = 6$ and $x^{2} + y^{2}-6x +8=0$ are given. Then the equation of the circle through their points of intersection and the point $\brak{1, 1}$ is \hfill {(1980)}
    \begin{multicols}{2}
\begin{enumerate}
    	\item $x^{2}+y^{2}-6x+4=0$ 
    	\item $x^{2}+y^{2}-3x+1=0$
    	\item $x^{2}+y^{2}-4y+2=0$
    	\item none of these
    \end{enumerate}
\end{multicols}
    \item The equation of circle passing through $\brak{1, 1}$ and points of intersection of the circles $x^{2}+y^{2}+13x-3y=0$ and $2x^{2}+2y^{2}+4x-7y-25=0$ is
    \hfill {(1983 )}
    \begin{multicols}{2}
\begin{enumerate}
    	\item $4x^{2}+4y^{2}-30x-10y-25=0$
    	\item $4x^{2}+4y^{2}+30x-13y-25=0$
    	\item $4x^{2}+4y^{2}-17x-10y+25=0$
    	\item none of these
    \end{enumerate}
\end{multicols}
    \item If the two circles $(x-1)^{2} + (y-3)^{2} = r^{2}$ and $x^{2}+y^{2}-8x+2y+8=0$ intersect in two distinct points,  then \hfill {(1989)} 
    \begin{multicols}{2}
\begin{enumerate}
    	\item $2<r<8$
    	\item $r<2$
    	\item $r=2$
    	\item $r>2$
    \end{enumerate}
\end{multicols}
    \item The circles $x^{2}-10x+16=0$ and $x^{2}+y^{2}=r^{2}$ intersect each other in the two distinct points\\ if
    \hfill {(1994)}
    \begin{multicols}{2}
\begin{enumerate}
    	\item $r<2$
    	\item $r>8$
    	\item $2<r<8$
    	\item $2\leq r\leq8$
    \end{enumerate}
\end{multicols}
     \item If the circles $x^2+y^2+2x+2ky+k=0$ intersect orthogonally, then $k$ is
        \hfill$\brak{2000S}$
    \begin{multicols}{2}
\begin{enumerate}
        \item 2 or $-\frac{3}{2}$
        \item -2 or $-\frac{3}{2}$
        \item 2 or $\frac{3}{2}$
        \item $-$2 or $\frac{3}{2}$
    \end{enumerate}
    \end{multicols}
        \item Let $PQ$ and $RS$ be tangents at the extremities of the diameter PR of a circle of radius $r$. If $PS$ and $RQ$ intersect at a point $\vec{X}$ on the circumference of the circle,  then $2r$ equals
        \hfill$\brak{2001S}$
        \begin{multicols}{2}
\begin{enumerate}
    \item $\sqrt{PQ.RS}$
     \item $\brak{PQ+RS}$
    \item $2PQ.RS/(PQ+RS)$
     \item {$\sqrt{(PQ^2+RS^2)}$}/2
     \end{enumerate}
\end{multicols}
     \item If the tangent at the point $\vec{P}$ on the circle $x^2+y^2+6x+6y=2$ meets a straight line $5x-2y+6=0$ at a point on the y-axis,  then the length of $PQ$ is 
             \hfill$\brak{2002S}$
             \begin{multicols}{2}
\begin{enumerate}
             \item 4
             \item 2$\sqrt5$
             \item 5
             \item 3$\sqrt5$
\end{enumerate}
\end{multicols}
             \item A line $y=mx+1$ intersects the circle$(x-3)^2+(y+2)^2=25$ at the points $\vec{P}$ and $\vec{Q}$. if the mid point of the line segment $PQ$ has x-coordinate $-\frac{3}{5}$,  then which one of the following options is correct?
                 \hfill$\brak{ 2019}$
                 \begin{multicols}{2}
\begin{enumerate}
                 \item $2\le m<4$
                 \item $-3\le m<-1$
                 \item $4\le m<6$
                 \item $6\le m<8$
             \end{enumerate}
\end{multicols}
\item Circles with radii 3, 4 and 5 touch each other externally. If $\vec{P}$ is the point of intersection of tangents to these circles at their points of contact,  find the distance of $\vec{P}$ from the points of contact.
	           \hfill(2005) 
    \item No tangent can be drawn from point $\brak{5/2,  1}$ to circumcircle of triangle with vertices $\brak{1,  \sqrt{3}}$,  $\brak{1,  -\sqrt{3}}$,  and $\brak{3,  -\sqrt{3}}$.
    \hfill{(1985 )}
\item Tangents are drawn from point $\brak{17,  7}$ to the circle $x^2+y^2=169$.\\
STATEMENT-$1$:The tangents are mutually perpendicular.\\
STATEMENT-$2$:The locus of all points from which mutually perpendicular tangents can be drawn to a given circle is $x^2+y^2=338$. \hfill(2007)
\begin{enumerate}
\item Statement-$1$ is True, statement-$2$ is True; Statement-$2$ is a correct explantion for Statement-$1$.
\item Statement-$1$ is True, statement-$2$ is True; Statement-$2$ is NOT a correct explantion for Statement-$1$.
\item Statement-$1$ is True,  Statement-$2$ is False
\item Statement-$1$ is False,  Statement-$2$ is True.
\end{enumerate}
%
\item The lines $3x-4y+4=0$ and $6x-8y-7=0$ are tangents to the same circle. The radius of the circle is \rule{1cm}{0.01pt}.
	\hfill\brak{1984}
\item Let $x^{2}+y^{2}-4x-2y-11=0$ be a circle. A pair of tangents from the point $\brak{4, 5}$ with a pair of radii form a quadrilateral of area
\rule{1cm}{0.01pt}.
%
	\hfill\brak{1985}
\item The area of the triangle formed by the tangents from the point $\brak{4, 3}$ to the circle $x^{2}+y^{2}=9$ and the line joining their point of contact is
\rule{1cm}{0.01pt}.
%
	\hfill\brak{1987}
\item The area formed by the positive $X$ axis and the normal and the tangent to the circle $x^{2}+y^{2}=4$ at $\brak{1, \sqrt{3}}$ is
\rule{1cm}{0.01pt}.
%
	\hfill\brak{1989}
\item The chords of contact of the pair of tangents drawn from each point on the line $2x+y=4$ to $x^{2}+y^{2}=1$ pass through the point \rule{1cm}{0.01pt}.
%
	\hfill\brak{1997}
\item The centres of two circles $C_1$ and $C_2$ each of unit radius are at a distance of $6$ units from each other. Let $\vec{P}$ be the midpoint of the line segment joining the centres of $C_1$ and $C_2$ and $C$ be a circle touching circles $C_1$ and $C_2$ externally.If a common tangent to $C_1$ and $C$ passing through $\vec{P}$ is also a common tangent to $C_2$ and $C$,  then the radius of circle $C$ is  \rule{1cm}{0.01pt}.\hfill(2009)
\item Consider a family of circles which are passing through the point \brak{-1, 1},  and are tangent to $X$ axis. If \brak{h, k} are the coordinate of the centre of the circles,  then the set of values of $k$ is given by the interval
\hfill{(2007)}
\begin{multicols}{4}
\begin{enumerate}
\item $\frac{-1}{2} \le k \le \frac{1}{2}$
\item $k \le \frac{1}{2}$
\item $o \le k \le \frac{1}{2}$
\item $k \ge \frac{1}{2}$
\end{enumerate}
\end{multicols}
\item The equations of the tangents drawn from the origin to the circle $x^2+y^2-2rx-2hy+h^2=0$, are
    \hfill$\brak{1988}$
\begin{multicols}{2}
\begin{enumerate}
    \item $x=0$
    \item $y=0$
    \item $(h^2-r^2)x-2rhy=0$
    \item $(h^2-r^2)x+2rhy=0$
\end{enumerate}
\end{multicols}
\item The number of common tangents to the circles $x^2+y^2=4$ and $x^2+y^2-6x-8y=24$ is 
    \hfill$\brak{1998}$
\begin{multicols}{4}
\begin{enumerate}
    \item 0
    \item 1 
    \item 3
    \item 4
\end{enumerate}
\end{multicols}
    \item The angle between the pair of tangents drawn from the point $\vec{P}$ to the circle $x^{2}+y^{2}+4x-6y+9\sin^{2}{\alpha}+13\cos^{2}{\alpha}=0$ is $2\alpha$. The equation of the locus of the point $\vec{P}$ is
    \hfill {(1996 )}
    \begin{multicols}{2}
\begin{enumerate}
    	\item $x^{2}+y^{2}+4x-6y+4=0$
    	\item $x^{2}+y^{2}+4x-6y-9=0$
    	\item $x^{2}+y^{2}+4x-6y-4=0$
    	\item $x^{2}+y^{2}+4x-6y+9=0$
    \end{enumerate}
\end{multicols}
         \item Tangents drawn from the point $\vec{P}\brak{1, 8}$ to the circle $x^2+y^2-6x-4y-11=0$ touch the circle at the points $\vec{A}$ and $\vec{B}$. The equation of the circumcircle of the triangle $PAB$ is
             \hfill$\brak{2009}$
             \begin{multicols}{2}
\begin{enumerate}
             \item $x^2+y^2+4x-6y+19=0$
             \item $x^2+y^2-4x-10y+19=0$
             \item $x^2+y^2-4x+6y-29=0$
             \item $x^2+y^2-4x-6y+19=0$
             \end{enumerate}
\end{multicols}
             \item The locus of the mid-point of the chord of contact of tangents drawn from points lying on the straight line $4x-5y=20$ to the circle $x^2+y^2=9$ is
                 \hfill$\brak{2012}$
             \begin{multicols}{2}
\begin{enumerate}
                 \item $20\brak{x^2+y^2}-36x+45y=0$
                 \item $20\brak{x^2+y^2}+36x-45y=0$
                 \item $36\brak{x^2+y^2}-20x+45y=0$
                 \item $36\brak{x^2+y^2}+20x-45y=0$
             \end{enumerate}
             \end{multicols}
\item Let $\vec{A}$ be the centre of circle $x^2+y^2-2x-4y-20=0$. Suppose that the tangents at the points $\vec{B}$\brak{1, 7} and $\vec{D}$\brak{4, -2} on the circle meet at point $\vec{C}$. Find the area of the quadrilateral $ABCD$.
%
\hfill {\brak{1981 }}
\item Two circles,  each of radius 5 units,  touch each other at \brak{1, 2}. If the equation of common tangent is $4x+3y=10$,  find the equations of circles.
%
\hfill {\brak{1991 }}
\item Find the coordinates of the point at which the circles $x^2+y^2-4x-2y=-4$ and $x^2+y^2-12x-8y=-36$ touch each other. Also find equations common tangents touching the circles in the distinct points.                        
%
\hfill(1993)
\item $C_{1}$ and $C_{2}$ are two concentric circles,  the radius of $C_{2}$ being twice that of $C_{1}$. From a point $\vec{P}$ on $C_{2}$,  tangents $PA$ and $PB$ are drawn to $C_{1}$. Prove that the centroid of the triangle $PAB$ lies on $C_{1}$.
	           \hfill(1998)
%
%
%
%
\item Let $T_{1}$,  $T_{2}$ be two tangents drawn from $\brak{2, 0}$ onto the circle $C$:$x^2+y^2=1$. Determine the circles touching $C$ and having $T_{1}$,  $T_{2}$ as their pair of tangents. Further,  find the equations of all possible common tangents to these circles,  when taken two at a time.
                  \hfill(1999)
%
%
%
%
\item Let $2x^2+y^2-3xy=0$ be the equation of pair of tangents drawn from the origin $O$ to a circle of radius 3 with the centre in the first quadrant. If $\vec{A}$ is one of the points of contact,  find the length of $OA$.                   \hfill(2001)
%
%
%
\item For the circle $x^2+y^2=r^2$,  find the value of $r$ for which the area enclosed by the tangents drawn from the point $\vec{P}$$\brak{6, 8}$ to the circle and the chord of contact is maximum.
%
%
\hfill(2003) 
%
\item Let $C_{1}$ and $C_{2}$ be two circles with $C_{2}$ lying inside $C_{1}$. A circle $C$ lying inside $C_{1}$ touches $C_{1}$ internally and $C_{2}$ externally. Identify the locus of centre of $C$.                                \hfill(2001)
%
%
%
%
%
%
\item Find the equation of circle touching the line $2x+3y+1=0$ at $\brak{1, -1}$ and cutting orthogonally the circle having line segment joining \brak{0, 3} and \brak{-2, -1} as diameter.
%
%
\hfill(2004)     
\item Find the equations of the circle passing through \brak{-4, 3} and touching the lines $x+y=2$ and $x-y=2$
%
\hfill {\brak{1981 }}
\item Lines $5x+12y-10=0$ and $5x-12y-40=0$ touch a Circle $C_1$ of diameter 6. If the centre of $C_1$ lies in the first quadrant,  find the equation of circle $C_2$ which is concentric with $C_1$ and cuts intecepts of length 8 on these lines
%
\hfill {\brak{1986 }}
\item A circle touches the line $y=x$ at a point $\vec{P}$ such that $OP=4\sqrt{2}$,  where O is the origin. The circle contains the point \brak{-10, 2} in its interior and the length of its chord on the line $x+y=0$ is $6\sqrt{2}$. Determine the equation of circle.
%
\hfill {\brak{1990 }}
\item If a circle passes through the point \brak{a, b} and cuts the circle $x^2+y^2=4$ orthogonally,  then the locus of its centre is
\hfill{(2004)}
\begin{multicols}{2}
\begin{enumerate}
\item $2ax-2by-\brak{a^2+b^2+4}=0$
\item $2ax+2by-\brak{a^2+b^2+4}=0$
\item $2ax-2by+\brak{a^2+b^2+4}=0$
\item $2ax+2by+\brak{a^2+b^2+4}=4$
\end{enumerate}
\end{multicols}
\item Intercept on the line $y=x$ by the circle $x^2+y^2-2x=0$ is $AB$. Equation of the circle on $AB$ as a diameter is 
\hfill{(2004)}
\begin{multicols}{2}
\begin{enumerate}
\item $x^2+y^2+x-y=0$
\item $x^2+y^2-x+y=0$
\item $x^2+y^2+x+y=0$
\item $x^2+y^2-x-y=0$
\end{enumerate}
\end{multicols}
\item If the pair of lines $ax^2+2(a+b)xy+by^2=0$ lie along diameters of a circle and divide the circle into four sectors such that the area of one of the sectors is thrice the area of another sector then
\hfill{(2005)}
\begin{multicols}{2}
\begin{enumerate}
\item $3a^2-10ab+3b^2=0$
\item $3a^2-2ab+3b^2=0$
\item $3a^2+10ab+3b^2=0$
\item $3a^2+2ab+3b^2=0$
\end{enumerate}
\end{multicols}
\item The differential equation of the family of circles with fixed radius 5 units and centre on the line $y=2$ is
\begin{multicols}{2}
\begin{enumerate}
\item $\brak{x-2}y^{\prime 2}=25-\brak{y-2}^2$
\item $\brak{y-2}y^{\prime 2}=25-\brak{y-2}^2$
\item $\brak{y-2}^2y^{\prime 2}=25-\brak{y-2}^2$
\item $\brak{x-2}^2y^{\prime 2}=25-(y-2)^2$
\end{enumerate}
\end{multicols}
\item Circle(s) touching the $X$ axis at a distance 3 from the origin and having an intercept of length $2\sqrt{7}$ on the y-axis is \brak{\text{are}}
%
\hfill{\brak{ 2013}}
\begin{multicols}{2}
\begin{enumerate}
\item $x^2 + y^2 - 6x + 8y + 9 = 0$
\item $x^2 + y^2 - 6x + 7y + 9 = 0$
\item $x^2 + y^2 - 6x - 8y + 9 = 0$
\item $x^2 + y^2 - 6x - 7y + 9 = 0$
\end{enumerate}
\end{multicols}
\item A circle $S$ passes through the point \brak{0, 1} and is orthogonal to the circle $(x-1)^2+y^2=16$ and $x^2+y^2=1$. Then
%
\hfill {\brak{ 2014}}
\begin{multicols}{2}
\begin{enumerate}
\item Radius of $S$ is $8$
\item Radius of $S$ is $7$
\item Centre of $S$ is \brak{-7, 1}
\item Centre of $S$ is \brak{-8, 1}
\end{enumerate}
\end{multicols}
    \item If a circle is passing through the point $\brak{a, b}$ and it is cutting the circle $x^{2}+y^{2}=k^{2}$ orthogonally,  then the equation of the locus of its centre \\ is 
    \hfill {(1988)}
    \begin{multicols}{2}
\begin{enumerate}
    	\item $2ax + 2by - (a^{2}+b^{2}+k^{2}) = 0$
    	\item $2ax + 2by - (a^{2}-b^{2}+k^{2}) = 0$
    	\item $x^{2} + y^{2}-3ax-4by+ (a^{2}+b^{2}-k^{2}) = 0$
    	\item $x^{2} + y^{2}-2ax-3by+ (a^{2}-b^{2}-k^{2}) = 0$
    \end{enumerate}
\end{multicols}
     \item The centre of the circle inscribed in square formed by the lines $x^2-8x+12=0$ and $y^2-14y+45=0$,  is
         \hfill$\brak{2003}$
         \begin{multicols}{2}
\begin{enumerate}
         \item $\brak{4, 7}$
         \item $\brak{7, 4}$
         \item $\brak{9, 4}$
         \item $\brak{4, 9}$
     \end{enumerate}
     \end{multicols}
%
    \item The line $x+3y = 0$ is a diameter of the circle $ x^{2} + y^{2} - 6x +2y = 0$.
    \hfill{(1989 )}
\end{enumerate}
