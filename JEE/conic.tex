\begin{enumerate}[label=\thesubsection.\arabic*.,ref=\thesubsection.\theenumi]
\item STATEMENT-1: The curve $y=\frac{-x^2}{2}+x+1$ is symmetric with respect to the line $x=1$.
	\\
STATEMENT-2: A Parabola is symmetric about its axis.
\hfill(2007)
			\begin{multicols}{2}
\begin{enumerate}
    \item Statement-1 is True,Statement-2 is True;Statement-2 is a correct explanation for Statement-1\item  Statement-1 is True,Statement-2 is True;Statement-2 is NOT a correct explanation for Statement-1\item Statement-1 is True,Statement-2 is False\item Statement-1 is False,Statement-2 is True
\end{enumerate}\end{multicols}
\item A parabola has the origin as its focus and $x=2$ as the directrix. Then the vertex of the parabola is at\hfill(2009)
	\begin{multicols}{4}
\begin{enumerate}
    \item $\brak{0,2}$
    \item $\brak{1,0}$
    \item $\brak{0,1}$
    \item $\brak{2,0}$
\end{enumerate}\end{multicols}
\item The ellipse $x^2+y^2 = 4$ is inscribed in a rectangle aligned with the coordinate axes, which in turn is inscribed in another ellipse that passes through the point$\brak{4,0}$. Then the equation of the ellipse is:\hfill(2009)
	\begin{multicols}{2}
\begin{enumerate}
    \item $x^2+12y^2+16$
    \item $4x^2+48y^2=48$
    \item $4x^2+64y^2=48$
    \item $x^2+16y^2=16$
\end{enumerate}\end{multicols}
\item Equation of the ellipse whose axes are the coordinates and which passes through the point $\brak{-3,1}$ and has eccentricity $\sqrt{\frac{2}{5}}$ is\hfill(2011)
	\begin{multicols}{2}
\begin{enumerate}
    \item $5x^2+3y^2-48=0$
    \item $3x^2+5y^2-15=0$
    \item $5x^2+3y^2-32=0$
    \item $3x^2+5y^2-32=0$
\end{enumerate}\end{multicols}
\item An ellipse is drawn by taking a diameter of the circle $(x-1)^2+y^2=1$ as its semi minor axis and a diameter of the circle $x^2+(y-2)^2=4$ as semi-major axis. If the centre of the ellipse is at the origin and its axes are the coordinate axes, then the equation of the ellipse is 

\hfill(2012)
		\begin{multicols}{2}
\begin{enumerate}
    \item $4x^2+y^2=4$ 
    \item $x^2+4y^2=8$
    \item $4x^2+y^2=8$
    \item $x^2+4y^2=1$
\end{enumerate}\end{multicols}
\item The equation of the circle passing through the foci of the ellipse $\frac{x^2}{16}+\frac{y^2}{9}=1$, and having centre at $\brak{0,3}$ is

\hfill( 2013)
			\begin{multicols}{2}
\begin{enumerate}
    \item $x^2+y^2-6y-7=0$
    \item $x^2+y^2-6y+7=0$
    \item $x^2+y^2-6y-5=0$
    \item $x^2+y^2-6y+5=0$
\end{enumerate}\end{multicols}
\item Let $\vec{O}$ be the vertex and $\vec{Q}$ be any point on the parabola, $x^2=8y$. If the point $\vec{P}$ divides the line segment $OQ$ internally in the ratio \brak{1:3}, then the locus of $\vec{P}$ is

\hfill( 2015)
				\begin{multicols}{4}
\begin{enumerate}
    \item $y^2=2x$
    \item $x^2=2y$
    \item $x^2=y$
    \item $y^2=x$
\end{enumerate}\end{multicols}
\item Let $\vec{P}$ be the point on the parabola, $y^2=8x$ which is a minimum distance from the centre $\vec{C}$ of the circle, passing through $\vec{C}$ and having its centre at $\vec{P}$ is

\hfill( 2016)
					\begin{multicols}{2}
\begin{enumerate}
    \item $x^2+y^2-\frac{x}{4}+2y-24=0$ 
    \item $x^2+y^2-4x+9y+18=0$ 
    \item $x^2+y^2-4x+8y+12=0$
    \item $x^2+y^2-x+4y-12=0$
\end{enumerate}\end{multicols}
\item The eccentricity of the hyperbola whose length of the latus rectum is equal to $8$ and the length of its conjugate axis is equal to half of the distance between its foci, is 

\hfill(2016)
						\begin{multicols}{2}
\begin{enumerate}
    \item $\frac{2}{\sqrt{3}}$
    \item $\sqrt{3}$
    \item $\frac{4}{3}$
    \item $\frac{4}{\sqrt{3}}$
\end{enumerate}\end{multicols}

\item Match the conics in column 1 with the statements/expressions in column 2\hfill{(2009)}

\begin{multicols}{2}
\textbf{Column I}
\begin{enumerate}
    \item  Circle
    \item  Parabola
    \item  Ellipse 
    \item  Hyperbola  
\end{enumerate} 
\columnbreak
\textbf{Column II}
\begin{enumerate}
	\item The locus of the point ${(h,k)}$ for which the line $hx + ky =1$ touches the circle $x^2 + y^2 = 4$
    \item Points z in the complex plane satisfying $|z + 2|- |z - 2|= \pm 3$
    \item Points of the conic have parametric representations $x=\sqrt{3}
	    \brak{\frac{1 - t^2}{1 + t^2}}$, $ y = \frac{2t}{1 + t^2}$

    \item The eccentricity of the conic lies in the interval $1 \leq x < \infty$

\end{enumerate}
\end{multicols}


 \item Let $H : \frac{x^2}{a^2}-\frac{y^2}{b^2}= 1$, where $a>b>0$, be a hyperbola in $XY$-plane whose conjugate axis $LM$ subtends an angle of $60\degree$ at one of its vertices $\vec{N}$. Let the area of the triangle $LMN$ be 4$\sqrt{3}$.
Match the following.
\begin{multicols}{2}
\textbf{List 1}
\begin{enumerate}
    \item The length of the conjugate axis of $H$ is
    \item The eccentricity of $H$ is
    \item The distance between the foci of $H$ is
    \item The length of the latus rectum of $H$ is
\end{enumerate}
\columnbreak
\textbf{List 2}
\begin{enumerate}
    \item 8
    \item $\frac{4}{\sqrt{3}}$
    \item $\frac{2}{\sqrt{3}}$
    \item 4
\end{enumerate}
\end{multicols}
\item If $\vec{P}=\brak{x,y}$, $\vec{F}_{1}=\brak{3,0}$, $\vec{F}_{2}=\brak{-3,0}$ and $16x^2+25y^2=400$, then ${PF}_{1}+{PF}_{2}$ equals \hfill(1998)
	\begin{multicols}{4}
	\begin{enumerate}
		\item $8$
		\item $6$
		\item $10$
		\item $12$
	\end{enumerate}\end{multicols}

\item Let a hyperbola passes through the focus of the ellipse $\frac{x^2}{25}+\frac{y^2}{16}=1$. The transverse and conjugate axes of this hyperbola coincide with the major and minor axes of the given ellipse, also the product of eccentricities of given ellipse and hyperbola is $1$, then \hfill (2006)
	\begin{multicols}{2}
	\begin{enumerate}
		\item the equation of hyperbola is $\frac{x^2}{9}-\frac{y^2}{16}=1$
		\item the equation of hyperbola is $\frac{x^2}{9}-\frac{y^2}{25}=1$
		\item focus of hyperbola is $\brak{5,0}$
		\item vertex of hyperbola is $\brak{5\sqrt{3},0}$
	\end{enumerate}\end{multicols}

\item Let $\vec{P}\brak{x_{1}, y_{1}}$ and $\vec{Q}\brak{x_{2},y_{2}}$, $y_{1}<0,y_{2}<0$, be the end points of the latus rectum of the ellipse $x^2+4y^2=4$. The equations of parabolas with latus rectum ${PQ}$ are \hfill(2008)
	\begin{multicols}{2}
	\begin{enumerate}
		\item $x^2+2\sqrt{3}y=3+\sqrt{3}$
		\item $x^2-2\sqrt{3}y=3+\sqrt{3}$
			\columnbreak
		\item $x^2+2\sqrt{3}y=3-\sqrt{3}$
		\item $x^2-2\sqrt{3}y=3-\sqrt{3}$
	\end{enumerate}\end{multicols}

\item In a triangle ${ABC}$ with fixed base ${BC}$, the vertex $\vec{A}$ moves such that
	$$\cos{B}+\cos{C}=4\sin^2{\frac{A}{2}}.$$
		If $a,b$ and $c$ denote the lengths of the sides of the triangle opposite to the angles $A,B$ and $C$, respectively, then \hfill(2009)\\
		\begin{enumerate}
			\item $b+c=4a$
			\item $b+c=2a$
			\item locus of the point $\vec{A}$ is an ellipse
			\item locus of the point $\vec{A}$ is a pair of straight lines
		\end{enumerate}

	\item Let the eccentricity of the hyperbola $\frac{x^2}{a^2}-\frac{y^2}{b^2}=1$ be reciprocal to that of the elipse $x^2+4y^2=4$. If the hyperbola
	passes through a focus of the elipse, then 
		\hfill(2011)
		 \begin{enumerate}
			\item the equation of the hyperbola is $\frac{x^2}{3}-\frac{y^2}{2}=1$
			\item a focus of the hypebola is $(2,0)$
			\item the eccentricity of the hyperbola is $\sqrt{\frac{5}{3}}$
			\item the equation of the hyperbola is $x^2-3y^2=3$
		 \end{enumerate}
	\item Let $\vec{P}$ and $\vec{Q}$ be distinct points on the parabola $y^2=2x$ such 
		that a circle with $PQ$ as diameter passes through the vertex
		$\vec{O}$ of the parabola. If $\vec{P}$ lies in the first quadrant and the area
		of the triangle  \(\Delta \)$OPQ$ is 3$\sqrt{2}$, then which of the following is
		(are) the coordinates of $\vec{P}$?  
		\hfill(2015)
				\begin{multicols}{4}
		 \begin{enumerate}
			\item $ \brak{4,2\sqrt{2}}$
			\item $\brak{9,3\sqrt{2}}$
			\item $ \brak{\frac{1}{4},\frac{1}{\sqrt{2}}}$
			\item  $ \brak{1,\sqrt{2}}$
		 \end{enumerate}\end{multicols}
    \item The equation $\frac{x^2}{1-r}-\frac{y^2}{1+r}=1,r > 1$
represents
          \hfill  \brak{1981}
					\begin{multicols}{2}
\begin{enumerate}
    \item an ellipse    \item       a hyperbola
   \item a circle     \item    none of there 
\end{enumerate}\end{multicols}
\item Each of the four inequalities give below defines a region in $xy$ plane.One of these four regions does not have the following property.For any two points   $\brak{\frac{x_1+x_2}{2},\frac{y_1+y_2}{2}}$   is also in region.The inequality defining this region is:
         \hfill \brak{1981}
						\begin{multicols}{2}
\begin{enumerate}
    \item $x^2+2y^2\le1$
    \item Max $\abs{x},\abs{y}$ $\le1$
    \item $x^2-y^2\le1$
    \item $y^2-x\le0$
\end{enumerate}\end{multicols}
\item The equation $2x^2+3y^2-8x-18y+35=k$ represents
        \hfill \brak{1994}
							\begin{multicols}{2}
\begin{enumerate}
    \item no locus if $k\textless0$
    \item an ellipse if $k\textless0$
    \item a point if $k=0$
    \item a hyperbola if $k\textgreater0$ 
\end{enumerate}\end{multicols}
\item Let $E$ be the ellipse $\frac{x^2}{9}+\frac{y^2}{4}=1$ and $C$ be the circle $x^2+y^2=9$. Let $\vec{P}$ and $\vec{Q}$ be the points $\brak{1,2}$ and $\brak{2,1}$ respectively.Then
        \hfill \brak{1994}
								\begin{multicols}{2}
\begin{enumerate}
    \item $\vec{Q}$ lies inside $C$ but outside $E$
    \item $\vec{Q}$ lies outside both $C$ and $E$
    \item $\vec{P}$ lies inside both $C$ and $E$
    \item $\vec{P}$ lies inside $C$ but outside $E$ 
\end{enumerate}\end{multicols}
\item The radius of the circle passing through the foci of the ellipse $\frac{x^2}{16}+\frac{y^2}{9}=1$ and having its centre at $\brak{0,3}$ is
       \hfill \brak{1995}
									\begin{multicols}{4}
\begin{enumerate}
    \item $4$
    \item $3$
    \item $\sqrt{\frac{1}{2}}$
    \item $\frac{7}{2}$
\end{enumerate}\end{multicols}
\item The curve described parametrically by $x=t^2+t+1$, $y=t^2-t+1$ represents
     \hfill\brak{1999}
										\begin{multicols}{2}
\begin{enumerate}
    \item a pair of straight lines
    \item an ellipse
    \item a parabola
    \item a hyperbola
\end{enumerate}\end{multicols}
\item If the line $x-1=0$ is the directrix of parabola $y^2-kx+8=0$, then one of the values of $K$ is
      \hfill\brak{2000}
											\begin{multicols}{2}
\begin{enumerate}
    \item $\frac{1}{8}$
    \item $8$
    \item $4$
    \item $\frac{1}{4}$ 
\end{enumerate}\end{multicols}
    \item The equation of the directrix of the parabola $y^2+4y+4x+2=0$ is 
     \hfill \brak{2001}
												\begin{multicols}{2}
\begin{enumerate}
    \item $x=-1$
    \item $x=1$
    \item $x=-\frac{3}{2}$
     \item $x=\frac{3}{2}$
\end{enumerate}\end{multicols}
\item The locus of the mid-point of the line segment joining the focus to a moving point on the parabola $y^{2} = 4ax$ is another parabola with directrix \hfill{(2002)}
\begin{multicols}{4}
 \begin{enumerate}
    \item x=-a
    \item x=-a/2
    \item x=a
    \item x=a/2
 \end{enumerate}
\end{multicols}
\item For hyperbola $\frac{x^{2}}{\cos^{2}\alpha}-\frac{y^{2}}{\sin^{2}\alpha}=1$ which of the following remains constant with change in $\alpha$
\hfill{(2003)}
\begin{multicols}{2}
\begin{enumerate}
    \item abscissae of vertices
    \item abscissae of foci
    \item eccentricity
    \item directrix
\end{enumerate}
\end{multicols}
\item The axis of the parabola is along the line $y=x$ and the distance of its vertex and focus from  origin are $\sqrt2$ and $2\sqrt2$  respectively. If the vertex and focus both lie in the first quadrant,then the equation of the parabola is \hfill{(2006)}
\begin{multicols}{2}
\begin{enumerate}
    \item $(x+y)^{2}=(x-y-2)$
    \item $(x-y)^{2}=(x+y-2)$
    \item $(x-y)^{2}=4(x+y-2)$
    \item $(x-y)^{2}=8(x+y-2)$
\end{enumerate}
\end{multicols}
\item A hyperbola, having the transverse axis of length $2\sin\theta$, is confocal with the ellipse $3x^{2}+4y^{2}=12$. Then its equation is \hfill{(2007)}
	\begin{multicols}{2}
\begin{enumerate}
    \item ${x^{2}\cosec^{2}\theta}-{y^{2}\sec^{2}\theta=1}$
    \item $x^{2}\sec^{2}\theta-y^{2}\cosec^{2}\theta=1$
    \item $x^{2}\sin^{2}\theta-y^{2}\cos^{2}\theta=1$
    \item $x^{2}\cos^{2}\theta-y^{2}\sin^{2}\theta=1$
\end{enumerate}\end{multicols}
\item Let $a$ and $b$ be non-zero real numbers. Then, the equation $(ax^{2}+by^{2}+c)(x^{2}-5xy+6y^{2}=0)$ represents \hfill{(2008)}
\begin{enumerate}
    \item four straight lines, when $c=0$ and $a, b$ are of the same sign.
    \item two straight lines and a circle, when $a=b$,and $c$ is of sign opposite to that of $a$.
    \item two straight lines and a hyperbola, when $a$ and $b$ are of the same sign and $c$ is of opposite to that of $a$.
    \item a circle and an ellipse, when $a$ and $b$ are of the same sign and $c$ is of sign opposite to that of $a$.
\end{enumerate}
\item Consider a branch of the hyperbola $x^{2}-2y^{2}-2\sqrt{2}x-4\sqrt{2}y-6=0$ with vertex at a point  $\vec{A}$. Let  $\vec{B}$  be one of the end points of its latus rectum. If $\vec{C}$ is the focus of the hyperbola nearest to the point  $\vec{A}$, then the area of the triangle $ABC$ is \hfill{(2008)}
\begin{multicols}{4}
\begin{enumerate}
    \item $1-\sqrt{\frac{2}{3}}$
    \item $\sqrt{\frac{3}{2}}-1$
    \item $1+\sqrt{\frac{2}{3}}$
    \item $\sqrt{\frac{3}{2}}+1$
\end{enumerate}
\end{multicols}
\item The line passing through the extremity $\vec{A}$ of the major axis and extremity $\vec{B}$ of the minor axis of the ellipse $x^{2}+9y^{2}=9$ meets its auxillary circle at the point $\vec{M}$. Then the area of the triangle with the vertices at $\vec{A}$, $\vec{M}$ and the origin $\vec{O}$ is \hfill{(2009)}
\begin{multicols}{4}
\begin{enumerate}
    \item $\frac{31}{10}$
    \item $\frac{29}{10}$
    \item $\frac{21}{10}$
    \item $\frac{27}{10}$
\end{enumerate}\end{multicols}
	%	
\item The locus of the orthocentre of the triangle formed by the lines
		
			$$\brak{1+p}x-py+p\brak{1+p}=0,$$
			$$\brak{1+q}x-qy+q\brak{1+q}=0,$$
		and $y=0$, where $p \neq q$, is
		\hfill(2009)
		\begin{multicols}{4}
\begin{enumerate}
	\item a hyperbola
	\item a parabola
	\item an ellipse
	\item a straight line
\end{enumerate}\end{multicols}
%
	\item Let $\brak{x,y}$ be any point on the parabola $y^2=4x$. Let $\vec{P}$ be the point that divides the line segment from $\brak{0,0}$ to $\brak{x,y}$ in the ratio $1:3$. Then the locus of $\vec{P}$ is  \hfill(2011)
		\begin{multicols}{4}
		\begin{enumerate}
			\item $x^2=y$
			\item $y^2=2x$
				\columnbreak
			\item $y^2=x$
			\item $x^2=2y$
		\end{enumerate}\end{multicols}
%
	\item The ellipse $E_{1}$:$\frac{x^2}{9}+\frac{y^2}{4}=1$ is inscribed in a rectangle $\vec{R}$ whose sides are parallel to the coordinate axes. Another ellipse $E_{2}$ passing through the point $\brak{0,4}$ circumscribes the rectangle $\vec{R}$. The eccentricity of the ellipse $E_{2}$ is \hfill(2012)
%
		\begin{multicols}{4}
		\begin{enumerate}
			\item $\frac{\sqrt{2}}{2}$
			\item $\frac{\sqrt{3}}{2}$
				\columnbreak
			\item $\frac{1}{2}$
			\item $\frac{3}{4}$
		\end{enumerate}\end{multicols}
%
\item  Two sets $A$ and $B$ are as under:
$A=\cbrak{\brak{a,b}\in\mathbb{R}\times\mathbb{R}:\abs{a-5}<1\text{ and} \abs{b-5}<1}$
$B=\cbrak{\brak{a,b}\in\mathbb{R}\times\mathbb{R}:4(a-6)^2\text{+}9(b-5)^2\leq36}$
    \hfill{( 2018)}
			\begin{multicols}{2}
	\begin{enumerate}
		\item $A\subset B$ 
		\item $A\cap B$
		\item neither $A\subset B$ nor $B\subset A$
		\item $B\subset A$
	\end{enumerate}\end{multicols}
\item Axis of a parabola lies along $X$ axis. If its vertex and focus are at a distance $2$ and $4$ respectively from origin, on the positive $X$ axis then which of the following points does not lie on it? 
     \hfill{( 2018)} 
				\begin{multicols}{4}
	\begin{enumerate}
    		\item $\brak{5,2\sqrt6}$
    		\item $\brak{8,6}$
    		\item $\brak{6,4\sqrt2}$
    		\item $\brak{4,-4}$
	\end{enumerate}\end{multicols}
\item Let $0<\theta<\pi/2$. If the eccentricty of the hyperbola $\frac{x^2}{\cos^2{\theta}} - \frac{y^2}{\sin^2{\theta}} = 1$ is greater than $2$, then the length of its latus rectum lies in the interval
	\hfill{( 2019)}
					\begin{multicols}{4}
	\begin{enumerate}
    		\item $\brak{5,\infty}$
    		\item $\lbrak{\frac{3}{2}},\rsbrak{3}$
    		\item $\lbrak{2},\rsbrak{3}$ 
    		\item $\lbrak{1},\rsbrak{\frac{3}{2}}$
	\end{enumerate} \end{multicols}
\item  The equation of the locus of the point whose distances from the point $\Vec{P}$ and the line $AB$ are equal, is

	\begin{multicols}{2}
\begin{enumerate}
     \item $9x^2+y^2-6xy-54x-62y+241=0$
     \item $x^2+9y^2+6xy-54x-62y-241=0$
     \item $9x^2+9y^2-6xy-54x-62y-241=0$
     \item $x^2+y^2-2xy+27x+31y-120=0$
\end{enumerate}\end{multicols}
\item Let $\vec{P}$ be a point on the ellipse $\frac{x^2}{a^2}+\frac{y^2}{b^2}=1$, $0<b<a$. Let the line parallel to the $X$ axis passing through $\vec{P}$ meet the circle $x^2+y^2=a^2$ at the point $\vec{Q}$ such that $\vec{P}$ and $\vec{Q}$ are on the same side of the $X$ axis. For two positive real numbers $r$ and $s$, find the locus of the point $\vec{R}$ on $PQ$ such that $PR
= r$ as $\vec{P}$ varies over the ellipse. \hfill\brak{2001}
 \item Let $a$ and $b$ be positive real numbers such that $a > 1$ and $b < a$. Let $\vec{P}$ be a point in the first quadrant that lies on the hyperbola $\frac{x^2}{a^2} - \frac{y^2}{b^2}$. Suppose the tangent to the hyperbola at $\vec{P}$ passes through the point (1, 0), and suppose the normal to the hyperbola at $\vec{P}$ cuts off equal intercepts on the coordinate axes. Let $\triangle$ denote the area of the triangle formed by the tangent at $\vec{P}$, the normal at $\vec{P}$ and the $X$ axis. If e denotes the eccentricity of the hyperbola, then which of the following statements is/are TRUE?
\hfill (2020)
\begin{enumerate}	
 \item $1< e <\sqrt{2}$
 \item $\sqrt{2}< e <2$
 \item $\triangle = a^4$
 \item $\triangle = b^4$
\end{enumerate}	

	\end{enumerate}
