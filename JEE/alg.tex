         \item %48 
		 Let $\vec{O}$ be the origin and let PQR be an arbitrary triangle. The point $\vec{S}$ is such that $OP\cdot OQ$+$OR\cdot OS$=$OR\cdot OP$+$OQ\cdot OS$=$OQ\cdot OR$+$OP\cdot OS$\\
Then the triangle PQR has $\vec{S}$ as its \hfill{\brak{JEE Adv. 2017}}\\
\begin{enumerate}
        \item Centroid                             
        \item Circumcentre                           
        \item Incentre            
        \item Orthocenter\\          
\end{enumerate}
\item From a point $\vec{O}$ inside the $\triangle ABC$, perpendiculars $OD, OE, OF$ are drawn to the sides $BC, CA, AB$ respectively. Prove that the perpendiculars from $\vec{A,B,C}$ to the sides $EF, FD, DE$ are concurrent. \hfill{\brak{1978}}
	\item $A_1,A_2,......A_n$ are the vertices of a regular plane polygon with n sides and $\vec{O}$ is its centre. Show that
	$\sum_{i=1}^{n-1}\brak{OA_i\times OA_{i+1}}= \brak{1-n}\brak{OA_2\times OA_1}$
		\hfill{\brak{1982-2Marks}}\\
	\item If $\vec{A,B,C,D}$ are any four points in space, prove that-
		$\abs{AB\times CD+ BC\times AD+CA\times BD}=$\\ $4$\brak{area\ of\ triangle\ ABC}  \hfill{\brak{1987-2Marks}}\\
	\item Tangent at a point $\vec{P_{1}}$ {other than \brak{0, 0}} on the curve $y=x^{3}$ meets the curve again at $\vec{P_{2}}$. The tangent at $\vec{P_{2}}$ meets the curve again at $\vec{P_{1}}$, and so on. Show that the abscissae of $\vec{P_{1}, P_{2}, P_{3} \dots P_{n}}$, form a G.P. Also find the ratio. $\frac{\sbrak{area(\Delta P_{1}P_{2}P_{3})}}{\sbrak{area(\Delta P_{2}P_{3}P_{4})}}$

	\hfill{(1993 - 5 Marks)}
\item Using co-ordinate geometry, prove that the three altitudes of any triangle are concurrent.
	\hfill{(1998)}
