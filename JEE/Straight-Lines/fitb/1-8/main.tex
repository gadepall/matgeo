\iffalse
\title{Straight Lines and Pair of Straight Lines}
\author{EE24BTECH11041 - Mohit}
\section {fitb}
\fi
	\item The area enclosed within the curve $\abs{x}+\abs{y} =1$ is \dots
    \hfill{1981-2 Marks}
    \item $y = 10^x $ is the reflection of $y=\log x$ in the line whose equation is \dots
    \hfill{1982-2 Marks}
    \item The set of lines $ax+by+c=0$,where $3a+2b+4c=0$ concurrent at the point \dots
    \hfill{1982-2 Marks}
    \item Given the points $\vec{A}\brak{0,4}$ and $\vec{B}\brak{0,-4}$,the equation of the locus of the point $\vec{p}\brak{x,y}$,such that \\
    $\abs{AP-BP}=6$ is \dots
    \hfill{1983-1 Marks}
    \item If $a,b$ and $c$ are in A.P, then the straight line $ax +by +c=0$ will always pass through a fixed point whose coordinate are \dots
    \hfill{1984-2 Marks}
    \item The orthocentre of the triangle formed by the lines $x+y=1,2x +3y=6$ and $4x-y+4=0$ lies in the quadrant number \dots
    \hfill{1985-2 Marks}
    \item Let the algebric sum of the perpendicular distances from the points $\brak{2,0},\brak{0,2}$ and $\brak{1,1}$ to a variable straight line be zero;then the line passes through a fixed point whose coordinates are \dots
    \hfill{1991-2 Marks}
    \item the vertices of a triangle are $\vec{A}\brak{-1,-7}$,$\vec{B}\brak{5,1}$ and $\vec{C}\brak{1,10}$. The equation of the bisector of the angle $\angle{ABC}$ is \dots
    \hfill{1993-2 marks}

