	\begin{enumerate}
\item Tangents are drawn from point $\brak{17,  7}$ to the circle $x^2+y^2=169$.\\
STATEMENT-$1$:The tangents are mutually perpendicular.\\
STATEMENT-$2$:The locus of all points from which mutually perpendicular tangents can be drawn to a given circle is $x^2+y^2=338$. \hfill(2007)
\begin{enumerate}
\item Statement-$1$ is True, statement-$2$ is True; Statement-$2$ is a correct explantion for Statement-$1$.
\item Statement-$1$ is True, statement-$2$ is True; Statement-$2$ is NOT a correct explantion for Statement-$1$.
\item Statement-$1$ is True,  Statement-$2$ is False
\item Statement-$1$ is False,  Statement-$2$ is True.
\end{enumerate}

\item Consider 
\begin{align*}
	L_1:2x+3y+p-3&=0
	\\
               L_2:2x+3y+p+3&=0
\end{align*}
where $p$ is a real number, and 
		$$	C: x^2+y^2+6x-10y+30=0$$
STATEMENT-$1$:If line $L_1$ is a chord of circle C, then line $L_2$ is not always a diameter of circle C\\
STATEMENT-$2$:If line $L_1$ is a diameter of circle C, then line $L_2$ is not a chord  of circle C. \hfill(2008)
\begin{enumerate}
\item Statement-$1$ is True, statement-$2$ is True;Statement-$2$ is a correct explantion for Statement-$1$.
\item Statement-$1$ is True, statement-$2$ is True;Statement-$2$ is NOT a correct explantion for Statement-$1$.
\item Statement-$1$ is True, Statement-$2$ is False
\item Statement-$1$ is False, Statement-$2$ is True.	
\end{enumerate}
	\item If $\vec{A}$ and $\vec{B}$ are points in the plane such that  $\frac{PA}{PB}=K$(constant) for all $P$ on a given circle,  then the value of $K$ cannot be equal to \rule{1cm}{0.01pt}.
		\hfill\brak{1982}
\item The points of intersection of the line $4x-3y-10=0$ and the circle $x^{2}+y^{2}-2x+4y-20=0$ are    \rule{1cm}{0.01pt}.
	\hfill\brak{1983}
\item The lines $3x-4y+4=0$ and $6x-8y-7=0$ are tangents to the same circle. The radius of the circle is \rule{1cm}{0.01pt}.
	\hfill\brak{1984}
\item Let $x^{2}+y^{2}-4x-2y-11=0$ be a circle. A pair of tangents from the point $\brak{4, 5}$ with a pair of radii form a quadrilateral of area
\rule{1cm}{0.01pt}.
%
	\hfill\brak{1985}
\item From the origin chords are drawn  to the circle $\brak{x-1}^{2}+y^{2}=1$. The equation of the locus of the mid-points of these chords is
\rule{1cm}{0.01pt}.
	\hfill\brak{1985}
\item The equation of the line passing through the points of intersection of the circles\\ $3x^{2}+3y^{2}-2x+12y-9=0$ and $x^{2}+y^{2}+6x+2y-15=0$ is
\rule{1cm}{0.01pt}.
	\hfill\brak{1986}
\item From the point $\vec{A}\brak{0, 3}$ on the circle \\            $x^{2}+4x+\brak{y-3}^{2}=0$,  a chord AB is drawn and extended to a point M such that $AM=2AB$. The equation of the locus of M is 
%
	\hfill\brak{1986}
\item The area of the triangle formed by the tangents from the point $\brak{4, 3}$ to the circle $x^{2}+y^{2}=9$ and the line joining their point of contact is
%
	\hfill\brak{1987}
\item If the circle $C_1:x^{2}+y^{2}=16$ intersects another circle $C_2$ of radius 5 in such a manner that common chord is of maximum length and has a slope equal to $\frac{3}{4}$,  then the coordinates of the centre of $C_2$ are
%
	\hfill\brak{1988}
\item The area formed by the positive x-axis and the normal and the tangent to the circle $x^{2}+y^{2}=4$ at $\brak{1, \sqrt{3}}$ is
%
	\hfill\brak{1989}
\item If a circle passes through the points of intersection of the coordinate axes with the lines $\lambda x-y+1=0$ and $x-2y+3=0$,  then the value of $\lambda =$
%
	\hfill\brak{1991}
\item The equation of the locus of the mid-points of the circle $4x^{2}+4y^{2}-12x+4y+1=0$ that subtend an angle of $\frac{2\pi}{3}$ at its centre is
%
	\hfill\brak{1993}
\item The intercept of the line $y=x$ by the circle $x^{2}+y^{2}-2x=0$ is AB. Equation of the circle with AB as a diameter is
%
	\hfill\brak{1996}
\item For each natural number k,  let $C_k$ denote the circle with radius k centimetres and centre at the origin. On the circle $C_k$,  $\alpha-particle$ moves k centimetres in the counter-clockwise direction. After completing its motion on $C_k$,  the particle moves to $C_{k+1}$ in the radial direction. The motion of the particle continues in this manner. The particle starts at $\brak{1, 0}$. If the particle crosses the positive direction of the x-axis for the first time on the circle $C_n$ \\then n=
%
	\hfill\brak{1997}
\item The chords of contact of the pair of tangents drawn from each point on the line $2x+y=4$ to $x^{2}+y^{2}=1$ pass through the point
%
	\hfill\brak{1997}
\item The centres of two circles $C_1$ and $C_2$ each of unit radius are at a distance of $6$ units from each other. Let $P$ be the midpoint of the line segment joining the centres of $C_1$ and $C_2$ and $C$ be a circle touching circles $C_1$ and $C_2$ externally.If a common tangent to $C_1$ and $C$ passing through P is also a common tangent to $C_2$ and $C$,  then the radius of circle $C$ is \hfill(2009)
\item The straight line $2x-3y=1$ divides the circular region $x^2+y^2\leq6$ into two parts.\\
If  S  is \{ $\brak{2, 3/4}, \brak{5/2, 3/4}, \brak{1/4, -1/4}, \brak{1/8, 1/4}$ \}  then the  number of point(s) in S lying inside the smaller part is \hfill(2011)
\item For how many values of $p$,  the circle $x^2+y^2+2x+4y-p=0$ and the coordinate axes have exactly three common points? \hfill( 2017)
\\
\item Let the point $\vec{B}$ be the reflection of the point $\vec{A}\brak{2, 3}$ with respect to the line $8x-6y-23=0$. Let $T_A$ and $T_B$ be circles of radii $2$ and $1$ with centres $\vec{A}$ and $\vec{B}$ respectively. Let T be a common tangent to the circles $T_A$ and $T_B$ such that both the circles are on the same side of T.If C is the point of intersection of T and the line passing through $\vec{A}$ and $\vec{B}$, then the length of the line segment AC is \hfill( 2019)
\item If the chord $y=mx+1$ of the circle $x^2+y^2=1$ subtends an angle of measure \( 45^\circ \) at the major segment of the circle then the value of $m$ is \hfill(2002)
\begin{enumerate}
\item$2\pm\sqrt{2}$
\item$-2\pm\sqrt{2}$
\item$-1\pm\sqrt{2}$
\item none of this
\end{enumerate}
\item The centres of a set of circles, each of radius $3$,  lie on the circle $x^2+y^2=25$.The locus of any point in the set is \hfill(2002)
\begin{enumerate}
\item$4$ $\leq$$x^2+y^2$ $\leq$ $64$
\item$x^2+y^2\leq25$
\item$x^2+y^2\geq25$
\item$3$ $\leq$ $x^2+y^2$ $\leq$ $9$
\end{enumerate}
\item The centre of the circle passing through $\brak{0, 0} and \brak{1, 0}$ and touching the circle $x^2+y^2=9$ is \hfill(2002)
\begin{enumerate}
\item$\brak{1/2, 1/2}$
\item$\brak{1/2, -\sqrt{2}}$
\item$\brak{3/2, 1/2}$
\item$\brak{1/2, 3/2}$
\end{enumerate}
\item The equation of a circle with origin as a centre and passing through equilateral triangle whose median is of length $3a$ is \hfill(2002)
\begin{enumerate}
\item$x^2+y^2=9a^2$
\item$x^2+y^2=16a^2$
\item$x^2+y^2=4a^2$
\item$x^2+y^2=a^2$
\end{enumerate}
\item If the two circles $(x-1)^2+(y-3)^2=r^2$ and $x^2+y^2-8x+2y+8=0$ intersect in two distinct points,  then \hfill(2003)
\begin{enumerate}
\item$r>2$
\item$2<r<8$
\item$r<2$
\item$r=2$
\end{enumerate}
\item The lines $2x-3y=5$ and $3x-4y=7$ are diameters of a circle having area as 154 sq.units. Then the equation of the circle is
\hfill{(2003)}
\begin{enumerate}
\item $x^2+y^2-2x+2y=62$
\item $x^2+y^2+2x-2y=62$
\item $x^2+y^2+2x-2y=47$
\item $x^2+y^2-2x+2y=47$
\end{enumerate}
\item If a circle passes through the point \brak{a, b} and cuts the circle $x^2+y^2=4$ orthogonally,  then the locus of its centre is
\hfill{(2004)}
\begin{enumerate}
\item $2ax-2by-\brak{a^2+b^2+4}=0$
\item $2ax+2by-\brak{a^2+b^2+4}=0$
\item $2ax-2by+\brak{a^2+b^2+4}=0$
\item $2ax+2by+\brak{a^2+b^2+4}=4$
\end{enumerate}
\item A variable circle passes through the fixed point $\vec{A}\brak{p, q}$ and touches $x$-axis. The locus of the other end of the diameter through $\vec{A}$is
\hfill{(2004)}
\begin{multicols}{2}
\begin{enumerate}
\item $\brak{y-q}^2=4px$
\item $\brak{x-q}^2=4py$ 
\item $\brak{y-p}^2=4qx$
\item $\brak{x-p}^2=4qy$
\end{enumerate}
\end{multicols}
\item If the lines $2x+3y+1=0$ and $3x-y-4=0$ lie along diameter of a circle of circumference $10\pi$,  then the equation of the circle is 
\hfill{(2004)}
\begin{enumerate}
\item $x^2+y^2+2x-2y-23=0$
\item $x^2+y^2-2x-2y-23=0$
\item $x^2+y^2+2x+2y-23=0$
\item $x^2+y^2-2x+2y-23=0$
\end{enumerate}
\item Intercept on the line $y=x$ by the circle $x^2+y^2-2x=0$ is $AB$. Equation of the circle on $AB$ as a diameter is 
\hfill{(2004)}
\begin{enumerate}
\item $x^2+y^2+x-y=0$
\item $x^2+y^2-x+y=0$
\item $x^2+y^2+x+y=0$
\item $x^2+y^2-x-y=0$
\end{enumerate}
\item If the circles $x^2+y^2+2ax+cy+a=0$ and $x^2+y^2-3ax+dy-1=0$ intersect in two distinct points $\vec{P}$ and $\vec{Q}$ then the line $5x+by-a=0$ passes through $\vec{P}$ and $\vec{Q}$ for
\hfill{(2005)}
\begin{enumerate}
\item exactly one value of $a$
\item no value of $a$
\item infinitely many values of $a$
\item exactly two values of $a$
\end{enumerate}
\item A circle touches the $x$-axis and also touches the circle with centre at \brak{0, 3} and radius 2. The locus of the centre of the circle is
\hfill{(2005)}
\begin{multicols}{2}
\begin{enumerate}
\item an ellipse
\item a circle 
\item a hyperbola
\item a parabola
\end{enumerate}
\end{multicols}
\item If a circle passes through the point \brak{a, b} and cuts the circle $x^2+y^2=p^2$ orthogonally,  then the equation of the locus of its centre is 
\hfill{(2005)}
\begin{enumerate}
\item $x^2+y^2-3ax-4by+\brak{a^2+b^2-p^2}=0$
\item $2ax+2by-\brak{a^2-b^2+p^2}=0$
\item $x^2+y^2-2ax-3by+\brak{a^2-b^2-p^2}=0$
\item $2ax+2by-\brak{a^2+b^2+p^2}=0$
\end{enumerate}
\item If the pair of lines $ax^2+2(a+b)xy+by^2=0$ lie along diameters of a circle and divide the circle into four sectors such that the area of one of the sectors is thrice the area of another sector then
\hfill{(2005)}
\begin{enumerate}
\item $3a^2-10ab+3b^2=0$
\item $3a^2-2ab+3b^2=0$
\item $3a^2+10ab+3b^2=0$
\item $3a^2+2ab+3b^2=0$
\end{enumerate}
\item If the lines $3x-4y-7=0$ and $2x-3y-5=0$ are two diameters of a circle of area $49\pi$ square units,  the equation of the circle is 
\hfill{(2006)}
\begin{enumerate}
\item $x^2+y^2+2x-2y-47=0$
\item $x^2+y^2+2x-2y-62=0$
\item $x^2+y^2-2x+2y-62=0$
\item $x^2+y^2-2x+2y-47=0$
\end{enumerate}
\item Let $\vec{C}$ be the circle with centre \brak{0, 0} and radius 3 units. The equation of the locus of the mid points of the chords of the circle $\vec{C}$ that subtend an angle of $\frac{2\pi}{3}$ at its centre is
\hfill{(2006)}
\begin{multicols}{2}
\begin{enumerate}
\item $x^2+y^2=\frac{3}{2}$
\item $x^2+y^2=1$
\item $x^2+y^2=\frac{27}{4}$
\item $x^2+y^2=\frac{9}{4}$
\end{enumerate}
\end{multicols}
\item Consider a family of circles which are passing through the point \brak{-1, 1},  and are tangent to $x$-axis. If \brak{h, k} are the coordinate of the centre of the circles,  then the set of values of $k$ is given by the interval
\hfill{(2007)}
\begin{multicols}{2}
\begin{enumerate}
\item $\frac{-1}{2} \le k \le \frac{1}{2}$
\item $k \le \frac{1}{2}$
\item $o \le k \le \frac{1}{2}$
\item $k \ge \frac{1}{2}$
\end{enumerate}
\end{multicols}
\item The point diametrically opposite to the point $\vec{P}\brak{1, 0}$ on the circle $x^2+y^2+2x+2y-3=0$ is 
\hfill{(2008)}
\begin{multicols}{2}
\begin{enumerate}
\item \brak{3, -4}
\item \brak{-3, 4}
\item \brak{-3, -4}
\item \brak{3, 4}
\end{enumerate}
\end{multicols}
\item The differential equation of the family of circles with fixed radius 5 units and centre on the line $y=2$ is
\begin{enumerate}
\item $\brak{x-2}y^{\prime 2}=25-\brak{y-2}^2$
\item $\brak{y-2}y^{\prime 2}=25-\brak{y-2}^2$
\item $\brak{y-2}^2y^{\prime 2}=25-\brak{y-2}^2$
\item $\brak{x-2}^2y^{\prime 2}=25-(y-2)^2$
\end{enumerate}
\item If $\vec{P}$ and $\vec{Q}$ are the points of intersection of the circles $x^2+y^2+3x+7y+2p-5=0$ and $x^2+y^2+2x+2y-p^2=0$ then there is a circle passing through $\vec{P},  \vec{Q}$ and \brak{1, 1} for:
\hfill{(2009)}
\begin{enumerate}
\item all except one value of $p$
\item all except two values of $p$
\item exactly one value of $p$
\item all value of $p$
\end{enumerate}
    \item Let the circles $C_{1}$ : $x^2+y^2=9$ and $C_{2}$ : $\brak{x-3}^2+\brak{y-4}^2 = 16$,  intersect at the points $X$ and $Y$. Suppose that another circle $C_{3}$ : $\brak{x-h}^2+\brak{y-k}^2=r^2$ satisfies the following condition:\\
 
 
 \begin{enumerate}[label=(\roman*)]
 \item Centre of $C_{3}$ is collinear with the centres of $C_{1}$ and $C_{2}$
 \item $C_{1}$ and $C{2}$ both lie inside $C_{3}$,  and

 \item $C_{3}$ touches $C_{1}$ at M and $C_{2}$ at N
\end{enumerate}
Let the line through X and Y intersect $C_{3}$ at Z and W,  and let a common tangent of $C_{1}$ and $C_{3}$ be a tangent to the parabola $x^2=8\alpha y.$\\

There are some expressions given in the List-1 whose values are given in List -2 below
\newpage
			
\begin{multicols}{2}
\textbf{Column 1}
\begin{enumerate}[label=(\Alph*)]           
\item $2h+k$                                  
\item $\frac{\text{Length of ZW}}{\text{length of XY}}$     
\item $\frac{\text{Area of triangle MZN}}{\text{Area of triangle ZMW}}$                     
\end{enumerate}
\columnbreak
 \textbf{Column 2}
 \begin{enumerate}[label=(\alph*),  start=16]
 \item 6
 \item $\sqrt{6}$
 \item $\frac{5}{4}$                           
 \item $\frac{21}{5}$                          
 \item $2\sqrt{6}$                             
 \item $\frac{10}{3}$                         
 \end{enumerate}
\end{multicols}
		
 Which of the following is the only CORRECT combination $?$
\begin{enumerate}[label=(\alph*)]  

\item (I), (U)
\item (I), (S)
 \item (II), (T)    
 \item (II), (Q)

 \end{enumerate}


\item Let the circles $C_{1}$ : $x^2+y^2=9$ and $C_{2}$ : $\brak{x-3}^2+\brak{y-4}^2=16$,  intersect at the points X and Y. Suppose that another circle $C_{3}$ : $\brak{x-h}^2+\brak{y-k}^2=r^2$ satisfies the following conditions: 

\begin{enumerate}[label=(\roman*)]                       
\item Centre of $C_{3}$ is collinear with centres of $C_{1}$ and $C_{2}$                         
\item $C_{1}$ and $C_{2}$ both lie inside $C_{3}$,  and                                            
\item $C_{3}$ touches $C_{1}$ at M and $C_{2}$ at N
\end{enumerate}                              
Let the line through X and Y intersect $C_{3}$ at Z and W,  and let a common tangent of $C_{1}$ and $C_{3}$ be a tangent to the parabola $x^2=8\alpha y$.
There are some expressions given in the List- 1 whose values are given in List - 2 below
		

\begin{multicols}{2}


\textbf{Column{1}}

\begin{enumerate}[label=(\Alph*)]

 \item $2h+k$
 \item $\frac{\text{Length of ZW}}{\text{length of XY}}$
\item $\frac{\text{Area of triangle MZN}}{\text{Area of triangle ZMW}}$             
\end{enumerate}    
\columnbreak
 \textbf{Column{2}}

 \begin{enumerate}[label=(\alph*),  start=16]                                              
 \item 6                                      
 \item $\sqrt{6}$                              
 \item $\frac{5}{4}$                          
 \item $\frac{21}{5}$                          
 \item $2\sqrt{6}$               
 \item $\frac{10}{3}$                         
 \end{enumerate}   
\end{multicols}
 Which of the following is the only INCORRECT combination?


\begin{enumerate}[label=(\alph*)] 
 \item (IV), (S)
 \item (I), (P)
 \item (III), (R)    
 \item (IV), (U)

  
  \end{enumerate}
\item The equations of the tangents drawn from the origin to the circle $x^2+y^2-2rx-2hy+h^2=0$, are
    \hfill$\brak{1988}$
    \begin{multicols}{2}
\begin{enumerate}
    \item $x=0$
    \item $y=0$
    \item $(h^2-r^2)x-2rhy=0$
    \item $(h^2-r^2)x+2rhy=0$
\end{enumerate}
\end{multicols}
\item The number of common tangents to the circles $x^2+y^2=4$ and $x^2+y^2-6x-8y=24$ is 
    \hfill$\brak{1998}$
    \begin{multicols}{2}
\begin{enumerate}
    \item 0
    \item 1 
    \item 3
    \item 4
\end{enumerate}
\end{multicols}
\item If the circle $x^2+y^2=a^2$ intersects the hyperbola $xy=c^2$ in four points $P\brak{x_1, y_1}$, $Q\brak{x_2, y_2}$, $R\brak{x_3, y_3}$, $S\brak{x_4, y_4}$, then

    \hfill$\brak{1998}$
\begin{multicols}{2}
\begin{enumerate}
    \item $x_1+x_2+x_3+x_4=0$
    \item $y_1+y_2+y_3+y_4=0$
    \item $x_1x_2x_3x_4=c^4$
    \item $y_1y_2y_3y_4=c^4$

\end{enumerate}
\end{multicols}
\item Circle(s) touching the x-axis at a distance \brak{3} from the origin and having an intercept of length $2\sqrt{7}$ on the y-axis is \brak{\text{are}}

\hfill{\brak{ 2013}}
\begin{enumerate}
\begin{multicols}{2}
\item $x^2 + y^2 - 6x + 8y + 9 = 0$
\item $x^2 + y^2 - 6x + 7y + 9 = 0$
\item $x^2 + y^2 - 6x - 8y + 9 = 0$
\item $x^2 + y^2 - 6x - 7y + 9 = 0$
\end{multicols}
\end{enumerate}
\item A circle $S$ passes through the point \brak{0, 1} and is orthogonal to the circle $(x-1)^2+y^2=16$ and $x^2+y^2=1$. Then

\hfill {\brak{ 2014}}
\begin{enumerate}
\begin{multicols}{2}
\item Radius of $S$ is $8$
\item Radius of $S$ is $7$
\item Centre of $S$ is \brak{-7, 1}
\item Centre of $S$ is \brak{-8, 1}
\end{multicols}
\end{enumerate}
\item Let $RS$ be the diameter of the Circle $x^{2} + y^{2} = 1$,  where $\vec{S}$ is the point \brak{1, 0}. Let $\vec{P}$ be a variable point \brak{\text{other than R and S}} on the circle and tangents to the circle at $\vec{S}$ and $\vec{P}$ meet at the point $\vec{Q}$. The normal to the circle at $\vec{P}$ intersects a line drawn through $\vec{Q}$ parallel to $RS$ at point $\vec{E}$. Then the locus of $\vec{E}$ passes through the point\brak{\text{s}}

\hfill {\brak{ 2016}}
\begin{enumerate}
\begin{multicols}{2}
	\item \brak{\frac{1}{3},  \frac{1}{\sqrt{3}}}
	\item \brak{\frac{1}{4},  \frac{1}{2}}
	\item \brak{\frac{1}{3},  -\frac{1}{\sqrt{3}}}
	\item \brak{\frac{1}{4},  -\frac{1}{2}}
\end{multicols}
\end{enumerate}
\item Let $T$ be a line passing through the points $\vec{P}$\brak{-2, 7} and $\vec{Q}$\brak{2, -5}. Let $F_1$ be the set of all pairs of circles \brak{S_1, S_2} such that $T$ is tangent to $S_1$ at $\vec{P}$ and tangent to $S_2$ at $\vec{Q}$,  and also such that $S_1$ and $S_2$ touch each other at a point,  say $\vec{M}$. Let $E_1$ be the set representing the locus of $\vec{M}$ as the pair \brak{S_1, S_2} varies in $F_1$. Let the set of all straight line segments joining a pair of distinct points of $E_1$ and passing through the point $\vec{R}$\brak{1, 1} be $F_2$. Then which of the following statements is (are) TRUE?

\hfill{\brak{ 2018}}
\begin{enumerate}
\begin{multicols}{2}
\item The point \brak{-2, 7} lies on $E_1$
\item The point \brak{\frac{4}{5},  \frac{7}{5}} does \textbf{NOT} lie on $E_1$
\item The point \brak{\frac{1}{3}, 1} lies on $E_1$
\item The point \brak{0,  \frac{3}{2}} does not lie on $E_1$
\end{multicols}
\end{enumerate}
    \item A square is inscribed in the circle $x^{2} + y^{2} - 2x +4y +3= 0.$ Its sides are parellel to the coordinate axes. The one vertex of the square is \hfill {(1980)}
    \begin{multicols}{2}
    	\begin{enumerate}
    		\item $\brak{1+\sqrt{2},  -2}$ 
    		\item $\brak{1-\sqrt{2},  -2}$
    		\item $\brak{1,  -2 +\sqrt{2}}$
    		\item none of these
    	\end{enumerate}
    \end{multicols}
    \item Two circles $x^{2} + y^{2} = 6$ and $x^{2} + y^{2}-6x +8=0$ are given. Then the equation of the circle through their points of intersection and the point $\brak{1, 1}$ is \hfill {(1980)}
    \begin{enumerate}
    	\item $x^{2}+y^{2}-6x+4=0$ 
    	\item $x^{2}+y^{2}-3x+1=0$
    	\item $x^{2}+y^{2}-4y+2=0$
    	\item none of these
    \end{enumerate}
    \item The centre of the circle passing through the point $\brak{0.1}$ and touching the curve y = $x^{2}$ at \\ $\brak{2, 4}$.
    \hfill {(1983 )}
    \begin{multicols}{2}
    	\begin{enumerate}
    		\item $\brak{\frac{-16}{5}, \frac{27}{10}}$
    		\item $\brak{\frac{-16}{7}, \frac{53}{10}}$
    		\item $\brak{\frac{-16}{5}, \frac{53}{10}}$
    		\item none of these
    	\end{enumerate}
    \end{multicols}
    \item The equation of circle passing through $\brak{1, 1}$ and points of intersection of the circles $x^{2}+y^{2}+13x-3y=0$ and $2x^{2}+2y^{2}+4x-7y-25=0$ is
    \hfill {(1983 )}
    \begin{enumerate}
    	\item $4x^{2}+4y^{2}-30x-10y-25=0$
    	\item $4x^{2}+4y^{2}+30x-13y-25=0$
    	\item $4x^{2}+4y^{2}-17x-10y+25=0$
    	\item none of these
    \end{enumerate}
    \item The locus of the midpoint of a chord of the circle $x^{2}+y^{2}=4$ which subtends a right angle at the origin is \hfill {(1984 - 2 mark)}
    \begin{enumerate}
    	\item $x+y=2$
    	\item $x^{2}+y^{2}=1$
    	\item $x^{2}+y^{2}=2$
    	\item $x+y=1$
    \end{enumerate}
    \item If a circle is passing through the point $\brak{a, b}$ and it is cutting the circle $x^{2}+y^{2}=k^{2}$ orthogonally,  then the equation of the locus of its centre \\ is 
    \hfill {(1988 - 2 mark)}
    \begin{enumerate}
    	\item $2ax + 2by - (a^{2}+b^{2}+k^{2}) = 0$
    	\item $2ax + 2by - (a^{2}-b^{2}+k^{2}) = 0$
    	\item $x^{2} + y^{2}-3ax-4by+ (a^{2}+b^{2}-k^{2}) = 0$
    	\item $x^{2} + y^{2}-2ax-3by+ (a^{2}-b^{2}-k^{2}) = 0$
    \end{enumerate}
    \item If the two circles $(x-1)^{2} + (y-3)^{2} = r^{2}$ and $x^{2}+y^{2}-8x+2y+8=0$ intersect in two distinct points,  then \hfill {(1989 - 2 mark)} 
    \begin{enumerate}
    	\item $2<r<8$
    	\item $r<2$
    	\item $r=2$
    	\item $r>2$
    \end{enumerate}
    \item The lines $2x-3y=5$ and $3x-4y=7$ are diameters of a circle of area $154$ sq. units. The equation of this circle is\hfill {(1989 - 2 mark)}
    \begin{enumerate}
    	\item $x^{2}+y^{2}+2x-2y=62$
    	\item $x^{2}+y^{2}+2x-2y=47$
    	\item $x^{2}+y^{2}-2x+2y=47$
    	\item $x^{2}+y^{2}-2x+2y=62$
    \end{enumerate}
    \item The centre of the circle passing through the points $\brak{0, 0}$, $\brak{1, 0}$ and touching the circle $x^{2}+y^{2}=9$ is
    \hfill {(1992 )}
    \begin{enumerate}
    	\item $\brak{\frac{3}{2}, \frac{1}{2}}$
    	\item $\brak{\frac{1}{2}, \frac{3}{2}}$
    	\item $\brak{\frac{1}{2}, -2\frac{1}{2}}$
    	\item none of these
    \end{enumerate}
    \item The locus of the centre of a circle,  which touches the circle is $x^{2}+y^{2}-6x-6y+14=0$ and also touches the y-axis,  is given by the equation: \hfill {(1993 )}
    \begin{enumerate}
    	\item $x^{2}-6x-10y+14=0$
    	\item $x^{2}-10x-6y+14=0$
    	\item $y^{2}-6x-10y+14=0$
    	\item $y^{2}-10x-6y+14=0$
    \end{enumerate}
    \item The circles $x^{2}-10x+16=0$ and $x^{2}+y^{2}=r^{2}$ intersect each other in the two distinct points\\ if
    \hfill {(1994)}
    \begin{enumerate}
    	\item $r<2$
    	\item $r>8$
    	\item $2<r<8$
    	\item $2\leq r\leq8$
    \end{enumerate}
    \item The angle between the pair of tangents drawn from the point $\vec{P}$ to the circle $x^{2}+y^{2}+4x-6y+9\sin^{2}{\alpha}+13\cos^{2}{\alpha}=0$ is $2\alpha$. The equation of the locus of the point $\vec{P}$ is
    \hfill {(1996 )}
    \begin{enumerate}
    	\item $x^{2}+y^{2}+4x-6y+4=0$
    	\item $x^{2}+y^{2}+4x-6y-9=0$
    	\item $x^{2}+y^{2}+4x-6y-4=0$
    	\item $x^{2}+y^{2}+4x-6y+9=0$
    \end{enumerate}
    \item If two distinct chords,  drawn from the point $\brak{p, q}$ on the circle $x^{2}+y^{2}=px+qy$ (where $pq \neq 0$) are bisected by the x-axis,  then which are true
    \hfill {(1999 )}
    \begin{enumerate}
    	\item $p^{2}=q^{2}$
    	\item $p^{2}=8q^{2}$ 
    	\item $p^{2}<8q^{2}$
    	\item $p^{2}>8q^{2}$
    \end{enumerate}
        \item The triangle $PQR$ is inscribed in the circle $x^2+y^2=25$.If $Q$ and $R$ have co-ordinates $\brak{3, 4}$ and $\brak{-4, 3}$ respectively, then $\angle$QPR is equal to  
    \hfill$\brak{2000S}$
    
    
    \begin{multicols}{2}
    \begin{enumerate}
    
        \item $\frac{\pi}{2}$
        \item $\frac{\pi}{3}$
        \item $\frac{\pi}{4}$
        \item $\frac{\pi}{6}$
    \end{enumerate}
    \end{multicols}
     \item If the circles $x^2+y^2+2x+2ky+k=0$ intersect orthogonally, then $k$ is
        \hfill$\brak{2000S}$
    \begin{multicols}{2}
    \begin{enumerate}
        \item 2 or $-\frac{3}{2}$
        \item -2 or $-\frac{3}{2}$
        \item 2 or $\frac{3}{2}$
        \item $-$2 or $\frac{3}{2}$
    \end{enumerate}
    \end{multicols}
    \item Let $AB$ be a chord of the circle $x^2+y^2=r^2$ subtending a right angle at the centre.then the locus of the centroid of the triangle $PAB$  as $P$ moves on the circle is 
        \hfill$\brak{2001S}$
        \begin{multicols}{2}
    \begin{enumerate}
        \item a parabola
        \item a circle
        \item an ellipse
        \item a pair of straight lines
        \end{enumerate}
        \end{multicols}
        \item Let $PQ$ and $RS$ be tangents at the extremities of the diameter PR of a circle of radius $r$. If $PS$ and $RQ$ intersect at a point $X$ on the circumference of the circle,  then $2r$ equals
        \hfill$\brak{2001S}$
        \begin{multicols}{2}
        \begin{enumerate}
    \item $\sqrt{PQ.RS}$
     \item $\brak{PQ+RS}$
    \item $2PQ.RS/(PQ+RS)$
     \item {$\sqrt{(PQ^2+RS^2)}$}/2
     \end{enumerate}
     \end{multicols}
     \item If the tangent at the point $P$ on the circle $x^2+y^2+6x+6y=2$ meets a straight line $5x-2y+6=0$ at a point on the y-axis,  then the length of $PQ$ is 
             \hfill$\brak{2002S}$
             \begin{multicols}{2}
         \begin{enumerate}
             \item 4
             \item 2$\sqrt5$
             \item 5
             \item 3$\sqrt5$
\end{enumerate}
\end{multicols}
     \item The centre of the circle inscribed in square formed by the lines $x^2-8x+12=0$ and $y^2-14y+45=0$,  is
         \hfill$\brak{2003S}$
         \begin{multicols}{2}
     \begin{enumerate}
         \item $\brak{4, 7}$
         \item $\brak{7, 4}$
         \item $\brak{9, 4}$
         \item $\brak{4, 9}$
     \end{enumerate}
     \end{multicols}
     \item If one of the diameters of the circle $x^2+y^2-2x-6y+6=0$ is a chord to the circle with the centre $\brak{2, 1}$, then the radius of the circle is 
         \hfill$\brak{2004S}$
         \begin{multicols}{2}
     \begin{enumerate}
         \item $\sqrt3$
         \item $\sqrt2$
         \item 3
         \item 2
     \end{enumerate}
     \end{multicols}
     \item A circle is given by $x^2+$\brak{y-1}$^2=1$,  another circle $C$ touches it externally and also the x-axis,  then the locus of its centre is
         \hfill$\brak{2005S}$
         \begin{multicols}{2}
     \begin{enumerate}
         \item \{$\brak{x, y}$:$x^2=4y$\} $\bigcup$ \{$\brak{x, y}$:y$\le$0\}
         \item \{$\brak{x, y}$:$x^2+(y-1)^2=4$\} $\bigcup$ \{$\brak{x, y}$:y$\le$0\}
         \item \{$\brak{x, y}$:$x^2=y$\} $\bigcup$ \{$\brak{0, y}$:y$\le$0\}
         \item \{$\brak{x, y}$:$x^2=4$y\} $\bigcup$ \{$\brak{0, y}$:y$\le$0\}
         \end{enumerate}
         \end{multicols}
         \item Tangents drawn from the point $P\brak{1, 8}$ to the circle $x^2+y^2-6x-4y-11=0$ touch the circle at the points $\vec{A}$ and $\vec{B}$. The equation of the circumcircle of the triangle $PAB$ is
             \hfill$\brak{2009}$
             \begin{multicols}{2}
         \begin{enumerate}
             \item $x^2+y^2+4x-6y+19=0$
             \item $x^2+y^2-4x-10y+19=0$
             \item $x^2+y^2-4x+6y-29=0$
             \item $x^2+y^2-4x-6y+19=0$
             \end{enumerate}
             \end{multicols}
             \item The circle passing through the point $\brak{-1, 0}$ and touching the y-axis at $\brak{0, 2}$ also passes through the point
                 \hfill$\brak{2011}$
                 \begin{multicols}{2}
             \begin{enumerate}
                 \item $\brak{-\frac{3}{2}, 0}$
                 \item $\brak{-\frac{5}{2}, 2}$
                 \item $\brak{-\frac{3}{2}, \frac{5}{2}}$
                 \item $\brak{-4, 0}$
             \end{enumerate}
             \end{multicols}
             \item The locus of the mid-point of the chord of contact of tangents drawn from points lying on the straight line $4x-5y=20$ to the circle $x^2+y^2=9$ is
                 \hfill$\brak{2012}$
                 \begin{multicols}{2}
             \begin{enumerate}
                 \item $20\brak{x^2+y^2}-36x+45y=0$
                 \item $20\brak{x^2+y^2}+36x-45y=0$
                 \item $36\brak{x^2+y^2}-20x+45y=0$
                 \item $36\brak{x^2+y^2}+20x-45y=0$
             \end{enumerate}
             \end{multicols}
             \item A line $y=mx+1$ intersects the circle$(x-3)^2+(y+2)^2=25$ at the points $P$ and $Q$. if the mid point of the line segment $PQ$ has x-coordinate $-\frac{3}{5}$,  then which one of the following options is correct?
                 \hfill$\brak{ 2019}$
                 \begin{multicols}{2}
             \begin{enumerate}
                 \item $2\le m<4$
                 \item $-3\le m<-1$
                 \item $4\le m<6$
                 \item $6\le m<8$
             \end{enumerate}
             \end{multicols}
\item $ABCD$ is a square of side length $2$ units.$C_1$ is the circle touching all the sides of the square $ABCD$ and $C_2$ is the $circumcircle$ of square $ABCD$.L is a fixed line in same plane and R is a fixed point.\\
\begin{enumerate}
\item If $P$ is any point of $C_1$ and $Q$ is another point on $C_2$, then $\frac{PA^2+PB^2+PC^2+PD^2}{QA^2+QB^2+QC^2+QD^2}$

\hfill(2006-5M, -2)
\begin{enumerate}
\item $0.75$
\item $1.25$
\item $1$
\item $0.5$
\end{enumerate}
\item If a circle is such that it touches the line L and the circle $C_1$ externally, such that both the circles are on the same side of the line, then locus of centre of the circle 

\hfill(2006-5M, -2)
\begin{enumerate}
\item ellipse
\item hyperbola
\item parabola
\item circle
\end{enumerate}
\item A line L'through $\vec{A}$ is drawn parallel to BD.Point $S$ moves such that its distances from the line BD and the vertex $\vec{A}$ are equal.If locus of $S$ cuts L' at $T_2$ and $T_3$ and AC at $T_1$, then area of $\Delta T_1T_2T_3$ is

\hfill(2006-5M, -2)
\begin{enumerate}
\item $1/2$ sq.units
\item $2/3$ sq.units
\item $1$ sq.units
\item $2$ sq.units
\end{enumerate}

\item A circle $C$ of radius $1$ unit is inscribed in an equilateral triangle $PQR$.The points of contact of C with sides PQ, QR, RP are D, E, F respectively.The line PQ is given by the equation $\sqrt{3}x+y-6=0$ and the point D is $\brak{3\sqrt{3}/2,  3/2}$.Further, it is given that the origin and the centre of C are on same side of line PQ.\\

\item The equation of circle C is\hfill(2008)
\end{enumerate}
\begin{enumerate}
\item $(x-2\sqrt{3})^2 + (y-1)^2=1$
\item $(x-2\sqrt{3})^2 + (y+1/2)^2=1$
\item $(x-\sqrt{3})^2 + (y-1)^2=1$
\item $(x-\sqrt{3})^2 + (y+1)^2=1$
\end{enumerate}
\item Find the equation of the circle whose radius is 5 and which touches the circle $x^2+y^2-2x-4y-20=0$ at the point \brak{5, 5}

\hfill {\brak{1978}}
\item Let $\vec{A}$ be the centre of circle $x^2+y^2-2x-4y-20=0$. Suppose that the tangents at the points $\vec{B}$\brak{1, 7} and $\vec{D}$\brak{4, -2} on the circle meet at point $\vec{C}$. Find the area of the quadrilateral $ABCD$.

\hfill {\brak{1981 }}
\item Find the equations of the circle passing through \brak{-4, 3} and touching the lines $x+y=2$ and $x-y=2$

\hfill {\brak{1981 }}
\item Through a fixed point \brak{h, k} secants are drawn to the circle $x^2+y^2=r^2$. Show that the locus of the mid-points of the secants intercepted is $x^2+y^2=hx+ky$

\hfill {\brak{1983 }}
\item The abscissa of two points $\vec{A}$ and $\vec{B}$ are roots of the equation $x^2+2ax-b^2=0$ and their ordinates are roots of the equation $x^2+2px-q^2=0$. Find the equation and the radius of the circle with $AB$ as diameter.

\hfill {\brak{1984 }}
\item Lines $5x+12y-10=0$ and $5x-12y-40=0$ touch a Circle $C_1$ of diameter 6. If the centre of $C_1$ lies in the first quadrant,  find the equation of circle $C_2$ which is concentric with $C_1$ and cuts intecepts of length 8 on these lines

\hfill {\brak{1986 }}
\item Let a given Line $L_1$ intersects the $x$ and $y$ axes at $\vec{P}$ and $\vec{Q}$ respectively. Let another line $L_2$,  perpendicular to $L_1$,  cut the $x$ and $y$ axes at $\vec{R}$ and $\vec{S}$,  respectively. Show that the locus of the point of intersection of $PS$ and $QR$ is a circle passing through origin.

\hfill {\brak{1987 - 3 marks}}
\item The circle $x^2+y^2-4x-y+4=0$ is inscribed in a triangle which has two of its sides along the co-ordinate axes. The locus of circumcentre of the triangle is $x+y-xy+k(x^2+y^2)\textsuperscript{1/2}$. Find $k$.

\hfill {\brak{1987 }}
\item If $\brak{ m_i,  \frac{1}{m_i}},  m_i > 0,  i = 1,  2,  3,  4$ are four distinct points on a circle,  then show that $m_1m_2m_3m_4=1$

\hfill {\brak{1989 }}
\item A circle touches the line $y=x$ at a point $\vec{P}$ such that $OP=4\sqrt{2}$,  where O is the origin. The circle contains the point \brak{-10, 2} in its interior and the length of its chord on the line $x+y=0$ is $6\sqrt{2}$. Determine the equation of circle.

\hfill {\brak{1990 }}
\item Two circles,  each of radius 5 units,  touch each other at \brak{1, 2}. If the equation of common tangent is $4x+3y=10$,  find the equations of circles.

\hfill {\brak{1991 }}
	\item Let a circle be given by $2x\brak{x-a}+y\brak{2y-b}=0$, $\brak{a\neq0, b\neq0}$.Find the condition on $a$ and $b$ if two chords,  each bisected by the x-axis, can be drawn to the circle from $\brak{a, \frac{b}{2}}$.                         

\hfill(1992- 6 Marks)




\item Consider a family of circles passing through two fixed points $\vec{A}$$\brak{3, 7}$ and $\vec{B}$$\brak{6, 5}$. Show that chords in which the circle $x^2+y^2-4x-6y-3=0$ cuts the members of the family are concurrent at a point. Find the coordinate of this point.
	        
\hfill(1993)





\item Find the coordinates of the point at which the circles $x^2+y^2-4x-2y=-4$ and $x^2+y^2-12x-8y=-36$ touch each other. Also find equations common tangents touching the circles in the distinct points.                        

\hfill(1993)


\item Find the intervals of values of a for which the line $y+x=0$ bisects two chords drawn from a point $\brak {\frac{1+\sqrt{2}a}{2}, \frac{1-\sqrt{2}a}{2}}$ to the circle $2x^2+2y^2-\brak{1+\sqrt{2}a}x-\brak{1-\sqrt{2}a}y=0$.  

\hfill(1996)





\item A circle passes through three points $\vec{A}$, $\vec{B}$ and C with the line segment AC as its diameter. A line passing through $\vec{A}$ intersects the chord BC at point D inside the circle. If angles $DAB$ and $CAB$ are $\alpha$ and $\beta$ respectively and the distance between the point $\vec{A}$ and midpoint of the line segment DC is $d$,  prove that the area of the circle is $\frac{\pi d^2 \cos^2{\alpha} }{\cos^2{\alpha}+\cos^2{\beta}+ 2\cos{\alpha} \cos{\beta} \cos{\brak{\beta-\alpha}}}$                

\hfill(1996)





\item Let $C$ be any circle with centre $\brak{0, \sqrt{2}}$. Prove that at the most two rational points can be there on $C$.(A rational point is a point both of whose coordinates are rational numbers)
	           
\hfill(1997)




\item $C_{1}$ and $C_{2}$ are two concentric circles,  the radius of $C_{2}$ being twice that of $C_{1}$. From a point $P$ on $C_{2}$,  tangents $PA$ and $PB$ are drawn to $C_{1}$. Prove that the centroid of the triangle $PAB$ lies on $C_{1}$.
	           \hfill(1998)




\item Let $T_{1}$,  $T_{2}$ be two tangents drawn from $\brak{2, 0}$ onto the circle $C$:$x^2+y^2=1$. Determine the circles touching $C$ and having $T_{1}$,  $T_{2}$ as their pair of tangents. Further,  find the equations of all possible common tangents to these circles,  when taken two at a time.
                  \hfill(1999)




\item Let $2x^2+y^2-3xy=0$ be the equation of pair of tangents drawn from the origin $O$ to a circle of radius 3 with the centre in the first quadrant. If $\vec{A}$ is one of the points of contact,  find the length of $OA$.                   \hfill(2001)




\item Let $C_{1}$ and $C_{2}$ be two circles with $C_{2}$ lying inside $C_{1}$. A circle $C$ lying inside $C_{1}$ touches $C_{1}$ internally and $C_{2}$ externally. Identify the locus of centre of $C$.                                \hfill(2001)



\item For the circle $x^2+y^2=r^2$,  find the value of $r$ for which the area enclosed by the tangents drawn from the point $P$$\brak{6, 8}$ to the circle and the chord of contact is maximum.


\hfill(2003) 





\item Find the equation of circle touching the line $2x+3y+1=0$ at $\brak{1, -1}$ and cutting orthogonally the circle having line segment joining \brak{0, 3} and \brak{-2, -1} as diameter.


\hfill(2004)     




\item Circles with radii 3, 4 and 5 touch each other externally. If $P$ is the point of intersection of tangents to these circles at their points of contact,  find the distance of $P$ from the points of contact.
	           \hfill(2005) 
    \item No tangent can be drawn from point $\brak{5/2,  1}$ to circumcircle of triangle with vertices $\brak{1,  \sqrt{3}}$,  $\brak{1,  -\sqrt{3}}$,  and $\brak{3,  -\sqrt{3}}$.
    \hfill{(1985 )}
    \item The line $x+3y = 0$ is a diameter of the circle $ x^{2} + y^{2} - 6x +2y = 0$
    \hfill{(1989 )}


\end{enumerate}
