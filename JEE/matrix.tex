\begin{enumerate}[label=\thesubsection.\arabic*.,ref=\thesubsection.\theenumi]
\item
%2nd question
	If the system of linear equations $$x+2ay+az = 0$$ $$x+3by+bz = 0$$ $$x+4cy+cz = 0$$ has a non-zero solution, then a,b,c

	\hfill (2003)

\begin{multicols}{2}
	\begin{enumerate}
                \item satisfy $a+ 2b+3c = 0$
                \item are in A.P
                \item are in G.P
                \item are in H.P
        \end{enumerate} 	
\end{multicols}
%
\item
%4th question
        If $\vec{A}=\myvec{a & b \\ b & a}$ and $\vec{A}^{2}=\myvec{\alpha & \beta \\ \beta & \alpha}$, then 
	\hfill (2003)
       %                                      
\begin{multicols}{2}
        \begin{enumerate}
		\item $\alpha=2ab, \beta=a^{2}+b^{2}$
                \item $\alpha=a^{2}+b^{2}, \beta=ab$
                \item $\alpha=a^{2}+b^{2}, \beta=2ab$
                \item $\alpha=a^{2}+b^{2}, \beta=a^{2}-b^{2}$
        \end{enumerate}
\end{multicols}
%
			\newcommand{\adj}[1]{\text{adj}\brak{#1}}
%
    \item The number of $3\times3$ non-singular matrices with four entries as $1$ and all other entries as $0$, is 
	\hfill{\brak{2010}}
%
\begin{multicols}{4}
	\begin{enumerate}
                \item $5$ 
		\item $6$
		\item atleast $7$
		\item less than $4$ 
	\end{enumerate}
\end{multicols}
    %Question22
    \item Let $\vec{A}$ be a $2\times2$ matrix with non-zero entries and let $\vec{A}^2 = \vec{I}$, where $\vec{I}$ is $2\times2$ identity matrix. Define 
	\newline
	$Tr\brak{\vec{A}}$- sum of diagonal elements of $\vec{A}$ and
	\newline
	$\abs{\vec{A}}$ - determinant of matrix $\vec{A}$.
	\newline
	{Statement - 1:} $Tr\brak{\vec{A}} = 0$.
	\newline
	{Statement - 2:} $\abs{\vec{A}} = 1$
	\hfill{\brak{2010}}
	\begin{enumerate}
		\item Statement - 1 is true, Statement - 2 is true; Statement - 2 is \textbf{not} a correct explanation for Statement-1. 
	    	\item Statement - 1 is true, Statement - 2 is false. 
	    	\item Statement - 1 is false, Statement - 2 is true.
	    	\item Statement - 1 is true, Statement - 2 is true; Statement - 2 is a correct explanation for Statement-1. 
	\end{enumerate}
%
    %Question23
    \item The system of linear equations
	\hfill{\brak{2010}}{\parfillskip0pt\par}
	\begin{align*}
		x_1 + 2x_2 + x_3 &= 3\\
		2x_1 + 3x_2 + x_3 &= 3\\
		3x_1 + 5x_2 + 2x_3 &= 1
	\end{align*}
\begin{multicols}{2}
	\begin{enumerate}
		\item exactly $3$ solutions
	    	\item a unique solution
	    	\item no solution
	    	\item infinite number of solutions
	\end{enumerate}
\end{multicols}

    %Question24
    \item The number of values of $k$ for which the linear equations $4x + ky + 2z = 0$, $kx + 4y + z = 0$ and $2x + 2y + z=0$ possess a non zero solution is 
	\hfill {\brak{2011}}{\parfillskip0pt\par}
\begin{multicols}{2}
        \begin{enumerate}
                \item $2$
                \item $1$
                \item zero
                \item $3$
        \end{enumerate}
\end{multicols}
%
    %Question25
    \item Let $\vec{A}$ and $\vec{B}$ be two symmetrix matrices of order $3$.
	\newline
	{Statement - 1:} $\vec{A}\brak{\vec{BA}}$ and $\brak{\vec{AB}}\vec{A}$ are symmetric matrices. 
	\newline
	{Statement - 2:} $\vec{AB}$ is symmetric matrix if matrix multiplication of $\vec{A}$ with $\vec{B}$ is commutative.
%
	\begin{enumerate}
		\item Statement - 1 is true, Statement - 2 is true; Statement - 2 is \textbf{not} a correct explanation for Statement-1. 
	    	\item Statement - 1 is true, Statement - 2 is false. 
	    	\item Statement - 1 is false, Statement - 2 is true.
	    	\item Statement - 1 is true, Statement - 2 is true; Statement - 2 is a correct explanation for Statement-1. 
	\end{enumerate}
%
    %Question26
	\item Let \begin{align*}
	\vec{A} = \myvec{ 
		1&0&0\\
		2&1&0\\
		3&2&1
	}
	\end{align*} If $\vec{u_1}$ and $\vec{u_2}$ are column matrices such that
	\begin{align*}
		\vec{Au_1} = \myvec{1\\0\\0},\
		\vec{Au_2} = \myvec{1\\0\\0},
	\end{align*} then $\vec{u_1} + \vec{u_2}$ is equal to
	\hfill{\brak{2012}}
\begin{multicols}{4}
        \begin{enumerate}
	    	\item $ \myvec{-1\\1\\0} $ 
		\item $ \myvec{-1\\1\\-1} $ 
		\item $ \myvec{-1\\-1\\0} $
		\item $ \myvec{1\\-1\\-1} $ 
        \end{enumerate}
\end{multicols}
%
    %Question27
	\item Let $\vec{P}$ and $\vec{Q}$ be $3\times3$ matrices $\vec{P}\neq \vec{Q}$. If $\vec{P}^3=\vec{Q}^3$ and $\vec{P}^2\vec{Q}=\vec{Q}^2\vec{P}$ then determinant of $\brak{\vec{P}^2+\vec{Q}^2}$ is equal to
	\hfill{\brak{2012}}
\begin{multicols}{4}
        \begin{enumerate}
                \item $-2$
                \item $1$
                \item $0$
                \item $-1$
        \end{enumerate}
\end{multicols}
%
    %Question28
	\item If \begin{align*}
		\vec{P} = \myvec{1&\alpha&3\\
		1&3&3\\
		2&4&4}
	\end{align*} is the adjoint of a $3\times3$ matrix $\vec{A}$ and $\abs{\vec{A}} = 4$, then $\alpha$ is equal to
	\hfill{\brak{2014}}
\begin{multicols}{4}
        \begin{enumerate}
                \item $4$
                \item $11$
                \item $5$
                \item $0$
        \end{enumerate}
\end{multicols}
%
    %Question30
	\item If $\vec{A}$ is a $3\times3$ non-singular matrix such that $\vec{A}\vec{A}^{\prime}=\vec{A}^{\prime}\vec{A}$ and $\vec{B}=\vec{A}^{-1}\vec{A}^{\prime}$, then $\vec{B}\vec{B}^{\prime}$ equals
	\hfill {\brak{2014}}{\parfillskip0pt\par}
\begin{multicols}{4}
	\begin{enumerate}
	    	\item $\vec{B}^{-1}$
		\item $\brak{\vec{B}^{-1}}^{\prime}$
		\item $\vec{I}+\vec{B}$ 
		\item $\vec{I}$
        \end{enumerate}
\end{multicols}
%
    %Question31
    \item The set of all values of $\lambda$ for which the system of linear equations
	\begin{align*}
		2x_1-2x_2+x_3 &= \lambda x_1\\
		2x_1-3x_2+2x_3 &= \lambda x_2\\
		-x_1+2x_2 &= \lambda x_3
	\end{align*}
	has a non-trivial solution
	\hfill{\brak{2015}}
\begin{multicols}{2}
	\begin{enumerate}
		\item contains two elements
		\item contains more than two elements
		\item is an empty set
		\item is a singleton
	\end{enumerate}
\end{multicols}
%

    %Question32
	\item If \begin{align*} \vec{A} = \myvec{1&2&2\\2&1&-2\\a&2&b} \end{align*} is a matrix satisfying the equation $\vec{A}\vec{A}^{\top} = 9\vec{I}$, where $\vec{I}$ is $3\times3$ identity matrix, then the ordered part $\brak{a,b}$ is equal to:
	\hfill {\brak{2015}}{\parfillskip0pt\par}
\begin{multicols}{4}
	\begin{enumerate}
	    	\item $\brak{2,1}$ 
	    	\item $\brak{-2,-1}$ 
	    	\item $\brak{2,-1}$ 
	   	\item $\brak{-2,1}$ 
	\end{enumerate}
\end{multicols}
    %Question33
	\item The system of linear equations 
	\begin{align*}
		x+\lambda y-z &= 0\\
		\lambda x-y-z &= 0\\
		x+y-\lambda z &= 0
	\end{align*}
	has a non-trivial solution for
	\hfill{\brak{2016}}
\begin{multicols}{2}
	\begin{enumerate}
		\item exactly two values of $\lambda$ 
		\item exactly three values of $\lambda$ 
		\item inifinitely many values of $\lambda$
		\item exactly one value of $\lambda$ 
	\end{enumerate}
\end{multicols}
    %Question34
\item If \begin{align*} \vec{A} = \myvec{5a&-b\\3&2}\end{align*} and $\vec{A} \adj{\vec{A}} = \vec{A}\vec{A}^{\top}$, then $5a + b$ is equal to
	\hfill{\brak{2016}}
\begin{multicols}{4}
	\begin{enumerate}
	    	\item $4$ 
	    	\item $13$
	    	\item $-1$
	   	\item $5$ 
	\end{enumerate}
\end{multicols}
    %Question35
	\item Let $k$ be an integer such that triangle with vertices $\brak{k, -3k}$, $\brak{5,k}$, $\brak{-k,2}$ has area $28$ sq. units. Then the orthocentre of this triangle is at the point
	\hfill{\brak{2017}}
\begin{multicols}{4}
	\begin{enumerate}
	    	\item $\brak{2,\frac{1}{2}}$ 
	    	\item $\brak{2,\frac{-1}{2}}$ 
	     	\item $\brak{1,\frac{3}{4}}$ 
	    	\item $\brak{1,\frac{-3}{4}}$ 
	\end{enumerate}
\end{multicols}
\item If  $\vec{A}=\myvec{
    2 & -3 \\
    -4 & 1
}$
    , then $\adj{3\vec{A}^2+12\vec{A}}$ is equal to
\hfill{(2017)}
\begin{enumerate}
\begin{multicols}{2}
    \item $\myvec{
    72 & -63 \\
    -84 & 51
}$ 
\columnbreak
    \item $\myvec{
    72 & -84 \\
    -63 & 51 
}$ \\
\end{multicols}
\begin{multicols}{2}
    \item $\myvec{
    51 & 63 \\
    84 & 72
}$ \\

    \item $\myvec{
    51 & 84 \\
    63 & 72
    
}$
\end{multicols}
\end{enumerate} 
\item If the system of linear equations 
\begin{align*}x+ky+3z=0 \\
3x+ky-2z=0 \\
2x+4y-3z=0 \end{align*}
has a non-zero solution $\brak{x,y,z}$, then $\frac{xz}{y^2}$ is equal to 
\hfill{(2018)}
\begin{enumerate}
\begin{multicols}{4}
    \item 10
    \item -30
    \item 30
    \item -10
    \end{multicols}
\end{enumerate}
\item The system of linear equations 
\hfill{(2019)}
\begin{align*}x+y+z=2 \\
2x+3y+2z=5 \\
2x+3y+\brak{a^2-1}z=a+1 \end{align*} 
\begin{multicols}{2}
\begin{enumerate}
    \item is consistent when $a=4$
    \item has a unique solution for $|a|= \sqrt{3}$
    \item has infinitely many solutions for $a=4$
    \item is consistent when $|a|= \sqrt{3}$
\end{enumerate}
\end{multicols}
\item If $\vec{A}= \myvec{
    \cos{\theta} & -\sin{\theta} \\
    \sin{\theta} & \cos{\theta}
}$, then the matrix $\vec{A}^{-50}$ when $\theta=\frac{\pi}{12}$, is equal to: 
\hfill{(2019)}
\begin{enumerate}
\begin{multicols}{2}
    \item $\myvec{
   \frac{1}{2}  & -\frac{\sqrt{3}}{2}  \\
    \frac{\sqrt{3}}{2} & \frac{1}{2}
}$ \\ 
\columnbreak
    \item $\myvec{
   \frac{\sqrt{3}}{2}  & -\frac{1}{2}  \\
    \frac{1}{2} & \frac{\sqrt{3}}{2}
}$ \\ 
\end{multicols}
\begin{multicols}{2}

    \item $\myvec{
   \frac{\sqrt{3}}{2}  & \frac{1}{2}  \\
    -\frac{1}{2} & \frac{\sqrt{3}}{2}
}$ \\

    \item $\myvec{
   \frac{1}{2}  & \frac{\sqrt{3}}{2}  \\
    -\frac{\sqrt{3}}{2} & \frac{1}{2}
}$
\end{multicols}
\end{enumerate} 
\item If \begin{align*}\myvec{
    1 & 1 \\
    0 & 1
}.\myvec{
    1 & 2 \\
    0 & 1
}.\myvec{
    1 & 3 \\
    0 & 1
}\dots \myvec{
    1 & n-1 \\
    0 & 1
}=\myvec{
    1 & 78 \\
    0 & 1
}.\end{align*}
then the inverse of $\myvec{
    1 & n \\
    0 & 1
}$ is 
\hfill{(2019)} 
\begin{enumerate}
\begin{multicols}{2}
    \item $\myvec{
        1 & 0 \\
        12 & 1 
    }$
    \columnbreak
    \item $\myvec{
        1 & -13 \\
        0 & 1 
    }$
    \end{multicols}
    \begin{multicols}{2}
    \item $\myvec{
        1 & -12 \\
        0 & 1 
    }$
     \item $\myvec{
        1 & 0 \\
        13 & 1 
    }$   
    \end{multicols}
\end{enumerate}
	\item Let $\vec{A}=\myvec{
			0 & 0 & -1 \\
			0 & -1 & 0 \\
			-1 & 0 & 0}$. The only correct statement about the matrix $\vec{A}$ is \hfill{(2004)}
\begin{multicols}{2}
		\begin{enumerate}
			\item $\vec{A}^2=\vec{I}$
			\item $\vec{A} = \brak{-1}\vec{I}$, where $\vec{I}$ is a unit matrix
			\item $\vec{A}^{-1}$ does not exist
			\item $\vec{A}$ is a zero matrix
		\end{enumerate}
\end{multicols}
%6
	\item Let $\vec{A}=\myvec{
			1 & -1 & 1 \\
			2 & 1 & -3 \\
			1 & 1 & 1}$, and $10\vec{B}=\myvec{
			4 & 2 & 2 \\
			-5 & 0 & \alpha \\
			1 & -2 & 3}$. If $\vec{B}$ is the inversion of matrix $\vec{A}$, then $\alpha$ is \hfill{(2004)}
\begin{multicols}{4}
		\begin{enumerate}
			\item 5
			\item -1
			\item 2
			\item -2
		\end{enumerate}
\end{multicols}
%7
	\item If $\vec{A}^2-\vec{A}+\vec{I}=0$, then the inverse of $\vec{A}$ is \hfill{(2005)}
\begin{multicols}{4}
		\begin{enumerate}
			\item $\vec{A}+\vec{I}$
			\item $\vec{A}$
			\item $\vec{A}-\vec{I}$
			\item $\vec{I}-\vec{A}$
		\end{enumerate}
\end{multicols}
%9
	\item The system of equations 
\begin{align*}
	\alpha x+y+z  &= \alpha -1  \\
		x+ \alpha y+z &= \alpha -1 \\
		x+y+ \alpha z &= \alpha -1 
\end{align*}
		has infinite solutions, if $\alpha$ is \hfill{(2005)}
\begin{multicols}{4}
		\begin{enumerate}
			\item -2
			\item either -2 or 1
			\item not -2
			\item 1
		\end{enumerate}
\end{multicols}
	\item If $\vec{A}$ and $\vec{B}$ are square matrices of size $n \times n$ such that $\vec{A}^2-\vec{B}^2=\brak{\vec{A}-\vec{B}}\brak{\vec{A}+\vec{B}}$, then which of the following will be always true? \hfill{(2006)}
\begin{multicols}{2}
		\begin{enumerate}
			\item $\vec{A}=\vec{B}$
			\item $\vec{A}\vec{B}=\vec{B}\vec{A}$
			\item either of $\vec{A}$ or $\vec{B}$ is zero matrix
			\item either of $\vec{A}$ or $\vec{B}$ is identity matrix
		\end{enumerate}
\end{multicols}
%13
	\item Let $\vec{A}=\myvec{
			1 & 2 \\
			3 & 4}$ and $\vec{B}=\myvec{
			a & 0 \\
			0 & b}$, a,b $\in$ N. Then \hfill{(2006)}
		\begin{enumerate}
			\item there cannot exist any $\vec{B}$ such that $\vec{A}\vec{B}=\vec{B}\vec{A}$
			\item there exist more than one but finite number of $\vec{B}$'s such that $\vec{A}\vec{B}=\vec{B}\vec{A}$
			\item there exists exactly one $\vec{B}$ such that $\vec{A}\vec{B}=\vec{B}\vec{A}$
			\item there exist infinitely many $\vec{B}$'s such that $\vec{A}\vec{B}=\vec{B}\vec{A}$
		\end{enumerate}
	\item Let $\vec{A}$ be a $2 \times 2$ matrix with real entries. Let $\vec{I}$ be the $2 \times 2$ identity matrix. Denote by tr\brak{\vec{A}},the sum of diagonal entries of $\vec{A}$. Assume that $\vec{A}^2 = \vec{I}$. \hfill{(2008)} \\
		Statement-1 : If $\vec{A} \neq \vec{I}$ and $\vec{A} \neq -\vec{I}$,then det\brak{\vec{A}}=-1 \\
		Statement-2 : If $\vec{A} \neq \vec{I}$ and $\vec{A} \neq -\vec{I}$,then $tr\brak{\vec{A}} \neq 0$.
		\begin{enumerate}
			\item Statement-1 is false, Statement-2 is true
			\item Statement-1 is true, Statement-2 is true;Statement-2 is a correct explanation for Statement-1
			\item Statement-1 is true, Statement-2 is true;Statement-2 is not a correct explanation for Statement-1
			\item Statement-1 is true, Statement-2 is false
		\end{enumerate}
%17
	\item Let a,b,c be any real numbers. Suppose that there are real numbers x,y,z not all zero such that $x=cy+bz$, $y=az+cx$, and $z=bx+ay$. Then $a^2+b^2+c^2+2abc$ is equal to \hfill{(2008)}
\begin{multicols}{4}
		\begin{enumerate}
			\item 2
			\item -1
			\item 0
			\item 1
		\end{enumerate}
\end{multicols}
%18
	\item Let $\vec{A}$ be a square matrix all of whose entries are integers. Then which of the following is true? \hfill{(2008)}
		\begin{enumerate}
			\item If $det\brak{\vec{A}}\neq\pm1$, then $\vec{A}^{-1}$ exists but all its entries are not necessarily integers
			\item If $det\brak{\vec{A}}\neq\pm1$, then $\vec{A}^{-1}$ exists and all its entries are non integers
			\item If $det\brak{\vec{A}}=\pm1$, then $\vec{A}^{-1}$ exists but all its entries are integers
			\item If $det\brak{\vec{A}}=\pm1$, then $\vec{A}^{-1}$ need not exist
		\end{enumerate}
%19
		\item 
			Let $\vec{A}$ and $\vec{B}$ be $3\times3$ matrices of real numbers, where $\vec{A}$ is symmetric, $\vec{B}$ is skew-symmetric and $(\vec{A}+\vec{B})(\vec{A}-\vec{B})=(\vec{A-B})(\vec{A+B})$. If $\vec{(AB)}^\top=(-1)^k\vec{AB}$, where $\vec{(AB)}^\top$ is the transpose of the matrix $\vec{AB}$, then the possible values of $k$ are 
			\rule{1cm}{0.01pt}.
		\hfill(2008)
    \item 
        Let $\vec{M}$ and $\vec{N}$ be two $3 \times 3$ non-singulr skew-symmetric matrices such that $\vec{M}\vec{N}=\vec{N}\vec{M}$. If $\vec{P}^{\top}$ denotes the transpose of $\vec{P}$, then $\vec{M}^2\vec{N}^2\brak{\vec{M}^{\top}\vec{N}^{-1}}^{-1} \brak{\vec{M}\vec{N}^{-1}}^{\top}$ is equal to 
\begin{multicols}{4}
        \begin{enumerate}
            \item $\vec{M}^2$
            \item $-\vec{N}^2$
            \item $-\vec{M}^2$
            \item $\vec{M}\vec{N}$
        \end{enumerate}
\end{multicols}
        \hfill (2011)
    \item 
        If the adjoint of a $3 \times 3$ matrix $\vec{P}$ is 
                $\myvec{	
                    1&4&4\\
                    2&1&7\\
                    1&1&3
                }$
        , then the possible value(s) of the determinant of $\vec{P}$ is (are)
            \hfill (2012)
\begin{multicols}{4}
            \begin{enumerate}
                \item $-2$
                \item $-1$
                \item $1$
                \item $2$
            \end{enumerate}
\end{multicols}
    \item 
        For $3 \times 3$ matrices $\vec{M}$ and $\vec{N}$, which of the following statement(s) is (are) NOT correct?
            \begin{enumerate}
                \item $\vec{N}^{\top}\vec{M}\vec{N}$ is symmetric or skew symmetric, according as $\vec{M}$ is symmetric or skew symmetric
                \item $\vec{M}\vec{N}-\vec{N}\vec{M}$ is skew symmetric for all matrices $\vec{M}$ and $\vec{N}$.
                \item $\vec{M}\vec{N}$ is symmetric for all symmetric matrices $\vec{M}$ and $\vec{N}$.
                \item (adj$\vec{M}$)(adj$\vec{N}$) = adj($\vec{M}\vec{N}$) for all invertible matrices $\vec{M}$ and $\vec{N}$.
            \end{enumerate}
            \hfill (2013)
    \item 
        Let $\omega$ be a complex cube root of unity with $\omega \neq 1 $ and $\vec{P}={p_{ij}}$ be a $n \times n$ matrix with $p_{ij} = \omega^{i+j}$. Then $p^2 \neq 0$, when $n=$
        \hfill (2013)
\begin{multicols}{4}
        \begin{enumerate}
            \item $57$
            \item $55$
            \item $58$
            \item $56$
        \end{enumerate}
\end{multicols}
    \item 
        Let $\vec{M}$ be a $2 \times 2$ symmetric matrix with integer entries. Then $\vec{M}$ is invertible if
            \begin{enumerate}
                \item The first column of $\vec{M}$ is the transpose of the second row of $\vec{M}$
                \item The second row of $\vec{M}$ is the transpose of the first column of $\vec{M}$
                \item $\vec{M}$ is a diagonal matrix with non-zero entries in the main diagonal
                \item The product of entries in the main diagonal of $\vec{M}$ is not the square of an integer
            \end{enumerate}
            \hfill (2014)
    \item
        Let $\vec{M}$ and $\vec{N}$ be two $3 \times 3$ matrices such that $\vec{M}\vec{N}=\vec{N}\vec{M}$. Further, if $\vec{M} \neq \vec{N}^2$ and $\vec{M}^2 = \vec{N}^4$, then
            \hfill (2014)
            \begin{enumerate}
                \item determinant of $(\vec{M}^2 + \vec{N}^2)$ is $0$
                \item there is $3 \times 3$ non-zero matrix $\vec{U}$ such that $(\vec{M}^2+\vec{M}\vec{N}^2)\vec{U}$ is the zero matrix
                \item determinant of $(\vec{M}^2 + \vec{M}\vec{N}^2) \geq 1$
                \item determinant of $(\vec{M}^2 + \vec{M}\vec{N}^2)\vec{U}$ equals the zero matrix then $\vec{U}$ is the zero matrix
            \end{enumerate}
    \item
        Let $\vec{X}$ and $\vec{Y}$ be two arbitrary, $3 \times 3$, non-zero, skew-symmetric matrices and $Z$ be an arbitrary $3 \times 3$, non-zero, symmetric matrix. Then which of the following matrices is (are) skew symmetric?
        \hfill (2015)
\begin{multicols}{4}
        \begin{enumerate}
            \item $\vec{Y}^3Z^4 -Z^4\vec{Y}^3$
            \item $\vec{X}^{44} + \vec{Y}^{44}$
            \item $\vec{X}^4Z^3 -Z^3\vec{X}^4$
            \item $\vec{X}^{23} + \vec{Y}^{23}$
        \end{enumerate}
\end{multicols}
    \item 
        Let $\vec{P} = 
            \myvec{		
                3 & -1 & -2\\
                2 & 0 & \alpha\\
                3 & -5 & 0
            }$,
        where $\alpha \in \mathbb{R}$. Suppose $\vec{Q}= \mathrm{\sbrak{q_{ij}}}$ is a matrix such that $\vec{P}\vec{Q}=k\vec{I}$, where $k \in \mathbb{R}, k \neq 0$ and $\vec{I}$ is the identity matrix of order $3$. If $q_{23} = -\frac{k}{8}$ and det\brak{\vec{Q}}$= \frac{k^2}{2}$, then
        \hfill (2016)
\begin{multicols}{2}
        \begin{enumerate}
            \item $a=0, k=8$
            \item $4a-k+8=0$
            \item det $\brak{\vec{P} \adj{\vec{Q}}} = 2^9$
            \item det $\brak{\vec{Q} \adj{\vec{P}}} = 2^{13}$
        \end{enumerate}
\end{multicols}
    \item
        Let $a, \lambda, \mu, \in \mathbb{R}$. Consider the system of linear equations $$ax+2y=\lambda$$ $$3x-2y=\mu$$ Which of the following statement(s) is (are) correct?
        \begin{enumerate}
            \item If $a=-3$, then the system has infinitely many solutions for all value of $\lambda$ and $\mu$.
            \item If $a \neq -3$, then the system has unique solution for all values of $\lambda$ and $\mu$.
            \item If $\lambda + \mu = 0$, then the system has infinitely many solutions for $a = -3$.
            \item If $\lambda + \mu \neq 0$, then the system has no solution for $a = -3$
        \end{enumerate}
        \hfill (2016)
    \item 
        Which of the following is (are) not the square of a $3 \times 3$ matrix with real entries?
\begin{multicols}{2}
            \begin{enumerate}
                \itemsep0.4em
                \item 
                    $\myvec{
                        1&0&0\\
                        0&1&0\\
                        0&0&1
                    }$\\
                \item 
                    $\myvec{
                        1&0&0\\
                        0&1&0\\
                        0&0&-1
                    }$\\
                \item 
                    $\myvec{
                        1&0&0\\
                        0&-1&0\\
                        0&0&-1
                    }$\\
                \item 
                    $\myvec{
                        -1&0&0\\
                        0&-1&0\\
                        0&0&-1
                    }$\\
            \end{enumerate}
\end{multicols}
            \hfill (2017)	
\item Let S be set of all column matrix $\myvec{b_1\\b_2\\b_3}$ such that $b_1, b_2, b_3 \in \mathbb{R}$ and the system of equations (in real variables)\begin{align*}-x+2y+5z=b_1\\2x-4y+3z=b_2\\x-2y+2z=b_3\end{align*} has at least one solution. Then, which of the following system(s) (in real variables) has (have) at least one solution for each $\myvec{b_1\\b_2\\b_3}\in S$ \hfill(2018)
%
\begin{enumerate}
    \item $x+2y+3z=b_1, 4y+5z=b_2 \text{ and }x+2y+6z=b_3$
    \item $x+y+3z=b_1, 5x+2y+6z=b_2\text{ and }-2x-y-3z=b_3$
    \item $-x+2y-5z=b_1,2x-4y+10z=b_2\text{ and }x-2y+5z=b_3$
    \item $sx+2y+5z=b_1,2x+3z=b_2,x+4y-5z=b_3$\\[2pt]
\end{enumerate}
%
\item Let $\vec{M}=\myvec{
    0 & 1 & a\\
    1 & 2 & 3\\
    3 & b & 1
}$ and $\adj{\vec{M}}=\myvec{
    -1 & 1 & -1\\
    8 & -6 & 2\\
    -5 & 3 & 1
}$ where $a$ and $b$ are real numbers. Which of the following options is/are correct? \hfill (2019)
\begin{multicols}{2}
\begin{enumerate}
    \item $a+b=3$
    \item $\mathop{det}(\mathop{adj}\vec{M}^2)=81$
    \item $(adj\vec{M})^{-1}+adj\vec{M}^{-1}=-\vec{M}$
    \item If $\vec{M}\myvec{
        \alpha\\
        \beta\\
        \gamma
} = \myvec{
        1\\
        2\\
        3
}$ then $\alpha-\beta+\gamma=3$\\[2pt]
\end{enumerate}
\end{multicols}
\item Let
$$
\vec{P}_1=\vec{I}=\myvec{
    1 & 0 & 0\\
    0 & 1 & 0\\
    0 & 0 & 1
},  \vec{P}_2 = \myvec{
    1 & 0 & 0\\
    0 & 0 & 1\\
    0 & 1 & 1
},\vec{P}_3 = \myvec{
    0 & 1 & 0\\
    1 & 0 & 0\\
    0 & 0 & 1
}, \vec{P}_4 = \myvec{
    0 & 1 & 0\\
    0 & 0 & 1\\
    1 & 0 & 0
},$$
$$\vec{P}_5 = \myvec{
    0 & 0 & 1\\
    1 & 0 & 0\\
    0 & 1 & 0
}, \vec{P}_6 = \myvec{
    0 & 0 & 1\\
    0 & 1 & 0\\
    1 & 0 & 1
}
\text{ and } \displaystyle \vec{X} = \sum_{k=1}^{6}\vec{P}_k\myvec{
    2 & 1 & 3\\
    1 & 0 & 2\\
    3 & 2 & 1
}{\vec{P}_k}^{\top}$$
Where ${\vec{P}_k}^{\top}$ denotes the transpose of matrix $\vec{P}_k$. Then which of the following options is/are correct? \hfill (2019)

\begin{enumerate}
	\item $\vec{X}$ is a symmetric matrix
	\item The sum of diagonal elements of $\vec{X}$ is 18
	\item $\vec{X}-30\vec{I}$ is an invertible matrix
	\item If $\vec{X}\myvec{1\\1\\1}=\alpha\myvec{1\\1\\1},$ then $\alpha$ is 30
\end{enumerate}
%
\item Let $x\in R$ and let
$$\vec{P} = \myvec{1&1&1\\0&2&2\\0&0&3},\vec{Q} = \myvec{2&x&x\\0&4&0\\x&x&5} \text{ and }\vec{R}=\vec{P}\vec{Q}\vec{P}^{-1}$$
Then which of the following options is/are correct? \hfill (2019)
%
\begin{enumerate}
\item $\det \vec{R} = \det\myvec{2&x&x\\0&4&0\\x&x&5}+8,$ for all $x\in R$
\item For $x=1$, there exists a unit vector $\alpha\hat{i}+\beta\hat{j}+\gamma\hat{k}$ for which $\vec{R}\myvec{\alpha\\\beta\\\gamma}=\myvec{0\\0\\0}$
		\item There exists a real number $x$ such that $\vec{P}\vec{Q} = \vec{Q}\vec{P}$
		\item For $x=0$, if $\vec{R}=\myvec{1\\a\\b}=6\myvec{1\\a\\b},$ then a+b=5\\[2pt]
\end{enumerate}
	\item Consider the set ${A}$ of all determinants of order 3 with entries
		0 or 1 only. Let ${B}$ be the subset of ${A}$ consisting of all
		determinants with value 1. Let $C$ be the subset of ${A}$ consisting
		of all determinants with value -1. Then
		\hfill (1981)
		\begin{multicols}{2}
			\begin{enumerate}
				\item $C$ is empty
				\item $B$ has as many elements as $C$
				\item $\vec{A} = B \cup C$
				\item $B$ has twice as many elements as $C$
			\end{enumerate}
		\end{multicols}
	\item Let $a, b, c$ be the real numbers. Then following system of
		equations in $x, y$ and $z$ 
		$$
		\frac{x^2}{a^2} + \frac{y^2}{b^2} - \frac{z^2}{c^2} = 1	,
		\frac{x^2}{a^2} - \frac{y^2}{b^2} + \frac{z^2}{c^2} = 1	,
		-\frac{x^2}{a^2} + \frac{y^2}{b^2} + \frac{z^2}{c^2} = 1
		$$ has	
		\hfill (1995)

		\begin{multicols}{2}
			\begin{enumerate}
				\item no solution
				\item unique solution
				\columnbreak
				\item infinitely many solutions
				\item finitely many solutions
			\end{enumerate}
		\end{multicols}
%
	\item If $\vec{A}$ and $\vec{B}$ are square matrices of equal degree, then which
		one is correct among the followings?
		\hfill (1995)
%
		\begin{multicols}{2}
			\begin{enumerate}
				\item $\vec{A} + \vec{B} = \vec{B} + \vec{A}$
				\item $\vec{A} + \vec{B} = \vec{A} - \vec{B}$
				\columnbreak
				\item $\vec{A} - \vec{B} = \vec{B} - \vec{A}$
				\item $\vec{A}\vec{B} = \vec{B}\vec{A}$
			\end{enumerate}
		\end{multicols}
\item If the system of equations
    \begin{align*}
	    x-ky-z&=0 ,\\ kx-y-z&=0, \\ x+y-z&=0
    \end{align*} has a non-zero  solution,  then the possible values of $k$ are 
    \hfill{\brak{2000}}
    \begin{multicols}{4}
        \begin{enumerate}
            \item -1,2 \columnbreak
            \item 1,2 \columnbreak
            \item 0,1 \columnbreak
            \item -1,1
        \end{enumerate}
    \end{multicols}
\item The number of values of $k$ for which the system of equations 
    \begin{align*}
    \brak{k+1}x + 8y&=4k; \\ kx +\brak{k+3}y&=3k-1 \end{align*} has infinitely many solutions is 
    \hfill{\brak{2002}}
    \begin{enumerate}
            \begin{multicols}{4}
            \item 0 \columnbreak
            \item 1 \columnbreak
            \item 2 \columnbreak
            \item infinte
            \end{multicols}
    \end{enumerate}
%
\item If $\vec{A}=$
    \myvec{
        \alpha & 0 \\
        1 & 1
    } and $\vec{B}=$ \myvec{
        1 & 0 \\
        5 & 1
    }, then value of $\alpha$ for which $\vec{A}^2 = \vec{B}$, is
    \hfill{\brak{2003}}
    \begin{enumerate}
            \begin{multicols}{2}
            \item 1 \columnbreak
            \item 4 
            \end{multicols}
            \begin{multicols}{2}

            \item 2 \columnbreak
            \item infinite
            \end{multicols}
    \end{enumerate}
%
\item If the system of equations $x + ay = 0, az + y =0$ and $ax + z =0$ has infinite solutions, then the value of $a$ is 
    \hfill{\brak{2003}}
    \begin{enumerate}
            \begin{multicols}{2}
            \item -1 \columnbreak
            \item 1
            \end{multicols}
            \begin{multicols}{2}

            \item 0 \columnbreak
            \item no real values
            \end{multicols}
    \end{enumerate}
%
\item  Given \begin{align*} 2x-y+2z=2,\\x-2y+z=-4,\\x+y+\lambda z=4 \end{align*} then the value of $\lambda$ such that the given system of equation has NO solution, is
%
        \hfill{\brak{2004}}
        \begin{enumerate}
                \begin{multicols}{4}
                \item 3 \columnbreak
                \item 1 \columnbreak
                \item 0 \columnbreak
                \item -3
                \end{multicols}
        \end{enumerate}
    \item If $\vec{A}=$ \myvec{
            \alpha & 2\\
            2 & \alpha
        } and $\mydet{\vec{A}^3}=125$ then the value $\alpha$ is
        \hfill{\brak{2004}}
        \begin{enumerate}
                \begin{multicols}{4}
                \item $\pm 1$ \columnbreak
                \item $\pm 2$ \columnbreak
                \item $\pm 3$ \columnbreak
                \item $\pm 5$
                \end{multicols}
        \end{enumerate}
%
    \item $\vec{A}=$ \myvec{1 & 0 & 0 \\ 0 & 1&1 \\ 0 &-2 &4} and $\vec{I}=$ \myvec{1 & 0 &0\\ 0 & 1 & 0\\0 & 0 & 1} 
        and $\vec{A}^{-1} = $ \myvec{\frac{1}{6}\brak{\vec{A}^2 + c\vec{A} +d\vec{I}}}, then the value of $c$ and $d$ are \hfill{\brak{2005}}
        \begin{enumerate}
                \begin{multicols}{2}
                \item\brak{-6,-11} \columnbreak
                \item\brak{6,11}
                \end{multicols}
                \begin{multicols}{2}
                \item\brak{-6,11} \columnbreak
                \item\brak{6,-11}
                \end{multicols}
        \end{enumerate}
%
    \item If $\vec{P}=$ 
        \myvec{
            \frac{\sqrt{3}}{2} & \frac{1}{2}\\
            -\frac{1}{2} & \frac{\sqrt{3}}{2}
	} and $\vec{A} = \myvec{ 1& 1 \\ 0 & 1}$ and $\vec{Q} = \vec{P}\vec{A}\vec{P}^{\top}$ and $x=\vec{P}^{\top}\vec{Q}^{2005}\vec{P}$ then $x$ is equal to 
%
            \begin{enumerate}
                \item \myvec{ 1 & 2005\\0 & 1 }
                \item $\myvec{ 4 + 2005\sqrt{3} & 6015 \\ 2005 & 4 - 2005\sqrt{3}}$
                \item $\frac{1}{4}\myvec{2 + \sqrt{3} & 1 \\ -1 & 2 -\sqrt{3}}$
                \item $\frac{1}{4}\myvec{2005 & 2 - \sqrt{3} \\ 2 + \sqrt{3} & 2005}$
            \end{enumerate}		
            \item The number of $3 \times 3$ matrices $\vec{A}$ whose entries are either $0$ or $1$ and for which the system $\vec{A}\myvec{x\\y\\z}=\myvec{1\\0\\0}$ has exactly two distinct solutions is \hfill{\brak{2008}}
%
                \begin{enumerate}
                        \begin{multicols}{4}
                        \item 0 \columnbreak
                        \item $2^9 - 1$ \columnbreak
                        \item 168 \columnbreak
                        \item 2
                        \end{multicols}
                \end{enumerate}
            \item Let $\omega \neq 1$ be a cube root of unity and $S$ be the set of all non-singular matrices of the form 
                \begin{align*}
                    \myvec{
                        1 & a & b \\
                        \omega & 1 & c\\
                        \omega^2 & \omega & 1
                    }
                \end{align*} where each of $a,b$ and $c$ is either $\omega$ or $\omega^2$. Then the number of distinct matrices in the set $S$ is
                \hfill{\brak{2008}}

                \begin{enumerate}

                        \begin{multicols}{4}
                        \item 2\columnbreak
                        \item 6\columnbreak
                        \item 4\columnbreak
                        \item 8
                        \end{multicols}
                \end{enumerate}

            \item Let $\vec{P} = \myvec{a_{ij}}$ be a $3 \times 3$ matrix and let $\vec{Q} =  \myvec{b_{ij}}$, where $b_{ij} = 2^{i+j}a_{ij}$ for $1 \le i,j \le 3$. If the determinant of $\vec{P}$ is 2, then the determinant of the matrix $\vec{Q}$ is 
                \hfill{\brak{2012}}
                \begin{enumerate}

                        \begin{multicols}{4}
                        \item $2^{10}$ \columnbreak
                        \item $2^{11}$ \columnbreak
                        \item $2^{12}$\columnbreak
                        \item $2^{13}$
                        \end{multicols}
                \end{enumerate}

            \item If $\vec{P}$ is a $3 \times 3$ matrix such that $\vec{P}^{\top} = 2\vec{P} +\vec{I}$, where $\vec{P}^{\top}$ is the transpose of $\vec{P}$ and $\vec{I}$ is the $3 \times 3$ identity matrix, then there exists a column matrix $\vec{X}=\myvec{x\\y\\z} \neq \myvec{0\\0\\0}$
                \hfill{\brak{2012}}
                \begin{enumerate}

                        \begin{multicols}{2}
                        \item $\vec{P}\vec{X}=\myvec{0\\0\\0}$ \columnbreak
                        \item $\vec{P}\vec{X}=\vec{X}$
                        \end{multicols}
                        \begin{multicols}{2}
                        \item $\vec{P}\vec{X}=2\vec{X}$ \columnbreak
                        \item $\vec{P}\vec{X} = -\vec{X}$
                        \end{multicols}
                \end{enumerate}

            \item Let $\vec{P}=\myvec{1&0&0\\4&1&0\\16&4&1}$ and $\vec{I}$ be the identity matrix of order 3. If $\vec{Q} = \myvec{q_{ij}}$ is a matrix such that $\vec{P}^{50} -\vec{Q} =\vec{I}$, then $\frac{q_{31}+q_{32}}{q_{21}}$ equals
                \hfill{\brak{JEE Adv. 2016}}
                \begin{enumerate}

                        \begin{multicols}{4}
                        \item52 \columnbreak
                        \item103 \columnbreak
                        \item201 \columnbreak
                        \item205 
                        \end{multicols}
                \end{enumerate}

    \item
        How many $3 \times 3$ matrices $\vec{M}$ with entries from $\brak{0,1,2}$ are there, for which the sum of the diagonal entries of $\vec{M}^{\top}\vec{M}$ is $5$?
            \hfill (2017)
\begin{multicols}{4}
            \begin{enumerate}
                \item $126$
                \item $198$
                \item $162$
                \item $135$
            \end{enumerate}
\end{multicols}
        \item
        Let $\vec{M}= \myvec{
            \sin^4 \brak{\theta} & -1 -\sin^2 \brak{\theta}\\
            1+\cos^2 \brak{\theta} & \cos^4 \brak{\theta}
            } = 
            \alpha \vec{I} + \beta \vec{M}^{-1}$\\
        Where $\alpha = \alpha \brak{\theta}$ and $\beta = \beta \brak{\theta}$ are real numbers, and $\vec{I}$ is the $2 \times 2$ identity matrix. If $\mathrm{a^*}$ is the minimum of the set $\brak{\alpha \brak{\theta} : \theta \in [0, 2 \pi)}$ and $\mathrm{b^*}$ is the minimum of the set $\brak{\beta \brak{\theta} : \theta \in [0, 2 \pi)}$. Then the value of $\mathrm{a^*} + \mathrm{b^*}$ is
            \hfill (2019)
\begin{multicols}{4}
            \begin{enumerate}
                \itemsep0.4em
                \item $-\frac{31}{16}$
                \item $-\frac{17}{16}$
                \item $-\frac{37}{16}$
                \item $-\frac{29}{16}$
            \end{enumerate}
\end{multicols}
	\item Let $\vec{A} = \myvec{
		1 & 0 & 0 \\
		2 & 1 & 0 \\
	3 & 2 & 1 }$ and $\vec{U_1}$, $\vec{U_2}$ and $\vec{U_3}$ are columns of a $3\times3$ matrix $\vec{U}$. If column matrices $\vec{U_1}$, $\vec{U_2}$ and $\vec{U_3}$ satisfy
	$\vec{AU_1} = \myvec{
		1 \\
		0 \\
		0 }$,
	$\vec{AU_2} = \myvec{
		2 \\
		3 \\
		0 }$,
	$\vec{AU_3} = \myvec{
		2 \\
		3 \\
		1 }$ evaluate as directed in the following questions.
			\hfill(2006)
%
	\begin{enumerate}
		\item The value $\abs{\vec{U}}$ is
\begin{multicols}{4}
			\begin{enumerate}
				\item $3$
				\item $-3$
			\item $\frac{3}{2}$
				\item $2$
			\end{enumerate}
\end{multicols}
		\item The sum of the elements of the matrix $\vec{U^{-1}}$ is
\begin{multicols}{4}
			\begin{enumerate}
				\item $-1$
				\item $0$
				\item $1$
				\item $3$
			\end{enumerate}
\end{multicols}
%
		\item The value of $\myvec{ 3 & 2 & 0 }
				\vec{U}
				\myvec{ 3 \\ 2 \\ 0 }$ is
\begin{multicols}{4}
			\begin{enumerate}
				\item $5$
				\item $\frac{5}{2}$
				\item $4$
				\item $\frac{3}{2}$
			\end{enumerate}
\end{multicols}
	\end{enumerate}
	%\bigskip

	%{\centering PASSAGE - 2 \par}

	%\bigskip

	\item Let $\mathcal{A}$ be the set of all $3\times3$ symmetric matrices all of whose entries are either $0$ or $1$. Five of these entries are $1$ and four of them are $0$.
			\hfill(2009)
%
	\begin{enumerate}
		\item The number of matrices in $\mathcal{A}$ is
\begin{multicols}{2}
			\begin{enumerate}
				\item less than $4$
				\item at least $4$ but less than $7$
				\item at least $7$ but less than $10$
				\item at least $10$
			\end{enumerate}
\end{multicols}
%		
		\item The number of matrices $\vec{A}$ in $\mathcal{A}$ for which the system of linear equations
			$$\vec{A}\myvec{x \\ y \\z}=\myvec{1 \\ 0 \\ 0}$$
			has a unique solution, is
%
			\begin{enumerate}
				\item less than 4
				\item at least 4 but less than 7
				\item at least 7 but less than 10
				\item at least 10
			\end{enumerate}
%
		\item The number of matrices $\vec{A}$ in $\mathcal{A}$ for which the system of linear equations
			$$\vec{A}\myvec{x \\ y \\ z}=\myvec{1 \\ 0 \\ 0}$$
			is inconsistent, is
\begin{multicols}{2}
			\begin{enumerate}
				\item $0$
				\item more than $2$
				\item $2$
				\item $1$
			\end{enumerate}
\end{multicols}
	\end{enumerate}
	%\bigskip

	%{\centering PASSAGE - 3 \par}

	%\bigskip

	\item Let $p$ be an odd prime number and $\vec{T_p}$ be the following set of $2\times2$ matrices 
	$$\vec{T}_p=\cbrak{\vec{A}=\myvec{a & b \\ c & d}:a,b,c\in\cbrak{0,1,2,\dots,p-1}}$$

			\hfill(2010)
	\begin{enumerate}
		\item The number of $\vec{A}$ in $\vec{T}_p$ such that $\vec{A}$ is either symmetric or skew-symmetric or both, and det($\vec{A}$) divisibly by $p$ is
\begin{multicols}{4}
			\begin{enumerate}
				\item $\brak{p-1}^2$
				\item $2\brak{p-1}$
				\item $\brak{p-1}^2+1$
				\item $2p-1$
			\end{enumerate}
\end{multicols}


		\item The number of $\vec{A}$ in $\vec{T}_p$ such that the trace of $\vec{A}$ is not divisible by $p$ but det($\vec{A}$) is divisible by $p$ is

			\sbrak{\text{\textbf{Note: }The trace of a matrix is the sum of its diagonal entries.}}
\begin{multicols}{4}
			\begin{enumerate}
				\item $\brak{p-1}\brak{p^2-p+1}$
				\item $p^3-\brak{p-1}^2$
				\item $\brak{p-1}^2$
				\item $\brak{p-1}\brak{p^2-2}$
			\end{enumerate}
\end{multicols}
%
		\item The number of $\vec{A}$ in $\vec{T}_p$ such that det($\vec{A}$) is not divisible by $p$ is 
\begin{multicols}{4}
			\begin{enumerate}
				\item $2p^2$
				\item $p^3-5p$
				\item $p^3-3p$
				\item $p^3-p^2$
			\end{enumerate}
\end{multicols}

	\end{enumerate}
\item For what value of $k$ do the following system of equations possess a non trivial (i.e., not all zero) solution over the set of rationals ${Q}$?\begin{align*}
x+ky+3z=0\\3x+ky-2z=0\\2x+3y-4z=0\end{align*} For that value of $k$, find all the solutions of the system. \hfill (1979)
\item Consider the system of linear equations in $x, y, z$\begin{align*}(\sin 3\theta) x-y+z=0\\(\cos 2\theta)x+4y+3z=0\\2x+7y+7z=0\end{align*} Find the values of $\theta$ for which this system has non trivial solutions. \hfill (1986)
\item Let $\lambda\text{ and }\alpha$ be real. Find the set of all values of $\lambda$ for which the system of linear equations \begin{align*}\lambda x+(\sin\alpha)y+(\cos\alpha)=0\\ x+(\cos\alpha)y+(\sin\alpha)z=0\\-x+(\sin\alpha)y-(\cos\alpha)z=0\end{align*} has a non trivial solution. For $\lambda = 1$, find all values of $\alpha$. \hfill (1993 )
	\item If matrix 
		$\vec{A} = \myvec{
			a & b & c \\
			b & c & a \\
			c & a & b } $
		where $a,b,c$ are real positive numbers, $abc=1$ and $\vec{A}^\top\vec{A}=\vec{I}$, then find the value of $a^3+b^3+c^3$.
		\hfill(2003)
	%4
	\item If $\vec{M}$ is a $3\times3$ matrix, where det $\vec{M}=1$ and $\vec{M}\vec{M}^\top=\vec{I}$, where $\vec{I}$ is an identity matrix, prove that det$(\vec{M}-\vec{I})=0$.
		\hfill(2004)

	%5
	\item If $\vec{A} = \myvec{
			a & 1 & 0 \\
			1 & b & d \\
			1 & b & c }$ ,
		$\vec{B} = \myvec{
			a & 1 & 1 \\
			0 & d & c \\
			f & g & h }$ ,
		$\vec{U} = \myvec{
			f \\
			g \\
			h }$ ,
		$\vec{V} = \myvec{
			a^2 \\
			0 \\
			0 }$ ,
		$\vec{X} = \myvec{
			x \\
			y \\
			z }$
		and $\vec{AX}=\vec{U}$ has infinitely many solutions, prove that $\vec{BX}=\vec{V}$ has no unique solution. Also show that if $afd$ $\neq0$, then $\vec{BX}=\vec{V}$ has no solution.
		\hfill(2004)

\item   Let $M$ be a $3 \times 3$ invertible matrix with real entries and let $I$ denote the $3 \times 3$ identity matrix. If $ M^{-1} = \text{adj}(\text{adj } M)$ then which of the following statements is/are ALWAYS TRUE?
		\hfill(2020)
\begin{multicols}{4}
    \begin{enumerate}
        \item  $M = I$
        \item  $\det M = 1$
        \item  $M^2 = I$
        \item  $(\text{adj } M)^2 = I$
    \end{enumerate}
\end{multicols}
%
\item The trace of a square matrix is defined to be the sum of its diagonal entries. If $\vec{A}$ is a 2 $\times$ 2 matrix, such that the trace of $\vec{A}$ is 3 and the trace of $\vec{A}^3$ is -18, then the value of the determinant of $\vec{A}$ is \rule{1cm}{0.1pt}.
\hfill (2020)
%
    \item For any $3 \times 3$ matrix $\vec{M}$, let $|\vec{M}|$ denote the determinant of $\vec{M}$. Let  
    \[
    \vec{E} = \begin{pmatrix} 1 & 2 & 3 \\ 2 & 3 & 4 \\ 8 & 13 & 18 \end{pmatrix}, \quad \vec{P} = \begin{pmatrix} 1 & 0 & 0 \\ 0 & 0 & 1 \\ 0 & 1 & 0 \end{pmatrix}, \quad \vec{F} = \begin{pmatrix} 1 & 3 & 2 \\ 8 & 18 & 13 \\ 2 & 4 & 3 \end{pmatrix}
    \]  
    If $\vec{Q}$ is a nonsingular matrix of order $3 \times 3$, then which of the following statements is (are) TRUE?  
    \hfill (2021)
\begin{enumerate}
         \item  $\vec{F} = \vec{P}\vec{E}\vec{P}$ and $\vec{P}^2 = \begin{pmatrix} 1 & 0 & 0 \\ 0 & 1 & 0 \\ 0 & 0 & 1 \end{pmatrix}$  
         \item  $|\vec{E}\vec{Q} + \vec{P}\vec{F}\vec{Q}^{-1}| = |\vec{E}\vec{Q}| + |\vec{P}\vec{F}\vec{Q}^{-1}|$  
         \item  $|(\vec{E}\vec{F})^3| > |\vec{E}\vec{F}|^2$  
         \item  Sum of the diagonal entries of $\vec{P}^{-1}\vec{E}\vec{P} +\vec{F}$ is equal to the sum of diagonal entries of $\vec{E} + \vec{P}^{-1}\vec{F}\vec{P}$
    \end{enumerate}

    \item For any $3 \times 3$ matrix $\vec{M}$, let $|\vec{M}|$ denote the determinant of $\vec{M}$. Let $\vec{I}$ be the $3 \times 3$ identity matrix. Let $\vec{E}$ and $\vec{F}$ be two $3 \times 3$ matrices such that $(\vec{I} - \vec{E}\vec{F})$ is invertible. If $\vec{G} = (\vec{I} - \vec{E}\vec{F})^{-1}$, then which of the following statements is (are) TRUE?  
    \hfill (2021)
    \begin{multicols}{2}
\begin{enumerate}
         \item $|\vec{F}\vec{E}| = |\vec{I} - \vec{F}\vec{E}| \cdot |\vec{F}\vec{G}\vec{E}|$  
         \item $(\vec{I} - \vec{F}\vec{E})(\vec{I} + \vec{F}\vec{G}\vec{E}) = \vec{I}$  
         \item $\vec{E}\vec{F}\vec{G} = \vec{G}\vec{E}\vec{F}$  
         \item $(\vec{I} - \vec{F}\vec{E})(\vec{I} - \vec{F}\vec{G}\vec{E}) = \vec{I}$
    \end{enumerate}
\end{multicols}
\item If 
$$
M = \myvec{ 
\frac{5}{2} & \frac{3}{2} \\ [1ex]
-\frac{3}{2} & \frac{1}{2} 
},
$$
then which of the following matrices is equal to $ M^{2022} $?
\hfill (2022)
\begin{multicols}{2} \begin{enumerate}
 \item $\myvec{ 3034 & 3033 \\ -3033 & -3032 }$
 \item $\myvec{3034 & -3033 \\ 3033 & -3032 }$
 \item $\myvec{ 3033 & 3032 \\ -3032 & -3031 }$
 \item $\myvec{ 3032 & 3031 \\ -3031 & -3030}$
\end{enumerate} \end{multicols}
\item Let $\beta$ be a real number. Consider the matrix  
	\begin{align*}
		\vec{A} = \myvec{ 
\beta & 0 & 1 \\ 
2 & 1 & -2 \\ 
3 & 1 & -2 
		}
	\end{align*}
If $\vec{A}^7 - (\beta - 1) \vec{A}^6 - \beta \vec{A}^5$ is a singular matrix, then the value of $9\beta$ is \rule{1cm}{0.1pt}.
\hfill (2022)
\item Let $\vec{M} = \brak{ a_{ij} }, \, a_{ij} \in \{1,2,3\}$ be the $3 \times 3$ matrix such that $a_{ij} = 1$ if $j + 1$ is divisible by $i$, otherwise $a_{ij} = 0$. Then which of the following statements is(are) true?  
\hfill (2023)
\begin{enumerate}  
\item $\vec{M}$ is invertible  
\item There exists a nonzero column matrix $\myvec{a_1 \\ a_2 \\ a_3}$ such that $\vec{M} \myvec{a_1 \\ a_2 \\ a_3} = \myvec{-a_1 \\ -a_2 \\ -a_3}$  
\item The set $\{ \vec{X} \in \mathbb{R}^3 : \vec{M}\vec{X} = \vec{0} \} \neq \{\vec{0}\}$, where $\vec{0}= \myvec{0 \\ 0 \\ 0}$  
\item The matrix $(\vec{M} - 2\vec{I})$ is invertible, where $\vec{I}$ is the $3 \times 3$ identity matrix  
\end{enumerate}
\item Let $$\vec{R} =  \myvec{ a & b \\ c & d } : a, b, c, d \in \{0,3,5,7,11,13,17,19\}. $$
Then the number of invertible matrices in $\vec{R}$ is
\rule{1cm}{0.1pt}.
\hfill (2023)
%
\item Let $$S = \{ \vec{A} =  
	\myvec{  
    0 & 1 & c \\  
    1 & a & d \\  
    1 & b & e}
    : a, b, c, d, e \in \{0,1\} \text{ and } |\vec{A}| \in \{-1,1\}  
    \},$$  
    where $|\vec{A}|$ denotes the determinant of $\vec{A}$.  
    Then the number of elements in $S$ is 
    \rule{1cm}{0.1pt}.
\hfill (2024)
\item Let $\alpha$ and $\beta$ be the distinct roots of the equation $x^2 + x - 1 = 0$. Consider the set $T = \{1, \alpha, \beta\}$.
For a $3 \times 3$ matrix $\vec{M} = \brak{a_{ij}}_{3 \times 3}$, define $R_i = a_{i1} + a_{i2} + a_{i3}$ and $C_j = a_{1j} + a_{2j} + a_{3j}$ for $i = 1, 2, 3$ and $j = 1, 2, 3$.
Match the following
\begin{multicols}{2}
\begin{enumerate}[label=\Alph*]
	\item  The number of matrices $\vec{M} = \brak{a_{ij}}_{3 \times 3}$ with all entries in $T$ such that $R_i = C_j = 0$ for all $i, j$, is 
	\item The number of symmetric matrices $\vec{M} = \brak{a_{ij}}_{3 \times 3}$ with all entries in $T$ such that $C_j = 0$ for all $j$, is 
	\item Let $\vec{M} = \brak{a_{ij}}_{3 \times 3}$ be a skew-symmetric matrix such that $a_{ij} \in T$ for $i > j$. Then the number of elements in the set 
$$\cbrak{ \begin{pmatrix} x \\ y \\ z \end{pmatrix} : x, y, z \in \mathbb{R}, \vec{M} = \begin{pmatrix} 0 & a_{12} & 0 \\ -a_{12} & 0 & a_{23} \\ 0 & -a_{23} & 0 \end{pmatrix} }$$
is 
	\item Let $\vec{M} = \brak{a_{ij}}_{3 \times 3}$ be a matrix with all entries in $T$ such that $R_i = 0$ for all $i$. Then the absolute value of the determinant of $\vec{M}$ is 
\end{enumerate}
\columnbreak
	\begin{enumerate}[label=\arabic*]
	\item 1  		
	\item 12 		
	\item infinite 	
	\item 6 		
	\item 0 		
\end{enumerate}\end{multicols}
The correct option is
    \hfill (2024)
\begin{enumerate}
\item $A \rightarrow 4$, $B \rightarrow 2$, $C \rightarrow 5$, $D \rightarrow 1$
\item $A \rightarrow 2$, $B \rightarrow 4$, $C \rightarrow 1$, $D \rightarrow 5$
\item $A \rightarrow 2$, $B \rightarrow 4$, $C \rightarrow 3$, $D \rightarrow 5$
\item $A \rightarrow 1$, $B \rightarrow 5$, $C \rightarrow 3$, $D \rightarrow 4$
\end{enumerate}

\end{enumerate}
