	\begin{enumerate}
\item The line $2x+y=1$ is the tangent to the hyperbola $\frac{x^2}{a^2}-\frac{y^2}{b^2}=1$. If this line passes through the point of intersection of the nearest directrix and the X-axis, then the eccentricity of the hyperbola is
\rule{1cm}{0.1pt}.
\hfill(2010)
\iffalse
\item A Vertical line passing through point $(h,0)$ intersects the ellipse   at the points  $\Vec{P}$ and $\Vec{Q}$. Let the tangents to the ellipse at $\Vec{P}$ and $\Vec{Q}$ meet at the points $\Vec{R}$. If $\Delta(h)$= area of the triangle $PQR$, $\Delta_1$= ma
then 
\hfill(2013)
\begin{multicols}{2}
\begin{enumerate}
    \item $g(x)$ is continuous but not differentiable at a
    \item $g(x)$ is differentiable on R
    \item $g(x)$ is continuous but not differentiable at b
    \item $g(x)$ is continuous and differentiable either(a) or (b) but not both 
    \end{enumerate}
\end{multicols}
    \fi
\item If the normal of the parabola $y^2=4x$ drawn at the end points of its latusrectum are the tangents of th circle $(x-3)^2+(y+2)^2=r^2$, then the value of $r^2$ is \rule{1cm}{0.1pt}.
\hfill(2015)
\iffalse
\item Suppose that the focii of the ellipse $\frac{x^2}{9}+\frac{y^2}{5}=1$ are $(f_1,0)$ and ($f_2$,0) where $f_1>0$ and $f_1<0$.Let $P_1$ and $P_2$ be two parabolas with a common vertex at $(0,0)$ and with foci at ($f_1$,0) and (2$f_2$,0),respectively. Let $T_1$ be a tangent to $P_1$ which passes through (2$f_2$,0) and $T_2$ be a tangent to $P_2$ which passes through ($f_1$,0).If $m_1$ is the slope of $T_1$ and $m_2$ is the slope of $T_2$,then the value of
\hfill(2015)
\fi
\item If two tangents drawn from a point $\vec{P}$ to the parabola $y^2=4x$ are at right angles, then the locus of $\vec{P}$ is\hfill(2010)
\begin{multicols}{4}
\begin{enumerate}
    \item $2x+1=0$
    \item $x=-1$
    \item $2x-1=0$
    \item $x=1$
\end{enumerate}
\end{multicols}
\item Statement -1 : An equation of a common tangent to the  parabola $y^2=16\sqrt{3}x$ and the ellipse $2x^2+y^2=4$ is $y=2x+2\sqrt{3}$\\
Statement -2 : If the line $y=mx+\frac{4\sqrt{3}}{m}$ $\brak{m \neq 0}$is a common tangent to the parabola $y^2=16\sqrt{3}x$ and the ellipse $2x^2+y^2=4$, then $m$ satisfies $m^4+2m^2=24$\hfill(2012)
\begin{multicols}{2}
\begin{enumerate}
    \item Statement-1 is false, statement-2 is true.
    \item Statement-1 is true,  statement-2 is true; statement-2 is correct explanation for statement-1.
    \item Statement-1 is true, statement-2 is true; statement-2 is not a correct explanation for statement-1.
    \item Statement-1 is true, statement-2 is false.
\end{enumerate}
\end{multicols}
\item {Given: A circle, $2x^2+2y^2=5$ and a parabola, $y^2=4\sqrt{5}x$.\\
Statement-1: An equation of a common tangent to these curves is $y=x+\sqrt{5}$.\\
Statement -2: If the line, $y=mx+\frac{\sqrt{5}}{m}$ $\brak{m \neq 0}$ is their common tangent, then $m$ satisfies $m^4-3m^2+2=0$.}
\hfill(2013)
\begin{multicols}{2}
\begin{enumerate}
    \item Statement-1 is true; Statement-2 is true; Statement-2 is a correct explanation for Statement-1.
    \item Statement-1 is true; Statement-2 is true; Statement-2 is not a correct explanation for Statement-1.
    \item Statement-1 is true; Statement-2 is false.
    \item Statement-1 is false; Statement-2 is true.
\end{enumerate}
\end{multicols}
\item The locus of the foot of perpendicular drawn from the centre of the ellipse $x^2+3y^2=6$ on an tangent to it is
\hfill(2014)
\begin{multicols}{2}
\begin{enumerate}
    \item $(x^2+y^2)^2=6x^2+2y^2$
    \item $(x^2+y^2)^2=6x^2-2y^2$
    \item $(x^2-y^2)^2=6x^2+2y^2$
    \item $(x^2-y^2)^2=6x^2-2y^2$
\end{enumerate}
\end{multicols}
\item The slope of the line touching both the parabolas $y^2=4x$ and $x^2=-32y$ is
\hfill(2014)
\begin{multicols}{4}
\begin{enumerate}
    \item $\frac{1}{8}$
    \item $\frac{2}{3}$
    \item $\frac{1}{2}$
    \item $\frac{3}{2}$
\end{enumerate}
\end{multicols}
\item The area (in sq. units) of the quadrilateral formed by the tangents at the end points of the latera recta to the ellipse $\frac{x^2}{9}+\frac{y^2}{5}=1$ is \hfill(2015)
\begin{multicols}{4}
\begin{enumerate}
    \item $\frac{27}{2}$
    \item $27$
    \item $\frac{27}{4}$
    \item $18$
\end{enumerate}
\end{multicols}
\item Match the statements in Column I with the properties in Column II.
	\hfill{(2007)}
\begin{multicols}{2}
\textbf{Column I}
\begin{enumerate}
    \item  Two intersecting circles
    \item Twomutually external circles
    \item Two circles, one strictly inside the other 
    \item  Two branches of a hyperbola  
\end{enumerate}
\textbf{Column II}
\begin{enumerate}
    \item have a common tangent 
    \item havea common normal 
    \item do not have a common tangent
    \item do not have a common normal
\end{enumerate}
\end{multicols}
\item line L : $y = mx + 3$ meets $Y$ axis at $\vec{E}(0,3)$ and the arc of the arc of the parabola $y^2 = 16x$, $ 0\leq y \leq6$ at the point $\vec{F}(x_o,y_o)$. The tangent to the parabola at $\vec{F}(x_o,y_o)$ intersects the $Y$ axis at $\vec{G}(0,y_1)$. The slope $m$ of the line $L$ is chosen such that the area of the triangle EFG has a local maximum.\hfill{(2013)} 

Match List 1 with List 2 and select the correct answer using the code given below the list 
\begin{multicols}{2}
\textbf{List 1}
\begin{enumerate}
    \item m=
    \item Maximum area of $\Delta EFG$
    \item $y_o=$
    \item $y_1=$
\end{enumerate}
\textbf{List 2}
\begin{enumerate}
    \item $\frac{1}{2}$
    \item 4
    \item 2
    \item 1
\end{enumerate}
\end{multicols}
\item  Columns 1, 2 and 3 contain conics and points of contact, respectively.
\begin{table}[h!]
\centering
\begin{tabular}{|c|c|c|c|}
\hline
Sno & 1       & 2 & 3 \\
\hline
	1 & $x^2 + y^2 = a^2$ & $my = m^2x + a$ & $\brak{\frac{a}{m^2}}$, $\frac{2a}{m}$ \\
	2 & $x^2 + a^2y^2 = a^2  $      & $y = mx +a\sqrt{m^2 + 1}$  & $\brak{\frac{-ma}{\sqrt{m^2 + 1}},\frac{a}{\sqrt{m^2 + 1}}}   $   \\
	3 & $y^2=4ax $  & $y = mx + \sqrt{a^2m^2 - 1}$ & $\brak{\frac{-a^2m}{\sqrt{a^2m^2 + 1}}, \frac{1}{\sqrt{a^2m^2 + 1}}}$    \\
	4 & $x^2 - a^2y^2 = a^2$     & $y = mx + \sqrt{a^2m^2 - 1 }$  & $\brak{\frac{-a^2m}{\sqrt{a^2m^2 - 1}}, \frac{-1}{\sqrt{a^2m^2 - 1}}}$      \\

\hline
\end{tabular}
\end{table}
%
\begin{enumerate}
\item For $a =\sqrt{2}$, if a tangent is drawn to a suitable conic (Column 1) at the point of contact $(-1, 1)$, then which of the following
options is the only correct combination for obtaining its equation?\hfill{(2009)}
%
\begin{multicols}{4}
\begin{enumerate}
	\item $(1,1,1)$
    \item $(1,2,2)$
    \item $(2,2,2)$
    \item $(3,1,1)$
\end{enumerate}
\end{multicols}

\item If a tangent to a suitable conic (Column 1) is found to be $y = x + 8$ and its point of contact is $(8, 16)$, then which of the following
options is the only correct combination?\hfill {(2018)}
%
\begin{multicols}{4}
\begin{enumerate}
    \item $(1,2,2)$
    \item $(2,4,3)$
    \item $(3,1,1)$
    \item $(3,2,2)$
\end{enumerate}
\end{multicols}
%
\item  The tangent to a suitable conic (Column 1) at $(\sqrt{3},\frac{1}{2})$ is found to be $\sqrt{3}x + 2y = 4$, then which of the following options is the only correct option ? 
%
\begin{multicols}{4}
\begin{enumerate}
    \item $(4,3,4)$
    \item $(4,4,4)$
    \item $(2,3,3)$
    \item $(2,4,3)$
\end{enumerate}
\end{multicols}
\end{enumerate}
\item On the ellipse $4x^2+9y^2=1$, the points at which the tangents are parallel to the line $8x=9y$ are \hfill(1999)
	\begin{multicols}{2}
\begin{enumerate}
			\begin{multicols}{4}
		\item $\brak{\frac{2}{5},\frac{1}{5}}$
		\item $\brak{-\frac{2}{5},\frac{1}{5}}$
			\columnbreak
		\item $\brak{-\frac{2}{5},-\frac{1}{5}}$
		\item $\brak{\frac{2}{5},-\frac{1}{5}}$
			\end{multicols}
	\end{enumerate}
\end{multicols}
%
\item The equations of the common tangents to the parabola $y=x^2$ and $y=-\brak{x-2}^2$ is/are \hfill(2006)\\
	\begin{multicols}{4}
\begin{enumerate}
		\item $y=4\brak{x-1}$
		\item $y=0$
		\item $y=-4\brak{x-1}$
		\item $y=-30x-50$
	\end{enumerate}
\end{multicols}
	\item The tangent ${PT}$ and the normal ${PN}$ to the parabola $y^2=4ax$ at a point $\vec{T}$ and $\vec{N}$, respectively. The locus of the centroid of the traingle ${PTN}$ is a parabola whose

		\hfill(2009)
		\begin{multicols}{2}
\begin{enumerate}
			\item vertex is $\brak{\frac{2a}{3},0}$
			\item directrix is $x=0$
				\columnbreak
			\item latus rectum is $\frac{2a}{3}$
			\item focus is $\brak{a,0}$
		\end{enumerate}
\end{multicols}
	\item Tangents are drawn to the hyperbola $\frac{x^2}{9}-\frac{y^2}{4}=1$, parallel to the straight line $2x-y=1$. The points of contact of the tangents to the hyperbola
		are  
		\hfill(2012)
		
		 \begin{multicols}{4}
\begin{enumerate}
			\item $\brak{\frac{9}{2\sqrt{2}},\frac{1}{\sqrt{2}}}$ 
			\item $ \brak{\frac{-9}{2\sqrt{2}},\frac{-1}{\sqrt{2}}}$
			\item $\brak{3\sqrt{3},-2\sqrt{2}}$
			\item $ \brak{-3\sqrt{3},2\sqrt{2}}$
		
		 \end{enumerate}
\end{multicols}
	\item Consider the hyperbola H:$x^2-y^2=1$ and a circle S with 
		centre $\vec{N}(x_2,0)$. Suppose that H and S touch each other at a 
	      point $\vec{P}(x_1,y_1)$ with $x_1>0$ and $y_1>0$. The common tangent to H and S at $\vec{P}$ intersects the x-axis at point $\vec{M}$. If $(l,m)$ is the centroid of the triangle $PMN$, then correct expressions(s) is(are)
	      
	      \hfill(2015)
	      
	       \begin{multicols}{2}
\begin{enumerate}
		      \item $\frac{dl}{dx_1}=1-\frac{1}{3x^2}$ for $x_1>1$ 
		      \item $\frac{dm}{dx_1}=\frac{x_1}{3\sqrt{x_1^2-1}}$ for $x_1>1$ 
		      \item $\frac{dl}{dx_1}=1+\frac{1}{3x^2}$ for $x_1>1$
		      \item $\frac{dm}{dy_1}=\frac{1}{3}$ for $y_1>0$ 
	       \end{enumerate}
\end{multicols}
      \item The circle $C_1$:$x^2+y^2=3$, with centre at $\vec{O}$, intersects the parabola $x^2=2y$ at the point $\vec{P}$ in the first quadrant. Let the tangent to the circle $C_1$, at $\vec{P}$ touch the other two circles $C_2$ and $C_3$ at $\vec{R_2}$ and $\vec{R}_3$, respectively. Suppose $C_2$ and $C_3$ have equal radii $2\sqrt{3}$ and the centres $\vec{Q}_2$ and $\vec{Q}_3$, respectively. If $\vec{Q}_2$ and $\vec{Q}_3$ lie on the $Y$ axis, then 

	      \hfill(2016)
	      
%	       \begin{multicols}{2}
\begin{enumerate}
		      \item $Q_2Q_3=12$
		      \item $R_2R_3=4\sqrt{6}$
		      \item area of the triangle $OR_2R_3$ is $6\sqrt{2}$
		      \item area of the triangle $PQ_2Q_3$ is $4\sqrt{2}$
	       \end{enumerate}
%\end{multicols}
      \item Let $\vec{P}$ be the point on the parabola $y^2=4x$ which is at the shortest distance from the center S of the circle $x^2+y^2-4x-16y+64=0$. Let $\vec{Q}$ be the point on the circle
	      dividing the line segment $SP$ internally. Then 
	      \hfill(2016)
	     % 
	       \begin{multicols}{2}
\begin{enumerate}
		      \item $SP=2\sqrt{5}$
		      \item $SQ:QP=(\sqrt{5}+1):2$
		      \item the x-intercept of the normal to the parabola at $\vec{P}$ is $6$
		      \item the slope of the tangent to the circle at $\vec{Q}$ is $\frac{1}{2}$
		      
	       \end{enumerate}
\end{multicols}
      \item If $2x-y+1=0$ is a tangent to the hyperbola $\frac{x^2}{a^2}-\frac{y^2}{16}=1$ then which of the following cannot be sides of a right angled triangle? 
	      \hfill(2017)
	       \begin{multicols}{4}
\begin{enumerate}
		      \item $a,4,1$
		      \item $a,4,2$
		      \item $2a,8,1$
		      \item $2a,4,1$
	       \end{enumerate}
\end{multicols}
%
      \item Consider two straight lines, each of which is tangent to both the circle $x^2+y^2=\frac{1}{2}$
	      and the parabola $y^2=4x$. Let these lines intersect at the point $\vec{Q}$. Consider the elipse whose center is at orgin $\vec{O}(0,0)$ and whose semi-major axis is $OQ$.
	      If the length of the minor axis of the elipse is $\sqrt{2}$, then which of the following statement(s) is(are) TRUE? 
	      \hfill(2018)
%	      
\begin{enumerate}
		      \item For the elipse, the eccentricity is $\frac{1}{\sqrt{2}}$ and the length of the latus rectum is $1$
%
		      \item For the elipse, the eccentricity is $\frac{1}{2}$ and the length of the latus rectum is $\frac{1}{2}$
		      \item The area of the region bounded by the elipse between the lines $x=\frac{1}{\sqrt{2}}$ and $x=1$ is $\frac{1}{4\sqrt{2}}(\pi-2)$
		      \item The area of the region bounded by the elipse between the line $x=\frac{1}{\sqrt{2}}$ and $x=1$ is $\frac{1}{16}(\pi-2)$
	       \end{enumerate}
%
\item If $x=9$ is the chord of contact of the hyperbola $x^2-y^2=9$, then the equation of the corresponding pair of tangents is
    \hfill \brak{1999}
\begin{multicols}{2}
\begin{enumerate}
    \item $9x^2-8y^2+18x-9=0$
    \item $9x^2-8y^2-18x+9=0$
    \item $9x^2-8y^2-18x-9=0$
    \item $9x^2-8y^2+18x+9=0$
\end{enumerate}
\end{multicols}
\item The equation of the common tangent touching the circle $\brak{x-3}^2-kx+8=0$ and the parabola $y^2=4x$ above the $X$ axis is 
      \hfill\brak{2000}
\begin{multicols}{2}
\begin{enumerate}
    \item $\sqrt{3}y=3x+1$
    \item $\sqrt{3}y=-\brak{x+3}$
    \item $\sqrt{3}y=x+3$
    \item $\sqrt{3}y=-\brak{3x+1}$
\end{enumerate}
\end{multicols}
\item If a $>$ 2b $>$ 0 then the positive value of $m$ for which       $y=mx-b\sqrt{1+m^{2}} $ is a common tangent to $x^{2} + y^{2} = b^{2} $ and  $(x-a)^{2} + y^{2} = b^{2}$ is   \hfill {(2002)}
\begin{multicols}{4}
\begin{enumerate}
    \item $\frac{2b}{\sqrt{a^{2}-4b^{2}}}$
    \item $\frac{2b}{a-2b}$
    \item $\frac{\sqrt{a^{2}-4b^{2}}}{2b}$
    \item $\frac{b}{a-2b}$
\end{enumerate}
\end{multicols}
\item The equation of the common tangent to the curves $y^{2}=8x$ and $xy=-1$ is \hfill{(2002)}
\begin{multicols}{4}
\begin{enumerate}
    \item $3y=9x+2$
    \item $y=2x+1$
    \item $2y=x+8$
    \item $y=x+2$
\end{enumerate}
\end{multicols}
\item The area of the quadrilateral formed by the tangents at the end points of the latus rectum to the ellipse $\frac{x^{2}}{9}+\frac{y^{2}}{5}=1$, in sq. units is \hfill{(2003)}
 \begin{multicols}{4}
\begin{enumerate}
	\item $\frac{27}{4}$ 
    \item 9 
    \item $\frac{27}{2}$ 
    \item 27 
 \end{enumerate}
\end{multicols}
\item The focal chord to $y^{2}=16x$ is tangent to $(x-6)^{2}+y^{2}=2$, then the possible values of the slope of this chord, are \hfill{(2003)}
\begin{multicols}{4}
\begin{enumerate}
    \item ${-1,1}$
    \item ${-2,2}$
    \item ${-2,-1/2}$
    \item ${2,-1/2}$
\end{enumerate}
\end{multicols}
\item If tangents are drawn to ellipse $x^{2}+2y^{2}=2$,then the locus of the mid-point of the intercept made by the tangents between the coordinate axes is 
\hfill{(2004)}
\begin{multicols}{4}
\begin{enumerate}
    \item $\frac{1}{2x^{2}}+\frac{1}{4y^{2}}$ 
    \item $\frac{1}{4x^{2}}+\frac{1}{2x^{2}}$ 
    \item $\frac{x^{2}}{2}+\frac{y^{2}}{4}=1$
    \item $\frac{x^{2}}{4}+\frac{y^{2}}{2}=1$ 
\end{enumerate}
\end{multicols}
\item The angle between the tangents drawn from the point ${(1,4)}$ to the parabola $y^{2}=4x$ is 
\hfill{(2004)}
\begin{multicols}{4}
\begin{enumerate}
    \item $\pi/6$ 
    \item $\pi/4$ 
    \item $\pi/3$
    \item $\pi/2$
\end{enumerate}
\end{multicols}
\item If the line $2x+\sqrt{6}y=2$ touches the hyperbola $x^{2}-2y^{2}=4$, then the point of contact is \hfill{(2004)}
\begin{multicols}{2}
\begin{enumerate}
    \item $\brak{-2,\sqrt{6}}$
    \item $\brak{-5,2\sqrt{6}}$
    \item $\brak{\frac{1}{2},\frac{1}{\sqrt{6}}}$
    \item $\brak{4,-\sqrt{6}}$
\end{enumerate}
\end{multicols}
\item The minimum area of the triangle formed by the tangent to the $\frac{x^{2}}{a^{2}}+\frac{y^{2}}{b^{2}}=1$ and coordinate axes in sq. units is \hfill{(2005)}
\begin{multicols}{4}
\begin{enumerate}
    \item $ab$ 
    \item $\frac{a^{2}+b^{2}}{2}$ 
    \item $\frac{(a+b)^{2}}{2}$ 
    \item $\frac{a^{2}+ab+b^{2}}{3}$ 
\end{enumerate}
\end{multicols}
\item Tangent to the curve $y=x^{2}+6$ at a point ${(1,7)}$ touches the circle $x^{2}+y^{2}+16x+12y+c=0$ at a point Q.Then the coordinates of Q are \hfill{(2005)}
\begin{multicols}{2}
\begin{enumerate}
    \item${(-6,-11)}$\\\\
    \item${(-9,-13)}$
    \item${(-10,-15)}$\\\\
    \item${(-6,-7)}$
\end{enumerate}
\end{multicols}
	\item The common tangents to the circie $x^2+y^2=2$ and the parabola $y^2=8x$ touch the circle at the points $\vec{P}$, $\vec{Q}$ and the parabola at the points $\vec{R}$, $\vec{S}$.Then the area of the quadrilateral $\vec{PQRS}$ is \hfill(2014)
		\begin{multicols}{2}
\begin{enumerate}
				\begin{multicols}{2}
			\item $3$
			\item $6$
				\columnbreak
			\item $9$
			\item $15$
				\end{multicols}
		\end{enumerate}
\end{multicols}
\item A hyperbola passes through point $\vec{P}\brak{\sqrt2,\sqrt2}$  and  has  foci  at $\brak{\pm2,0}$. Then  the  tangent  to  this  hyperbola at $\vec{P}$ also passes through the point :
      \hfill{( 2017)} 
	\begin{multicols}{2}
\begin{enumerate}
    		\item  $\brak{-\sqrt2,-\sqrt3}$
    		\item  $\brak{3\sqrt2,2\sqrt3}$
    		\item  $\brak{2\sqrt2,3\sqrt3}$
    		\item  $\brak{\sqrt3,\sqrt2}$
	\end{enumerate}
\end{multicols}
\item  The radius of a circle, having minimum area, which touches the curve $y=4-x^2$ and the lines, $y=\abs{x}$ is : 
   \hfill{( 2018)}
	\begin{multicols}{2}
\begin{enumerate}
     		\item $4\brak{\sqrt2+1}$
     		\item $2\brak{\sqrt2+1}$
     		\item $2\brak{\sqrt2-1}$
     		\item $4\brak{\sqrt2-1}$
	\end{enumerate}
\end{multicols}
\item Tangents are drawn to the hyperbola $4x^2-y^2=36$ at the points $\vec{P}$ and $\vec{Q}$. If  these tangents intersect  at the point $\vec{T}\brak{0,3}$ then the area (in sq.units) of $\Delta$ PTQ is:
     \hfill{( 2018)}
	\begin{multicols}{2}
\begin{enumerate}
     		\item $54\sqrt3$
     		\item $60\sqrt3$
     		\item $36\sqrt3$ 
     		\item $45\sqrt5$
	\end{enumerate}
\end{multicols}
\item Tangent and normal are drawn at $\vec{P}\brak{16,16}$ on the parabola $y^2=16x$,
which is intersect the axis of the parabola at $\vec{A}$ and $\vec{B}$, respectively. If $\vec{C}$ is the centre of the circle through the points $\vec{P}$, $\vec{A}$ and $\vec{B}$ and $\angle$ CPB=$\theta$, then the value of $\tan{\theta}$ is :
     \hfill{( 2018)}
	\begin{multicols}{2}
\begin{enumerate}
    		\item $2$
    		\item $3$
    		\item $\frac{4}{3}$
    		\item $\frac{1}{2}$
	\end{enumerate}
\end{multicols}
\item If the tangent at $\brak{1,7}$ to the curve $x^2=y-6$ touches the circle $x^2+y^2+16x+12y+c=0$ then the value of c is :
       \hfill{(JEEM 2018)}
	\begin{multicols}{2}
\begin{enumerate}
    		\item $185$
    		\item $85$
    		\item $95$
    		\item $195$
	\end{enumerate}
\end{multicols} 
\item Equation of a common tangent to the circle $x^2+y^2-6x=0$ and the parabola $y^2=4x$, is:
     \hfill{(  2019-9 Jan(M))}
	\begin{multicols}{2}
\begin{enumerate}
    		\item $2\sqrt{3}y=12x+1$ 
    		\item $\sqrt{3}y=x+3$
    		\item $2\sqrt{3}y=-x-12$ 
    		\item $\sqrt{3}y=3x+1$
	\end{enumerate}
\end{multicols}   
\item If the line $y=mx+7\sqrt{3}$ is normal to the hyperbola $\frac{x^2}{24}$-$\frac{y^2}{18}$ then a value of m is: 
     \hfill{(JEEM 2019-9 April(M))}
	\begin{multicols}{2}
\begin{enumerate}
    		\item $\frac{\sqrt{5}}{2}$ 
    		\item $\frac{\sqrt{15}}{2}$
    		\item $\frac{2}{\sqrt5}$
    		\item $\frac{3}{\sqrt5}$
	\end{enumerate}
\end{multicols}
\item[] Let $PQ$ be a focal chord of the parabola $y^2=4ax$.The tangents to the parabola at $\Vec{P}$ and $\Vec{Q}$ meet at a point lying on the line $y=2x+a$,$a>0$
\item Length of the chord $PQ$ is

\hfill(2013)        
\begin{multicols}{2}
\begin{enumerate}
    \item $7a$
    \item $5a$
    \item $2a$
    \item $3a$
\end{enumerate}
\end{multicols}
\item If $st=1$, then the tangent at $\Vec{P}$ and the normal at $\Vec{M}$ to the
parabola meet at a point whose ordinate is 
\begin{multicols}{2}
\begin{enumerate}
    \item $\frac{a(t^2+1)^2}{t^3}$
    \item $\frac{a(t^2+1)}{2t^3}$
    \item $\frac{1}{t}$
    \item $\frac{t^2-1}{t}$
    \end{enumerate}
\end{multicols}
\item If the tangents to the ellipse at $\Vec{M}$ and $\Vec{N}$ meet at $\Vec{R}$ and the normal to the parabola at $\Vec{M}$ meets the X-Axis at $\Vec{Q}$, the the ratio of area of the triangle $MQR$ to the area of the quadrilateral M$F_1$N$F_2$ is
\hfill(2016)
\begin{multicols}{2}
\begin{enumerate}
    \item $3:4$
    \item $4:5$
    \item $5:8$
    \item $2:3$
\end{enumerate}
\end{multicols}
    \item Suppose that the normals drawn at three different points on the parabola $y^2=4x$ pass through the point$(h,k)$. Show that $h>2$. 
		\hfill(1981-4 Marks)
		
	\item $\vec{A}$ is a point on the parabola $y^2=4ax$. The normal at $\vec{A}$ cuts the parabola again at point $\vec{B}$. If $AB$ subtends a right angle at the vertex of the parabola. Find the slope of $AB$.  
		\hfill(1982-4 Marks)
		
	\item Three normals are drawn from the point $(c,0)$ to the curve $y^2=x$. Show that $c$ must be greater than $\frac{1}{2}$. One normal is always the x-axis. Find $c$ for which the other two normals are perpendicular to each other. 
	      \hfill(1991-4 Marks)
\item Let '$d$' be the perpendicular distance from the centre of the ellipse $\frac{x^2}{a^2}+\frac{y^2}{b^2}=1$ to the tangent drawn at a point $\vec{P}$ on the ellipse. If $\vec{F_1}$ and $\vec{F_2}$ are the two $foci$ of the ellipse, then show that $\brak{PF_1-PF_2}^2=4a^2\brak{1-\frac{b^2}{d^2}}$. \hfill\brak{1995- 5 marks}

\item Points $\vec{A}$, $\vec{B}$ and $\vec{C}$ lie on a parabola $y^2=4ax$. The tangents to the parabola at $\vec{A}$, $\vec{B}$ and $\vec{C}$ taken in pairs, intersect at points $\vec{P}$, $\vec{Q}$ and $\vec{R}$. Determine the ratios of the areas of triangles $ABC$ and $PQR$. \hfill\brak{1996- 3 marks}

\item From a point $\vec{A}$ common tangents are drawn to the circle $x^2+y^2=\frac{a^2}{2}$ and the parabola $y^2=4ax$. Find the area of the quadrilateral formed by the common tangents, the chord of contact of the circle, and the chord of contact of the parabola. \hfill\brak{1996- 2 marks}

\item A tangent to the ellipse $x^2+4y^2=4$ meets the ellipse $x^2+2y^2=6$ at $\vec{P}$ and $\vec{Q}$. Prove that the tangents at $\vec{P}$ and $\vec{Q}$ of the ellipse $x^2+2y^2=6$ are at right angles. \hfill\brak{1997- 5 marks}

\item The angle between a pair of tangents drawn from a point $\vec{P}$ to the parabola $y^2=4ax$ is 45\degree. Show that the locus of the point $\vec{P}$ is a hyperbola. \hfill\brak{1998- 8 marks}

\item Consider the family of circles $x^2+y^2=r^2$, $2<r<5$. If in the first quadrant, the common tangent to a circle of this family and the ellipse $4x^2+25y^2=100$ meets the coordinate axes at $\vec{A}$ and $\vec{B}$, then find the equation of the locus of the midpoint of $AB$. \hfill\brak{1999- 10 marks}

\item Find the coordinates of all the points $\vec{P}$ on the ellipse $\frac{x^2}{a^2}+\frac{y^2}{b^2}$=1, for which the area of the triangle $PON$ is maximum, where $\vec{O}$ denotes the origin and $\vec{N}$, the foot of the perpendicular from $\vec{O}$ to the tangent at $\vec{P}$. \hfill\brak{1999- 10 marks}

\item Let $ABC$ be an equilateral triangle inscribed in the circle $x^2+y^2=a^2$. Suppose perpendiculars from $\vec{A}$, $\vec{B}$, $\vec{C}$ to the major axis of the ellipse $\frac{x^2}{a^2}+\frac{y^2}{b^2}$=1, $(a>b)$ meet the ellipse respectively at $\vec{P}$, $\vec{Q}$, $\vec{R}$ such that $\vec{P}$, $\vec{Q}$, $\vec{R}$ lie on the same side of the major axis as $\vec{A}$, $\vec{B}$, $\vec{C}$ respectively. Prove that the normals to the ellipse drawn at the points $\vec{P}$, $\vec{Q}$, and $\vec{R}$ are concurrent. \hfill\brak{2000- 7 marks}

\item Let $C_1$ and $C_2$ be respectively, the parabolas $x^2=y-1$ and $y^2=x-1$. Let $\vec{P}$ be any point on $C_1$ and $\vec{Q}$ be any point on $C_2$. Let $P_1$ and $Q_1$ be the reflections of $\vec{P}$ and $\vec{Q}$ respectively with respect to the line $y=x$. Prove that $P_1$ lies on $C_2$, $Q_1$ lies on $C_1$, and $PQ \geq \text{min}({PP_1, QQ_1})$. Hence or otherwise determine points $P_0$ and $Q_0$ on the parabolas $C_1$ and $C_2$ respectively such that $P_0Q_0 \leq PQ$ for all pairs of points $(\vec{P},\vec{Q})$ with $\vec{P}$ on $C_1$ and $\vec{Q}$ on $C_2$. \hfill\brak{2000- 10 marks}
\item Prove that, in an ellipse, the perpendicular from a focus upon any tangent and the line joining the center of the ellipse to the point of contact meet on the corresponding directrix. \hfill\brak{2002- 5 marks}

\item Normals are drawn from the point $\vec{P}$ with slopes $m_1, m_2, m_3$ to the parabola $y^2=4x$. If the locus of $\vec{P}$ with $m_1m_2=\alpha$ is a part of the parabola itself, then find $\alpha$. \hfill\brak{2003- 4 marks}

\item A tangent is drawn to the parabola $y^2-2y-4x+5=0$ at a point $P$ which cuts the directrix at the point $\vec{Q}$. A point $\vec{R}$ is such that it divides $QP$ externally in the ratio 1:2. Find the locus of the point $\vec{R}$. \hfill\brak{2004 - 4 marks}

\item Tangents are drawn from any point on the hyperbola $\frac{x^2}{9}-\frac{y^2}{4}=1$ to the circle $x^2+y^2=9$. Find the locus of the midpoint of the chord of contact. \hfill\brak{2005 - 4 marks}

\item Find the equation of the common tangent in the 1st quadrant to the circle $x^2+y^2=16$ and the ellipse $\frac{x^2}{25}+\frac{y^2}{4}$=1. Also, find the length of the intercept of the tangent between the coordinate axes. \hfill\brak{2005 - 4 marks} 

\item Let the point $\vec{B}$ be the reflection of the point $\vec{A}\brak{2, 3}$ with respect to the line $8x-6y-23=0$. Let $T_A$ and $T_B$ be circles of radii $2$ and $1$ with centres $\vec{A}$ and $\vec{B}$ respectively. Let T be a common tangent to the circles $T_A$ and $T_B$ such that both the circles are on the same side of T. If $\vec{C}$ is the point of intersection of T and the line passing through $\vec{A}$ and $\vec{B}$, then the length of the line segment $AC$ is  \rule{1cm}{0.01pt}.\hfill(2019)
    \item Let the circles $C_{1}$ : $x^2+y^2=9$ and $C_{2}$ : $\brak{x-3}^2+\brak{y-4}^2 = 16$,  intersect at the points $\vec{X}$ and $\vec{Y}$. Suppose that another circle $C_{3}$ : $\brak{x-h}^2+\brak{y-k}^2=r^2$ satisfies the following conditions
%
%
% \begin{multicols}{2}
\begin{multicols}{2}
\begin{enumerate}[label=(\roman*)]
 \item Centre of $C_{3}$ is collinear with the centres of $C_{1}$ and $C_{2}$
 \item $C_{1}$ and $C{2}$ both lie inside $C_{3}$,  and
%
 \item $C_{3}$ touches $C_{1}$ at $\vec{M}$ and $C_{2}$ at  $\vec{N}$ 
\end{enumerate}
\end{multicols}
%\end{multicols}
Let the line through $\vec{X}$ and $\vec{Y}$ intersect $C_{3}$ at Z and W,  and let a common tangent of $C_{1}$ and $C_{3}$ be a tangent to the parabola $x^2=8\alpha y.$\\ %
Match the expressions given in the List-1 with those in List -2 below
%
\begin{multicols}{2}
\textbf{Column 1}
\begin{multicols}{2}
\begin{multicols}{2}
\begin{enumerate}[label=(\Alph*)]           
\item $2h+k$                                  
\item $\frac{\text{Length of ZW}}{\text{length of XY}}$     
\item $\frac{\text{Area of triangle MZN}}{\text{Area of triangle ZMW}}$                     
\end{enumerate}
\end{multicols}
\end{multicols}
\columnbreak
 \textbf{Column 2}
 \begin{multicols}{2}
\begin{multicols}{2}
\begin{enumerate}[label=(\alph*),  start=16]
 \item 6
 \item $\sqrt{6}$
 \item $\frac{5}{4}$                           
 \item $\frac{21}{5}$                          
 \item $2\sqrt{6}$                             
 \item $\frac{10}{3}$                         
 \end{enumerate}
\end{multicols}
\end{multicols}
\end{multicols}
%
\item Let $RS$ be the diameter of the Circle $x^{2} + y^{2} = 1$,  where $\vec{S}$ is the point \brak{1, 0}. Let $\vec{P}$ be a variable point \brak{\text{other than R and S}} on the circle and tangents to the circle at $\vec{S}$ and $\vec{P}$ meet at the point $\vec{Q}$. The normal to the circle at $\vec{P}$ intersects a line drawn through $\vec{Q}$ parallel to $RS$ at point $\vec{E}$. Then the locus of $\vec{E}$ passes through the point\brak{\text{s}}
%
\hfill {\brak{ 2016}}
\begin{multicols}{4}
\begin{multicols}{2}
\begin{enumerate}
	\item \brak{\frac{1}{3},  \frac{1}{\sqrt{3}}}
	\item \brak{\frac{1}{4},  \frac{1}{2}}
	\item \brak{\frac{1}{3},  -\frac{1}{\sqrt{3}}}
	\item \brak{\frac{1}{4},  -\frac{1}{2}}
\end{enumerate}
\end{multicols}
\end{multicols}
\item Let $T$ be a line passing through the points $\vec{P}$\brak{-2, 7} and $\vec{Q}$\brak{2, -5}. Let $F_1$ be the set of all pairs of circles \brak{S_1, S_2} such that $T$ is tangent to $S_1$ at $\vec{P}$ and tangent to $S_2$ at $\vec{Q}$,  and also such that $S_1$ and $S_2$ touch each other at a point,  say $\vec{M}$. Let $E_1$ be the set representing the locus of $\vec{M}$ as the pair \brak{S_1, S_2} varies in $F_1$. Let the set of all straight line segments joining a pair of distinct points of $E_1$ and passing through the point $\vec{R}$\brak{1, 1} be $F_2$. Then which of the following statements is (are) TRUE?
%
\hfill{\brak{ 2018}}
\begin{multicols}{2}
\begin{multicols}{2}
\begin{enumerate}
\item The point \brak{-2, 7} lies on $E_1$
\item The point \brak{\frac{4}{5},  \frac{7}{5}} does \textbf{NOT} lie on $E_1$
\item The point \brak{\frac{1}{3}, 1} lies on $E_1$
\item The point \brak{0,  \frac{3}{2}} does not lie on $E_1$
\end{enumerate}
\end{multicols}
\end{multicols}
        \item Let $PQ$ and $RS$ be tangents at the extremities of the diameter PR of a circle of radius $r$. If $PS$ and $RQ$ intersect at a point $\vec{X}$ on the circumference of the circle,  then $2r$ equals
        \hfill$\brak{2001S}$
        \begin{multicols}{2}
\begin{multicols}{2}
\begin{enumerate}
    \item $\sqrt{PQ.RS}$
     \item $\brak{PQ+RS}$
    \item $2PQ.RS/(PQ+RS)$
     \item {$\sqrt{(PQ^2+RS^2)}$}/2
     \end{enumerate}
\end{multicols}
\end{multicols}
     \item If the tangent at the point $\vec{P}$ on the circle $x^2+y^2+6x+6y=2$ meets a straight line $5x-2y+6=0$ at a point on the $Y$ axis,  then the length of $PQ$ is 
             \hfill$\brak{2002S}$
             \begin{multicols}{2}
\begin{multicols}{2}
\begin{enumerate}
             \item 4
             \item 2$\sqrt5$
             \item 5
             \item 3$\sqrt5$
\end{enumerate}
\end{multicols}
\end{multicols}
\item Circles with radii 3, 4 and 5 touch each other externally. If $\vec{P}$ is the point of intersection of tangents to these circles at their points of contact,  find the distance of $\vec{P}$ from the points of contact.
	           \hfill(2005) 
    \item No tangent can be drawn from point $\brak{5/2,  1}$ to circumcircle of triangle with vertices $\brak{1,  \sqrt{3}}$,  $\brak{1,  -\sqrt{3}}$,  and $\brak{3,  -\sqrt{3}}$.
    \hfill{(1985 )}
\item Tangents are drawn from point $\brak{17,  7}$ to the circle $x^2+y^2=169$.\\
STATEMENT-$1$:The tangents are mutually perpendicular.\\
STATEMENT-$2$:The locus of all points from which mutually perpendicular tangents can be drawn to a given circle is $x^2+y^2=338$. \hfill(2007)
\begin{multicols}{2}
\begin{enumerate}
\item Statement-$1$ is True, statement-$2$ is True; Statement-$2$ is a correct explantion for Statement-$1$.
\item Statement-$1$ is True, statement-$2$ is True; Statement-$2$ is NOT a correct explantion for Statement-$1$.
\item Statement-$1$ is True,  Statement-$2$ is False
\item Statement-$1$ is False,  Statement-$2$ is True.
\end{enumerate}
\end{multicols}
%
\item The lines $3x-4y+4=0$ and $6x-8y-7=0$ are tangents to the same circle. The radius of the circle is \rule{1cm}{0.01pt}.
	\hfill\brak{1984}
\item Let $x^{2}+y^{2}-4x-2y-11=0$ be a circle. A pair of tangents from the point $\brak{4, 5}$ with a pair of radii form a quadrilateral of area
\rule{1cm}{0.01pt}.
%
	\hfill\brak{1985}
\item The area of the triangle formed by the tangents from the point $\brak{4, 3}$ to the circle $x^{2}+y^{2}=9$ and the line joining their point of contact is
\rule{1cm}{0.01pt}.
%
	\hfill\brak{1987}
\item The area formed by the positive $X$ axis and the normal and the tangent to the circle $x^{2}+y^{2}=4$ at $\brak{1, \sqrt{3}}$ is
\rule{1cm}{0.01pt}.
%
	\hfill\brak{1989}
\item The chords of contact of the pair of tangents drawn from each point on the line $2x+y=4$ to $x^{2}+y^{2}=1$ pass through the point \rule{1cm}{0.01pt}.
%
	\hfill\brak{1997}
\item The centres of two circles $C_1$ and $C_2$ each of unit radius are at a distance of $6$ units from each other. Let $\vec{P}$ be the midpoint of the line segment joining the centres of $C_1$ and $C_2$ and $C$ be a circle touching circles $C_1$ and $C_2$ externally.If a common tangent to $C_1$ and $C$ passing through $\vec{P}$ is also a common tangent to $C_2$ and $C$,  then the radius of circle $C$ is  \rule{1cm}{0.01pt}.\hfill(2009)
\item Consider a family of circles which are passing through the point \brak{-1, 1},  and are tangent to $X$ axis. If \brak{h, k} are the coordinate of the centre of the circles,  then the set of values of $k$ is given by the interval
\hfill{(2007)}
\begin{multicols}{4}
\begin{multicols}{2}
\begin{enumerate}
\item $\frac{-1}{2} \le k \le \frac{1}{2}$
\item $k \le \frac{1}{2}$
\item $o \le k \le \frac{1}{2}$
\item $k \ge \frac{1}{2}$
\end{enumerate}
\end{multicols}
\end{multicols}
\item The equations of the tangents drawn from the origin to the circle $x^2+y^2-2rx-2hy+h^2=0$, are
    \hfill$\brak{1988}$
\begin{multicols}{2}
\begin{multicols}{2}
\begin{enumerate}
    \item $x=0$
    \item $y=0$
    \item $(h^2-r^2)x-2rhy=0$
    \item $(h^2-r^2)x+2rhy=0$
\end{enumerate}
\end{multicols}
\end{multicols}
\item The number of common tangents to the circles $x^2+y^2=4$ and $x^2+y^2-6x-8y=24$ is 
    \hfill$\brak{1998}$
\begin{multicols}{4}
\begin{multicols}{2}
\begin{enumerate}
    \item 0
    \item 1 
    \item 3
    \item 4
\end{enumerate}
\end{multicols}
\end{multicols}
    \item The angle between the pair of tangents drawn from the point $\vec{P}$ to the circle $x^{2}+y^{2}+4x-6y+9\sin^{2}{\alpha}+13\cos^{2}{\alpha}=0$ is $2\alpha$. The equation of the locus of the point $\vec{P}$ is
    \hfill {(1996 )}
    \begin{multicols}{2}
\begin{multicols}{2}
\begin{enumerate}
    	\item $x^{2}+y^{2}+4x-6y+4=0$
    	\item $x^{2}+y^{2}+4x-6y-9=0$
    	\item $x^{2}+y^{2}+4x-6y-4=0$
    	\item $x^{2}+y^{2}+4x-6y+9=0$
    \end{enumerate}
\end{multicols}
\end{multicols}
         \item Tangents drawn from the point $\vec{P}\brak{1, 8}$ to the circle $x^2+y^2-6x-4y-11=0$ touch the circle at the points $\vec{A}$ and $\vec{B}$. The equation of the circumcircle of the triangle $PAB$ is
             \hfill$\brak{2009}$
             \begin{multicols}{2}
\begin{multicols}{2}
\begin{enumerate}
             \item $x^2+y^2+4x-6y+19=0$
             \item $x^2+y^2-4x-10y+19=0$
             \item $x^2+y^2-4x+6y-29=0$
             \item $x^2+y^2-4x-6y+19=0$
             \end{enumerate}
\end{multicols}
\end{multicols}
             \item The locus of the mid-point of the chord of contact of tangents drawn from points lying on the straight line $4x-5y=20$ to the circle $x^2+y^2=9$ is
                 \hfill$\brak{2012}$
             \begin{multicols}{2}
\begin{multicols}{2}
\begin{enumerate}
                 \item $20\brak{x^2+y^2}-36x+45y=0$
                 \item $20\brak{x^2+y^2}+36x-45y=0$
                 \item $36\brak{x^2+y^2}-20x+45y=0$
                 \item $36\brak{x^2+y^2}+20x-45y=0$
             \end{enumerate}
\end{multicols}
             \end{multicols}
\item Let $\vec{A}$ be the centre of circle $x^2+y^2-2x-4y-20=0$. Suppose that the tangents at the points $\vec{B}$\brak{1, 7} and $\vec{D}$\brak{4, -2} on the circle meet at point $\vec{C}$. Find the area of the quadrilateral $ABCD$.
%
\hfill {\brak{1981 }}
\item Two circles,  each of radius 5 units,  touch each other at \brak{1, 2}. If the equation of common tangent is $4x+3y=10$,  find the equations of circles.
%
\hfill {\brak{1991 }}
\item Find the coordinates of the point at which the circles $x^2+y^2-4x-2y=-4$ and $x^2+y^2-12x-8y=-36$ touch each other. Also find equations common tangents touching the circles in the distinct points.                        
%
\hfill(1993)
\item $C_{1}$ and $C_{2}$ are two concentric circles,  the radius of $C_{2}$ being twice that of $C_{1}$. From a point $\vec{P}$ on $C_{2}$,  tangents $PA$ and $PB$ are drawn to $C_{1}$. Prove that the centroid of the triangle $PAB$ lies on $C_{1}$.
	           \hfill(1998)
%
%
%
%
\item Let $T_{1}$,  $T_{2}$ be two tangents drawn from $\brak{2, 0}$ onto the circle $C$:$x^2+y^2=1$. Determine the circles touching $C$ and having $T_{1}$,  $T_{2}$ as their pair of tangents. Further,  find the equations of all possible common tangents to these circles,  when taken two at a time.
                  \hfill(1999)
%
%
%
%
\item Let $2x^2+y^2-3xy=0$ be the equation of pair of tangents drawn from the origin $O$ to a circle of radius 3 with the centre in the first quadrant. If $\vec{A}$ is one of the points of contact,  find the length of $OA$.                   \hfill(2001)
%
%
%
\item For the circle $x^2+y^2=r^2$,  find the value of $r$ for which the area enclosed by the tangents drawn from the point $\vec{P}$$\brak{6, 8}$ to the circle and the chord of contact is maximum.
%
%
\hfill(2003) 
%
\item Let $C_{1}$ and $C_{2}$ be two circles with $C_{2}$ lying inside $C_{1}$. A circle $C$ lying inside $C_{1}$ touches $C_{1}$ internally and $C_{2}$ externally. Identify the locus of centre of $C$.                                \hfill(2001)
%
%
%
%
%
%
\item Find the equation of circle touching the line $2x+3y+1=0$ at $\brak{1, -1}$ and cutting orthogonally the circle having line segment joining \brak{0, 3} and \brak{-2, -1} as diameter.
%
%
\hfill(2004)     
\item Find the equations of the circle passing through \brak{-4, 3} and touching the lines $x+y=2$ and $x-y=2$
%
\hfill {\brak{1981 }}
\item Lines $5x+12y-10=0$ and $5x-12y-40=0$ touch a Circle $C_1$ of diameter 6. If the centre of $C_1$ lies in the first quadrant,  find the equation of circle $C_2$ which is concentric with $C_1$ and cuts intecepts of length 8 on these lines
%
\hfill {\brak{1986 }}
\item A circle touches the line $y=x$ at a point $\vec{P}$ such that $OP=4\sqrt{2}$,  where O is the origin. The circle contains the point \brak{-10, 2} in its interior and the length of its chord on the line $x+y=0$ is $6\sqrt{2}$. Determine the equation of circle.
%
\hfill {\brak{1990 }}
     \item The point of intersection of the tangents at the ends of the latus rectum of the parabola $y^2=4x$ is \ldots.
    \hfill\brak{1994}
    
    \item An ellipse has eccentricity $\frac{1}{2}$ and one focus at the point $p\brak{\frac{1}{2},1}$.Its one directrix is common tangent,nearer to the point $P$, to the circle $x^2+y^2=1$ and the hyperbola $x^2-y^2=1$.The equation of the ellipse, in the standard form, is\dots.
    \hfill\brak{1992} 
\item Consider the parabola $y^2=8x$ . Let $\Delta_1$ be the area of the triangle formed by the end points of its latus rectum and the point $\Vec{P}$$(\frac{1}{2},2)$ on the parabola and $\Delta_2$ be the area of the triangle formed by drawing tangents at $\Vec{P}$ and at the end points of the latus rectum.Then $\frac{\Delta_1}{\Delta_2}$ is 
\hfill(2011)
\item A circle touches the $X$ axis and also touches the circle with centre at \brak{0, 3} and radius 2. The locus of the centre of the circle is
\hfill{(2005)}
\begin{multicols}{2}
\begin{multicols}{2}
\begin{enumerate}
\item an ellipse
\item a circle 
\item a hyperbola
\item a parabola
\end{enumerate}
\end{multicols}
\end{multicols}
\item Circle(s) touching the $X$ axis at a distance 3 from the origin and having an intercept of length $2\sqrt{7}$ on the $Y$ axis is \brak{\text{are}}
%
\hfill{\brak{ 2013}}
\begin{multicols}{2}
\begin{multicols}{2}
\begin{enumerate}
\item $x^2 + y^2 - 6x + 8y + 9 = 0$
\item $x^2 + y^2 - 6x + 7y + 9 = 0$
\item $x^2 + y^2 - 6x - 8y + 9 = 0$
\item $x^2 + y^2 - 6x - 7y + 9 = 0$
\end{enumerate}
\end{multicols}
\end{multicols}
    \item The number of the values of $c$ such that the straight line $y=4x+c$ touches the curves $(x^2/4)+y^2=1$ is \hfill(1998)\\
	\begin{multicols}{2}
\begin{enumerate}
			\begin{multicols}{2}
	\item $0$
	\item $1$
		\columnbreak
	\item $2$
	\item infinite
			\end{multicols}
	\end{enumerate}
\end{multicols}
\item Let $P\brak{a\sec\theta,b\tan\theta}$ and $Q\brak{a\sec\phi,b\tan \phi}$,where $\theta+\phi=\pi/2$, be two points on the hyperbola $\frac{x^2}{a^2}-\frac{y^2}{b^2}=1$.If $\brak{h,k}$ is the point 0f intersection of the normals at $P$ and $Q$, then $K$ equal to 
      \hfill \brak{1999}
\begin{multicols}{2}
\begin{enumerate}
    \item $\frac{a^2+b^2}{a}$
    \item $-\brak{\frac{a^2+b^2}{a}}$
    \item $\frac{a^2+b^2}{b}$
    \item $-\brak{\frac{a^2+b^2}{b}}$
\end{enumerate}
\end{multicols}
    \item The normal at a point $\vec{P}$ on the ellipse $x^2 +4y^2=16$ meets the $x$-axis at $\vec{Q}$. If $\vec{M}$ is the mid point of the line segment $\vec{PQ}$, then the locus of $\vec{M}$ interests the latusrectums of the given ellipse at the points
	\hfill (2009)
		\begin{multicols}{2}
\begin{enumerate}
				\begin{multicols}{2}
			\item $\brak{\pm\frac{3\sqrt{5}}{2},\pm\frac{2}{7}}$
			\item $\brak{\pm\frac{3\sqrt{5}}{2},\pm\sqrt{\frac{19}{4}}}$
				\columnbreak
			\item $\brak{\pm2\sqrt{3},\pm\frac{1}{7}}$
			\item $\brak{\pm2\sqrt{3},\pm\frac{4\sqrt{3}}{7}}$
				\end{multicols}
		\end{enumerate}
\end{multicols}

\item Let $\vec{P}\brak{6,3}$ be a point on the hyperbola $\frac{x^2}{a^2}-\frac{y^2}{b^2}=1$. If the normal at the point $\vec{P}$ intersects the $x$-axis at $\brak{9,0}$, then the eccentricity of the hyperbola is 
	\hfill (2011)
		\begin{multicols}{2}
\begin{enumerate}
				\begin{multicols}{2}
			\item$\sqrt{\frac{5}{2}}$
			\item$\sqrt{\frac{3}{2}}$
				\columnbreak
			\item$\sqrt{2}$
			\item$\sqrt{3}$
				\end{multicols}
		\end{enumerate}
\end{multicols}
\item The normal to the curve, $x^2+2xy-3y^2=0$, at $(0,1)$\hfill(2015)
\begin{multicols}{2}
\begin{enumerate}
    \item meets the curve again in the third quadrant.
    \item meets the curve again in the fourth quadrant.
    \item doesn't meet the curve again.
    \item meets the curve again in the second quadrant.
\end{enumerate}
\end{multicols}
	\item   Match the following: $\vec{(3,0)}$ is the pt. from which three normals are drawn to the parabola $y^2 = 4x$ which meet the parabola in the points P, Q and R. Then \hfill{(2006 - 6M)}
\begin{multicols}{2}
\textbf{Column I}
\begin{multicols}{2}
\begin{enumerate}
    \item Area of $\Delta$POR 
    \item Radius of circumcircle of $\Delta$PQR
    \item Centroid of $\Delta$POR 
    \item  Circumcentre of $\Delta$PQR 
\end{enumerate}
\end{multicols} 

\textbf{Column II}
\begin{multicols}{2}
\begin{enumerate}
    \item 2
    \item $\frac{5}{2}$
    \item $\vec{(\frac{5}{2},0)}$
    \item $\vec{(\frac{2}{3},0)}$
\end{enumerate}
\end{multicols}
\end{multicols}

	\item Let L be a normal to the parabola $y^2=4x$. If L passes through the point $(9,6)$, then the L is given by 
		\hfill(2010)
		
		 \begin{multicols}{2}
\begin{enumerate}
			\item $y-x+3=0$
			\item $y+3x-33=0$
			\item $y+x-15=0$
			\item $y-2x+12=0$
		 \end{enumerate}
\end{multicols}

\item If $x+y=k$ is normal $y^2=12x$,then $K$ is
     \hfill\brak{2000s}
\begin{multicols}{2}
\begin{enumerate}
    \item $3$
    \item $9$
    \item $-9$
    \item $-3$
\end{enumerate}
\end{multicols}
\end{enumerate}
