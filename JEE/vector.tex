	\begin{enumerate}
        \item Consider 3 points 
            \begin{align*}
                \vec{P}=\brak{-\sin \brak{\beta - \alpha}, - \cos\beta}, \vec{Q} = \brak{\cos \brak{\beta - \alpha}, \sin\beta}
            \end{align*} and 
            \begin{align*} \vec{R} = \brak{\cos \brak{\beta - \alpha + \theta}, \sin\brak{\beta - \theta}} \end{align*} where $0<\alpha,\beta,\theta<\frac{\pi}{4}$. Then, 
  \begin{multicols}{2}
                \begin{enumerate}
                    \item $\vec{P}$ lies on the line segment $RQ$
                    \item $\vec{Q}$ lies on the line segment $PR$
                    \item $\vec{R}$ lies on the line segment $QP$
                    \item $\vec{P}$,$\vec{Q}$,$\vec{R}$ are non-collinear
                \end{enumerate}
  \end{multicols}
\hfill \brak{2008}
    \item Let $\vec{A}$, $\vec{B}$, $\vec{C}$ be vectors of length 3,4,5 respectively. Let $\vec{A} \perp \vec{B}+\vec{C}$, $\vec{B}\perp \vec{C}+\vec{A}$ and $\vec{C} \perp \vec{A}+\vec{B}$. Find  the length of $\vec{A}+\vec{B}+\vec{C}$. 
    \hfill (1981)
    \item Find the unit vector perpendicular to the plane determined by $\vec{P}(1,-1,2), \vec{Q}(2,0,-1)$ and $\vec{R}(0,2,1)$.
    \hfill (1983)
    \item Find the area of the triangle whose vertices are $\vec{A}(1,-1,2), \vec{B}(2,0,-1), \vec{C}(3,-1,2)$. 
    \hfill (1983)
    \item $\vec{A}, \vec{B}, \vec{C}$ and $\vec{D}$, are four points in a plane respectively such that $(\vec{A}-\vec{D})\cdot(\vec{B}-\vec{C})=(\vec{B}-\vec{D})\cdot(\vec{C}-\vec{A})=0$.
	    The point $\vec{D}$, then, is the \rule{1cm}{0.01pt} of $\triangle ABC$.
    \hfill (1984)
\item If $\vec{A}, \vec{B}, \vec{C}$ are three non-coplanar vectors, then $\frac{\vec{A}\cdot\vec{B}\times\vec{C}}{\vec{C}\times\vec{A}\cdot\vec{B}}+\frac{\vec{B}\cdot\vec{A}\times\vec{C}}{\vec{C}\cdot\vec{A}\times\vec{B}}=$.
\hfill (1985)
\item $\vec{A}=(1,1,1)$, $\vec{C}=(0,1,-1)$ are given vectors, then find $\vec{B}$ that satisfies $\vec{A}\times\vec{B}=\vec{C}$ and $\vec{A}\cdot\vec{B}=3$. 
	\hfill (1985)
\item Let $\vec{b}=4\hat{i}+3\hat{j}$ and $\vec{c}$ be two vectors perpendicular to each other in the xy-plane.All vectors in the same plane having projections 1 and 2 along $\vec{b}$ and $\vec{c}$, respectively, are given by \rule{1cm}{0.01pt}.
\hfill{(1987-2 marks)}
\item The components of a vectors $\vec{a}$ along and  perpendicular to a non-zero vector $\vec{b}$ are \rule{1cm}{0.01pt} and \rule{1cm}{0.01pt} respectively.
\hfill (1988)
\item Given that $\vec{a}=(1,1,1),\vec{c}=(0,1,-1),\vec{a}\cdot\vec{b}=3$ and $\vec{a}\times\vec{b}=\vec{c}, \vec{b}= \rule{1cm}{0.01pt}$.
\hfill(1991)
\item A unit vector coplanar with $\hat{i}+\hat{j}+2\hat{k}$ and $\hat{i}+2\hat{j}+\hat{k}$ and perpendicular to $\hat{i}+\hat{j}+\hat{k}$ is \rule{1cm}{0.01pt}.
\hfill (1992)
\item A unit vector perpendicular to the plane determined by the points $\vec{P}(1,-1,2), \vec{Q}(2,0,-1)$ and $\vec{R}(0,2,1)$ is \rule{1cm}{0.01pt}.
\hfill (1994)
\item If $\vec{b}$ and $\vec{c}$ are any two non-collinear unit vectors and $\vec{a}$ is any vector, then $(\vec{a}\cdot\vec{b})\vec{b}+(\vec{a}\cdot\vec{c})\vec{c}+\frac{\vec{a}\cdot(\vec{b}\times\vec{c})}{\abs{\vec{b}\times\vec{c}}}(\vec{b}\times\vec{c})$=\rule{1cm}{0.01pt}.
\item %1
		Let $\vec{a}=a_1\hat{i}+a_2\hat{j}+a_3\hat{k}$, $\vec{b}=b_1\hat{i}+b_2\hat{j}+b_3\hat{k}$ and $\vec{c}=c_1\hat{i}+c_2\hat{j}+c_3\hat{k}$ be three non-zero vectors such that $\vec{c}$ is a unit vector perpendicular to both the vectors $\vec{a}$ and $\vec{b}$. If the angle between $\vec{a}$ and $\vec{b}$ is $\frac{\pi}{6}$, then 
$$
	\mydet{
a_1 & a_2 & a_3 \\
b_1 & b_2 & b_3 \\
c_1 & c_2 & c_3
}^2
$$
		is equal to \rule{1cm}{0.01pt}. \hfill \brak{1986}
\item %2
	The number of vectors of unit length perpendicular to vectors $\vec{a}=\brak{1,1,0}$ and $\vec{b}=\brak{0,1,1}$ is \hfill \brak{1987}
  \begin{multicols}{4}
		\begin{enumerate}
			\item one
		        \item two
			\item three
			\item infinite
			\item None of these
		\end{enumerate}
  \end{multicols}
\item %3
	Let $\vec{a}=2\hat{i}-\hat{j}+\hat{k}$, $\vec{b}=\hat{i}+2\hat{j}-\hat{k}$ and $\vec{c}=\hat{i}+2\hat{j}-2\hat{k}$ be three vectors. A vector in the plane of $\vec{b}$ and $\vec{c}$, whose projection on $\vec{a}$ is of magnitude $\sqrt{2/3}$, is: \hfill{\brak{1993-2Marks}}\\
		\begin{enumerate}
			\item $2\hat{i}+3\hat{j}-3\hat{k}$
			\item $2\hat{i}+3\hat{j}+3\hat{k}$
			\item $-2\hat{i}-\hat{j}+5\hat{k}$
			\item $2\hat{i}+\hat{j}+5\hat{k}$
		\end{enumerate}
\item %4
	The vector $\frac{1}{3}\brak{2\hat{i}-2\hat{j}+\hat{k}}$ is \hfill{\brak{1994}}\\
		\begin{enumerate}
			\item a unit vector
			\item makes an angle with the vector
			\item parallel to the vector $-\hat{i}+\hat{j}-\frac{1}{2}\hat{k}$
			\item perpendicular to the vector $3\hat{i}+2\hat{j}-2\hat{k}$
                \end{enumerate}
\item %5
	If $\vec{a}=\hat{i}+\hat{j}+\hat{k}$,$\vec{b}=4\hat{i}+3\hat{j}+4\hat{k}$ and $\vec{c}=\hat{i}+\alpha\hat{j}+\beta\hat{k}$ are linearly dependent vectors and $\abs{c}=\sqrt{3}$, then \hfill{\brak{1998-2Marks}}\\
		\begin{enumerate}
			\item $\alpha=1,\beta=-1$
			\item $\alpha=1,\beta=\pm1$
			\item $\alpha=-1,\beta=\pm1$
			\item $\alpha=\pm1,\beta=1$
		\end{enumerate} 
\item %6
	For three vectors $\vec{u,v,w}$ which of the following expression is not equal to any of the remaining three? \hfill{\brak{1998-2Marks}} 
		\begin{enumerate}
			\item $\vec{u\cdot\brak{v\times w}}$
			\item $\vec{\brak{v\times w}\cdot u}$
			\item $\vec{v\cdot\brak{u\times w}}$
			\item $\vec{\brak{u\times v}\cdot w}$
		\end{enumerate}
\item %7
	Which of the following expressions are meaningful? \hfill{\brak{1998-2Marks}}
		\begin{enumerate}
			\item $\vec{u\brak{v \times w}}$
			\item $\vec{\brak{u \cdot v}\cdot w}$
			\item $\vec{\brak{u \cdot v}w}$
			\item $\vec{u\times\brak{v \cdot w}}$
		\end{enumerate}
	\item Let a and b be two non-collinear unit vectors. If $u=a-\brak{a\cdot b}b$ and $v=a\times b$, then $\abs{v}$ is \hfill{(1999-3 Marks)}
		\begin{enumerate}
			\item $\abs{u}$
			\item $\abs{u} + \abs{u \cdot a}$
			\item $\abs{u} + \abs{u \cdot b}$
			\item $\abs{u} +  u\cdot\brak{a+b}$
		\end{enumerate}
%9
	\item Let $\vec{A}$ be vector parallel to line of intersection of planes $P_1$ and $P_2$. Plane $P_1$ is parallel to the vectors $2\hat{j}+3\hat{k}$ and $4\hat{j}-3\hat{k}$ and that $P_2$ is
		parallel to $\hat{j}-\hat{k}$ and $3\hat{i}+3\hat{j}$, then the angle between vector $\vec{A}$ and a given vector $2\hat{i}+\hat{j}-2\hat{k}$ is \hfill{(2006-5M,-1)}
		\begin{enumerate}
			\item $\frac{\pi}{2}$
			\item $\frac{\pi}{4}$
			\item $\frac{\pi}{6}$
			\item $\frac{3\pi}{4}$
		\end{enumerate}
%10
	\item The vector(s) which is/are coplanar with vectors $\hat{i}+\hat{j}+2\hat{k}$ and $\hat{i}+2\hat{j}+\hat{k}$, and perpendicular to the vector $\hat{i}+\hat{j}+\hat{k}$ is/are \hfill{(2011)}
		\begin{enumerate}
			\item $\hat{j}-\hat{k}$
			\item $\hat{i}+\hat{j}$
			\item $\hat{i}-\hat{j}$
			\item $\hat{j}+\hat{k}$
		\end{enumerate}
%11
	\item Let $\vec{x},\vec{y}$ and $\vec{z}$ be three vectors each of magnitude $\sqrt{2}$ and the angle between each pair of them is $\frac{\pi}{3}$. If $\vec{a}$ is a non-zero vector perpendicular
		to $\vec{x}$ and $\vec{y}\times \vec{z}$ and $\vec{b}$ is a non-zero vector perpendicular to $\vec{y}$ and $\vec{z}\times \vec{x}$, then \hfill{(JEE Adv.2014)}
		\begin{enumerate}
			\item $\vec{b}=\brak{\vec{b} \cdot \vec{z}}\brak{\vec{z}-\vec{x}}$
			\item $\vec{a}=\brak{\vec{a} \cdot \vec{y}}\brak{\vec{y}-\vec{z}}$
			\item $\vec{a}\cdot\vec{b}=-\brak{\vec{a} \cdot \vec{y}}\brak{\vec{b}\cdot\vec{z}}$
			\item $\vec{a}=-\brak{\vec{a} \cdot \vec{y}}\brak{\vec{z}-\vec{y}}$
		\end{enumerate}
%15
%18
	\item Let $\triangle PQR$ be a triangle. Let $\vec{a}=\vec{QR}$, $\vec{b}=\vec{RP}$ and $\vec{c}=\vec{PQ}$. If $\abs{\vec{a}}=12$, $\abs{\vec{b}}=4\sqrt{3}$, $\vec{b}
		\cdot \vec{c}=24$, then which of the following is(are)true? \hfill{(JEE Adv.2015)}
		\begin{enumerate}
			\item $\frac{\abs{\vec{c}}^2}{2}-\abs{\vec{a}}=12$
			\item $\frac{\abs{\vec{c}}^2}{2}+\abs{\vec{a}}=30$
			\item $\abs{\vec{a}\times \vec{b}+\vec{c}\times \vec{a}}=48\sqrt{3}$
			\item $\vec{a}\cdot\vec{b}=-72$
		\end{enumerate}
%19
%20
	\item Let $\hat{u}=u_1\hat{i}+u_2\hat{j}+u_3\hat{k}$ be a unit vector in $\vec{R}^3$ and $\hat{w}=\frac{1}{\sqrt{6}}\brak{\hat{i}+\hat{j}+2\hat{k}}$. Given that there exists a vector $\vec{v}$ in $\vec{R}^3$ 
		such that $\abs{\hat{u} \times \vec{v}}=1$ and $\hat{w}\brak{\hat{u}\times \vec{v}}=1$. Which of the following statement(s) is(are) correct? \hfill{(JEE Adv.2016)}
		\begin{enumerate}
			\item there is exactly one choice for such $\vec{v}$
			\item There are infinitely many choices for such $\vec{v}$
			\item If $\hat{u}$ lies in the xy-plane then $\abs{u_1}= \abs{u_2}$
			\item If $\hat{u}$ lies in the xz-plane then $2\abs{u_1}=\abs{u_3}$
		\end{enumerate}
%21
\item The scalar $\vec{A}.\brak{\brak{\vec{B}+\vec{C})\times(\vec{A}+\vec{B}+\vec{C}}}$ equals:
\begin{enumerate}
\item  $0$
\item $\sbrak{\vec{A}\ \vec{B}\ \vec{C}} + \sbrak{\vec{B}\ \vec{C}\ \vec{A}}$
\item $\sbrak{\vec{A}\ \vec{B}\ \vec{C}}$
\item None of these
\end{enumerate}
\hfill (1981 - 2 Marks)

\item For non-zero vectors $\vec{a}, \vec{b}, \vec{c}$, $\abs{\brak{\vec{a}\times\vec{b}}.\vec{c}} = \abs{\vec{a}}\abs{\vec{b}}\abs{\vec{c}}$ holds if and only if
\begin{enumerate}
\item $\vec{a}.\vec{b}=0$, $\vec{b}.\vec{c}=0$
\item $\vec{b}.\vec{c}=0$, $\vec{c}.\vec{a}=0$
\item $\vec{c}.\vec{a}=0$, $\vec{a}.\vec{b}=0$
\item $\vec{a}.\vec{b}= \vec{b}.\vec{c}= \vec{c}.\vec{a}=0$
\end{enumerate}
\hfill (1982 - 2 Marks)

\item The volume of the parallelopiped whose sides are given by $OA =\vec{2i}-\vec{2j}$, $OB = \vec{i}+\vec{j}-\vec{k}$, $OC = \vec{3i}-\vec{k}$, is 
\begin{enumerate}
\item $\frac{4}{13}$
\item $4$
\item $\frac{2}{7}$
\item None of these
\end{enumerate}
\hfill (1983 - 1 Mark)

\item The points with position vectors $\vec{60i} + \vec{3j}$, $\vec{40i}-\vec{8j}$, $\vec{ai}-\vec{52j}$ are collinear if
\begin{enumerate}
\item $a=-40$
\item $a=40$
\item $a=20$
\item None of these
\end{enumerate}
\hfill (1983 - 1 Mark)

\item Let $\vec{a}, \vec{b}, \vec{c}$ be three non coplanar vectors and $\vec{p}, \vec{q},\vec{r}$ are vectors defined by the relations $\vec{p}=\frac{\vec{b}\times\vec{c}}{\sbrak{\vec{a}\ \vec{b}\ \vec{c}}}, \vec{q}=\frac{\vec{c}\times\vec{a}}{\sbrak{\vec{a}\ \vec{b}\ \vec{c}}},\vec{r}=\frac{\vec{a}\times\vec{b}}{\sbrak{\vec{a}\ \vec{b}\ \vec{c}}}$ then the value of the expression $\brak{\vec{a}+\vec{b}}.\vec{p}+\brak{\vec{b}+\vec{c}}.\vec{q}+\brak{\vec{c}+\vec{a}}.\vec{r}$ is equal to
\begin{enumerate}
\item $0$
\item $1$
\item $2$
\item $3$
\end{enumerate}
\hfill (1988 - 2 Marks)


\item Let $\vec{p}$ and $\vec{q}$ be the position vectors of $\vec{P}$ and $\vec{Q}$ respectively, with respect to $\vec{O}$ and $\abs{\vec{p}} = p$, $\abs{\vec{q}} = q$. The points $\vec{R}$ and $\vec{S}$ divide $PQ$ internally and externally in the ratio $2\colon3$ respectively. If $OR$ and $OS$ are perpendicular then
\begin{enumerate}
\item $9p^2 =4q^2$
\item $4p^2 = 9q^2$
\item $9p = 4q$
\item $4p = 9q$
\end{enumerate}
\hfill (1994)

\item Let $\alpha$, $\beta$, $\gamma$ be distinct real numbers. The points with position vectors $\vec{\alpha i}+ \vec{\beta j} + \vec{\gamma k}$, $\vec{\beta i}+ \vec{\gamma j}+ \vec{\alpha k}$, $\vec{\gamma i} + \vec{\alpha j} + \vec{\beta k}$
\begin{enumerate}
\item are collinear
\item form an equilateral triangle
\item form a scalene triangle
\item form a right angles triangle
\end{enumerate}
\hfill (1994)

\item Let $\vec{a=i-j}$, $\vec{b=j-k}$, $\vec{c=k-i}$. If $\vec{d}$ is a unit vector such that $\vec{a}.\vec{d} = 0 = \sbrak{\vec{b}\ \vec{c}\ \vec{d}}$, then $\vec{d}$ equals
\begin{enumerate}
\item $\pm \vec{\frac{{i+j-2k}}{\sqrt{6}}}$
\item $\pm \vec{\frac{{i+k-k}}{\sqrt{3}}}$
\item $\pm \vec{\frac{{i+j+k}}{\sqrt{3}}}$
\item $\pm \vec{k}$
\end{enumerate}
\hfill (1995S)

\item If $\vec{a},\vec{b},\vec{c}$ are non coplanar unit vectors such that $\vec{a}\times\brak{\vec{b}\times\vec{c}} = \frac{\brak{\vec{b}+\vec{c}}}{\sqrt{2}}$, then the angle between $\vec{a}$ and $\vec{b}$ is
\begin{enumerate}
\item $\frac{3\pi}{4}$
\item $\frac{\pi}{4}$
\item $\frac{\pi}{2}$
\item $\pi$
\end{enumerate}
\hfill (1995S)
  \item Let $\vec{u},\vec{v}$ and $\vec{w}$ be vectors such that $\vec{u}+\vec{v}+\vec{w} = 0$. If $\abs{\vec{u}}=3$,$\abs{\vec{v}}=4$ and $\abs{\vec{w}}=5$ , then $\vec{u} \cdots \vec{v}+\vec{v} \cdots \vec{w}+\vec{u} \cdot \vec{w}$ is
  \hfill (1995S)
 \begin{enumerate}
  \begin{multicols}{2}
  \item 47
  \item -25
  \item 0
  \item 25
  \end{multicols}
 \end{enumerate}   	
  \item If $\vec{a},\vec{b}$ and $\vec{c}$ are three non-coplanar vectors then 
  $\brak{\vec{a}+\vec{b}+\vec{c}} \cdots \sbrak{\brak{\vec{a}+\vec{b}} \times \brak{\vec{a}+\vec{c}}}$ equals
  \hfill (1995S)
  \begin{enumerate}
  \begin{multicols}{2}
    \item $0$
    \item $\sbrak{\vec{a}\ \vec{b}\ \vec{c}}$ 
    \item $2\sbrak{\vec{a}\ \vec{b}\ \vec{c}}$ 
   \item $-\sbrak{\vec{a}\ \vec{b}\ \vec{c}}$
  \end{multicols}
  \end{enumerate}
\item Let $\vec{a}=2\vec{i}+\vec{j}-2\vec{k}$ and $\vec{b}=\vec{i}+\vec{j}$. If $\vec{c}$ is  a vector such that $\vec{a} \rule{1cm}{0.01pt}. \vec{c}=\abs{\vec{c}}$,$\abs{\vec{c}-\vec{a}}=2\sqrt{2}$ and the angle between $\brak{\vec{a} \times \vec{b}}$ and $\vec{c}$ is $30\degree$, then $\abs{\brak{\vec{a} \times \vec{b}} \times \vec{c}} = $  
\hfill (1999 - 2 Marks)
\begin{enumerate}
\begin{multicols}{2}
    \item $\frac{2}{3}$
    \item $\frac{3}{2}$
    \item $2$
    \item $3$
\end{multicols}
\end{enumerate}
\item Let $\vec{a}=2\vec{i}+\vec{j}+\vec{k},\vec{b}=\vec{i}+2\vec{j}-\vec{k}$ and a unit vector $\vec{c}$ be coplanar. If $\vec{c}$ is perpendicular to $\vec{a}$, then $\vec{c}=$
\hfill (1999 - 2 Marks)
\begin{enumerate}
\begin{multicols}{2}
\item $\frac{1}{\sqrt{2}}\brak{-\vec{j}+\vec{k}}$
\item $\frac{1}{\sqrt{3}}\brak{-\vec{i}-\vec{j}-\vec{k}}$ \item $\frac{1}{\sqrt{5}}\brak{\vec{i}-2\vec{j}}$    
\item $\frac{1}{\sqrt{3}}\brak{\vec{i}-\vec{j}-\vec{k}}$  \end{multicols}
\end{enumerate}
\item If the vectors $\vec{a},\vec{b}$ and $\vec{c}$ from the sides $BC,CA$ and $AB$ respectively of a triangle $ABC$, then
\hfill (2000S)
\begin{enumerate}
\begin{multicols}{2}
\item $\vec{a} \cdot \vec{b}+\vec{b} \cdot \vec{c}+\vec{c} \cdot \vec{a} = 0$
\item $\vec{a} \cdot \vec{b}=\vec{b} \cdot \vec{c}=\vec{c} \rule{1cm}{0.01pt}. \vec{a}$
\item $\vec{a} \times \vec{b}=\vec{b} \times \vec{c}=\vec{c} \times \vec{a}$
\item $\vec{a} \times \vec{b}+\vec{b} \times \vec{c}+\vec{c} \times \vec{a}=0$
\end{multicols}
\end{enumerate}
\item Let the vectors $\vec{a},\vec{b},\vec{c}$ and $\vec{d}$ be such that $\brak{\vec{a} \times \vec{b}} \times \brak{\vec{c} \times \vec{d}} = 0$. Let $A$ and $B$ be planes determined by the pairs of vectors $\vec{a},\vec{b}$ and $\vec{c},\vec{d}$ respectively. Then the angle between $A$ and $B$ is 
\hfill (2000S)
\begin{enumerate}
\begin{multicols}{4}
 \item $0$
 \item $\frac{\pi}{4}$
 \item $\frac{\pi}{3}$
 \item $\frac{\pi}{2}$
\end{multicols}
\end{enumerate}
\item If $\vec{a},\vec{b}$ and $\vec{c}$ are unit coplanar vectors, then the scalar triple product $\sbrak{2\vec{a}-\vec{b},2\vec{b}-\vec{c},2\vec{c}-\vec{a}} = $
\hfill (2000S)
\begin{enumerate}
\begin{multicols}{4}
    \item $0$
    \item $1$
    \item $-\sqrt{3}$
    \item $\sqrt{3}$
\end{multicols}
\end{enumerate}
\item Let $\vec{a}= \vec{i}-\vec{k},\vec{b}=x\vec{i}+\vec{j}+\brak{1-x}\vec{k}$ and $\vec{c}= y\vec{i}+x\vec{j}+\brak{1+x-y}\vec{k}$. Then $\sbrak{\vec{a} \ \vec{b} \ \vec{c}}$ depends on 
\hfill (2001S)
\begin{enumerate}
\begin{multicols}{1}
\item only $x$
\item only $y$
\item Neither $x$ Nor $y$
\item both $x$ and $y$
\end{multicols}
\end{enumerate}
\item If $\vec{a},\vec{b}$ and $\vec{c}$ are unit vectors, then \\
$\abs{\vec{a}-\vec{b}}^2+\abs{\vec{b}-\vec{c}}^2+\abs{\vec{a}-\vec{b}}^2$ does not exceed 
\hfill (2001S)
\begin{enumerate}
\begin{multicols}{4}
    \item $4$
    \item $9$
    \item $8$
    \item $6$
\end{multicols}
\end{enumerate}
\item If $\vec{a}$ and $\vec{b}$ are two unit vectors such that $\vec{a}+2\vec{b}$ and $5\vec{a}-4\vec{b}$ are perpendicular to each other then the angle between $\vec{a}$ and $\vec{b}$ is 
\hfill (2002S)
\begin{enumerate}
\begin{multicols}{2}
    \item $45\degree$
    \item $60\degree$
    \item $\arccos{\frac{1}{3}}$
    \item $\arccos{\frac{2}{7}}$
\end{multicols}
\end{enumerate}
\item Let $\vec{V}= 2\vec{i}+\vec{j}-\vec{k}$ and $\vec{W}= \vec{i}+3\vec{k}$. If $\vec{U}$ is a unit vector, then the maximum value of the scalar triple product $\brak{U \ V \ W}$ is 
\hfill (2002S)
\begin{enumerate}
\begin{multicols}{2}
    \item $-1$
    \item $\sqrt{10}+\sqrt{6}$
    \item $\sqrt{59}$
    \item $\sqrt{60}$
\end{multicols}
\end{enumerate}
\item The value of 'a' so that the volume of parallelopiped formed by $\vec{i}+a\vec{j}+\vec{k},\vec{j}+a\vec{k}$ and $a\vec{i}+\vec{k}$ becomes minimum is 
\hfill (2003S)
\begin{enumerate}
\begin{multicols}{4}
    \item $-3$
    \item $3$
    \item $\frac{1}{\sqrt{3}}$
    \item $\sqrt{3}$
\end{multicols}
\end{enumerate}
\item If $\vec{a}=\vec{i}+\vec{j}+\vec{k}$, $\vec{a} \cdot \vec{b}=1$ and $\vec{a} \times \vec{b} = \vec{j} - \vec{k}$. Then $\vec{b}$ is 
\hfill (2004S)
\begin{enumerate}
\begin{multicols}{2}
    \item $\vec{i}-\vec{j}+\vec{k}$
    \item $2\vec{j}-\vec{k}$
    \item $\vec{i}$
    \item $2\vec{i}$
\end{multicols}
\end{enumerate}
    \item The unit vector which is orthogonal to the vector $3\hat{i} + 2\hat{j} + 6\hat{k}$ and is coplanar with vectors $2\hat{i} + \hat{j} + \hat{k}$ \text{ and } $\hat{i} - \hat{j} + \hat{k}$ is 
    \hfill{(2004S)}
    \begin{multicols}{2}
    	\begin{enumerate}
    		\item $\frac{2\hat{i} - 6\hat{j} + \hat{k}}{\sqrt{41}}$
    		\item $\frac{2\hat{i} - 3\hat{j}}{\sqrt{13}}$
    		\item $\frac{3\hat{i} - \hat{k}}{\sqrt{10}}$
    		\item $\frac{4\hat{i} + 3\hat{j} - 3\hat{k}}{\sqrt{34}}$
    	\end{enumerate}
    \end{multicols}
    \item A variable plane at a distance of one unit from the origin cuts the coordinate axes at $\vec{A}, \vec{B} \text{ and } \vec{C}$. If the centroid $\vec{D} \brak{x,y,z}$ of triangle ABC satisfies the relation $\frac{1}{x^{2}} + \frac{1}{y^{2}} + \frac{1}{z^{2}} = k$, then the value $k$ is
    \hfill{(2005S)}
    \begin{multicols}{2}
    	\begin{enumerate}
    		\item $3$
    		\item $1$
    		\item $\frac{1}{3}$
    		\item $9$
    	\end{enumerate}
    \end{multicols}
    \item If $\vec{a}, \vec{b}, \vec{c}$ are three non-zero, non-coplanar vectors and $\vec{b_1} = \vec{b} - \frac{\vec{b} \cdot \vec{a}}{|\vec{a}^2|} \vec{a}, \vec{b_2} = \vec{b} + \frac{\vec{b} \cdot \vec{a}}{|\vec{a}^2|} \vec{a}, \vec{c_1} = \vec{c} - \frac{\vec{c} \cdot \vec{a}}{|\vec{a}^2|} \vec{a} + \frac{\vec{b} \cdot \vec{c}}{|\vec{c}^2|} \vec{b_1}, \vec{c_2} = \vec{c} - \frac{\vec{c} \cdot \vec{a}}{|\vec{a}^2|} \vec{a} + \frac{\vec{b_1} \cdot \vec{c}}{|\vec{b_1}^2|} \vec{b_1}, \vec{c_3} = \vec{c} - \frac{\vec{c} \cdot \vec{a}}{|\vec{c}^2|} \vec{a} + \frac{\vec{b} \cdot \vec{c}}{|\vec{c}^2|} \vec{b_1}, \vec{c_4} = \vec{c} - \frac{\vec{c} \cdot \vec{a}}{|\vec{c}^2|} \vec{a} = \frac{\vec{b} \cdot \vec{c}}{|\vec{b}^2|} \vec{b_1}$, then the set of orthogonal vectors is 
    \hfill{(2005S)}
    \begin{multicols}{2}
    	\begin{enumerate}
    		\item $\brak{\vec{a}, \vec{b_1}, \vec{c_3}}$
    		\item $\brak{\vec{a}, \vec{b_1}, \vec{c_2}}$
    		\item $\brak{\vec{a}, \vec{b_1}, \vec{c_1}}$
    		\item $\brak{\vec{a}, \vec{b_2}, \vec{c_2}}$
    	\end{enumerate}
    \end{multicols}
    \item Let $\vec{a} = \hat{i} + 2\hat{j} + \hat{k}, \vec{b} = \hat{i}-\hat{j}+\hat{k}$ \text{ and } $\vec{c}= \hat{i}+\hat{j}-\hat{k}$. A vector in the plane of $\vec{a}$ \text{ and } $\vec{b}$ whose projection on $\vec{c}$ is $\frac{1}{\sqrt{3}}$, is
    \hfill{(2006-3M,-1)}
    \begin{enumerate}
    	\item $4\hat{i} - \hat{j} + 4\hat{k}$
    	\item $3\hat{i} + \hat{j} - 3\hat{k}$
    	\item $2\hat{i} + \hat{j} - 2\hat{k}$
    	\item $4\hat{i} + \hat{j} - 4\hat{k}$
    \end{enumerate}
    \item The number of distinct real values of $\lambda$, for which the vectors $-\lambda^{2}\hat{i} + \hat{j} + \hat{k}$, $\hat{i} - \lambda^{2}\hat{j} + \hat{k}$ \text{ and } $\hat{i} + \hat{j} - \lambda^{2}\hat{k}$ are coplanar, is
    \hfill{(2007 - 3marks)}
    \begin{multicols}{2} 
    	\begin{enumerate}
    		\item $1$
    		\item $2$
    		\item $3$
    		\item $4$
    	\end{enumerate}
    \end{multicols}
    \item let $\vec{a},\vec{b},\vec{c}$ be unit vectors such that $\vec{a}+\vec{b}+\vec{c}=\vec{0}$. Which of the following are correct?
    \hfill{(2007- 3marks)}
    \begin{enumerate}
    	\item $\vec{a} \times \vec{b} = \vec{b} \times \vec{c} = \vec{c} \times \vec{a} = \vec{0}$
    	\item $\vec{a} \times \vec{b} = \vec{b} \times \vec{c} = \vec{c} \times \vec{a} \neq \vec{0}$
    	\item $\vec{a} \times \vec{b} = \vec{b} \times \vec{c} = \vec{a} \times \vec{c} \neq \vec{0}$
    	\item $\vec{a} \times \vec{b}, \vec{b} \times \vec{c}, \vec{c} \times \vec{a}$ are mutually perpendicular
    \end{enumerate}
    \item The edges of a parallelopiped are of unit length and are parallel to non-coplanar unit vectors $\hat{a},\hat{b},\hat{c}$ such that $\hat{a} \cdot \hat{b}= \hat{b} \cdot \hat{c}= \hat{c} \cdot \hat{a}= \frac{1}{2}$. Then, the volume of the parallelopiped is 
    \hfill{(2008)}
    \begin{multicols}{2}
    	\begin{enumerate}
    		\item $\frac{1}{\sqrt{2}}$
    		\item $\frac{1}{2\sqrt{2}}$
    		\item $\frac{\sqrt{3}}{2}$
    		\item $\frac{1}{\sqrt{3}}$
    	\end{enumerate}
    \end{multicols}
    \item Let two non-collinear unit vectors $\hat{a}$ and $\hat{b}$ form an acute angle. A point $\vec{P}$ moves so that at any time $t$ the position vector $\overrightarrow{OP}$ (where $O$ is the origin) is given by $\hat{a}\cos{t} + \hat{b}\sin{t}$. When $\vec{P}$ is farthest from origin $\vec{O}$, let $\vec{M}$ be the length of $\overrightarrow{OP}$ and $\hat{u}$ be the unit vector along $\overrightarrow{OP}$. Then,
    \hfill{(2008)}
    \begin{enumerate}
    	\item $\hat{u} = \frac{\hat{a}+\hat{b}}{|\hat{a}+\hat{b}|} \text{ and } M = (1+\hat{a} \cdot \hat{b})^{1/2}$
    	\item $\hat{u} = \frac{\hat{a}-\hat{b}}{|\hat{a}-\hat{b}|} \text{ and } M = (1+\hat{a} \cdot \hat{b})^{1/2}$
    	\item $\hat{u} = \frac{\hat{a}+\hat{b}}{|\hat{a}+\hat{b}|} \text{ and } M = (1+2\hat{a} \cdot \hat{b})^{1/2}$
    	\item $\hat{u} = \frac{\hat{a}-\hat{b}}{|\hat{a}-\hat{b}|} \text{ and } M = (1+2\hat{a} \cdot \hat{b})^{1/2}$
    \end{enumerate}
    \item If $\vec{a}, \vec{b}, \vec{c},\text{ and } \vec{d}$ are unit vectors such that $(\vec{a} \times \vec{b}) \cdot (\vec{c} \times \vec{d}) = 1 \text{ and } \vec{a} \cdot \vec{c} = \frac{1}{2}$, then
    \hfill{(2009)}
    \begin{enumerate}
    	\item $\vec{a}, \vec{b}, \vec{c}$ are non-coplanar
    	\item $\vec{b}, \vec{c}, \vec{d}$ are non-coplanar
    	\item $\vec{b}, \vec{d}$ are non-parallel
    	\item $\vec{a}, \vec{d}$ are parallel and $\vec{b}, \vec{c}$ are parallel 
    \end{enumerate}
    \item Let $\vec{P}, \vec{Q}, \vec{R} and \vec{S}$ be the points on the plane with position vectors $-2\hat{i} -\hat{j},4\hat{i},3\hat{i}+3\hat{j} and -3\hat{i}+2\hat{j}$ respectively. The quadrilateral $PQRS$ must be a 
    \hfill{(2010)}
    \begin{enumerate}
    	\item parallelogram, which is neither a rhombus nor a rectangle 
    	\item square 
    	\item rectangle, but not a square
    	\item rhombus, but not a square 
    \end{enumerate}
	\item %41
		Two adjacent sides of a parallelogram ABCD are given by $AB = 2\hat{i}+10\hat{j}+11\hat{k}$ and $AD = \hat{i}+2\hat{j}+2\hat{k}$ \\
		The side AD is rotated by an acute angle $\alpha$ in the plane of the parallelogram so that AD becomes AD$^{\prime}$. If AD$^{\prime}$ makes a right angle with the side AB, then the cosine of the angle $\alpha$ is given by \hfill{\brak{2010}}\\
\begin{enumerate}
	\item $\frac{8}{9}$
	\item $\frac{\sqrt{17}}{9}$
	\item $\frac{1}{9}$
	\item $\frac{4\sqrt{5}}{9}$\\
\end{enumerate}
        \item %42
		Let $\vec{a}=\hat{i}+\hat{j}+\hat{k}$, $\vec{b}=\hat{i}-\hat{j}+\hat{k}$ and $\vec{c}=\hat{i}-\hat{j}-\hat{k}$ be three vectors. A vector $\vec{v}$ in the plane of $\vec{a}$ and $\vec{b}$ , whose projection on $\vec{c}$ is $\frac{1}{\sqrt{3}}$ , is given by \hfill{\brak{2011}}\\
\begin{enumerate}
	\item $\hat{i}-3\hat{j}+3\hat{k}$
	\item $-3\hat{i}-3\hat{j}-\hat{k}$
	\item $3\hat{i}-\hat{j}+3\hat{k}$
	\item $\hat{i}+3\hat{j}-3\hat{k}$\\
\end{enumerate}

         \item %45 
		 If $\vec{a}$ and $\vec{b}$ are vectors such that $\abs{\vec{a}+\vec{b}}$=$\sqrt{29}$ and $\vec{a}\times\brak{2\hat{i}+3\hat{j}+4\hat{k}}$ = $\brak{2\hat{i}+3\hat{j}+4\hat{k}}\times\vec{b}$, then a possible value of $\brak{\vec{a}+\vec{b}}\cdot\brak{-7\hat{i}+2\hat{j}+3\hat{k}}$ is \hfill{\brak{2012}}\\
\begin{enumerate}
        \item $0$                             
        \item $3$                           
        \item $4$            
        \item $8$\\          
\end{enumerate}
         \item %48 
		 Let $\vec{O}$ be the origin and let PQR be an arbitrary triangle. The point $\vec{S}$ is such that $OP\cdot OQ$+$OR\cdot OS$=$OR\cdot OP$+$OQ\cdot OS$=$OQ\cdot OR$+$OP\cdot OS$\\
Then the triangle PQR has $\vec{S}$ as its \hfill{\brak{JEE Adv. 2017}}\\
\begin{enumerate}
        \item Centroid                             
        \item Circumcentre                           
        \item Incentre            
        \item Orthocenter\\          
\end{enumerate}
\item From a point $\vec{O}$ inside the triangle ABC, perpendiculars OD,OE,OF are drawn to the sides BC,CA,AB respectively. Prove that the perpendiculars from $\vec{A,B,C}$ to the sides EF,FD,DE are concurrent. \hfill{\brak{1978}}\\

	\item $A_1,A_2,......A_n$ are the vertices of a regular plane polygon with n sides and $\vec{O}$ is its centre. Show that
	$\sum_{i=1}^{n-1}\brak{OA_i\times OA_{i+1}}= \brak{1-n}\brak{OA_2\times OA_1}$
		\hfill{\brak{1982-2Marks}}\\

	\item Find all values of $\lambda$ such that $x,y,z\neq\brak{0,0,0}$ and $\brak{\hat{i}+\hat{j}+3\hat{k}}x+\brak{3\hat{i}-3\hat{j}+\hat{k}}y+\brak{-4\hat{i}+5\hat{j}}z=\lambda\brak{x\hat{i}+y\hat{j}+z\hat{k}}$ where $\hat{i},\hat{j},\hat{k}$ are unit vectors along the coordinate axes. \hfill{\brak{1982-3Marks}}\\
	\item A vector $\vec{A}$ has components $A_1,A_2,A_3$ in a right-handed rectangular Cartesian coordinate system oxyz. The coordinate system is rotated about the x-axis through an angle $\frac{\pi}{2}$. Find the components of A in the new coordinate system, in terms of $A_1,A_2,A_3$. \hfill{\brak{1983-2Marks}}\\

	\item The position vectors of the points $\vec{A,B,C\ \text{and}\ D}$ are $\brak{3\hat{i}-2\hat{j}-\hat{k}}$,$\brak{2\hat{i}+3\hat{j}-4\hat{k}}$,$\brak{-\hat{i}+\hat{j}+2\hat{k}}$ and $\brak{4\hat{i}+5\hat{j}+\lambda\hat{k}}$, respectively. If the points $\vec{A,B,C\ \text{and}\ D}$ lie on a plane, find the value of $\lambda$. \hfill{\brak{1986-2.5Marks}}\\

	\item If $\vec{A,B,C,D}$ are any four points in space, prove that-
		$\abs{AB\times CD+ BC\times AD+CA\times BD}=$\\ $4$\brak{area\ of\ triangle\ ABC}  \hfill{\brak{1987-2Marks}}\\


	\item Let OACB be a parallelogram with $\vec{O}$ at the origin and OC a diagonal. Let $\vec{D}$ be the midpoint of OA. Using vector methods prove that BD and CO intersect in the same ratio. Determine this ratio. \hfill{\brak{1988-3Marks}}\\ 

\item If vectors $\vec{a},\vec{b},\vec{c}$ are coplanar, show that
	$
		\mydet {
\vec{a} & \vec{b} & \vec{c} \\
\vec{a}\cdot\vec{a} & \vec{a}\cdot\vec{b} & \vec{a}\cdot\vec{c} \\
\vec{b}\cdot\vec{a} & \vec{b}\cdot\vec{b} & \vec{b}\cdot\vec{c}
 }=\vec{0}
		$ \hfill{\brak{1989-2Marks}}\\

	\item In a triangle OAB, $\vec{E}$ is the midpoint of BO and $\vec{D}$ is a point on AB such that AD:DB=2:1. If OD and AE intersect at $\vec{P}$, determine the ratio OP:PD  using vector methods. \hfill{\brak{1989-4 Marks}}\\

	\item Let $\vec{A}=2\hat{i}+\hat{k},\vec{B}=\hat{i}+\hat{j}+\hat{k}$, and $\vec{C}=4\hat{i}-3\hat{j}+7\hat{k}$. Determine a vector $\vec{R}$ satisfying $\vec{R\times B}=\vec{C\times B}$ and $\vec{R}\cdot\vec{A}=0$ \hfill{\brak{1990-3Marks}}\\

	\item Determine the value of 'c' so that for all real $x$, the vector $cx\hat{i}-6\hat{j}-3\hat{k}$ and $x\hat{i}+2\hat{j}+2cx\hat{k}$ make an obtuse angle with each other. \hfill{\brak{1991-4Marks}}\\

	\item In a triangle ABC,$\vec{D}$ and $\vec{E}$ are points on BC and AC respectively, such that $BD=2DC$ and $AE=3EC$. Let $\vec{P}$ be the point of intersection of AD and BE. Find BP/PE using vector methods. \hfill{\brak{1993-5Marks}}\\

	\item If the vectors $\vec{b,c,d}$ are not coplanar, then prove that the vector $\vec{\brak{a\times b}\times\brak{c\times d}}+\vec{\brak{a\times c}\times\brak{d\times b}}+\vec{\brak{a\times d}\times\brak{b\times c}}$ is parallel to $\vec{a}$. \hfill{\brak{1994-4Marks}}\\

	\item The position vectors of the vertices $\vec{A,B\ \text{and}\ C}$ of a tetrahedron ABCD are $\vec{\hat{i}+\hat{j}+\hat{k},\hat{i}}$ and $\vec{3\hat{i}}$ respectively. The altitude from vertex $\vec{D}$ to the opposite face ABC meets the median line through $\vec{A}$ of the triangle ABC at a point $\vec{E}$. If the length of the side AD is $4$ and the volume of the tetrahedron is$\frac{2\sqrt{2}}{3}$, find the position vector of the point $\vec{E}$ for all its possible positions. \hfill{\brak{1996-5Marks}}

\item Let $\vec{A}$, $\vec{B}$ and $\vec{C}$ be unit vectors suppose that $\vec{A}.\vec{B} = \vec{A}.\vec{C}=0$, and that the angle between $\vec{B}$ and $\vec{C}$ is $\frac{\pi}{6}$. Then $\vec{A}=\pm2\brak{\vec{B}\times\vec{C}}$.
\hfill (1981 - 2 Marks)

\item If $\vec{X}.\vec{A}=0$ , $\vec{X}.\vec{B}=0$, $\vec{X}.\vec{C}=0$ for some non-zero vector $\vec{X}$, then $\sbrak{\vec{A}\ \vec{B}\ \vec{C}}=0$\\
\hfill (1983 - 1 Mark)

\item The points with position vectors $\vec{a+b}$, $\vec{a-b}$ and $\vec{a+kb}$ are collinear for all real values of $\vec{k}$.
\hfill (1984 - 1 Mark)

\item For any three vectors $\vec{a}, \vec{b}$ and $\vec{c}$, $\brak{\vec{a}-\vec{b}}.\brak{{\brak{\vec{b}-\vec{c)}\times(\vec{c}-\vec{a}}}} = 2\vec{a}.(\vec{b}\times\vec{c}).$\\
\hfill (1989 - 1 Mark)
\item The points $\brak{-a,-b}$, $\brak{0,0}$, $\brak{a,b}$ and $\brak{a^2,ab}$\hfill{$\brak{1979}$}
\begin{enumerate}
      \item Collinear
    \item  Vertices of a parallelogram
    \item Vertices of a rectangle
    \item None of these
\end{enumerate}
\item  The point $\brak{4,1}$ undergoes the following three transformations successively.
\hfill{$\brak{1980}$}
\begin{enumerate}
    \item Reflection about the line $y=x$.
    \item Translation through a distance $2$ units along the positive direction of x-axis.
    \item Rotation through an angle $\frac{\pi}{4}$ about the origin in the counter clockwise direction.
\end{enumerate}
Then the final position of the point is given by the coordinates.
\begin{enumerate} 
    \item $\brak{\frac{1}{\sqrt{2}},\frac{7}{\sqrt{2}}}$
    \item $\brak{-\sqrt{2},7\sqrt{2}}$
    \item $\brak{\sqrt{2},7\sqrt{2}}$
    \item $\brak{-\frac{1}{\sqrt{2}},\frac{7}{\sqrt{2}}}$
\end{enumerate}
\end{enumerate}
