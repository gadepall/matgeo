\iffalse
  \title{Assignment 1}
  \author{Srihaas Gunda}
  \section{paragraph}
\fi



\item $ABCD$ is a square of side length $2$ units.$C_1$ is the circle touching all the sides of the square $ABCD$ and $C_2$ is the $circumcircle$ of square $ABCD$.L is a fixed line in same plane and R is a fixed point.\\
\begin{enumerate}
\item If $P$ is any point of $C_1$ and $Q$ is another point on $C_2$,then $\frac{PA^2+PB^2+PC^2+PD^2}{QA^2+QB^2+QC^2+QD^2}$

\hfill(2006-5M,-2)
\begin{enumerate}
\item $0.75$
\item $1.25$
\item $1$
\item $0.5$
\end{enumerate}
\item If a circle is such that it touches the line L and the circle $C_1$ externally,such that both the circles are on the same side of the line,then locus of centre of the circle 

\hfill(2006-5M,-2)
\begin{enumerate}
\item ellipse
\item hyperbola
\item parabola
\item circle
\end{enumerate}
\item A line L'through A is drawn parallel to BD.Point $S$ moves such that its distances from the line BD and the vertex A are equal.If locus of $S$ cuts L' at $T_2$ and $T_3$ and AC at $T_1$,then area of $\Delta T_1T_2T_3$ is

\hfill(2006-5M,-2)
\begin{enumerate}
\item $1/2$ sq.units
\item $2/3$ sq.units
\item $1$ sq.units
\item $2$ sq.units
\end{enumerate}

\item A circle $C$ of radius $1$ unit is inscribed in an equilateral triangle $PQR$.The points of contact of C with sides PQ,QR,RP are D,E,F respectively.The line PQ is given by the equation $\sqrt{3}x+y-6=0$ and the point D is $\brak{3\sqrt{3}/2 , 3/2}$.Further,it is given that the origin and the centre of C are on same side of line PQ.\\

\item The equation of circle C is\hfill(2008)
\end{enumerate}
\begin{enumerate}
\item $(x-2\sqrt{3})^2 + (y-1)^2=1$
\item $(x-2\sqrt{3})^2 + (y+1/2)^2=1$
\item $(x-\sqrt{3})^2 + (y-1)^2=1$
\item $(x-\sqrt{3})^2 + (y+1)^2=1$
\end{enumerate}

