\iffalse                
\title{8.circle}
\author{EE24BTECH11027-satwikagv}
\section{mains}
\fi
%begin{enumerate}[start=6]
\item The lines $2x-3y=5$ and $3x-4y=7$ are diameters of a circle having area as 154 sq.units. Then the equation of the circle is
\hfill{(2003)}
\begin{enumerate}
\item $x^2+y^2-2x+2y=62$
\item $x^2+y^2+2x-2y=62$
\item $x^2+y^2+2x-2y=47$
\item $x^2+y^2-2x+2y=47$
\end{enumerate}
\item If a circle passes through the point \brak{a,b} and cuts the circle $x^2+y^2=4$ orthogonally, then the locus of its centre is
\hfill{(2004)}
\begin{enumerate}
\item $2ax-2by-\brak{a^2+b^2+4}=0$
\item $2ax+2by-\brak{a^2+b^2+4}=0$
\item $2ax-2by+\brak{a^2+b^2+4}=0$
\item $2ax+2by+\brak{a^2+b^2+4}=4$
\end{enumerate}
\item A variable circle passes through the fixed point $\vec{A}\brak{p,q}$ and touches $x$-axis. The locus of the other end of the diameter through $A$is
\hfill{(2004)}
\begin{multicols}{2}
\begin{enumerate}
\item $\brak{y-q}^2=4px$
\item $\brak{x-q}^2=4py$ 
\item $\brak{y-p}^2=4qx$
\item $\brak{x-p}^2=4qy$
\end{enumerate}
\end{multicols}
\item If the lines $2x+3y+1=0$ and $3x-y-4=0$ lie along diameter of a circle of circumference $10\pi$, then the equation of the circle is 
\hfill{(2004)}
\begin{enumerate}
\item $x^2+y^2+2x-2y-23=0$
\item $x^2+y^2-2x-2y-23=0$
\item $x^2+y^2+2x+2y-23=0$
\item $x^2+y^2-2x+2y-23=0$
\end{enumerate}
\item Intercept on the line $y=x$ by the circle $x^2+y^2-2x=0$ is $AB$. Equation of the circle on $AB$ as a diameter is 
\hfill{(2004)}
\begin{enumerate}
\item $x^2+y^2+x-y=0$
\item $x^2+y^2-x+y=0$
\item $x^2+y^2+x+y=0$
\item $x^2+y^2-x-y=0$
\end{enumerate}
\item If the circles $x^2+y^2+2ax+cy+a=0$ and $x^2+y^2-3ax+dy-1=0$ intersect in two distinct points $\vec{P}$ and $\vec{Q}$ then the line $5x+by-a=0$ passes through $\vec{P}$ and $\vec{Q}$ for
\hfill{(2005)}
\begin{enumerate}
\item exactly one value of $a$
\item no value of $a$
\item infinitely many values of $a$
\item exactly two values of $a$
\end{enumerate}
\item A circle touches the $x$-axis and also touches the circle with centre at \brak{0,3} and radius 2. The locus of the centre of the circle is
\hfill{(2005)}
\begin{multicols}{2}
\begin{enumerate}
\item an ellipse
\item a circle 
\item a hyperbola
\item a parabola
\end{enumerate}
\end{multicols}
\item If a circle passes through the point \brak{a,b} and cuts the circle $x^2+y^2=p^2$ orthogonally, then the equation of the locus of its centre is 
\hfill{(2005)}
\begin{enumerate}
\item $x^2+y^2-3ax-4by+\brak{a^2+b^2-p^2}=0$
\item $2ax+2by-\brak{a^2-b^2+p^2}=0$
\item $x^2+y^2-2ax-3by+\brak{a^2-b^2-p^2}=0$
\item $2ax+2by-\brak{a^2+b^2+p^2}=0$
\end{enumerate}
\item If the pair of lines $ax^2+2(a+b)xy+by^2=0$ lie along diameters of a circle and divide the circle into four sectors such that the area of one of the sectors is thrice the area of another sector then
\hfill{(2005)}
\begin{enumerate}
\item $3a^2-10ab+3b^2=0$
\item $3a^2-2ab+3b^2=0$
\item $3a^2+10ab+3b^2=0$
\item $3a^2+2ab+3b^2=0$
\end{enumerate}
\item If the lines $3x-4y-7=0$ and $2x-3y-5=0$ are two diameters of a circle of area $49\pi$ square units, the equation of the circle is 
\hfill{(2006)}
\begin{enumerate}
\item $x^2+y^2+2x-2y-47=0$
\item $x^2+y^2+2x-2y-62=0$
\item $x^2+y^2-2x+2y-62=0$
\item $x^2+y^2-2x+2y-47=0$
\end{enumerate}
\item Let $\vec{C}$ be the circle with centre \brak{0,0} and radius 3 units. The equation of the locus of the mid points of the chords of the circle $\vec{C}$ that subtend an angle of $\frac{2\pi}{3}$ at its centre is
\hfill{(2006)}
\begin{multicols}{2}
\begin{enumerate}
\item $x^2+y^2=\frac{3}{2}$
\item $x^2+y^2=1$
\item $x^2+y^2=\frac{27}{4}$
\item $x^2+y^2=\frac{9}{4}$
\end{enumerate}
\end{multicols}
\item Consider a family of circles which are passing through the point \brak{-1,1}, and are tangent to $x$-axis. If \brak{h,k} are the coordinate of the centre of the circles, then the set of values of $k$ is given by the interval
\hfill{(2007)}
\begin{multicols}{2}
\begin{enumerate}
\item $\frac{-1}{2} \le k \le \frac{1}{2}$
\item $k \le \frac{1}{2}$
\item $o \le k \le \frac{1}{2}$
\item $k \ge \frac{1}{2}$
\end{enumerate}
\end{multicols}
\item The point diametrically opposite to the point $\vec{P}\brak{1,0}$ on the circle $x^2+y^2+2x+2y-3=0$ is 
\hfill{(2008)}
\begin{multicols}{2}
\begin{enumerate}
\item \brak{3,-4}
\item \brak{-3,4}
\item \brak{-3,-4}
\item \brak{3,4}
\end{enumerate}
\end{multicols}
\item The differential equation of the family of circles with fixed radius 5 units and centre on the line $y=2$ is
\begin{enumerate}
\item $\brak{x-2}y^{\prime 2}=25-\brak{y-2}^2$
\item $\brak{y-2}y^{\prime 2}=25-\brak{y-2}^2$
\item $\brak{y-2}^2y^{\prime 2}=25-\brak{y-2}^2$
\item $\brak{x-2}^2y^{\prime 2}=25-(y-2)^2$
\end{enumerate}
\item If $\vec{P}$ and $\vec{Q}$ are the points of intersection of the circles $x^2+y^2+3x+7y+2p-5=0$ and $x^2+y^2+2x+2y-p^2=0$ then there is a circle passing through $\vec{P}, \vec{Q}$ and \brak{1,1} for:
\hfill{(2009)}
\begin{enumerate}
\item all except one value of $p$
\item all except two values of $p$
\item exactly one value of $p$
\item all value of $p$
\end{enumerate}
%end{enumerate}

