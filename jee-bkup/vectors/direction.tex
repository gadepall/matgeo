
\iffalse
  \title{Vector-Algebra}
  \author{KOTHAPALLI AKHIL}
  \section{fitb}
\fi
%\begin{enumerate}
    \item Let $\Vec{A}$, $\vec{B}$, $\vec{C}$ be vectors of length 3,4,5 respectively .Let $\vec{A}$ be perpendicular to $\vec{B}$+$\vec{C}$,$\vec{B}$ to $\vec{C}$+$\vec{A}$ and $\vec{C}$ to $\vec{A}$+$\vec{B}$. The the length of vector $\vec{A}$+$\vec{B}$+$\vec{C}$ is
    \hfill{(1981-2marks)}
    \item The unit vector perpendicular to the plane determined by $\vec{P}$$\$(1,-1,2)$,$\vec{Q}$$(2,0,-1)$ and $\vec{R}$$(0,2,1)$ is
    \hfill{(1983-1mark)}
    \item The area of the triangle whose vertices are $\vec{A}$$(1,-1,2)$, $\vec{B}$$(2,0,-1)$, $\vec{C}$$(3,-1,2)$ is
    \hfill{(1983-1 mark)}
    \item $\vec{A}$,$\vec{B}$,$\vec{C}$ and $\vec{D}$, are four points in a plane with position vectors $\vec{a}$,$\vec{b}$,$\vec{c}$ and $\vec{d}$ respectively such that $(\vec{a}-\vec{d})\cdot(\vec{b}-\vec{c})=(\vec{b}-\vec{d})\cdot(\vec{c}-\vec{a})=0$
    The point $\vec{D}$, then, is the \dots of the triangle ABC.
    \hfill{(1984-2 marks)}
    \item If $ 
 \begin{vmatrix}
a & a^2 & 1+a^3\\
b & b^2 & 1+b^3\\
c & c^2 & 1+c^3
\end{vmatrix}
=0$ and the vectors $\vec{A}$=(1,a,$a^2$),$\vec{B}$=(1,b,$b^2$),$\vec{C}$=(1,c,$c^2$), are co-planar, then the product $abc$=\dots
\hfill{(1985-2 marks)}
\item If $\vec{A}$$\vec{B}$$\vec{C}$ are the three non-coplanar vectors, then- $\frac{\vec{A}\cdot\vec{B}\times\vec{C}}{\vec{C}\times\vec{A}\cdot\vec{B}}+\frac{\vec{B}\cdot\vec{A}\times\vec{C}}{\vec{C}\cdot\vec{A}\times\vec{B}}=$
\hfill{(1985-2 marks)}
\item $\vec{A}=(1,1,1)$, $\vec{C}=(0,1,-1)$ are given vectors, then a vector $\vec{B}$ satisfying the given equations $\vec{A}\times\vec{B}=\vec{C}$ and $\vec{A}\cdot\vec{B}=3$ \dots
\hfill{1985-2 marks}
\item If the vectors $a\hat{i}+\hat{j}+\hat{k}$,$\hat{i}+b\hat{j}+\hat{k}$ and $\hat{i}+\hat{j}+c\hat{k}$ $(a\neq b\neq c\neq 1)$ are co-planar, then the value of the $\frac{1}{(1-a)}$+$\frac{1}{(1-b)}$+$\frac{1}{(1-c)}$=\dots
\hfill{(1987-2 marks)}
\item Let $b=4\hat{i}+3\hat{j}$ and $\vec{c}$ be two vectors perpendicular to each other in the xy-plane.All vectors in the same plane having projections 1 and 2 along $\vec{b}$ and $\vec{c}$, respectively, are given by \dots
\hfill{(1987-2 marks)}
\item The components of a vectors $\vec{a}$ along and  perpendicular to a non-zero vector $\vec{b}$ are \dots and \dots respectively.
\hfill{(1988-2 marks)}
\item Given that $\vec{a}=(1,1,1)$, $\vec{c}=(0,1,-1)$,$\vec{a}\cdot\vec{b}=3$ and $\vec{a}\times\vec{b}=\vec{c}$, then $\vec{b}=$ \dots
\hfill{(1991-2 marks)}
\item A unit vector coplanar with $\hat{i}+\hat{j}+2\hat{k}$ and $\hat{i}+2\hat{j}+\hat{k}$ and perpendicular to $\hat{i}+\hat{j}+\hat{k}$ is \dots
\hfill{(1992-2 marks)}
\item A unit vector perpendicular to the plane determined by the points $\vec{P}$$(1,-1,2)$,$\vec{Q}2,0,-1)$ and $\vec{R}$(0,2,1) is \dots
\hfill{(1994-2 marks)}
\item A non-zero vector $\vec{a}$ is parallel to the line of intersection of the plane determined by the vectors $\hat{i}$,$\hat{i}+\hat{j}$ and the plane determined by the vectors $\hat{i}-\hat{j}$,$\hat{i}+\hat{k}$.The angle between $\vec{a}$ and the vector $\hat{I}-2\hat{j}+2\hat{k}$ is \dots
\hfill{(1996-2marks)}
\item If $\vec{b}$ and $\vec{c}$ are any two non-collinear unit vectors and $\vec{a}$ is any vector, then $(\vec{a}\cdot\vec{b})\vec{b}+(\vec{a}\cdot\vec{c})\vec{c}+\frac{\vec{a}\cdot(\vec{b}\times\vec{c})}{\abs{\vec{b}\times\vec{c}}}(\vec{b}\times\vec{c})$=\dots

%\end{enumerate}

% \end{enumerate}
