\iffalse
\title{Assignment}
\author{Y.Harsha Vardhan Reddy}
\section{mains}
\fi
% \begin{enumerate}
    \item Let $k$ be an integer such that triangle with vertices $\brak{k,-3k}$, $\brak{5,k}$ and $\brak{-k,2}$ has area $28 sq. units$. Then the orthocentre of the triangle is at the point :
    \hfill{[JEE M 2017]}
    \begin{enumerate}
    \begin{multicols}{2}
        \item \brak{2,\frac{1}{2}}
        \columnbreak
        \item \brak{2,-\frac{1}{2}}
        \end{multicols}
        \begin{multicols}{2}
        \item \brak{1,\frac{3}{4}}
        \item \brak{1,-\frac{3}{4}}
        \end{multicols}
    \end{enumerate}
    \item Let $\omega$ be a complex number such that $2\omega + 1=z$ where $z=\sqrt{-3}$. If \begin{align}\mydet{
1 & 1  & 1 \\
1 & -\omega^2-1 & \omega^2 \\
1 & \omega^2 & \omega^7 
}=3k \end{align}, then k is equal to:
\hfill{[JEE M 2017]}
\begin{enumerate}
\begin{multicols}{2}
    \item 1 
    \columnbreak
    \item -$z$
    \end{multicols}
    \begin{multicols}{2}
    \item $z$
    \item -1
    \end{multicols}
\end{enumerate}
\item If A= $\myvec{
    2 & -3 \\
    -4 & 1
}$
    , then $adj\brak{3A^2+12A}$ is equal to:
\hfill{[JEE M 2017]}
\begin{enumerate}
\begin{multicols}{2}
    \item $\myvec{
    72 & -63 \\
    -84 & 51
}$ 
\columnbreak

    \item $\myvec{
    72 & -84 \\
    -63 & 51 
}$ \\
\end{multicols}
\begin{multicols}{2}
    \item $\myvec{
    51 & 63 \\
    84 & 72
}$ \\

    \item $\myvec{
    51 & 84 \\
    63 & 72
    
}$
\end{multicols}
\end{enumerate} 
\item If \begin{align} \mydet{
x-4 & 2x  & 2x\\
2x & x-4 & 2x \\
 2x & 2x & x-4 
} =\brak{A+Bx}\brak{x-A}^2\end{align}, then the ordered pair \brak{A,B} is equal to: 
\hfill{[JEE M 2018]}
\begin{enumerate}
\begin{multicols}{4}
    \item \brak{-4,3}
    
    \item \brak{-4,5}
    \item \brak{4,5}
    \item \brak{-4,-5}
    \end{multicols}
\end{enumerate}
\item If the system of linear equations \\
\begin{align}x+ky+3z=0 \\
3x+ky-2z=0 \\
2x+4y-3z=0 \end{align}\\
has a non-zero solution $\brak{x,y,z}$,then $\frac{xz}{y^2}$ is equal to :
\hfill{[JEE M 2018]}
\begin{enumerate}
\begin{multicols}{4}
    \item 10
    \item -30
    \item 30
    \item -10
    \end{multicols}
\end{enumerate}
\item The system of linear equations \\
\begin{align}x+y+z=2 \\
2x+3y+2z=5 \\
2x+3y+\brak{a^2-1}z=a+1 \end{align} 
\hfill{[JEE M2019-9 Jan(M)]}
\begin{enumerate}

    

    \item is consistent when $a=4$
    \item has a unique solution for $|a|= \sqrt{3}$
    \item has infinitely many solutions for $a=4$
    \item is consistent when $|a|= \sqrt{3}$
    
\end{enumerate}
\item If $A= \myvec{
    \cos{\theta} & -\sin{\theta} \\
    \sin{\theta} & \cos{\theta}
}$, then the matrix $A^{-50}$ when $\theta=\frac{\pi}{12}$, is equal to: 
\hfill{[JEE M 2019-9 Jan(M)]}
\begin{enumerate}
\begin{multicols}{2}
    \item $\myvec{
   \frac{1}{2}  & -\frac{\sqrt{3}}{2}  \\
    \frac{\sqrt{3}}{2} & \frac{1}{2}
}$ \\ 
\columnbreak
    \item $\myvec{
   \frac{\sqrt{3}}{2}  & -\frac{1}{2}  \\
    \frac{1}{2} & \frac{\sqrt{3}}{2}
}$ \\ 
\end{multicols}
\begin{multicols}{2}

    \item $\myvec{
   \frac{\sqrt{3}}{2}  & \frac{1}{2}  \\
    -\frac{1}{2} & \frac{\sqrt{3}}{2}
}$ \\

    \item $\myvec{
   \frac{1}{2}  & \frac{\sqrt{3}}{2}  \\
    -\frac{\sqrt{3}}{2} & \frac{1}{2}
}$
\end{multicols}
\end{enumerate} 
\item If \begin{align}\myvec{
    1 & 1 \\
    0 & 1
}.\myvec{
    1 & 2 \\
    0 & 1
}.\myvec{
    1 & 3 \\
    0 & 1
}\dots \myvec{
    1 & n-1 \\
    0 & 1
}=\myvec{
    1 & 78 \\
    0 & 1
}.\end{align}\\
then the inverse of $\myvec{
    1 & n \\
    0 & 1
}$ is 
\hfill{[JEE M2019-9 April(M)]} 
\begin{enumerate}
\begin{multicols}{2}
    \item $\myvec{
        1 & 0 \\
        12 & 1 
    }$
    \columnbreak
    \item $\myvec{
        1 & -13 \\
        0 & 1 
    }$
    \end{multicols}
    \begin{multicols}{2}
    \item $\myvec{
        1 & -12 \\
        0 & 1 
    }$
     \item $\myvec{
        1 & 0 \\
        13 & 1 
    }$   
    \end{multicols}
\end{enumerate}
\item Let $\alpha$ and $\beta$ be the roots of the equation $$x^2+x+1=0$$. Then for $y\ne0$ in R,\\
\begin{align}\mydet{
   y+1 & \alpha & \beta \\
    \alpha & y+\beta & 1 \\
    \beta & 1 & y+\alpha
} \end{align} is equal to : 
\hfill{[JEE M 2019-9 April(M)]} 
\begin{enumerate}
\begin{multicols}{2}
    \item $y(y^2-1)$
    \columnbreak
    \item $y(y^2-3)$
    \end{multicols}
    \begin{multicols}{2}
    \item $y^3$
    \item $y^3-1$
    \end{multicols}
\end{enumerate}

% \end{enumerate}
