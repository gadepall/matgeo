\iffalse
\title{Matrices and Determinants}
\author{EE24BTECH11002 - AGAMJOT SINGH} 
\section{mains}
\fi

\newcommand{\adj}[1]{$adj\brak{#1}$}

    %Question20
    \item Let $a,b,c$ be such that $b\brak{a+c}\neq0$ if
	\begin{align*}
		\mydet{
			a & a+1 & a-1\\
			b & b+1 & b-1\\
			c & c-1 & c+1\\
		}
		+
		\mydet{
			a+1 & b+1 & c-1\\
			a-1 & b-1 & c+1\\
			\brak{-1}^{n+2}a & \brak{-1}^{n+1}b & \brak{-1}^{n}c\\
		}	
		= 0
	\end{align*},
		then the value of $n$ is:
	\hfill{\brak{2009}}
	\begin{enumerate}
                \item any even integer
		\item any odd integer 
		\item any integer 
		\item zero 
	\end{enumerate}

    %Question21
    \item The number of $3\times3$ non-singular matrices with four entries as $1$ and all other entries as $0$, is 
	\hfill{\brak{2010}}{\parfillskip0pt\par}


	\begin{enumerate}
                \item $5$ 
		\item $6$
		\item atleast $7$
		\item less than $4$ 
	\end{enumerate}

    %Question22
    \item Let $\vec{A}$ be a $2\times2$ matrix with non-zero entries and let $\vec{A}^2 = \vec{I}$, where $\vec{I}$ is $2\times2$ identity matrix. Define 
	\newline
	$Tr\brak{\vec{A}}$- sum of diagonal elements of $\vec{A}$ and
	\newline
	$\abs{\vec{A}}$ - determinant of matrix $\vec{A}$.
	\newline
	\textbf{Statement - 1:} $Tr\brak{\vec{A}} = 0$.
	\newline
	\textbf{Statement - 2:} $\abs{\vec{A}} = 1$

	\hfill{\brak{2010}}
	\begin{enumerate}
		\item Statement - 1 is true, Statement - 2 is true; Statement - 2 is \textbf{not} a correct explanation for Statement-1. 
	    	\item Statement - 1 is true, Statement - 2 is false. 
	    	\item Statement - 1 is false, Statement - 2 is true.
	    	\item Statement - 1 is true, Statement - 2 is true; Statement - 2 is a correct explanation for Statement-1. 
	\end{enumerate}

    %Question23
    \item Consider the system of linear equations;
	\begin{align*}
		x_1 + 2x_2 + x_3 &= 3\\
		2x_1 + 3x_2 + x_3 &= 3\\
		3x_1 + 5x_2 + 2x_3 &= 1
	\end{align*}
	\hfill{\brak{2010}}{\parfillskip0pt\par}
	\begin{enumerate}
		\item exactly $3$ solutions
	    	\item a unique solution
	    	\item no solution
	    	\item infinite number of solutions
	\end{enumerate}

    %Question24
    \item The number of values of k for which the linear equations $4x + ky + 2z = 0$, $kx + 4y + z = 0$ and $2x + 2y + z=0$ possess a non zero solution is 
	\hfill {\brak{2011}}{\parfillskip0pt\par}
        \begin{enumerate}
                \item $2$
                \item $1$
                \item zero
                \item $3$
        \end{enumerate}

    %Question25
    \item Let $\vec{A}$ and $\vec{B}$ be two symmetrix matrices of order $3$.
	\newline
	\textbf{Statement - 1:} $\vec{A}\brak{\vec{BA}}$ and $\brak{\vec{AB}}\vec{A}$ are symmetric matrices. 
	\newline
	\textbf{Statement - 2:} $\vec{AB}$ is symmetric matrix if matrix multiplication of $\vec{A}$ with $\vec{B}$ is commutative.

	\begin{enumerate}
		\item Statement - 1 is true, Statement - 2 is true; Statement - 2 is \textbf{not} a correct explanation for Statement-1. 
	    	\item Statement - 1 is true, Statement - 2 is false. 
	    	\item Statement - 1 is false, Statement - 2 is true.
	    	\item Statement - 1 is true, Statement - 2 is true; Statement - 2 is a correct explanation for Statement-1. 
	\end{enumerate}

    %Question26
	\item Let \begin{align*}
	\vec{A} = \myvec{ 
		1&0&0\\
		2&1&0\\
		3&2&1
	}
	\end{align*} If $\vec{u_1}$ and $\vec{u_2}$ are column matrices such that
	\begin{align*}
		\vec{Au_1} = \myvec{1\\0\\0}
	\end{align*} and 
	\begin{align*}
		\vec{Au_2} = \myvec{1\\0\\0}
	\end{align*}, then $\vec{u_1} + \vec{u_2}$ is equal to:
	\hfill{\brak{2012}}
        \begin{enumerate}
	    	\item \begin{align*} \myvec{-1\\1\\0} \end{align*} 
		\item \begin{align*} \myvec{-1\\1\\-1} \end{align*} 
		\item \begin{align*} \myvec{-1\\-1\\0} \end{align*}
		\item \begin{align*} \myvec{1\\-1\\-1} \end{align*} 
        \end{enumerate}

    %Question27
	\item Let $\vec{P}$ and $\vec{Q}$ be $3\times3$ matrices $\vec{P}\neq \vec{Q}$. If $\vec{P}^3=\vec{Q}^3$ and $\vec{P}^2\vec{Q}=\vec{Q}^2\vec{P}$ then determinant of $\brak{\vec{P}^2+\vec{Q}^2}$ is equal to
	\hfill{\brak{2012}}
        \begin{enumerate}
                \item $-2$
                \item $1$
                \item $0$
                \item $-1$
        \end{enumerate}

    %Question28
	\item If \begin{align*}
		\vec{P} = \myvec{1&\alpha&3\\
		1&3&3\\
		2&4&4}
	\end{align*} is the adjoint of a $3\times3$ matrix $\vec{A}$ and $\abs{\vec{A}} = 4$, then $\alpha$ is equal to:
	\hfill{\brak{JEE M 2014}}
        \begin{enumerate}
                \item $4$
                \item $11$
                \item $5$
                \item $0$
        \end{enumerate}

    %Question29
	\item If $\alpha,\beta\neq 0$ and $f\brak{n} = \alpha^n + \beta^n$ and
		\newline
		\begin{align*}
			\mydet{
				3 & 1+f\brak{1} & 1+f\brak{2}\\
				1+f\brak{1} & 1+f\brak{2} & 1+f\brak{3}\\
				1+f\brak{2} & 1+f\brak{3} & 1+f\brak{4}
			}
		\end{align*}

		$=K\brak{1-\alpha}^2\brak{1-\beta}^2\brak{\alpha-\beta}^2$,
		then $K$ is equal to

	\hfill{\brak{JEE M 2014}}
        \begin{enumerate}
                \item $1$
                \item $-1$
                \item $\alpha\beta$
		\item $\frac{1}{\alpha\beta}$
        \end{enumerate}


    %Question30
	\item If $\vec{A}$ is a $3\times3$ non-singular matrix such that $\vec{A}\vec{A}^{\prime}=\vec{A}^{\prime}\vec{A}$ and $\vec{B}=\vec{A}^{-1}\vec{A}^{\prime}$, then $\vec{B}\vec{B}^{\prime}$ equals:
	\hfill {\brak{JEE M 2014}}{\parfillskip0pt\par}
	\begin{enumerate}
	    	\item $\vec{B}^{-1}$
		\item $\brak{\vec{B}^{-1}}^{\prime}$
		\item $\vec{I}+\vec{B}$ 
		\item $\vec{I}$
        \end{enumerate}


    %Question31
    \item The set of all values of $\lambda$ for which the system of linear equations:
	\begin{align*}
		2x_1-2x_2+x_3 &= \lambda x_1\\
		2x_1-3x_2+2x_3 &= \lambda x_2\\
		-x_1+2x_2 &= \lambda x_3
	\end{align*}
	has a non-trivial solution

	\hfill{\brak{JEE M 2015}}
	\begin{enumerate}
		\item contains two elements
		\item contains more than two elements
		\item is an empty set
		\item is a singleton
	\end{enumerate}


    %Question32
	\item If \begin{align*} \vec{A} = \myvec{1&2&2\\2&1&-2\\a&2&b} \end{align*} is a matrix satisfying the equation $\vec{A}\vec{A}^T = 9\vec{I}$, where $\vec{I}$ is $3\times3$ identity matrix, then the ordered part $\brak{a,b}$ is equal to:
	\hfill {\brak{JEE M 2015}}{\parfillskip0pt\par}
	\begin{enumerate}
	    	\item $\brak{2,1}$ 
	    	\item $\brak{-2,-1}$ 
	    	\item $\brak{2,-1}$ 
	   	\item $\brak{-2,1}$ 
	\end{enumerate}


    %Question33
	\item The system of linear equations 
	\begin{align*}
		x+\lambda y-z &= 0\\
		\lambda x-y-z &= 0\\
		x+y-\lambda z &= 0
	\end{align*}
	has a non-trivial solution for:

	\hfill{\brak{JEE M 2016}}
	\begin{enumerate}
		\item exactly two values of $\lambda$ 
		\item exactly three values of $\lambda$ 
		\item inifinitely many values of $\lambda$
		\item exactly one value of $\lambda$ 
	\end{enumerate}


    %Question34
\item If \begin{align*} \vec{A} = \myvec{5a&-b\\3&2}\end{align*} and $\vec{A} \adj{\vec{A}} = \vec{A}\vec{A}^{T}$, then $5a + b$ is equal to: 
	\hfill{\brak{JEE M 2016}}
	\begin{enumerate}
	    	\item $4$ 
	    	\item $13$
	    	\item $-1$
	   	\item $5$ 
	\end{enumerate}

    %Question35
	\item Let k be an integer such that triangle with vertices $\brak{k, -3k}$, $\brak{5,k}$, $\brak{-k,2}$ has area $28$ sq. units. Then the orthocentre of this triangle is at the point:
	\hfill{\brak{JEE M 2017}}
	\begin{enumerate}
	    	\item $\brak{2,\frac{1}{2}}$ 
	    	\item $\brak{2,\frac{-1}{2}}$ 
	     	\item $\brak{1,\frac{3}{4}}$ 
	    	\item $\brak{1,\frac{-3}{4}}$ 
	\end{enumerate}


    %Question36
    \item Let $\omega$ be a complex number such that $2\omega + 1 = z$ where $z = \sqrt{-3}$. If
	\begin{align*} \mydet{1&1&1\\1&-\omega^2-1&\omega^2\\1&\omega^2&\omega^7} = 3k \end{align*}, then $k$ is equal to:
	\hfill{\brak{JEE M 2017}}
	\begin{enumerate}[label={(\alph*)}]
	    	\item $1$ 
	    	\item $-z$
	     	\item $z$
	    	\item $-1$
	\end{enumerate}
