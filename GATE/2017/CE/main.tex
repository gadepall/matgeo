    \item The matrix $\vec{P}$ is the inverse of a matrix $\vec{Q}$. If $\vec{I}$ denotes the identity matrix, which one of the following options is correct?\hfill (CE 2017)
    \begin{multicols}{2}
    \begin{enumerate}
        \item $\vec{P}\vec{Q} = \vec{I}$ but $\vec{Q}\vec{P} \neq \vec{I}$
        \item $\vec{Q}\vec{P} = \vec{I}$ but $\vec{P}\vec{Q} \neq \vec{I}$
        \item $\vec{P}\vec{Q} = \vec{I}$ and $\vec{Q}\vec{P} = \vec{I}$
        \item $\vec{P}\vec{Q} - \vec{Q}\vec{P} = \vec{I}$
    \end{enumerate}
    \end{multicols}
    \item Consider the matrix \myvec{ 5 & -1 \\ 4 & 1 }. Which one of the following statements is TRUE for the eigenvalues and eigenvectors of this matrix? \hfill (CE 2017)
    \begin{enumerate}
        \item Eigenvalue 3 has a multiplicity of 2, and only one independent eigenvector exists.
        \item Eigenvalue 3 has a multiplicity of 2, and two independent eigenvectors exist.
        \item Eigenvalue 3 has a multiplicity of 2, and no independent eigenvector exists.
        \item Eigenvalues are 3 and -3, and two independent eigenvectors exist.
    \end{enumerate}
    \item Consider the following simultaneous equations (with $c_1$ and $c_2$ being constants)
    \begin{align*}
3x_1 + 2x_2 &= c_1
\\
4x_1 + x_2  &= c_2
    \end{align*}
    The characteristic equation for these simultaneous equations is
    \hfill (CE 2017)
    \begin{multicols}{2}
    \begin{enumerate}
        \item $\lambda^2 - 4\lambda - 5 = 0$  
        \item $\lambda^2 - 4\lambda + 5 = 0$  
        \item $\lambda^2 + 4\lambda - 5 = 0$  
        \item $\lambda^2 + 4\lambda + 5 = 0$  
    \end{enumerate}
    \end{multicols}
    \item If $ \vec{A} = \myvec{1 & 5 \\ 6 & 2} $ and $ \vec{B} = \myvec{ 3 & 7 \\ 8 & 4} $, then $ \vec{A}\vec{B}^{\top} $ is equal to
    \begin{multicols}{4}
    \begin{enumerate}
        \item $\myvec{ 38 & 28 \\ 32 & 56 }$  
        \item $\myvec{ 3 & 40 \\ 42 & 8 }$  
        \item $\myvec{ 43 & 27 \\ 34 & 50 }$  
        \item $\myvec{ 38 & 32 \\ 28 & 56 }$  
    \end{enumerate}
    \end{multicols}
   \hfill (CE 2017)
