\item
Consider the vector space $V = \{a_0 + a_1x + a_2x^2 : a_i \in \mathbb{R} \text{ for } i=0,1,2\}$ of polynomials of degree at most 2. Let $f: V \to \mathbb{R}$ be a linear functional such that $f(1+x)=0$, $f(1-x^2)=0$ and $f(x^2-x)=2$. Then $f(1+x+x^2)$ equals \rule{1cm}{0.01pt}.
\hfill (MA 2017)
\item
Let $\vec{A}$ be a $7 \times 7$ matrix such that $2\vec{A}^2 - \vec{A}^4 = \vec{I}$, where $\vec{I}$ is the identity matrix. If $\vec{A}$ has two distinct eigenvalues and each eigenvalue has geometric multiplicity 3, then the total number of nonzero entries in the Jordan canonical form of $\vec{A}$ equals \rule{1.5cm}{0.4pt}.
\hfill (MA 2017)
\item
If U and V are the null spaces of $\myvec {1 & 1 & 0 & 0 \\ 0 & 0 & 1 & 1 }$ and $\myvec{ 1 & 2 & 3 & 2 \\ 0 & 1 & 2 & 1 }$, respectively, then the dimension of the subspace $U+V$ equals \rule{1.5cm}{0.4pt}.
\hfill (MA 2017)
%27
\item
Given two $n \times n$ matrices $\vec{A}$ and $\vec{B}$ with entries in $\mathbb{C}$, consider the following statements
\begin{itemize}
\item[(P):] If $\vec{A}$ and $\vec{B}$ have the same minimal polynomial, then $\vec{A}$ is similar to $\vec{B}$.
\item[(Q):] If $\vec{A}$ has n distinct eigenvalues, then there exists $\vec{u} \in \mathbb{C}^n$ such that $\vec{u}, \vec{A}\vec{u}, \dots, \vec{A}^{n-1}\vec{u}$ are linearly independent.
\end{itemize}
Which of the above statements hold TRUE?
\hfill (MA 2017)
\begin{enumerate}
\begin{multicols}{4}
\item Both P and Q
\item Only P
\item Only Q
\item Neither P nor Q
\end{multicols}
\end{enumerate}
%28
\item
Let $\vec{A}=(a_{ij})$ be a $10 \times 10$ matrix such that $a_{ij}=1$ for $i \ne j$ and $a_{ii}=\alpha+1$, where $\alpha > 0$. Let $\lambda$ and $\mu$ be the largest and the smallest eigenvalues of $\vec{A}$, respectively. If $\lambda + \mu = 24$, then $\alpha$ equals \rule{1.5cm}{0.4pt}.
\hfill (MA 2017)
\item
Let $\vec{J}$ be the Jacobi iteration matrix of the linear system
$\myvec{ 1 & 2 & 1 \\ 2 & 1 & 2 \\ -4 & 2 & 1  } \myvec{ x \\ y \\ z } = \myvec{ 1 \\ 2 \\ 3 }$.
Consider the following statements:
\begin{itemize}
\item[(P):] One of the eigenvalues of $\vec{J}$ lies in the interval .
\item[(Q):] The Jacobi iteration converges for the above system.
\end{itemize}
Which of the above statements hold TRUE?
\hfill (MA 2017)
\begin{enumerate}
\begin{multicols}{4}
\item Both P and Q
\item Only P
\item Only Q
\item Neither P nor Q
\end{multicols}
\end{enumerate}
