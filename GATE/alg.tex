\begin{enumerate}[label=\thesubsection.\arabic*.,ref=\thesubsection.\theenumi]
         \item %48 
		 Let $\vec{O}$ be the origin and let PQR be an arbitrary triangle. The point $\vec{S}$ is such that $OP\cdot OQ$+$OR\cdot OS$=$OR\cdot OP$+$OQ\cdot OS$=$OQ\cdot OR$+$OP\cdot OS$.
Then the triangle PQR has $\vec{S}$ as its \hfill{\brak{2017}}
\begin{enumerate}
        \item Centroid                             
        \item Circumcentre                           
        \item Incentre            
        \item Orthocenter
\end{enumerate}
\item From a point $\vec{O}$ inside the $\triangle ABC$, perpendiculars $OD, OE, OF$ are drawn to the sides $BC, CA, AB$ respectively. Prove that the perpendiculars from $\vec{A,B,C}$ to the sides $EF, FD, DE$ are concurrent. \hfill{\brak{1978}}
	\item $A_1,A_2,\dots A_n$ are the vertices of a regular plane polygon with $n$ sides and $\vec{O}$ is its centre. Show that
	$$\sum_{i=1}^{n-1}\brak{OA_i\times OA_{i+1}}= \brak{1-n}\brak{OA_2\times OA_1}$$
		\hfill{\brak{1982}}
	\item If $\vec{A,B,C,D}$ are any four points in space, prove that
		$$\abs{AB\times CD+ BC\times AD+CA\times BD}=4\text{ar}\brak{\triangle ABC}$$  \hfill{\brak{1987}}
	\item Tangent at a point $\vec{P}_{1}$ (other than $\brak{0, 0}$) on the curve $y=x^{3}$ meets the curve again at $\vec{P}_{2}$. The tangent at $\vec{P}_{2}$ meets the curve again at $\vec{P}_{1}$, and so on. Show that the abscissae of $\vec{P}_{1}, \vec{P}_{2}, \vec{P}_{3} \dots \vec{P}_{n}$, form a G.P. Also find the ratio
		$$\frac{ar\brak{\triangle P_{1}P_{2}P_{3}}}{ar\brak{\triangle P_{2}P_{3}P_{4}}}$$
	\hfill{(1993)}
\item Using co-ordinate geometry, prove that the three altitudes of any triangle are concurrent.
	\hfill{(1998)}
\item $C_{1}$ and $C_{2}$ are two concentric circles,  the radius of $C_{2}$ being twice that of $C_{1}$. From a point $\vec{P}$ on $C_{2}$,  tangents $PA$ and $PB$ are drawn to $C_{1}$. Prove that the centroid of the triangle $PAB$ lies on $C_{1}$.
	           \hfill(1998)
	\end{enumerate}
