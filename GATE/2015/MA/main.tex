\item Let $T: \mathbb{R}^4 \to \mathbb{R}^4$ be a linear map defined by  
\begin{align*}
T(x, y, z, w) = (x + z,\ 2x + y + 3z,\ 2y + 2z, w).
\end{align*}
Then the rank of $T$ is equal to \underline{\hspace{2cm}}.
\hfill\brak{\text{MA 2015}} 
\item Let $\vec{M}$ be a $3 \times 3$ matrix and suppose that $1, 2$ and $3$ are the eigenvalues of $\vec{M}$. 
If 

\hfill\brak{\text{MA 2015}} 
\begin{align*}
\vec{M}^{-1} = \frac{\vec{M}^2 - \vec{M} + \vec{I}_3}{\alpha}
\end{align*}
for some scalar $\alpha \neq 0$, then $\alpha$ is equal to \underline{\hspace{2cm}}.
\item Let $\vec{M}$ be a $3 \times 3$ singular matrix and suppose that $2$ and $3$ are eigenvalues of $\vec{M}$. Then the number of linearly independent eigenvectors of $\vec{M}^3 + 2\vec{M} + \vec{I}_3$ is equal to \underline{\hspace{2cm}}. \hfill\brak{\text{MA 2015}} 
\item Let $\vec{M}$ be a $3 \times 3$ matrix such that $\vec{M} \myvec{ -2 \\ 1 \\ 0 } = \myvec{ 6 \\ -3 \\ 0 }$ and suppose that $\vec{M}^3 \myvec{ 1 \\ -1/2 \\ 0 } = \myvec{ \alpha \\ \beta \\ \gamma }$ for some $\alpha, \beta, \gamma \in \mathbb{R}$. Then $|\alpha|$ is equal to \underline{\hspace{2cm}}.
\hfill\brak{\text{MA 2015}} 
\item Let $W = \operatorname{Span} \left\{ \frac{1}{\sqrt{2}} (0,0,1,1), \frac{1}{\sqrt{2}} (1,-1,0,0) \right\}$ be a subspace of the Euclidean space $\mathbb{R}^4$. Then the square of the distance from the point $(1,1,1,1)$ to the subspace $W$ is equal to \underline{\hspace{2cm}}
\hfill\brak{\text{MA 2015}} 
\item Let $T : \mathbb{R}^4 \to \mathbb{R}^4$ be a linear map such that the null space of $T$ is
\begin{align*}
\left\{ (x, y, z, w) \in \mathbb{R}^4 : x+y+z+w = 0 \right\}
\end{align*}
and the rank of $(T - 4I_4)$ is $3$. If the minimal polynomial of $T$ is $x(x - 4)^\alpha$, then $\alpha$ is equal to \underline{\hspace{2cm}}
\hfill\brak{\text{MA 2015}} 
\item Let $\vec{M}$ be an invertible Hermitian matrix and let $x, y \in \mathbb{R}$ be such that $x^2 < 4y$. Then

\hfill\brak{\text{MA 2015}} 
\begin{enumerate}
     \item both $\vec{M}^2 + x\vec{M} + y\vec{I}$ and $\vec{M}^2 - x\vec{M} + y\vec{I}$ are singular
     \item  $\vec{M}^2 + x\vec{M} + y\vec{I}$ is singular but $\vec{M}^2 - x\vec{M} + y\vec{I}$ is non-singular
     \item  $\vec{M}^2 + x\vec{M} + y\vec{I}$ is non-singular but $\vec{M}^2 - x\vec{M} + y\vec{I}$ is singular
     \item  both $\vec{M}^2 + x\vec{M} + y\vec{I}$ and $\vec{M}^2 - x\vec{M} + y\vec{I}$ are non-singular
\end{enumerate}
\item Let \( V \) be a closed subspace of \( L^2[0,1] \) and let \( f, g \in L^2[0,1] \) be given by \( f(x) = x \) and \( g(x) = x^2 \).  
If \( V = \text{Span}\{f\} \) and \( Pg \) is the orthogonal projection of \( g \) on \( V \), then  
\begin{align*}
(g - Pg)(x), \ x \in [0,1], \ \text{is:} 
 \end{align*}
Options:
\hfill\brak{\text{MA 2015}}  
\begin{multicols}{4}
\begin{enumerate}
  \item $\frac{3}{4}x$
  \item $\frac{1}{4}x$
  \item $\frac{3}{4}x^2$
  \item $\frac{1}{4}x^2$
\end{enumerate}
\end{multicols}

