\item In a Cartesian coordinate system the vertices of a triangular plate are given by $\brak{-2, 1}$,$\brak{3, 4}$,and $\brak{-4, -8}$. The coordinates of the centre of gravity of the plate are
\hfill\brak{\text{MN 2010}}
\begin{multicols}{4}
\begin{enumerate}
\item $\brak{3,4}$
\item $\brak{7,12}$ 
\item $\brak{-1,-1}$
\item $\brak{-3,-4}$
\end{enumerate}
\end{multicols}
\item Two determinants of order $n$ are multiplied. The order of the resultant determinant is
\hfill\brak{\text{MN 2010}}
\begin{multicols}{4}
\begin{enumerate}
\item $n$
\item $2n$
\item $n^2$
\item $n/2$
\end{enumerate}
\end{multicols}
\item A force of $\vec{F} = 50\hat{i} - 50\hat{j}$ N is moved from the origin to the coordinate $\brak{4.0, 2.0}$. The work done in the process, is
\hfill\brak{\text{MN 2010}}
\begin{multicols}{4}
\begin{enumerate}
\item $75.6$
\item $85.5$
\item $90.2$
\item $100.0$
\end{enumerate}
\end{multicols}
\item  The value of $k$ for which the points $\brak{5,5},\brak{k,1},\brak{10,7}$ lie on a straight line is

\hfill\brak{\text{MN 2010}}
\begin{multicols}{4}
\begin{enumerate}
\item $-5$
\item $+5$
\item $-2$
\item $+2$
\end{enumerate}
\end{multicols}

