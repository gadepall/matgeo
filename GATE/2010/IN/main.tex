\item A real $n \times n$ matrix $\vec{A} = [a_{ij}]$ is defined as follows:
\begin{align*}
    a_{ij}=
    \begin{cases}
	    i, &  i = j \\
	    0 & \text{ otherwise}
    \end{cases}
\end{align*}
The summation of all eigenvalues of $\vec{A}$ is
\hfill{\brak{\text{IN 2010}}}
\begin{enumerate}
\begin{multicols}{4}
\item $\frac{\brak{n+1}}{2}$
\item $\frac{\brak{n-1}}{2}$
\item $\frac{n\brak{n+1}\brak{2n+1}}{6}$
\item $n^2$
\end{multicols}
\end{enumerate}
%
\item $\vec{X}$ and $\vec{Y}$ are non-zero square matrices of size $n \times n$. If $\vec{X}\vec{Y}=0_{n \times n}$ then
\hfill{\brak{\text{IN 2010}}}\begin{enumerate} \begin{multicols}{2}
    \item $\abs{\vec{X}}=0$ and $\abs{\vec{Y}}\neq0$
    \item $\abs{\vec{X}}\neq0$ and $\abs{\vec{Y}}=0$
    \item $\abs{\vec{X}}=0$ and $\abs{\vec{Y}}=0$
    \item $\abs{\vec{X}}\neq0$ and $\abs{\vec{Y}}\neq0$
\end{multicols} \end{enumerate}


\item 4-point DFT of a real discrete-time signal $x[n]$ of length 4 is given by $X[k]$, $n=0,1,2,3$ and $k=0,1,2,3$. It is given that $X[0]=5$, $X[1]=1+j1$, $X[2]=0.5$. $X[3]$ and $x[0]$ respectively are
\hfill{\brak{\text{IN 2010}}}
\begin{enumerate} \begin{multicols}{4}
    \item $1-j$, 1.875
    \item $1-j$, 1.500
    \item $1+j$, 1.875
    \item $0.1-j0.1$, 1.500
\end{multicols} 
\end{enumerate}


