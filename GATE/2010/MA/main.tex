\item If the nullity of the matrix 
\[
\myvec{ k & 1 & 2 \\ 1 & -1 & -2 \\ 1 & 1 & 4 }
\]
is $1$, then the value of $k$ is \underline{\hspace{2cm}}.
\hfill(MA 2010)
\begin{enumerate}
\begin{multicols}{4}
\item $-1$
\item $0$
\item $1$
\item $2$
\end{multicols}
\end{enumerate}
\item If a $3\times 3$ real skew-symmetric matrix has an eigenvalue $2i$, then one of the remaining eigenvalues is \underline{\hspace{2cm}}.
\hfill(MA 2010)
\begin{enumerate}
\begin{multicols}{4}
\item $\dfrac{1}{2i}$
\item $-\dfrac{1}{2i}$
\item $0$
\item $1$
\end{multicols}
\end{enumerate}
\item Let $T:\mathbb{R}^3\to\mathbb{R}^3$ be a linear transformation defined by $T(x,y,z)=(x+y,\,y+z,\,z-x)$. Then, an orthonormal basis for the range of $T$ is \underline{\hspace{2cm}}
\hfill(MA 2010)
\begin{enumerate}
\begin{multicols}{2}
    \item $\left\{\myvec{\tfrac{1}{\sqrt{2}},\tfrac{1}{\sqrt{2}},0},\;\myvec{\tfrac{1}{\sqrt{3}},-\tfrac{1}{\sqrt{3}},\tfrac{1}{\sqrt{3}}}\right\}$

\item $\left\{\myvec{\tfrac{1}{\sqrt{2}},-\tfrac{1}{\sqrt{2}},0},\;\myvec{\tfrac{1}{\sqrt{6}},\tfrac{1}{\sqrt{6}},\tfrac{2}{\sqrt{6}}}\right\}$

\item $\left\{\myvec{\tfrac{1}{\sqrt{2}},\tfrac{1}{\sqrt{2}},0},\;\myvec{\tfrac{1}{\sqrt{6}},-\tfrac{1}{\sqrt{6}},-\tfrac{2}{\sqrt{6}}}\right\}$

\item $\left\{\myvec{\tfrac{1}{\sqrt{2}},-\tfrac{1}{\sqrt{2}},0},\;\myvec{\tfrac{1}{\sqrt{3}},\tfrac{1}{\sqrt{3}},\tfrac{1}{\sqrt{3}}}\right\}$
\end{multicols}
\end{enumerate}
\item Let $T:P_3[0,1]\to P_2[0,1]$ be defined by $(Tp)(x)=p^{\prime\prime}(x)+p^{\prime}(x)$. Then the matrix representation of $T$ with respect to the bases $\{1,x,x^2,x^3\}$ and $\{1,x,x^2\}$ of $P_3[0,1]$ and $P_2[0,1]$ respectively is \underline{\hspace{2cm}}
\hfill(MA 2010)
\begin{enumerate}
\begin{multicols}{4}
\item $\myvec{0&0&0\\1&0&0\\2&2&0\\0&6&3}$
\item $\myvec{0&1&2&0\\0&0&2&6\\0&0&0&3}$
\item $\myvec{0&2&1&0\\6&2&0&0\\3&0&0&0}$
\item $\myvec{0&0&0\\0&0&1\\0&2&2\\3&6&0}$
\end{multicols}
\end{enumerate}
\item Consider the basis $\{u_1,u_2,u_3\}$ of $\mathbb{R}^3$, where $u_1=(1,0,0)$, $u_2=(1,1,0)$, $u_3=(1,1,1)$. Let $\{f_1,f_2,f_3\}$ be the dual basis of $\{u_1,u_2,u_3\}$ and $f$ be a linear functional defined by $f(a,b,c)=a+b+c$, $(a,b,c)\in\mathbb{R}^3$. If $f=\alpha_1f_1+\alpha_2f_2+\alpha_3f_3$, then $(\alpha_1,\alpha_2,\alpha_3)$ is \underline{\hspace{2cm}}
\hfill(MA 2010)
\begin{enumerate}
\begin{multicols}{4}
\item $(1,2,3)$
\item $(1,3,2)$
\item $(2,3,1)$
\item $(3,2,1)$
\end{multicols}
\end{enumerate}

