    \item Consider a $3 \times 3$ matrix with every element being equal to $1$. Its only non-zero eigenvalue is \underline{\hspace{2cm}}.
    \hfill{\brak{\text{EE 2016}}}
    \item Let the eigenvalues of a $2 \times 2$ matrix $\vec{A}$ be $1, -2$ with eigenvectors $\vec{x}_{1}$ and $\vec{x}_{2}$ respectively. Then the eigenvalues and eigenvectors of the matrix $\vec{A}^{2}-3\vec{A}+4\vec{I}$ would, respectively, be
    \hfill{\brak{\text{EE 2016}}}
    \begin{enumerate}
        \begin{multicols}{2}
            \item $2, 14; \vec{x}_{1}, \vec{x}_{2}$
            \item $2, 14; \vec{x}_{1}+\vec{x}_{2}, \vec{x}_{1}-\vec{x}_{2}$
            \item $2, 0; \vec{x}_{1}, \vec{x}_{2}$
            \item $2, 0; \vec{x}_{1}+\vec{x}_{2}, \vec{x}_{1}-\vec{x}_{2}$
        \end{multicols}
    \end{enumerate}
    \item Let $\vec{A}$ be a $4 \times 3$ real matrix with rank $2$. Which one of the following statement is TRUE?
    \hfill{\brak{\text{EE 2016}}}
    \begin{enumerate}
        \item Rank of $\vec{A}^{\top} \vec{A}$ is less than $2$.
        \item Rank of $\vec{A}^{\top} \vec{A}$ is equal to $2$.
        \item Rank of $\vec{A}^{\top} \vec{A}$ is greater than $2$.
        \item Rank of $\vec{A}^{\top} \vec{A}$ can be any number between $1$ and $3$.
    \end{enumerate}
    \item $\vec{A}$ $3 \times 3$ matrix $\vec{P}$ is such that, $\vec{P}^3=\vec{P}$. Then the eigenvalues of $\vec{P}$ are
    \hfill{\brak{\text{EE 2016}}}
    \begin{enumerate}
        \begin{multicols}{2}
            \item $1, 1, -1$
            \item $1, 0.5 + j0.866, 0.5 - j0.866$
            \item $1, -0.5 + j0.866, -0.5 - j0.866$
            \item $0, 1, -1$
        \end{multicols}
    \end{enumerate}
    \item Let $\vec{P} = \myvec{3 & 1 \\ 1 & 3}$. Consider the set S of all vectors $\myvec{x \\ y}$ such that $a^2+b^2=1$ where $\myvec{a \\ b} = \vec{P}\myvec{x \\ y}$. Then S is
    \hfill{\brak{\text{EE 2016}}}
        \begin{multicols}{2}
    \begin{enumerate}
        \item a circle of radius $\sqrt{10}$
        \item a circle of radius $\frac{1}{\sqrt{10}}$
        \item an ellipse with major axis along $\myvec{1 \\ 1}$
        \item an ellipse with minor axis along $\myvec{1 \\ 1}$
    \end{enumerate}
        \end{multicols}
    \item Consider a linear time-invariant system $\dot{\vec{x}}=\vec{A}\vec{x}$, with initial condition $\vec{x}(0)$ at $t=0$.
Suppose $\bm{\alpha}$ and $\bm{\beta}$ are eigenvectors of a $(2\times2)$ matrix $\vec{A}$ corresponding to
distinct eigenvalues $\lambda_1$ and $\lambda_2$, respectively. Then the response $\vec{x}(t)$ of the
system due to initial condition $\vec{x}(0)=\bm{\alpha}$ is
    \hfill{\brak{\text{EE 2016}}}
        \begin{multicols}{4}
    \begin{enumerate}
            \item $e^{\lambda_1 t}\bm{\alpha}$
            \item $e^{\lambda_2 t}\bm{\beta}$
            \item $e^{\lambda_2 t}\bm{\alpha}$
            \item $e^{\lambda_1 t}\bm{\alpha} + e^{\lambda_2 t}\bm{\beta}$
    \end{enumerate}
        \end{multicols}

