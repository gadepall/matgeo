\item If $\vec{A}\vec{B} = \vec{I}$ then
  \hfill{\brak{\text{MN 2016}}}
  \begin{enumerate}
\begin{multicols}{4}
      \item $\vec{B} = \vec{A}^{\top}$
      \item $\vec{A} = \vec{B}^{\top}$
      \item $\vec{B} = \vec{A}^{-1}$
      \item $\vec{B} = \vec{A}$
    \end{multicols}
  \end{enumerate}
\item Equations of two planes are $z = 4$ and $z = 4 + 3x$. The included angle between the two planes in degrees, is \rule{1cm}{0.01pt}.
\hfill{\brak{\text{MN 2016}}}

\item A force $\vec{P} = 2\hat{i} - 5\hat{j} + 6\hat{k}$ acts on a particle. The particle is moved from point A to point B, where the position vectors of $\vec{A}$ and $\vec{B}$ are $6\hat{i} + \hat{j} - 3\hat{k}$ and $4\hat{i} - 3\hat{j} - 2\hat{k}$ respectively. The work done is \rule{1cm}{0.01pt}.
\hfill{\brak{\text{MN 2016}}}
\item The value of $x$ in the simultaneous equations is \rule{1cm}{0.01pt}.
\hfill{\brak{\text{MN 2016}}}
\begin{align*}
3x + y + 2z &= 3 \\
2x - 3y - z &= -3 \\
x + 2y + z &= 4
\end{align*}

