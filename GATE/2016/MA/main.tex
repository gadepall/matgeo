\let\negmedspace\undefined
\let\negthickspace\undefined
\documentclass[journal]{IEEEtran}
\usepackage[a5paper, margin=10mm, onecolumn]{geometry}
%\usepackage{lmodern} % Ensure lmodern is loaded for pdflatex
\usepackage{tfrupee} % Include tfrupee package
\setlength{\headheight}{1cm} % Set the height of the header box
\setlength{\headsep}{0mm}     % Set the distance between the header box and the top of the text
\usepackage{gvv-book}
\usepackage{gvv}
\usepackage{cite}
\usepackage{amsmath,amssymb,amsfonts,amsthm}
\usepackage{algorithmic}
\usepackage{graphicx}
\usepackage{textcomp}
\usepackage{xcolor}
\usepackage{txfonts}
\usepackage{listings}
\usepackage{enumitem}
\usepackage{mathtools}
\usepackage{gensymb}
\usepackage{comment}
\usepackage[breaklinks=true]{hyperref}
\usepackage{tkz-euclide}
\usepackage{multicol}
\usepackage{listings}                                        
\def\inputGnumericTable{}                                 
\usepackage[latin1]{inputenc}                                
\usepackage{color}                                            
\usepackage{array}                                            
\usepackage{longtable}                                       
\usepackage{calc}                                             
\usepackage{multirow}   
\usepackage{hhline}
\usepackage{ifthen}                                           
\usepackage{lscape}
\usepackage{circuitikz}
\renewcommand{\thefigure}{\theenumi}
\renewcommand{\thetable}{\theenumi}
\setlength{\intextsep}{10pt} % Space between text and floats
\numberwithin{equation}{enumi}
\numberwithin{figure}{enumi}
\renewcommand{\thetable}{\theenumi}

\begin{document}
\bibliographystyle{IEEEtran}

\begin{center}
    \LARGE \textbf{GATE 2016 MA}
    
    \large \textbf{EE25BTECH11001 - AARUSH DILAWRI}
\end{center}

\begin{enumerate}
\item An apple costs $10$. An onion costs $8$.

Select the most suitable sentence with respect to grammar and usage.

\begin{enumerate}

\item The price of an apple is greater than an onion.
\item The price of an apple is more than onion.
\item The price of an apple is greater than that of an onion.
\item Apples are more costlier than onions.

\end{enumerate}

\hfill{\brak{\text{GATE MA 2016}}}

\item The Buddha said, ``Holding on to anger is like \underline{grasping} a hot coal with the intent of throwing it at someone else; you are the one who gets burnt.''

Select the word below which is closest in meaning to the word underlined above.

\begin{enumerate}
\begin{multicols}{4}
\item burning
\item igniting
\item clutching
\item flinging
\end{multicols}
\end{enumerate}

\hfill{\brak{\text{GATE MA 2016}}}

\item $M$ has a son $Q$ and a daughter $R$. He has no other children. $E$ is the mother of $P$ and daughter-in-law of $M$. How is $P$ related to $M$?

\begin{enumerate}
\begin{multicols}{2}
\item $P$ is the son-in-law of $M$.
\item $P$ is the grandchild of $M$.
\item $P$ is the daughter-in-law of $M$.
\item $P$ is the grandfather of $M$.
\end{multicols}
\end{enumerate}

\hfill{\brak{\text{GATE MA 2016}}}

\item The number that least fits this set \brak{324, 441, 97, 64} is \underline.

\begin{enumerate}
\begin{multicols}{4}
\item $324$
\item $441$
\item $97$
\item $64$
\end{multicols}
\end{enumerate}

\hfill{\brak{\text{GATE MA 2016}}}

\item It takes $10,\text{s}$ and $15,\text{s}$, respectively, for two trains travelling at different constant speeds to completely pass a telegraph post. The length of the first train is $120,\text{m}$ and that of the second train is $150,\text{m}$. The magnitude of the difference in the speeds of the two trains \brak{\text{in m/s}} is \underline.

\begin{enumerate}
\begin{multicols}{4}
\item $2.0$
\item $10.0$
\item $12.0$
\item $22.0$
\end{multicols}
\end{enumerate}

\hfill{\brak{\text{GATE MA 2016}}}

\item The velocity $V$ of a vehicle along a straight line is measured in m/s and plotted as shown with respect to time in seconds. At the end of the $7$ seconds, how much will the odometer reading increase by \brak{\text{in m}}?

\begin{enumerate}
\begin{multicols}{4}
\item $0$
\item $3$
\item $4$
\item $5$
\end{multicols}
\end{enumerate}

\hfill{\brak{\text{GATE MA 2016}}}

\item The overwhelming number of people infected with rabies in India has been flagged by the World Health Organization as a source of concern. It is estimated that inoculating $70\%$ of pets and stray dogs against rabies can lead to a significant reduction in the number of people infected with rabies.

Which of the following can be logically inferred from the above sentences?

\begin{enumerate}

\item The number of people in India infected with rabies is high.
\item The number of people in other parts of the world who are infected with rabies is low.
\item Rabies can be eradicated in India by vaccinating $70\%$ of stray dogs.
\item Stray dogs are the main source of rabies worldwide.

\end{enumerate}

\hfill{\brak{\text{GATE MA 2016}}}

\item A flat is shared by four first year undergraduate students. They agreed to allow the oldest of them to enjoy some extra space in the flat. Manu is two months older than Sravan, who is three months younger than Trideep. Pavan is one month older than Sravan. Who should occupy the extra space in the flat?

\begin{enumerate}
\begin{multicols}{4}
\item Manu
\item Sravan
\item Trideep
\item Pavan
\end{multicols}
\end{enumerate}

\hfill{\brak{\text{GATE MA 2016}}}

\item Find the area bounded by the lines $3x + 2y = 14$, $2x - 3y = 5$ in the first quadrant.

\begin{enumerate}
\begin{multicols}{4}
\item $14.95$
\item $15.25$
\item $15.70$
\item $20.35$
\end{multicols}
\end{enumerate}

\hfill{\brak{\text{GATE MA 2016}}}

\item A straight line is fit to a data set \brak{\ln x, y}. This line intercepts the abscissa at $\ln x = 0.1$ and has a slope of $-0.02$. What is the value of $y$ at $x = 5$ from the fit?

\begin{enumerate}
\begin{multicols}{4}
\item $-0.030$
\item $-0.014$
\item $0.014$
\item $0.030$
\end{multicols}
\end{enumerate}

\hfill{\brak{\text{GATE MA 2016}}}
\item Let ${X, Y, Z}$ be a basis of $\mathbb{R}^3.$ Consider the following statements P and Q:

(P) \cbrak{X + Y, Y + Z, X - Z} \text{is a basis of } $\mathbb{R}^3$

(Q) \cbrak{X + Y + Z, X + 2Y - Z, X - 3Z} \text{is a basis of } $\mathbb{R}^3$


\begin{enumerate}
\begin{multicols}{2}
\item both P and Q
\item only P
\item only Q
\item neither P nor Q
\end{multicols}
\end{enumerate}

\hfill{\brak{\text{GATE MA 2016}}}

\item Consider the following statements P and Q:

(P) : If $M = \myvec{1 & 1 & 1 \\ 1 & 2 & 4 \\ 1 & 3 & 9}$, then $M$ is singular.

(Q) : Let $S$ be a diagonalizable matrix. If $T$ is a matrix such that $S + S^T = Id$, then $T$ is diagonalizable.

Which of the above statements hold TRUE?

\begin{enumerate}
\begin{multicols}{2}
\item both P and Q
\item only P
\item only Q
\item neither P nor Q
\end{multicols}
\end{enumerate}

\hfill{\brak{\text{GATE MA 2016}}}

\item Consider the following statements P and Q:

(P) : If $M$ is an $n \times n$ complex matrix, then $\mathcal{R}(M) = (N(M^*))^\perp.$

(Q) : There exists a unitary matrix with an eigenvalue $\lambda$ such that $\abs{\lambda} < 1.$

Which of the above statements hold TRUE?

\begin{enumerate}
\begin{multicols}{2}
\item both P and Q
\item only P
\item only Q
\item neither P nor Q
\end{multicols}
\end{enumerate}

\hfill{\brak{\text{GATE MA 2016}}}

\item Consider a real vector space $V$ of dimension $n$ and a non-zero linear transformation $T : V \to V$. If dimension$\brak{T(V)} < n$ and $T^2 = \lambda T$, for some $\lambda \in \mathbb{R} \setminus \brak{0}$, then which of the following statements is TRUE?

\begin{enumerate}

\item determinant$\brak{T} = \abs{\lambda}^n$
\item There exists a non-trivial subspace $V_1$ of $V$ such that $T(X) = 0$ for all $X \in V_1$
\item $T$ is invertible
\item $\lambda$ is the only eigenvalue of $T$

\end{enumerate}

\hfill{\brak{\text{GATE MA 2016}}}

\item Let $S = [0, 1) \cup [2, 3]$ and $f : S \to \mathbb{R}$ be a strictly increasing function such that $f(S)$ is connected. Which of the following statements is TRUE?

\begin{enumerate}
\begin{multicols}{2}
\item $f$ has exactly one discontinuity
\item $f$ has exactly two discontinuities
\item $f$ has infinitely many discontinuities
\item $f$ is continuous
\end{multicols}
\end{enumerate}

\hfill{\brak{\text{GATE MA 2016}}}



\item Let $a_1 = 1$ and $a_n = a_{n-1} + 4, \ n \geq 2.$ Then,
\begin{align*}
\lim_{n \to \infty} \brak{ \frac{1}{a_1 a_2} + \frac{1}{a_2 a_3} + \cdots + \frac{1}{a_{n-1} a_n} }
\end{align*}
is equal to \underline

\hfill{\brak{\text{GATE MA 2016}}}

\item Maximum $\brak{x + y : (x, y) \in \overline{B(0,1)}}$ is equal to \underline

\hfill{\brak{\text{GATE MA 2016}}}

\item Let $a, b, c, d \in \mathbb{R}$ such that $c^2 + d^2 \neq 0$. Then, the Cauchy problem
\begin{align*}
a u_x + b u_y = e^{x+y}, \quad x, y \in \mathbb{R},
\end{align*}
\begin{align*}
u(x, y) = 0 \quad \text{on} \quad c x + d y = 0
\end{align*}
has a unique solution if

\begin{enumerate}
\begin{multicols}{2}
\item $a c + b d \neq 0$
\item $a d - b c \neq 0$
\item $a c - b d \neq 0$
\item $a d + b c \neq 0$
\end{multicols}
\end{enumerate}

\hfill{\brak{\text{GATE MA 2016}}

\item Let $u(x, t)$ be the d'Alembert's solution of the initial value problem for the wave equation
\begin{align*}
u_{tt} - c^2 u_{xx} = 0 \\
u(x, 0) = f(x), \;\; u_t(x, 0) = g(x),
\end{align*}
where $c$ is a positive real number and $f, g$ are smooth odd functions. Then, $u(0,1)$ is equal to \underline

\hfill{\brak{\text{GATE MA 2016}}}

\item Let the probability density function of a random variable $X$ be
\begin{align*}
f(x) = 
\begin{cases}
x & 0 \leq x < \frac{1}{2} \\
c(2x-1)^2 & \frac{1}{2} \leq x \leq 1 \\
0 & \text{otherwise}
\end{cases}
\end{align*}
Then, the value of $c$ is equal to \underline

\hfill{\brak{\text{GATE MA 2016}}}

\item Let $V$ be the set of all solutions of the equation $y'' + a y' + b y = 0$ satisfying $y(0) = y(1)$, where $a, b$ are positive real numbers. Then, dimension$\brak{V}$ is equal to \underline

\hfill{\brak{\text{GATE MA 2016}}}

\item Let $y'' + p(x) y' + q(x) y = 0, \; x \in (-\infty, \infty)$, where $p(x)$ and $q(x)$ are continuous functions. If $y_1(x) = \sin(x) - 2\cos(x)$ and $y_2(x) = 2\sin(x) + \cos(x)$ are two linearly independent solutions of the above equation, then $\abs{4 p(0) + 2 q(1)}$ is equal to \underline

\hfill{\brak{\text{GATE MA 2016}}}

\item Let $P_n(x)$ be the Legendre polynomial of degree $n$ and $I = \int_{-1}^1 x^k P_n(x)dx$, where $k$ is a non-negative integer. Consider the following statements P and Q:

(P) : $I = 0$ if $k < n$

(Q) : $I = 0$ if $n - k$ is an odd integer.

Which of the above statements hold TRUE?

\begin{enumerate}
\begin{multicols}{2}
\item both P and Q
\item only P
\item only Q
\item neither P nor Q
\end{multicols}
\end{enumerate}

\hfill{\brak{\text{GATE MA 2016}}}



\item Consider the following statements P and Q:

(P) : $x^2 y'' + x y' + \brak{x^2 - \frac{1}{4}} y = 0$ has two linearly independent Frobenius series solutions near $x=0$.

(Q) : $x^2 y'' + 3 \sin(x) y' + y = 0$ has two linearly independent Frobenius series solutions near $x=0$.

Which of the above statements hold TRUE?

\begin{enumerate}
\begin{multicols}{2}
\item both P and Q
\item only P
\item only Q
\item neither P nor Q
\end{multicols}
\end{enumerate}

\hfill{\brak{\text{GATE MA 2016}}}

\item Let the polynomial $x^4$ be approximated by a polynomial of degree $\leq 2$, which interpolates $x^4$ at $x = -1, 0$ and $1$. Then, the maximum absolute interpolation error over the interval $\brak{-1,1}$ is equal to \underline

\hfill{\brak{\text{GATE MA 2016}}}

\item Let $(z_n)$ be a sequence of distinct points in $D(0,1) = \brak{z \in \mathbb{C} : \abs{z} < 1}$ with $\lim_{n \to \infty} z_n = 0$. Consider the following statements P and Q:

(P) : There exists a unique analytic function $f$ on $D(0,1)$ such that $f(z_n) = \sin(z_n)$ for all $n$.

(Q) : There exists an analytic function $f$ on $D(0,1)$ such that $f(z_n) = 0$ if $n$ is even and $f(z_n) = 1$ if $n$ is odd.

Which of the above statements hold TRUE?

\begin{enumerate}
\begin{multicols}{2}
\item both P and Q
\item only P
\item only Q
\item neither P nor Q
\end{multicols}
\end{enumerate}

\hfill{\brak{\text{GATE MA 2016}}}

\item Let $\brak{\mathbb{R}, \tau}$ be a topological space with the cofinite topology. Every infinite subset of $\mathbb{R}$ is

\begin{enumerate}
\begin{multicols}{2}
\item Compact but NOT connected
\item Both compact and connected
\item NOT compact but connected
\item Neither compact nor connected
\end{multicols}
\end{enumerate}

\hfill{\brak{\text{GATE MA 2016}}}

\item Let $c_0 = \brak{ (x_n) : x_n \in \mathbb{R},\ x_n \to 0 }$ and $M = \brak{ (x_n) \in c_0 : x_1 + x_2 + \cdots + x_{10} = 0 }$. Then, dimension$\brak{c_0 / M}$ is equal to \underline

\hfill{\brak{\text{GATE MA 2016}}}

\item Consider $(\mathbb{R}^2, \|\cdot\|_{\infty})$, where $\|(x, y)\|_{\infty} = \max\brak{ |x|, |y| }$. Let $f : \mathbb{R}^2 \to \mathbb{R}$ be defined by $f(x, y) = \frac{x + y}{2}$ and $\tilde{f}$ the norm preserving linear extension of $f$ to $(\mathbb{R}^3, \|\cdot\|_{\infty})$. Then, $\tilde{f}(1, 1, 1)$ is equal to \underline

\hfill{\brak{\text{GATE MA 2016}}}

\item $f : [0, 1] \to [0, 1]$ is called a shrinking map if $|f(x) - f(y)| < |x - y|$ for all $x, y \in [0, 1]$ and a contraction if there exists an $a < 1$ such that $|f(x) - f(y)| \leq a|x - y|$ for all $x, y \in [0, 1]$.

Which of the following statements is TRUE for the function $f(x) = x - \frac{x^2}{2}$\ ?

\begin{enumerate}
\begin{multicols}{2}
\item is both a shrinking map and a contraction
\item is a shrinking map but NOT a contraction
\item is NOT a shrinking map but a contraction
\item is neither a shrinking map nor a contraction
\end{multicols}
\end{enumerate}

\hfill{\brak{\text{GATE MA 2016}}}

\item Let $\mathbb{M}$ be the set of all $n \times n$ real matrices with the usual norm topology. Consider the following statements P and Q:

(P) : The set of all symmetric positive definite matrices in $\mathbb{M}$ is connected.\\
(Q) : The set of all invertible matrices in $\mathbb{M}$ is compact.

Which of the above statements hold TRUE?

\begin{enumerate}
\begin{multicols}{2}
\item both P and Q
\item only P
\item only Q
\item neither P nor Q
\end{multicols}
\end{enumerate}

\hfill{\brak{\text{GATE MA 2016}}}

\item Let $X_1, X_2, \ldots, X_n$ be a random sample from the following probability density function for $0 < \mu < \infty, \; 0 < \alpha < 1$,
\begin{align*}
f(x; \mu, \alpha) = 
\begin{cases}
\frac{1}{\Gamma(\alpha)} (x - \mu)^{\alpha - 1} e^{-(x - \mu)}, & x > \mu \\
0, & \text{otherwise}
\end{cases}
\end{align*}
Here $\alpha$ and $\mu$ are unknown parameters. Which of the following statements is TRUE?

\begin{enumerate}

\item Maximum likelihood estimator of only $\mu$ exists
\item Maximum likelihood estimator of only $\alpha$ exists
\item Maximum likelihood estimators of both $\mu$ and $\alpha$ exist
\item Maximum likelihood estimator of neither $\mu$ nor $\alpha$ exists

\end{enumerate}

\hfill{\brak{\text{GATE MA 2016}}}

\item Suppose $X$ and $Y$ are two random variables such that $aX + bY$ is a normal random variable for all $a, b \in \mathbb{R}$. Consider the following statements P, Q, R, and S:

(P) : $X$ is a standard normal random variable. \\
(Q) : The conditional distribution of $X$ given $Y$ is normal. \\
(R) : The conditional distribution of $X$ given $X+Y$ is normal. \\
(S) : $X - Y$ has mean $0$.

Which of the above statements ALWAYS hold TRUE?

\begin{enumerate}
\begin{multicols}{2}
\item both P and Q
\item both Q and R
\item both Q and S
\item both P and S
\end{multicols}
\end{enumerate}

\hfill{\brak{\text{GATE MA 2016}}}

\item Consider the following statements P and Q:

(P) : If $H$ is a normal subgroup of order $4$ of the symmetric group $S_4$, then $S_4/H$ is abelian. \\
(Q) : If $Q = \brak{\pm 1, \pm i, \pm j, \pm k}$ is the quaternion group, then $Q/\brak{-1, 1}$ is abelian.

Which of the above statements hold TRUE?

\begin{enumerate}
\begin{multicols}{2}
\item both P and Q
\item only P
\item only Q
\item neither P nor Q
\end{multicols}
\end{enumerate}

\hfill{\brak{\text{GATE MA 2016}}}

\item Let $F$ be a field of order $32$. Then the number of non-zero solutions $(a, b) \in F \times F$ of the equation $x^2 + x y + y^2 = 0$ is equal to \underline

\hfill{\brak{\text{GATE MA 2016}}}

\item Let $\gamma = \brak{z \in \mathbb{C} : \abs{z} = 2}$ be oriented in the counter-clockwise direction. Let
\begin{align*}
I = \frac{1}{2\pi i}\oint_{\gamma} z^7 \cos \brak{ \frac{1}{z^2} } dz.
\end{align*}
Then, the value of $I$ is equal to \underline

\hfill{\brak{\text{GATE MA 2016}}}

\item Let $u(x, y) = x^3 + a x^2 y + b x y^2 + 2 y^3$ be a harmonic function and $v(x, y)$ its harmonic conjugate. If $v(0,0) = 1$, then $\abs{a + b + v(1,1)}$ is equal to \underline

\hfill{\brak{\text{GATE MA 2016}}}

\item Let $\gamma$ be the triangular path connecting the points $(0,0), (2,2)$ and $(0,2)$ in the counter-clockwise direction in $\mathbb{R}^2$. Then
\begin{align*}
I = \oint_{\gamma} \sin(x^3) dx + 6 x y \, dy
\end{align*}
is equal to \underline

\hfill{\brak{\text{GATE MA 2016}}}

\item Let $y$ be the solution of
\begin{align*}
y' + y = \abs{x}, \quad x \in \mathbb{R} \\
y(-1) = 0.
\end{align*}
Then $y(1)$ is equal to

\begin{enumerate}
\begin{multicols}{2}
\item $\frac{2}{e} - \frac{2}{e^2}$
\item $\frac{2}{e} - 2 e^2$
\item $2 - \frac{2}{e}$
\item $2 - 2e$
\end{multicols}
\end{enumerate}

\hfill{\brak{\text{GATE MA 2016}}}

\item Let $X$ be a random variable with the following cumulative distribution function:
\begin{align*}
F(x) = 
\begin{cases}
0 & x < 0 \\
x^2 & 0 \leq x < \frac{1}{2} \\
\frac{3}{4} & \frac{1}{2} \leq x < 1 \\
1 & x \geq 1 \\
\end{cases}
\end{align*}
Then $P\brak{ \frac{1}{4} < X < 1 }$ is equal to \underline

\hfill{\brak{\text{GATE MA 2016}}}

\item Let $y$ be the curve which passes through $(0,1)$ and intersects each curve of the family $y = c x^2$ orthogonally. Then $y$ also passes through the point

\begin{enumerate}
\begin{multicols}{2}
\item $(\sqrt{2}, 0)$
\item $(0, \sqrt{2})$
\item $(1, 1)$
\item $(-1, 1)$
\end{multicols}
\end{enumerate}

\hfill{\brak{\text{GATE MA 2016}}}

\item Let $S(x) = a_0 + \sum_{n=1}^\infty \brak{ a_n \cos (n x) + b_n \sin (n x)}$ be the Fourier series of the $2\pi$ periodic function defined by $f(x) = x^2 + 4 \sin(x)\cos(x),\ -\pi \leq x \leq \pi.$ Then 
\begin{align*}
\abs{ \sum_{n=0}^\infty a_n - \sum_{n=1}^\infty b_n }
\end{align*}
is equal to \underline

\hfill{\brak{\text{GATE MA 2016}}}

\item Let $y(t)$ be a continuous function on $[0, \infty)$. If
\begin{align*}
y(t) = t \brak{1 - 4\int_0^t y(x) dx } + 4 \int_0^t x y(x) dx,
\end{align*}
then $\int_0^{\pi/2} y(t)\, dt$ is equal to \underline

\hfill{\brak{\text{GATE MA 2016}}}

\item Let $S_n = \sum_{k=1}^n \frac{1}{k}$ and $I_n = \int_1^n \frac{x - \lfloor x \rfloor}{x^2} dx$. Then, $S_{10} + I_{10}$ is equal to

\begin{enumerate}
\begin{multicols}{2}
\item $\ln 10 + 1$
\item $\ln 10 - 1$
\item $\ln 10 - \frac{1}{10}$
\item $\ln 10 + \frac{1}{10}$
\end{multicols}
\end{enumerate}

\hfill{\brak{\text{GATE MA 2016}}}

\item For any $(x, y) \in \mathbb{R}^2 \setminus \overline{B(0,1)}$, let 
\begin{align*}
f(x, y) = \text{distance}\brak{(x, y), \overline{B(0,1)}} = \inf \brak{\sqrt{(x-x_1)^2 + (y-y_1)^2} : (x_1, y_1) \in \overline{B(0,1)} }
\end{align*}
Then, $\norm{\nabla f(3, 4)}$ is equal to \underline

\hfill{\brak{\text{GATE MA 2016}}}

\item Let $f(x) = \brak{ \int_0^x e^{-t^2} dt }^2$ and $g(x) = \int_0^x \frac{e^{-t^2} (1 + t^2)}{1 + t^2} dt$. Then $f'(\sqrt{\pi}) + g'(\sqrt{\pi})$ is equal to \underline

\hfill{\brak{\text{GATE MA 2016}}}

\item Let $M = \myvec{a & b & c \\ l & d & e \\ l & e & f}$ be a real matrix with eigenvalues $1, 0$ and $3$. If the eigenvectors corresponding to $1$ and $0$ are $(1, 1, 1)^T$ and $(1, -1, 0)^T$ respectively, then the value of $3f$ is equal to \underline

\hfill{\brak{\text{GATE MA 2016}}}

\item Let $M = \myvec{1 & 0 & 1 \\ 0 & 1 & 1 \\ 0 & 0 & 1}$ and $e^M = Id + M + \frac{1}{2!}M^2 + \frac{1}{3!}M^3 + \dots.$ If $e^M = [b_{ij}]$, then
\begin{align*}
\frac{1}{e} \sum_{i=1}^3 \sum_{j=1}^3 b_{ij}
\end{align*}
is equal to \underline

\hfill{\brak{\text{GATE MA 2016}}}

\item Let the integral $I = \int_0^4 f(x) \, dx$, where
\begin{align*}
f(x) = \begin{cases}
x & 0 \leq x \leq 2 \\
4 - x & 2 < x \leq 4
\end{cases}
\end{align*}
Consider the following statements P and Q:

(P) : If $I_2$ is the value of the integral obtained by the composite trapezoidal rule with two equal sub-intervals, then $I_2$ is exact.

(Q) : If $I_3$ is the value of the integral obtained by the composite trapezoidal rule with three equal sub-intervals, then $I_3$ is exact.

Which of the above statements hold TRUE?

\begin{enumerate}
\begin{multicols}{2}
\item both P and Q
\item only P
\item only Q
\item neither P nor Q
\end{multicols}
\end{enumerate}

\hfill{\brak{\text{GATE MA 2016}}}

\item The difference between the least two eigenvalues of the boundary value problem
\begin{align*}
y'' + \lambda y = 0, \qquad 0 < x < \pi \\
y(0) = 0, \qquad y'(\pi) = 0,
\end{align*}
is equal to \underline

\hfill{\brak{\text{GATE MA 2016}}}

\item The number of roots of the equation $x^2 - \cos(x) = 0$ in the interval $\left[ -\frac{\pi}{2}, \frac{n}{2} \right]$ is equal to \underline

\hfill{\brak{\text{GATE MA 2016}}}

\item For the fixed point iteration $x_{k+1} = g(x_k), \; k = 0, 1, 2, \ldots$, consider the following statements P and Q:

(P) : If $g(x) = 1 + \frac{2}{x}$ then the fixed point iteration converges to $2$ for all $x_0 \in [1, 100]$.

(Q) : If $g(x) = \sqrt{2 + x}$ then the fixed point iteration converges to $2$ for all $x_0 \in [0, 100]$.

Which of the above statements hold TRUE?

\begin{enumerate}
\begin{multicols}{2}
\item both P and Q
\item only P
\item only Q
\item neither P nor Q
\end{multicols}
\end{enumerate}

\hfill{\brak{\text{GATE MA 2016}}}

\item Let $T : \ell_2 \to \ell_2$ be defined by
\begin{align*}
T\brak{ (x_1, x_2, ..., x_n, \ldots) } = (x_2 - x_1, x_3 - x_2, ..., x_{n+1} - x_n, ...)
\end{align*}
Then

\begin{enumerate}
\begin{multicols}{2}
\item $\norm{T} = 1$
\item $\norm{T} \geq 2$ but bounded
\item $1 < \norm{T} \leq 2$
\item $\norm{T}$ is unbounded
\end{multicols}
\end{enumerate}

\hfill{\brak{\text{GATE MA 2016}}}

\item Minimize $w = x + 2y$ subject to
\begin{align*}
\begin{cases}
2x + y \geq 3 \\
x + y \geq 2 \\
x \geq 0, y \geq 0
\end{cases}
\end{align*}
Then, the minimum value of $w$ is equal to \underline

\hfill{\brak{\text{GATE MA 2016}}}

\item Maximize $w = 11x - z$ subject to
\begin{align*}
\begin{cases}
10x + y - z \leq 1 \\
2x - 2y + z \leq 2 \\
x, y, z \geq 0
\end{cases}
\end{align*}
Then, the maximum value of $w$ is equal to \underline

\hfill{\brak{\text{GATE MA 2016}}}

\item Let $X_1, X_2, X_3, \ldots$ be a sequence of i.i.d. random variables with mean $1$. If $N$ is a geometric random variable with the probability mass function $P(N = k) = \frac{1}{2^k}$, $k = 1, 2, 3, \ldots$, and it is independent of the $X_i$'s, then $E(X_1 + X_2 + \cdots + X_N)$ is equal to \underline

\hfill{\brak{\text{GATE MA 2016}}}

\item Let $X_1$ be an exponential random variable with mean $1$ and $X_2$ a gamma random variable with mean $2$ and variance $2$. If $X_1$ and $X_2$ are independently distributed, then $P(X_1 < X_2)$ is equal to \underline

\hfill{\brak{\text{GATE MA 2016}}}

\item Let $X_1, X_2, X_3, \ldots$ be a sequence of i.i.d. uniform $(0,1)$ random variables. Then, the value of
\begin{align*}
\lim_{n \to \infty} P\left( -\ln(1 - X_1) - \cdots - \ln(1 - X_n) \geq n \right)
\end{align*}
is equal to \underline

\hfill{\brak{\text{GATE MA 2016}}}

\item Let $X$ be a standard normal random variable. Then, $P\brak{X < 0 \mid \abs{X} = 1}$ is equal to

\begin{enumerate}
\begin{multicols}{2}
\item $\frac{\Phi(1) - \frac{1}{2}}{\Phi(2) - \frac{1}{2}}$
\item $\frac{\Phi(1) + \frac{1}{2}}{\Phi(2) + \frac{1}{2}}$
\item $\frac{\Phi(1) - \frac{1}{2}}{\Phi(2) + \frac{1}{2}}$
\item $\frac{\Phi(1) + \frac{1}{2}}{\Phi(2) + 1}$
\end{multicols}
\end{enumerate}

\hfill{\brak{\text{GATE MA 2016}}}

\item Let $X_1, X_2, X_3, \ldots, X_n$ be a random sample from the probability density function
\begin{align*}
f(x) = \theta \, a\, e^{-a x} + (1 - \theta) 2 a \, e^{-2 a x}, \quad x \geq 0 \text{; } 0 \text{ otherwise}
\end{align*}
where $a > 0,\ 0 \leq \theta \leq 1$ are parameters. Consider the following testing problem: $H_0: \theta = 1,\, a = 1$ versus $H_1: \theta = 0,\, a = 2$.

Which of the following statements is TRUE?

\begin{enumerate}
\begin{multicols}{2}
\item Uniformly Most Powerful test does NOT exist
\item Uniformly Most Powerful test is of the form $\sum_{i=1}^n X_i > c$, for some $0 < c < \infty$
\item Uniformly Most Powerful test is of the form $\sum_{i=1}^n X_i < c$, for some $0 < c < \infty$
\item Uniformly Most Powerful test is of the form $c_1 < \sum_{i=1}^n X_i < c_2$, for some $0 < c_1 < c_2 < \infty$
\end{multicols}
\end{enumerate}

\hfill{\brak{\text{GATE MA 2016}}}

\item Let $X_1, X_2, X_3, \ldots$ be a sequence of i.i.d. $N(\mu, 1)$ random variables. Then,
\begin{align*}
\lim_{n \to \infty} \frac{\sqrt{\pi}}{2n} \sum_{i=1}^n E\abs{X_i - \mu}
\end{align*}
is equal to \underline

\hfill{\brak{\text{GATE MA 2016}}}

\item Let $X_1, X_2, X_3, ..., X_n$ be a random sample from uniform $[1, \theta]$, for some $\theta > 1$. If $X_{(n)} = \text{Maximum}(X_1, X_2, ..., X_n)$, then the UMVUE of $\theta$ is

\begin{enumerate}
\begin{multicols}{2}
\item $\frac{n+1}{n} X_{(n)} + \frac{1}{n}$
\item $\frac{n+1}{n} X_{(n)} - \frac{1}{n}$
\item $\frac{n}{n+1} X_{(n)} + \frac{1}{n}$
\item $\frac{n}{n+1} X_{(n)} + \frac{n+1}{n}$
\end{multicols}
\end{enumerate}

\hfill{\brak{\text{GATE MA 2016}}}

\item Let $x_1 = x_2 = x_3 = 1, \ x_4 = x_5 = x_6 = 2$ be a random sample from a Poisson random variable with mean $\theta$, where $\theta \in \brak{1, 2}$. Then, the maximum likelihood estimator of $\theta$ is equal to \underline

\hfill{\brak{\text{GATE MA 2016}}}

\item The remainder when $98!$ is divided by $101$ is equal to \underline

\hfill{\brak{\text{GATE MA 2016}}}

\item Let $G$ be a group whose presentation is
\begin{align*}
G = \brak{ x, y \mid x^5 = y^2 = e, \quad x^2 y = y x }
\end{align*}
Then $G$ is isomorphic to

\begin{enumerate}
\begin{multicols}{2}
\item $\mathbb{Z}_5$
\item $\mathbb{Z}_{10}$
\item $\mathbb{Z}_2$
\item $\mathbb{Z}_{30}$
\end{multicols}
\end{enumerate}

\hfill{\brak{\text{GATE MA 2016}}}


\end{enumerate}
\end{document}