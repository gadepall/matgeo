\item Let ${X, Y, Z}$ be a basis of $\mathbb{R}^3.$ Which of the following statements P and Q are true?
\hfill{\brak{\text{MA 2016}}}
		\begin{enumerate}[label=\Alph*:, start=16]
			\item  $\cbrak{X + Y, Y + Z, X - Z}$ is a basis of  $\mathbb{R}^3$
			\item  $\cbrak{X + Y + Z, X + 2Y - Z, X - 3Z}$ is a basis of  $\mathbb{R}^3$
\end{enumerate}
\begin{enumerate}
\begin{multicols}{4}
\item both P and Q
\item only P
\item only Q
\item neither P nor Q
\end{multicols}
\end{enumerate}
\item Consider the following statements P and Q
		\begin{enumerate}[label=\Alph*:, start=16]
\item   If $\vec{M} = \myvec{1 & 1 & 1 \\ 1 & 2 & 4 \\ 1 & 3 & 9}$, then $\vec{M}$ is singular.
\item  Let $\vec{S}$ be a diagonalizable matrix. If $\vec{T}$ is a matrix such that $\vec{S} + \vec{S}^{\top} = \vec{I}d$, then $\vec{T}$ is diagonalizable.
\end{enumerate}
Which of the above statements hold TRUE?
\hfill{\brak{\text{MA 2016}}}
\begin{enumerate}
\begin{multicols}{4}
\item both P and Q
\item only P
\item only Q
\item neither P nor Q
\end{multicols}
\end{enumerate}
\item Consider the following statements P and Q:
		\begin{enumerate}[label=\Alph*:, start=16]
\item  If $\vec{M}$ is an $n \times n$ complex matrix, then $\mathcal{R}(\vec{M}) = (N(\vec{M}^*))^\perp.$
\item  There exists a unitary matrix with an eigenvalue $\lambda$ such that $\abs{\lambda} < 1.$
\end{enumerate}
Which of the above statements hold TRUE?
\hfill{\brak{\text{MA 2016}}}
\begin{enumerate}
\begin{multicols}{4}
\item both P and Q
\item only P
\item only Q
\item neither P nor Q
\end{multicols}
\end{enumerate}
\item Consider a real vector space $V$ of dimension $n$ and a non-zero linear transformation $T : V \to V$. If dimension$\brak{T(V)} < n$ and $T^2 = \lambda T$, for some $\lambda \in \mathbb{R} \setminus \brak{0}$, then which of the following statements is TRUE?
\hfill{\brak{\text{MA 2016}}}
\begin{enumerate}
\item determinant $\brak{T} = \abs{\lambda}^n$
\item There exists a non-trivial subspace $V_1$ of $V$ such that $T(X) = 0$ for all $X \in V_1$
\item $T$ is invertible
\item $\lambda$ is the only eigenvalue of $T$
\end{enumerate}
\item Let $y$ be the curve which passes through $(0,1)$ and intersects each curve of the family $y = c x^2$ orthogonally. Then $y$ also passes through the point
\hfill{\brak{\text{MA 2016}}}
\begin{enumerate}
\begin{multicols}{4}
\item $(\sqrt{2}, 0)$
\item $(0, \sqrt{2})$
\item $(1, 1)$
\item $(-1, 1)$
\end{multicols}
\end{enumerate}
\item Let $\vec{M} = \myvec{a & b & c \\ l & d & e \\ l & e & f}$ be a real matrix with eigenvalues $1, 0$ and $3$. If the eigenvectors corresponding to $1$ and $0$ are $(1, 1, 1)^{\top}$ and $(1, -1, 0)^{\top}$ respectively, then the value of $3f$ is equal to 
\rule{1cm}{0.01pt}.
\hfill{\brak{\text{MA 2016}}}
\item Let $\vec{M} = \myvec{1 & 0 & 1 \\ 0 & 1 & 1 \\ 0 & 0 & 1}$ and $e^\vec{M} = \vec{I}d + \vec{M} + \frac{1}{2!}\vec{M}^2 + \frac{1}{3!}\vec{M}^3 + \dots.$ If $e^\vec{M} = [b_{ij}]$, then
\begin{align*}
\frac{1}{e} \sum_{i=1}^3 \sum_{j=1}^3 b_{ij}
\end{align*}
is equal to \rule{1cm}{0.01pt}.
\hfill{\brak{\text{MA 2016}}}
