\item The convolution of $\vec{A} = \{4, -2, -1, 2\}$ with $\vec{B} = \{1, 0, -1\}$ gives
\hfill{\brak{\text{GG 2020}}}
\begin{multicols}{2}
\begin{enumerate}
    \item \{4, -2, -5, 4, 1, -2\}
    \item \{4, 2, -5, 0, 1, -2\}
    \item \{4, -2, -5, 0, -1, -2\}
    \item \{4, -2, 0, 4, 1, -2\}
\end{enumerate}
\end{multicols}
\item $\vec{P}$ and $\vec{R}$ are Jacobian matrices for two different geophysical inverse problems. If their generalized inverses are written as $\vec{P}^{-g} = (\vec{P}^T \vec{P})^{-1} \vec{P}^T$ and $\vec{R}^{-g} = \vec{R}^T(\vec{R} \vec{R}^T)^{-1}$, then
\hfill{\brak{\text{GG 2020}}}
\begin{enumerate}
    \item $\vec{P}$ represents an overdetermined problem and $\vec{R}$ represents an underdetermined problem
    \item $\vec{P}$ represents an underdetermined problem and $\vec{R}$ represents an overdetermined problem
    \item Both $\vec{P}$ and $\vec{R}$ represent overdetermined problems
    \item Both $\vec{P}$ and $\vec{R}$ represent underdetermined problems
\end{enumerate}
