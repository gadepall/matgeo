\item Consider the subspace \(W = \{[a]: a=0 \text{ if } i \text{ is even}\}\) of all \(10 \times 10\) real matrices. Then the dimension of \(W\) is
\hfill\brak{\text{MA 2008}}
\begin{multicols}{4}
\begin{enumerate}
\item 25
\item 50
\item 75
\item 100
\end{enumerate}
\end{multicols}
\item
Let $T:\mathbb{R}^4 \to \mathbb{R}^4$ be the linear map satisfying
\begin{align*}
    T(e_1) = e_2,\quad T(e_2) = e_3,\quad T(e_3) = 0,\quad T(e_4) = e_3,
\end{align*}
where $\{e_1, e_2, e_3, e_4\}$ is the standard basis of $\mathbb{R}^4$. Then
\hfill\brak{\text{MA 2008}}
\begin{multicols}{2}
\begin{enumerate}
\item $T$ is idempotent
\item $T$ is invertible
\item Rank $T = 3$
\item $T$ is nilpotent
\end{enumerate}

\end{multicols}
\item
Let 
\begin{align*}
    \vec{M} = \myvec{1 & 1 & 2 \\ 0 & 1 & 1 \\ 0 & 1 & 1}
\end{align*}
and $\vec{V} = \{ \vec{M} \vec{x}^{\top} : \vec{x} \in \mathbb{R}^3 \}$. Then an orthonormal basis for $\vec{V}$ is
\hfill\brak{\text{MA 2008}}
\begin{multicols}{2}
\begin{enumerate}
	\item $\left\{ \myvec{1 \\0 \\0},\, \myvec{0 \\ \frac{2}{\sqrt{5}} \\ \frac{1}{\sqrt{5}}},\, \myvec{\frac{2}{\sqrt{6}} \\ \frac{1}{\sqrt{6}} \\ \frac{1}{\sqrt{6}}} \right\}$
	\item $\left\{ \myvec{1\\0\\0},\, \myvec{0 \\ \frac{1}{\sqrt{2}} \\ \frac{1}{\sqrt{2}}} \right\}$
	\item $\left\{ \myvec{1\\0\\0},\, \myvec{\frac{1}{\sqrt{3}} \\ \frac{1}{\sqrt{3}} \\ \frac{1}{\sqrt{3}}},\, \myvec{\frac{2}{\sqrt{6}} \\ \frac{1}{\sqrt{6}} \\ \frac{1}{\sqrt{6}}} \right\}$
	\item $\left\{ \myvec{1\\0\\0},\, \myvec{0\\0\\1} \right\}$
\end{enumerate}
\end{multicols}{2}
\item
For any $n \in \mathbb{N}$, let $P_n$ denote the vector space of all polynomials with real coefficients and of degree at most $n$. Define $T:P_n \to P_{n+1}$ by
\begin{align*}
    T(p)(x) = p'(x) - \int_0^x p(t) \, dt.
\end{align*}
Then the dimension of the null space of $T$ is
\hfill\brak{\text{MA 2008}}
\begin{multicols}{4}
\begin{enumerate}
    \item $0$
    \item $1$
    \item $n$
    \item $n+1$
\end{enumerate}
\end{multicols}
\item
Let
\begin{align*}
    \vec{M} = \myvec{1 & 0 & 0 \\ 0 & \cos\theta & -\sin\theta \\ 0 & \sin\theta & \cos\theta}
\end{align*}
where $0 < \theta < \frac{\pi}{2}$. Let $\vec{V} = \{ \vec{u} \in \mathbb{R}^3 : \vec{M} \vec{u}^2 = \vec{u}^{\top} \}$. Then the dimension of $\vec{V}$ is
\hfill\brak{\text{MA 2008}}
\begin{multicols}{4}
\begin{enumerate}
    \item $0$
    \item $1$
    \item $2$
    \item $3$
\end{enumerate}
\end{multicols}


\item
The number of linearly independent eigenvectors of the matrix
\begin{align*}
   \myvec{2 & 2 & 0 & 0 \\ 2 & 1 & 0 & 0 \\ 0 & 0 & 3 & 0 \\ 0 & 0 & 0 & 4} 
\end{align*}

is
\hfill\brak{\text{MA 2008}}
\begin{multicols}{4}
\begin{enumerate}
    \item $1$
    \item $2$
    \item $3$
    \item $4$
\end{enumerate}
\end{multicols}
\item
Consider $\mathbb{R}^3$ with norm $\|\cdot\|$ and the linear transformation $T : \mathbb{R}^3 \to \mathbb{R}^3$ defined by the matrix
\begin{align*}
    \myvec{1 & 1 & 3 \\ 1 & 3 & 3 \\ 1 & 3 & 3}.
\end{align*}

Then the operator norm $\|T\|$ of $T$ is equal to
\hfill\brak{\text{MA 2008}}
\begin{multicols}{4}
\begin{enumerate}
    \item 6
    \item 7
    \item 8
    \item $\sqrt{42}$
\end{enumerate}
\end{multicols}
\item
Let $H = L^2([0, \pi])$ with the usual inner product. For $n \in \mathbb{N}$, let
\begin{align*}
    u_n(t) = \sqrt{\frac{2}{\pi}} \sin(n t), \quad t \in [0, \pi],
\end{align*}

and $E = \{u_n : n \in \mathbb{N}\}$. Then
\hfill\brak{\text{MA 2008}}

\begin{enumerate}
    \item $E$ is not a linearly independent subset of $H$
    \item $E$ is a linearly independent subset of $H$, but is not an orthonormal subset of $H$
    \item $E$ is an orthonormal subset of $H$, but is not an orthonormal basis for $H$
    \item $E$ is an orthonormal basis for $H$
\end{enumerate}

\item
Let
\begin{align*}
   \vec{N} =
\begin{pmatrix}
\frac{4}{5} & \frac{3}{5} & 0 \\
0 & 0 & 1 \\
\end{pmatrix}. 
\end{align*}
Then $\vec{N}$ is
\hfill\brak{\text{MA 2008}}
\begin{multicols}{4}
\begin{enumerate}
    \item non-invertible
    \item skew-symmetric
    \item symmetric
    \item orthogonal
\end{enumerate}
\end{multicols}

\item
If $\vec{M}$ is any $3 \times 3$ real matrix, then $\operatorname{trace}(\vec{N} \vec{M} \vec{N}^{\top})$ is equal to
\hfill\brak{\text{MA 2008}}
\begin{multicols}{2}
\begin{enumerate}
    \item $(\operatorname{trace}(\vec{N}))^2 \operatorname{trace}(\vec{M})$
    \item $2 \operatorname{trace}(\vec{N}) + \operatorname{trace}(\vec{M})$
    \item $\operatorname{trace}(\vec{M})$
    \item $(\operatorname{trace}(\vec{N}))^2 + \operatorname{trace}(\vec{M})$
\end{enumerate}
\end{multicols}
