\item The characteristic equation of a $(3 \times 3)$ matrix $\vec{P}$ is defined as
$$
\alpha(\lambda) = \lvert \lambda \vec{I} - \vec{P} \rvert 
= \lambda^3 + \lambda^2 + 2\lambda + 1 = 0.
$$ % /lvert is mod
If $\vec{I}$ denotes the identity matrix, then the inverse of matrix $\vec{P}$ will be
\hfill (EE 2008)
\begin{enumerate}
\begin{multicols}{4}
    \item $(\vec{P}^2 + \vec{P} + 2\vec{I})$
    \item $(\vec{P}^2 + \vec{P} + \vec{I})$
    \item $-(\vec{P}^2 + \vec{P} + \vec{I})$
    \item $-(\vec{P}^2 + \vec{P} + 2\vec{I})$
\end{multicols}
\end{enumerate}
\item If the rank of a $(5 \times 6)$ matrix $\vec{Q}$ is $4$, then which one of the following statements is correct?
\hfill (EE 2008)
\begin{enumerate}
    \item $\vec{Q}$ will have four linearly independent rows and four linearly independent columns
    \item $\vec{Q}$ will have four linearly independent rows and five linearly independent columns
    \item $\vec{Q} \vec{Q}^{\top}$ will be invertible
    \item $\vec{Q}^{\top} \vec{Q}$ will be invertible
\end{enumerate}
\item $\vec{A}$ is a $m \times n$ full rank matrix with $m>n$ and $\vec{I}$ is an identity matrix. Let matrix $\vec{A}^{+}=(\vec{A}^{\top}\vec{A})^{-1}\vec{A}^{\top}$. Then, which one of the following statements is {FALSE}?

\hfill (EE 2008)
\begin{multicols}{4}
\begin{enumerate}
    \item $\vec{A}\vec{A}^{+}\vec{A} = \vec{A}$
    \item $(\vec{A}\vec{A}^{+})^2 = \vec{A}\vec{A}^{+}$
    \item $\vec{A}^{+}\vec{A} = \vec{I}$
    \item $\vec{A} \vec{A}^{+}\vec{A} = \vec{A}^{+}$
\end{enumerate}
\end{multicols}
%
\item Let $\vec{P}$ be a $2\times 2$ real orthogonal matrix and $\vec{x}$ is a real vector $\myvec{x_1 & x_2}^{\top}$ with length $\|\vec{x}\| = (x_1^2+x_2^2)^{1/2}$. Then, which one of the following statements is correct?
\begin{enumerate}
    \item $\|\vec{Px}\| \le \|\vec{x}\|$ where at least one vector satisfies $\|\vec{Px}\| < \|\vec{x}\|$
    \item $\|\vec{Px}\| = \|\vec{x}\|$ for all vectors $\vec{x}$
    \item $\|\vec{Px}\| \ge \|\vec{x}\|$ where at least one vector satisfies $\|\vec{Px}\| > \|\vec{x}\|$
    \item No relationship can be established between $\|\vec{x}\|$ and $\|\vec{Px}\|$
\end{enumerate}
\hfill (EE 2008)
\item 
The state space equation of a system is described by
\begin{align*}
	\dot{\vec{x}} &= \vec{A}\vec{x} + \vec{B}\vec{u}\\
	y &= \vec{C}\vec{x}
\end{align*}
where $\vec{x}$ is state vector, $\vec{u}$ is input, ${y}$ is output and
$$
\vec{A}=\myvec{0&1\\0&-2},\
\vec{B}=\myvec{0\\1},\
\vec{C}=\myvec{1&0}.
	$$
The transfer function $G(s)$ of this system will be
\hfill (EE 2008) 
\begin{multicols}{4}
\begin{enumerate}
    \item $\dfrac{s}{(s+2)}$
    \item $\dfrac{s+1}{s(s-2)}$
    \item $\dfrac{s}{(s-2)}$
    \item $\dfrac{1}{s(s+2)}$
\end{enumerate}
\end{multicols}

