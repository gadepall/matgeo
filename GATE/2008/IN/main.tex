\item  Consider a discrete-time LTI system with input $x\sbrak{n} = \delta \sbrak{n} +\delta\sbrak{n-1} $ and impulse response $h\sbrak{n} =\delta\sbrak{n}-\delta\sbrak{n-1}$. The output of the system will be given by 

	\hfill{(IN 2008)}
\begin{multicols}{2}
  \begin{enumerate} 
        \item $\delta\sbrak{n}-\delta\sbrak{n-2}$
        \item $\delta\sbrak{n}-\delta\sbrak{n-1}$
        \item $\delta\sbrak{n-1}+\delta\sbrak{n-2}$
        \item $\delta\sbrak{n}+\delta\sbrak{n-1}+\delta\sbrak{n-2}$
  \end{enumerate}
  \end{multicols}
\item The state space representation of a system is given by 
\begin{align*}
\dot{\vec{x}} &= 
\myvec{
0 & 1 \\
0 & -3
} \vec{\vec{x}}
+ 
\myvec{
1 \\
0
} u
\\
y &= 
\myvec{
1 & 0
} \vec{x}
\end{align*}
%
The transfer function$\frac{Y\brak{s}}{U\brak{s}}$ of the system will be \hfill{(IN 2008)}
\begin{multicols}{4}
           \begin{enumerate} 
              \item $\frac{1}{s}$           
              \item $\frac{1}{s\brak{s+3}}$
              \item $\frac{1}{s+3}$
              \item $\frac{1}{s^2}$
            \end{enumerate}
            \end{multicols}

