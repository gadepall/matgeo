\item If the characteristic equation of a 3x3 matrix is $$\lambda^3 - \lambda^2 + \lambda - 1 =0,$$ then the matrix should be 
\hfill (XE 2008)
\begin{multicols}{4}
\begin{enumerate}
\item  Hermitian 
\item Unitary
 \item Skew symmetric
\item Identity
\end{enumerate}
\end{multicols}
\item The directional derivative at the point $\vec{P}(1,2,3)$ to the surface {\large $x^2 + \frac{y^2}{4} + \frac{z^2}{9} =3$} in the direction of the vector $\overrightarrow{OP}$ ,where $\vec{O}$ denotes the origin is
\hfill (XE 2008)
\begin{multicols}{4}
\begin{enumerate}
\item 0
\item {$\frac{2}{\sqrt{14}}$}
\item {$\frac{3}{\sqrt{14}}$}
\item {$\frac{6}{\sqrt{14}}$}
\end{enumerate}
\end{multicols}
\item The system of equations 
\hfill (XE 2008)
	\begin{align*}	 ax+by+a^2=0  \\ bx+ay-b^2=0 \\ x+y+a-b=0
	\end{align*}
\begin{enumerate}
\item  admits unique solution if $a=b\ne 0$
\item admits unique solution if $a=-b\ne 0$
\item admits unique solution if $a=b=0$
\item does not admit unique solution
\end{enumerate}
\item The matrix $$\myvec{l&0&\sin\theta\\0&1&m\\n&0&\cos\theta}$$ is orthogonal, if
\hfill (XE 2008)
\begin{multicols}{2}
\begin{enumerate}
\item $l=-\sin\theta, m=-\cos\theta, n=0$
\item $l=-\sin\theta, m=0, n=\cos\theta$
\item $l=\cos\theta, m=\sin\theta, n=0$
\item $l=-\cos\theta, m=0, n=\sin\theta$
\end{enumerate}
\end{multicols}

\item  On solving the system of equations
\begin{align*}
4x+z =5\\
x+2y+3z =1\\
-y-4z=3,
\end{align*}
by LU decomposition with $u_{ii} =1$ for $i =1,2,3$; the values of $u_{23}$ and $l_{33}$ are respectively
\hfill (XE 2008)
\begin{enumerate}
\item  1.375 and -4.250
\item  2.750 and -3.625
\item  1.375 and -2.625
\item 2.750 and -4.250
\end{enumerate}


\item The eigenvalues of the matrix $$\begin{pmatrix}0&-1\\1&0\end{pmatrix}$$ are
\hfill (XE 2008)
\begin{enumerate}
\begin{multicols}{4}
\item 1,-1          
\item $i,i$
\item 1,1    
\item $i,-i$
\end{multicols}
\end{enumerate}

\item One of the eigenvalues of a $3\times 3$ matrix $\vec{M}$ is 3. If the determinant of the matrix $\vec{M}$ is 24 and the trace is 9, then the smallest eigenvalue of the matrix $\vec{M}^{-1}$ is

\hfill (XE 2008)
\begin{enumerate}
\begin{multicols}{4}
\item 1/8 
\item 1/4 
\item 1/3 
\item 1/2
\end{multicols}
\end{enumerate}
\item The set of simultaneous equations 
	\begin{align*}4x-y&=15 
		\\
x+5y&=9\end{align*} 
is to be solved using Jacobi's iterative method. Starting with the initial values $x = 2, y =2$, the values of $x$ and $y$ after two iterations are, respectively,
\hfill (XE 2008)
\begin{enumerate}
\begin{multicols}{4}
\item  4.25, 0.95
\item  4.25, 1.4
\item  4.1, 0.95
\item 4.1, 1.4
\end{multicols}
\end{enumerate}

\item The lower triangular matrix $\vec{L}$ in the LU factorization of the matrix
$\begin{pmatrix}
    25&5&4\\10&8&16\\8&10&22
\end{pmatrix}$ is written as $\begin{pmatrix}
    1&0&0\\L_{21}&1&0\\L_{31}&L_{32}&1
\end{pmatrix}$. The element $L_{32}$ is 
\hfill (XE 2008)
\begin{enumerate}
\begin{multicols}{4}
\item 1.0 
\item 1.4 
\item 0.4 
\item 0.32
\end{multicols}
\end{enumerate}

