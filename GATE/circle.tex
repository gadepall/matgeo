\begin{enumerate}[label=\thesubsection.\arabic*.,ref=\thesubsection.\theenumi]
\item Consider 
\begin{align*}
	L_1:2x+3y+p-3&=0
	\\
               L_2:2x+3y+p+3&=0
\end{align*}
where $p$ is a real number, and 
		$$	C: x^2+y^2+6x-10y+30=0$$
STATEMENT-$1$:If line $L_1$ is a chord of circle C, then line $L_2$ is not always a diameter of circle C\\
STATEMENT-$2$:If line $L_1$ is a diameter of circle C, then line $L_2$ is not a chord  of circle C. \hfill(2008)
\begin{enumerate}
\item Statement-$1$ is True, statement-$2$ is True;Statement-$2$ is a correct explantion for Statement-$1$.
\item Statement-$1$ is True, statement-$2$ is True;Statement-$2$ is NOT a correct explantion for Statement-$1$.
\item Statement-$1$ is True, Statement-$2$ is False
\item Statement-$1$ is False, Statement-$2$ is True.	
\end{enumerate}
	\item If $\vec{A}$ and $\vec{B}$ are points in the plane such that  $\frac{PA}{PB}=K$(constant) for all $\vec{P}$ on a given circle,  then the value of $K$ cannot be equal to \rule{1cm}{0.01pt}.
		\hfill\brak{1982}
\item From the origin chords are drawn  to the circle $\brak{x-1}^{2}+y^{2}=1$. The equation of the locus of the mid-points of these chords is
\rule{1cm}{0.01pt}.
	\hfill\brak{1985}
\item From the point $\vec{A}\brak{0, 3}$ on the circle $$x^{2}+4x+\brak{y-3}^{2}=0,$$  a chord $AB$ is drawn and extended to a point $\vec{M}$ such that $AM=2AB$. The equation of the locus of $\vec{M}$ is 
%
	\hfill\brak{1986}
\item If a circle passes through the points of intersection of the coordinate axes with the lines $\lambda x-y+1=0$ and $x-2y+3=0$,  then the value of $\lambda =\rule{1cm}{0.01pt}.$
%
	\hfill\brak{1991}
\item The equation of the locus of the mid-points of the circle $4x^{2}+4y^{2}-12x+4y+1=0$ that subtend an angle of $\frac{2\pi}{3}$ at its centre is \rule{1cm}{0.01pt}.
%
	\hfill\brak{1993}
\item The intercept of the line $y=x$ by the circle $x^{2}+y^{2}-2x=0$ is $AB$. Equation of the circle with $AB$ as a diameter is \rule{1cm}{0.01pt}.
%
	\hfill\brak{1996}
\item For each natural number $k$,  let $C_k$ denote the circle with radius $k$ centimetres and centre at the origin. On the circle $C_k$,  a particle moves $k$ centimetres in the counter-clockwise direction. After completing its motion on $C_k$,  the particle moves to $C_{k+1}$ in the radial direction. The motion of the particle continues in this manner. The particle starts at $\brak{1, 0}$. If the particle crosses the positive direction of the $X$ axis for the first time on the circle $C_n$, then $n=$\rule{1cm}{0.01pt}.
%
	\hfill\brak{1997}
\item The straight line $2x-3y=1$ divides the circular region $x^2+y^2\leq6$ into two parts.\\
If  S  is \{ $\brak{2, 3/4}, \brak{5/2, 3/4}, \brak{1/4, -1/4}, \brak{1/8, 1/4}$ \}  then the  number of point(s) in S lying inside the smaller part is  \rule{1cm}{0.01pt}.\hfill(2011)
\item For how many values of $p$,  the circle $x^2+y^2+2x+4y-p=0$ and the coordinate axes have exactly three common points? \hfill( 2017)
\item If the chord $y=mx+1$ of the circle $x^2+y^2=1$ subtends an angle of measure \( 45^\circ \) at the major segment of the circle then the value of $m$ is \hfill(2002)
\begin{multicols}{4}
\begin{enumerate}
\item$2\pm\sqrt{2}$
\item$-2\pm\sqrt{2}$
\item$-1\pm\sqrt{2}$
\item none of this
\end{enumerate}
\end{multicols}
\item The centres of a set of circles, each of radius $3$,  lie on the circle $x^2+y^2=25$.The locus of any point in the set is \hfill(2002)
\begin{multicols}{2}
\begin{enumerate}
\item$4$ $\leq$$x^2+y^2$ $\leq$ $64$
\item$x^2+y^2\leq25$
\item$x^2+y^2\geq25$
\item$3$ $\leq$ $x^2+y^2$ $\leq$ $9$
\end{enumerate}
\end{multicols}
\item The centre of the circle passing through $\brak{0, 0}$ and $\brak{1, 0}$ and touching the circle $x^2+y^2=9$ is \hfill(2002)
\begin{multicols}{4}
\begin{enumerate}
\item$\brak{1/2, 1/2}$
\item$\brak{1/2, -\sqrt{2}}$
\item$\brak{3/2, 1/2}$
\item$\brak{1/2, 3/2}$
\end{enumerate}
\end{multicols}
\item The equation of a circle with origin as a centre and passing through equilateral triangle whose median is of length $3a$ is \hfill(2002)
\begin{multicols}{2}
\begin{enumerate}
\item$x^2+y^2=9a^2$
\item$x^2+y^2=16a^2$
\item$x^2+y^2=4a^2$
\item$x^2+y^2=a^2$
\end{enumerate}
\end{multicols}
\item The lines $2x-3y=5$ and $3x-4y=7$ are diameters of a circle having area as 154 sq.units. Then the equation of the circle is
\hfill{(2003)}
\begin{multicols}{2}
\begin{enumerate}
\item $x^2+y^2-2x+2y=62$
\item $x^2+y^2+2x-2y=62$
\item $x^2+y^2+2x-2y=47$
\item $x^2+y^2-2x+2y=47$
\end{enumerate}
\end{multicols}
\item A variable circle passes through the fixed point $\vec{A}\brak{p, q}$ and touches the $X$ axis. The locus of the other end of the diameter through $\vec{A}$ is
\hfill{(2004)}
\begin{multicols}{2}
\begin{enumerate}
\item $\brak{y-q}^2=4px$
\item $\brak{x-q}^2=4py$ 
\item $\brak{y-p}^2=4qx$
\item $\brak{x-p}^2=4qy$
\end{enumerate}
\end{multicols}
\item If the lines $2x+3y+1=0$ and $3x-y-4=0$ lie along diameter of a circle of circumference $10\pi$,  then the equation of the circle is 
\hfill{(2004)}
\begin{multicols}{2}
\begin{enumerate}
\item $x^2+y^2+2x-2y-23=0$
\item $x^2+y^2-2x-2y-23=0$
\item $x^2+y^2+2x+2y-23=0$
\item $x^2+y^2-2x+2y-23=0$
\end{enumerate}
\end{multicols}
\item If the lines $3x-4y-7=0$ and $2x-3y-5=0$ are two diameters of a circle of area $49\pi$ square units,  the equation of the circle is 
\hfill{(2006)}
\begin{multicols}{2}
\begin{enumerate}
\item $x^2+y^2+2x-2y-47=0$
\item $x^2+y^2+2x-2y-62=0$
\item $x^2+y^2-2x+2y-62=0$
\item $x^2+y^2-2x+2y-47=0$
\end{enumerate}
\end{multicols}
\item Let $\vec{C}$ be the circle with centre \brak{0, 0} and radius 3 units. The equation of the locus of the mid points of the chords of the circle $\vec{C}$ that subtend an angle of $\frac{2\pi}{3}$ at its centre is
\hfill{(2006)}
\begin{multicols}{4}
\begin{enumerate}
\item $x^2+y^2=\frac{3}{2}$
\item $x^2+y^2=1$
\item $x^2+y^2=\frac{27}{4}$
\item $x^2+y^2=\frac{9}{4}$
\end{enumerate}
\end{multicols}
\item The point diametrically opposite to the point $\vec{P}\brak{1, 0}$ on the circle $x^2+y^2+2x+2y-3=0$ is 
\hfill{(2008)}
\begin{multicols}{4}
\begin{enumerate}
\item \brak{3, -4}
\item \brak{-3, 4}
\item \brak{-3, -4}
\item \brak{3, 4}
\end{enumerate}
\end{multicols}
    \item A square is inscribed in the circle $x^{2} + y^{2} - 2x +4y +3= 0.$ Its sides are parellel to the coordinate axes. Then one vertex of the square is \hfill {(1980)}
    \begin{multicols}{2}
\begin{enumerate}
    		\item $\brak{1+\sqrt{2},  -2}$ 
    		\item $\brak{1-\sqrt{2},  -2}$
    		\item $\brak{1,  -2 +\sqrt{2}}$
    		\item none of these
    	\end{enumerate}
    \end{multicols}
    \item The locus of the midpoint of a chord of the circle $x^{2}+y^{2}=4$ which subtends a right angle at the origin is \hfill {(1984)}
    \begin{multicols}{2}
\begin{enumerate}
    	\item $x+y=2$
    	\item $x^{2}+y^{2}=1$
    	\item $x^{2}+y^{2}=2$
    	\item $x+y=1$
    \end{enumerate}
\end{multicols}
    \item The lines $2x-3y=5$ and $3x-4y=7$ are diameters of a circle of area $154$ sq. units. The equation of this circle is\hfill {(1989)}
    \begin{multicols}{2}
\begin{enumerate}
    	\item $x^{2}+y^{2}+2x-2y=62$
    	\item $x^{2}+y^{2}+2x-2y=47$
    	\item $x^{2}+y^{2}-2x+2y=47$
    	\item $x^{2}+y^{2}-2x+2y=62$
    \end{enumerate}
\end{multicols}
    \item The centre of the circle passing through the points $\brak{0, 0}$, $\brak{1, 0}$ and touching the circle $x^{2}+y^{2}=9$ is
    \hfill {(1992 )}
    \begin{multicols}{2}
\begin{enumerate}
    	\item $\brak{\frac{3}{2}, \frac{1}{2}}$
    	\item $\brak{\frac{1}{2}, \frac{3}{2}}$
    	\item $\brak{\frac{1}{2}, -2\frac{1}{2}}$
    	\item none of these
    \end{enumerate}
\end{multicols}
    \item The locus of the centre of a circle,  which touches the circle is $x^{2}+y^{2}-6x-6y+14=0$ and also touches the y-axis,  is given by the equation: \hfill {(1993 )}
    \begin{multicols}{2}
\begin{enumerate}
    	\item $x^{2}-6x-10y+14=0$
    	\item $x^{2}-10x-6y+14=0$
    	\item $y^{2}-6x-10y+14=0$
    	\item $y^{2}-10x-6y+14=0$
    \end{enumerate}
\end{multicols}
    \item If two distinct chords,  drawn from the point $\brak{p, q}$ on the circle $x^{2}+y^{2}=px+qy$ (where $pq \neq 0$) are bisected by the $X$ axis,  then which are true
    \hfill {(1999 )}
    \begin{multicols}{2}
\begin{enumerate}
    	\item $p^{2}=q^{2}$
    	\item $p^{2}=8q^{2}$ 
    	\item $p^{2}<8q^{2}$
    	\item $p^{2}>8q^{2}$
    \end{enumerate}
\end{multicols}
        \item The triangle $PQR$ is inscribed in the circle $x^2+y^2=25$. If $\vec{Q}$ and $\vec{R}$ have co-ordinates $\brak{3, 4}$ and $\brak{-4, 3}$ respectively, then $\angle$QPR is equal to  
    \hfill$\brak{2000}$
%
%
    \begin{multicols}{2}
\begin{enumerate}
%
        \item $\frac{\pi}{2}$
        \item $\frac{\pi}{3}$
        \item $\frac{\pi}{4}$
        \item $\frac{\pi}{6}$
    \end{enumerate}
    \end{multicols}
    \item Let $AB$ be a chord of the circle $x^2+y^2=r^2$ subtending a right angle at the centre. Then the locus of the centroid of the triangle $PAB$  as $\vec{P}$ moves on the circle is 
        \hfill$\brak{2001}$
        \begin{multicols}{2}
\begin{enumerate}
        \item a parabola
        \item a circle
        \item an ellipse
        \item a pair of straight lines
        \end{enumerate}
        \end{multicols}
     \item If one of the diameters of the circle $x^2+y^2-2x-6y+6=0$ is a chord to the circle with the centre $\brak{2, 1}$, then the radius of the circle is 
         \hfill$\brak{2004}$
         \begin{multicols}{2}
     \begin{multicols}{4}
\begin{enumerate}
         \item $\sqrt3$
         \item $\sqrt2$
         \item 3
         \item 2
     \end{enumerate}
\end{multicols}
     \end{multicols}
     \item A circle is given by $x^2+$\brak{y-1}$^2=1$,  another circle $C$ touches it externally and also the $X$ axis,  then the locus of its centre is
         \hfill$\brak{2005}$
     \begin{multicols}{2}
\begin{enumerate}
         \item \{$\brak{x, y}$:$x^2=4y$\} $\bigcup$ \{$\brak{x, y}$:y$\le$0\}
         \item \{$\brak{x, y}$:$x^2+(y-1)^2=4$\} $\bigcup$ \{$\brak{x, y}$:y$\le$0\}
         \item \{$\brak{x, y}$:$x^2=y$\} $\bigcup$ \{$\brak{0, y}$:y$\le$0\}
         \item \{$\brak{x, y}$:$x^2=4$y\} $\bigcup$ \{$\brak{0, y}$:y$\le$0\}
         \end{enumerate}
         \end{multicols}
             \item The circle passing through the point $\brak{-1, 0}$ and touching the y-axis at $\brak{0, 2}$ also passes through the point
                 \hfill$\brak{2011}$
             \begin{multicols}{4}
\begin{enumerate}
                 \item $\brak{-\frac{3}{2}, 0}$
                 \item $\brak{-\frac{5}{2}, 2}$
                 \item $\brak{-\frac{3}{2}, \frac{5}{2}}$
                 \item $\brak{-4, 0}$
             \end{enumerate}
\end{multicols}
\item $ABCD$ is a square of side length $2$ units. $C_1$ is the circle touching all the sides of the square $ABCD$ and $C_2$ is the $circumcircle$ of square $ABCD$. $L$ is a fixed line in same plane and $\vec{R}$ is a fixed point.
\hfill(2006)
\begin{enumerate}
\item If $\vec{P}$ is any point of $C_1$ and $\vec{Q}$ is another point on $C_2$, then $\frac{PA^2+PB^2+PC^2+PD^2}{QA^2+QB^2+QC^2+QD^2}$
%
\begin{multicols}{4}
\begin{enumerate}
\item $0.75$
\item $1.25$
\item $1$
\item $0.5$
\end{enumerate}
\end{multicols}
\item If a circle is such that it touches the line $L$ and the circle $C_1$ externally, such that both the circles are on the same side of the line, then locus of centre of the circle 
%
\begin{multicols}{4}
\begin{enumerate}
\item ellipse
\item hyperbola
\item parabola
\item circle
\end{enumerate}
\end{multicols}

\item A line $L'$ through $\vec{A}$ is drawn parallel to $BD$. Point $S$ moves such that its distances from the line $BD$ and the vertex $\vec{A}$ are equal. If locus of $S$ cuts $L'$ at $T_2$ and $T_3$ and $AC$ at $T_1$, then area of $\Delta T_1T_2T_3$ is
%
\begin{multicols}{4}
\begin{enumerate}
\item $1/2$ sq.units
\item $2/3$ sq.units
\item $1$ sq.units
\item $2$ sq.units
\end{enumerate}
\end{multicols}
\end{enumerate}
%
\item A circle $C$ of radius $1$ unit is inscribed in an equilateral triangle $PQR$. The points of contact of $C$ with sides $PQ, QR, RP$ are $\vec{D}, \vec{E}, \vec{F}$ respectively. The line $PQ$ is given by the equation $\sqrt{3}x+y-6=0$ and the point $\vec{D}$ is $\brak{3\sqrt{3}/2,  3/2}$. Further, it is given that the origin and the centre of $C$ are on same side of line $PQ$.
%
The equation of circle $C$ is\hfill(2008)
\begin{multicols}{2}
\begin{enumerate}
\item $(x-2\sqrt{3})^2 + (y-1)^2=1$
\item $(x-2\sqrt{3})^2 + (y+1/2)^2=1$
\item $(x-\sqrt{3})^2 + (y-1)^2=1$
\item $(x-\sqrt{3})^2 + (y+1)^2=1$
\end{enumerate}
\end{multicols}
\item Find the equation of the circle whose radius is 5 and which touches the circle $x^2+y^2-2x-4y-20=0$ at the point \brak{5, 5}.
%
\hfill {\brak{1978}}
\item Through a fixed point \brak{h, k} secants are drawn to the circle $x^2+y^2=r^2$. Show that the locus of the mid-points of the secants intercepted is $x^2+y^2=hx+ky$.
%
\hfill {\brak{1983 }}
\item The abscissa of two points $\vec{A}$ and $\vec{B}$ are roots of the equation $x^2+2ax-b^2=0$ and their ordinates are roots of the equation $x^2+2px-q^2=0$. Find the equation and the radius of the circle with $AB$ as diameter.
%
\hfill {\brak{1984 }}
\item Let a given Line $L_1$ intersects the $X$ and $Y$ axes at $\vec{P}$ and $\vec{Q}$ respectively. Let another line $L_2$,  perpendicular to $L_1$,  cut the $X$ and $Y$ axes at $\vec{R}$ and $\vec{S}$,  respectively. Show that the locus of the point of intersection of $PS$ and $QR$ is a circle passing through origin.
%
\hfill {\brak{1987}}
\item The circle $x^2+y^2-4x-y+4=0$ is inscribed in a triangle which has two of its sides along the co-ordinate axes. The locus of circumcentre of the triangle is $x+y-xy+k(x^2+y^2)\textsuperscript{1/2}$. Find $k$.
%
\hfill {\brak{1987 }}
\item If $\brak{ m_i,  \frac{1}{m_i}},  m_i > 0,  i = 1,  2,  3,  4$ are four distinct points on a circle,  then show that $m_1m_2m_3m_4=1$
%
\hfill {\brak{1989 }}
	\item Let a circle be given by $2x\brak{x-a}+y\brak{2y-b}=0$, $\brak{a\neq0, b\neq0}$. Find the condition on $a$ and $b$ if two chords,  each bisected by the $X$ axis, can be drawn to the circle from $\brak{a, \frac{b}{2}}$.                         
%
\hfill(1992)
%
%
%
%
\item Consider a family of circles passing through two fixed points $\vec{A}$$\brak{3, 7}$ and $\vec{B}$$\brak{6, 5}$. Show that chords in which the circle $x^2+y^2-4x-6y-3=0$ cuts the members of the family are concurrent at a point. Find the coordinate of this point.
%
\hfill(1993)
%
%
%
%
%
%
\item Find the intervals of values of $a$ for which the line $y+x=0$ bisects two chords drawn from a point $\brak {\frac{1+\sqrt{2}a}{2}, \frac{1-\sqrt{2}a}{2}}$ to the circle $2x^2+2y^2-\brak{1+\sqrt{2}a}x-\brak{1-\sqrt{2}a}y=0$.  
%
\hfill(1996)
%
\item A circle passes through three points $\vec{A}$, $\vec{B}$ and C with the line segment $AC$ as its diameter. A line passing through $\vec{A}$ intersects the chord $BC$ at point $\vec{D}$ inside the circle. If angles $DAB$ and $CAB$ are $\alpha$ and $\beta$ respectively and the distance between the point $\vec{A}$ and midpoint of the line segment $DC$ is $d$,  prove that the area of the circle is $\frac{\pi d^2 \cos^2{\alpha} }{\cos^2{\alpha}+\cos^2{\beta}+ 2\cos{\alpha} \cos{\beta} \cos{\brak{\beta-\alpha}}}$.        
%

\hfill(1996)
%
%
%
%
%
\item Let $C$ be any circle with centre $\brak{0, \sqrt{2}}$. Prove that at the most two rational points can be there on $C$. (A rational point is a point both of whose coordinates are rational numbers)
%
\hfill(1997)
%
%
%
%
%
%
%
%
%
%
%
 \item Let $\vec{O}$ be the centre of the circle $x^2 + y^2 = r^2$, where $r>\frac{\sqrt{5}}{2}$
. Suppose $PQ$ is a chord of this circle and the equation of the line passing through $\vec{P}$ and $\vec{Q}$ is $2x + 4y = 5$. If the centre of the circumcircle of the triangle $OPQ$ lies on the lies on the line $x + 2y = 4$, then the value of $r$ is \rule{1cm}{0.1pt}.
\hfill (2020) 
 \item Consider a triangle $\Delta$ whose two sides lie on the $X$ axis and the line $x + y + 1 = 0$. If the orthocenter of $\Delta$ is $(1, 1)$, then the equation of the circle passing through the vertices of the triangle $\Delta$ is
    \hfill (2021)
\begin{multicols}{2}
\begin{enumerate} 
	 \item  $x^2 + y^2 - 3x + y = 0$  
         \item  $x^2 + y^2 + x + 3y = 0$  
         \item  $x^2 + y^2 + 2y - 1 = 0$  
         \item  $x^2 + y^2 + x + y = 0$
    \end{enumerate}
\end{multicols}
\end{enumerate}
