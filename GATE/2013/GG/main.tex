        \item A singular value of an $m \times n$ matrix, $\vec{A}$, is defined as \hfill{\brak{\text{GG 2013}}}
            \begin{enumerate}
%                \begin{multicols}{2}
                    \item positive square root of eigenvalue of $\vec{A}\vec{A}^{\top}$
                    \item modulus of eigenvalue of $\vec{A}$
                    \item eigenvalue of $\vec{A}^{\top}\vec{A}$
                    \item square of eigenvalue of $\vec{A}$
%                \end{multicols}
            \end{enumerate}
In a linear inverse problem, the coefficient matrix, $\vec{A} = \myvec{ 2.00 & 2.01 \\ 2.01 & 2.00 }$.

        \item The eigenvalues of $\vec{A}$ are \hfill{\brak{\text{GG 2013}}}
            \begin{enumerate}
                \begin{multicols}{4}
                    \item $\brak{4.01\,, -0.01}$
                    \item $\brak{-4.01\,,-0.01}$
                    \item $\brak{4.01\,,0.01}$
                    \item $\brak{-4.01\,,1.01}$
                \end{multicols}
            \end{enumerate}
        
        \item If the elements of $\vec{A}$ are expressed up to first decimal place only, then the number of possible solution(s) of the resulting inverse problem is \hfill{\brak{\text{GG 2013}}}
            \begin{enumerate}
                \begin{multicols}{4}
                    \item $1$
                    \item $2$
                    \item $3$
                    \item $\infty$
                \end{multicols}
            \end{enumerate}

