    \item The possible set of eigenvalues of a $4 \times 4$ skew-symmetric orthogonal real matrix is
    \hfill (MA 2013)
    \begin{enumerate}
    \begin{multicols}{4}
        \item $\{\pm i\}$
        \item $\{\pm i, \pm 1\}$
        \item $\{\pm 1\}$
        \item $\{0, \pm i\}$
        \end{multicols}
    \end{enumerate}
%2
    \item Let $\vec{P}$ be a $2 \times 2$ complex matrix such that trace$(\vec{P})=1$ and det$(\vec{P})=-6$. Then, trace$(\vec{P}^4 - \vec{P}^3)$ is \underline{\hspace{1cm}}.
    \hfill (MA 2013)
    \item Let V be the real vector space of all polynomials in one variable with real coefficients and having degree at most 20. Define the subspaces 
	    \begin{align*}
W_1 &= \cbrak{p \in V : p(1)=0, p(1/2)=0, p(5)=0, p(7)=0}, \\
    W_2 &= \cbrak{p \in V : p(1/2)=0, p(3)=0, p(4)=0, p(7)=0}. 
	    \end{align*}
    Then the dimension of $W_1 \cap W_2$ is \underline{\hspace{1cm}}.
    \hfill (MA 2013)
%27
    \item Let M be the real vector space of $2 \times 3$ matrices with real entries. Let $T: M \rightarrow M$ be defined by 
	    $$T{\myvec{ x_1 & x_2 & x_3 \\ x_4 & x_5 & x_6 }} = \myvec{ -x_6 & x_4 & x_1 \\ x_3 & x_5 & x_2 }.$$
    The determinant of T is \underline{\hspace{1cm}}.
    \hfill (MA 2013)
%32
    \item The matrix $\vec{A} = \myvec{ 1 & 2 & 0 \\ 1 & 3 & 1 \\ 0 & 1 & 3 }$ can be decomposed uniquely into the product $\vec{A}=\vec{L}\vec{U}$, where $\vec{L} = \myvec{ 1 & 0 & 0 \\ l_{21} & 1 & 0 \\ l_{31} & l_{32} & 1 }$ and $\vec{U} = \myvec{ u_{11} & u_{12} & u_{13} \\ 0 & u_{22} & u_{23} \\ 0 & 0 & u_{33} }$. The solution of the system $\vec{L}\vec{X} = \myvec{1 & 2 & 2}^{\top}$ is
    \hfill (MA 2013)
    \begin{enumerate}
    \begin{multicols}{4}
        \item $\myvec{1&1&1}^{\top}$
        \item $\myvec{1&1 &0}^{\top}$
        \item $\myvec{0 &1& 1}^{\top}$
        \item $\myvec{1 &0& 1}^{\top}$
    \end{multicols}
    \end{enumerate}
    \item Let $\vec{B}$ be a real symmetric positive-definite $n \times n$ matrix. Consider the inner product on $\mathbb{R}^n$ defined by $\brak{ \vec{X}, \vec{Y} } = \vec{Y}^{\top} \vec{B}\vec{X}$. Let $\vec{A}$ be an $n \times n$ real matrix and let $T: \mathbb{R}^n \rightarrow \mathbb{R}^n$ be the linear operator defined by $T(\vec{X}) = \vec{A}\vec{X}$ for all $\vec{X} \in \mathbb{R}^n$. If S is the adjoint of T, then $S(\vec{X}) = \vec{C}\vec{X}$ for all $\vec{X} \in \mathbb{R}^n$, where $\vec{C}$ is the matrix
    \hfill (MA 2013)
    \begin{enumerate}
    \begin{multicols}{4}
        \item $\vec{B}^{-1}\vec{A}^{\top} \vec{B}$
        \item $\vec{B} \vec{A}^{\top} \vec{B}^{-1}$
        \item $\vec{B}^{-1}\vec{A}\vec{B}$
        \item $\vec{A}^{\top}$
    \end{multicols}
    \end{enumerate}
    \item Let M be the space of all $4 \times 3$ matrices with entries in the finite field of three elements. Then the number of matrices of rank three in M is
    \hfill (MA 2013)
    \begin{enumerate}
        \item $(3^4-3)(3^4-3^2)(3^4-3^3)$
        \item $(3^4-1)(3^4-2)(3^4-3)$
        \item $(3^4-1)(3^4-3)(3^4-3^2)$
        \item $3^4(3^4-1)(3^4-2)$
    \end{enumerate}
%42
    \item Let $V$ be a vector space of dimension $m \ge 2$. Let $T: V \rightarrow V$ be a linear transformation such that $T^{n+1}=0$ and $T^n \neq 0$ for some $n \ge 1$. Then which of the following is necessarily TRUE?
    \hfill (MA 2013)
    \begin{enumerate}
    \begin{multicols}{2}
        \item Rank$(T^n) \le$ Nullity$(T^n)$
        \item trace$(T) \neq 0$
        \item T is diagonalizable
        \item $n=m$
    \end{multicols}
    \end{enumerate}
