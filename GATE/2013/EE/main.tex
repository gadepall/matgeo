\item The equation 
\[
\myvec{
2 & -2 \\ -1 & 1
}
\myvec{
x_1 \\ x_2
}
=
\myvec{
1 \\ -1
}
\]
has  
\hfill (EE 2013)
\begin{multicols}{2}
\begin{enumerate}
\item No solution  
\item Only one solution 
\item Non-zero unique solution  
\item Multiple solutions  
\end{enumerate}
\end{multicols}
\item A matrix has eigenvalues $-1$ and $-2$. The corresponding eigenvectors are 
$\myvec{ 1 \\ -1 }, 
\myvec{ 1 \\ -2 }
$
respectively. The matrix is

\hfill (EE 2013)
\begin{enumerate}
\begin{multicols}{4}
\item \myvec{ 1 & 1 \\ -1 & -2}
\item \myvec{ 1 & 2 \\ -2 & -4 }
\item \myvec{ 0 & -2 \\ -1 & 0 }
\item \myvec{ 0 & 1 \\ -1 & -3 }
\end{multicols}
\end{enumerate}
\item A function 
\begin{align*}
    y = 5x^2 + 10x
\end{align*}
is defined over an open interval $x=(1,2)$. At least at one point in this interval, $\frac{dy}{dx}$ is exactly
\hfill (EE 2013)
\begin{enumerate}
\begin{multicols}{4}
\item 20
\item 25
\item 30
\item 35
\end{multicols}
\end{enumerate}
\item The set of values of $p$ for which the roots of the equation
\begin{align*}
    3x^{2} + 2x + p(p-1) = 0
\end{align*}
are of opposite sign is
\hfill (EE 2013)
\begin{enumerate}
\begin{multicols}{4}
\item $(-\infty,\,0)$
\item $(0,\,1)$
\item $(1,\,\infty)$
\item $(0,\,\infty)$
\end{multicols}
\end{enumerate}

