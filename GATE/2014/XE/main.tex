\begin{enumerate}
    \item If $1, 0$, and $-1$ are the eigenvalues of a $3\times3$ matrix $\vec{A}$, then the trace of $\vec{A}^2 + 5\vec{A}$ is equal to \underline{\hspace{2cm}}.
    \hfill{\brak{\text{XE 2014}}}
    \item If a cubic polynomial passes through the points $(0, 1)$, $(1, 0)$, $(2, 1)$ and $(3, 10)$, then it also passes through the point
    \hfill{\brak{\text{XE 2014}}}
    \begin{enumerate}
        \begin{multicols}{4}
            \item $(-2, -11)$
            \item $(-1, -2)$
            \item $(-1, -4)$
            \item $(-2, -23)$
        \end{multicols}
    \end{enumerate}
    \item The perimeter of a rectangle having the largest area that can be inscribed in the ellipse $$\frac{x^2}{8} + \frac{y^2}{32} = 1$$ is \underline{\hspace{2cm}}.
    \hfill{\brak{\text{XE 2014}}}
    \item If the work done in moving a particle once around a circle $x^2 + y^2 = 4$ under the force field $\vec{F}(x,y) = (2x-ay)\hat{i} + (2y+ax)\hat{j}$ is $16\pi$, then $|a|$ is equal to \underline{\hspace{2cm}}.

    \hfill{\brak{\text{XE 2014}}}
    \item Let $r$ and $s$ be real numbers. If $\vec{A}= \myvec{1 & 2 & 0 \\ 2 & 0 & 3 \\ r & s & 0}$ and $\vec{b} = \myvec{1 \\ 1 \\ s-1}$, then the system of linear equations $\vec{A}\vec{X} =\vec{b}$ has
    \hfill{\brak{\text{XE 2014}}}
    \begin{enumerate}
        \item no solutions for $s \neq 2r$.
        \item infinitely many solutions for $s = 2r \neq 2$.
        \item a unique solution for $s = 2r = 2$.
        \item infinitely many solutions for $s = 2r = 2$.
    \end{enumerate}
\end{enumerate}
