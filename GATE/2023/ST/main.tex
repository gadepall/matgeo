\item The area of the region bounded by the parabola $x = -y^2$ and the line $y = x + 2$ equals 

	\hfill(ST 2023) 
	\begin{multicols}{4}
\begin{enumerate}
    \item $\frac{3}{2}$
    \item $\frac{7}{2}$
    \item $\frac{9}{2}$
    \item $9$
\end{enumerate}
\end{multicols}
% Q12
\item Let $\vec{A}$ be a $3 \times 3$ real matrix having eigenvalues $1, 0, -1$. If $\vec{B} = \vec{A}^2 + 2\vec{A} + \vec{I}_3$, where $\vec{I}_3$ is the $3\times 3$ identity matrix, then which one of the following statements is true? 

	\hfill(ST 2023) 
	\begin{multicols}{2}
\begin{enumerate}
    \item $\vec{B}^3 - 5\vec{B}^2 + 4\vec{B} = 0$
    \item $\vec{B}^3 - 5\vec{B}^2 - 4\vec{B} = 0$
    \item $\vec{B}^3 + 5\vec{B}^2 - 4\vec{B} = 0$
    \item $\vec{B}^3 + 5\vec{B}^2 + 4\vec{B} = 0$
\end{enumerate}
\end{multicols}
\item Consider the following statements
	\begin{enumerate}[label=(\Roman*)]
\item Let $\vec{A}$ and $\vec{B}$ be two $n\times n$ real matrices. If $\vec{B}$ is invertible, then $\mathrm{rank}(\vec{B}\vec{A}) = \mathrm{rank}(\vec{A})$.
\item Let $\vec{A}$ be an $n\times n$ real matrix. If $\vec{A}^2 \vec{x} = \vec{b}$ has a solution for every $\vec{b} \in \mathbb{R}^n$, then $\vec{A}\vec{x} = \vec{b}$ also has a solution for every $\vec{b} \in \mathbb{R}^n$.
\end{enumerate}
Which of the above statements is/are true? \hfill(ST 2023) 
	\begin{multicols}{2}
\begin{enumerate}
    \item Only (I)
    \item Only (II)
    \item Both (I) and (II)
    \item Neither (I) nor (II)
\end{enumerate}
\end{multicols}
\item Let $\vec{A}$ be $2\times 2$ real matrix such that $\vec{A}\vec{B}=\vec{B}\vec{A}$ for all $2\times2$ real $\vec{B}$. If $\mathrm{trace}(\vec{A})$ equals $5$, then $\det(\vec{A})$  equals \underline{\hspace{3cm}}\hfill(ST 2023)
\item Let $\vec{A}$ be an $n\times n$ real matrix. Consider the following statements
	\begin{enumerate}[label=(\Roman*)]
		\item If $\vec{A}$ is symmetric, then there exists $c \ge 0$ such that $\vec{A} + c \vec{I}_n$ is symmetric and positive definite, where $\vec{I}_n$ is the $n\times n$ identity matrix. 
		\item  If $\vec{A}$ is symmetric and positive definite, then there exists a symmetric and positive definite $\vec{B}$ such that $\vec{A} = \vec{B}^2$. 
	\end{enumerate}
Which of the above statements is/are true?\hfill(ST 2023)
	\begin{multicols}{2}
\begin{enumerate}
\item Only (I)
\item Only (II)
\item Both (I) and (II)
\item Neither (I) nor (II)
\end{enumerate}
\end{multicols}
\item For subsets $\mathcal{T}, \mathcal{S} \subset \mathbb{R}^n$, let $L(U)$ denote their span. Which is NOT true?\hfill(ST 2023)
\begin{enumerate}
\item If $\mathcal{T}$ is proper subset of $\mathcal{S}$ then $L(\mathcal{T})$ is a proper subset of $L(\mathcal{S})$
\item $L(L(\mathcal{S})) = L(\mathcal{S})$
\item $L(\mathcal{T} \cup \mathcal{S}) = \cbrak{\vec{u}+\vec{v} : \vec{u} \in L(\mathcal{T}), \vec{v} \in L(\mathcal{S})}$
\item If $\bm{\alpha},\bm{\beta}$ and $\bm{\gamma}$ are three vectors in $R^n$ such that $\bm{\alpha}+2\bm{\beta}+3\bm{\gamma}=0$, then $L(\cbrak{\bm{\alpha},\bm{\beta}}) = L(\cbrak{\bm{\beta},\bm{\gamma}})$
\end{enumerate}
\item Let $\vec{A}$ be a $3\times 3$ real matrix such that
	\[
\vec{A}\myvec{0\\1\\1\\} = \myvec{4\\0\\0\\},
\vec{A}\myvec{1\\0\\1\\} = \myvec{0\\4\\0\\} \text{ and }
\vec{A}\myvec{1\\1\\0\\} = \myvec{0\\0\\4\\}.
\]
Which of the following statements is/are true?\hfill(ST 2023)
	\begin{multicols}{4}
\begin{enumerate}
\item $\vec{A}\myvec{1\\0\\0\\} = \myvec{2\\2\\-2\\}$
\item $\vec{A}\myvec{0\\1\\0\\} = \myvec{2\\-2\\2\\}$
\item $\vec{A}\myvec{1\\1\\1\\} = \myvec{2\\0\\2\\}$
\item $\vec{A}\myvec{1\\2\\3\\} = \myvec{8\\4\\0\\}$
\end{enumerate}
	\end{multicols}
\item Consider the orthonormal set 
\[
\vec{v}_1 = \begin{cases}
\myvec{ \frac1{\sqrt3}\\[2pt] -\frac1{\sqrt3}\\[2pt] \frac1{\sqrt3} }, \quad
\vec{v}_2 = \myvec{ \frac1{\sqrt6}\\[2pt] \frac2{\sqrt6}\\[2pt] \frac1{\sqrt6} },
\quad
\vec{v}_3 = \myvec{ \frac1{\sqrt2}\\[2pt] 0\\[2pt] -\frac1{\sqrt2} }.
\end{cases}
\]
with respect to the standard innner product on $R^3$.
If $\vec{u} = \myvec {a\\b\\c}$ is the vector such that inner products of $\vec{u}$ with $\vec{v}_1$,$\vec{v}_2$ and $\vec{v}_3$, respectively, then $a^2 + b^2 + c^2$  equals \underline{\hspace{1cm}}

\hfill(ST 2023)

