\item Let $\vec{T} : \mathbb{R}^{3}\to \mathbb{R}^{3}$ be a linear transformation satisfying
\[
\vec{T}\brak{1,0,0}=\brak{0,1,1},\qquad
\vec{T}\brak{1,1,0}=\brak{1,0,1},\qquad
\vec{T}\brak{1,1,1}=\brak{1,1,2}.
\]
Then
\hfill{\brak{\text{MA 2023}}}
\begin{enumerate}
\begin{multicols}{2}
\item $\vec{T}$ is one\mbox{-}one but $\vec{T}$ is NOT onto
\item $\vec{T}$ is one\mbox{-}one and onto
\item $\vec{T}$ is NEITHER one\mbox{-}one NOR onto
\item $\vec{T}$ is NOT one\mbox{-}one but $\vec{T}$ is onto
\end{multicols}
\end{enumerate}
%
\item Let $\vec{M}=\myvec{4 & -3\\ 1 & 0}$. Consider the following statements
\begin{enumerate}[label=\Alph*:,start=16]
	\item  $\vec{M}^{8}+\vec{M}^{12}$ is diagonalizable.
	\item  $\vec{M}^{7}+\vec{M}^{9}$ is diagonalizable.
\end{enumerate}
Which of the following statements is correct?
\hfill{\brak{\text{MA 2023}}}
\begin{multicols}{2}
\begin{enumerate}
\item P is TRUE and Q is FALSE
\item P is FALSE and Q is TRUE
\item Both P and Q are FALSE
\item Both P and Q are TRUE
\end{enumerate}
\end{multicols}
\item Let $\vec{T} : \mathbb{R}^{4}\to \mathbb{R}^{4}$ be a linear transformation and the null space of $\vec{T}$ be the subspace of $\mathbb{R}^{4}$ given by
\[
\{\, (x_{1},x_{2},x_{3},x_{4})\in \mathbb{R}^{4} : 4x_{1}+3x_{2}+2x_{3}+x_{4}=0 \,\}.
\]
If $\operatorname{Rank}\!\brak{\vec{T}-3\vec{I}}=3$, where $\vec{I}$ is the identity map on $\mathbb{R}^{4}$, then the minimal polynomial of $\vec{T}$ is
\hfill{\brak{\text{MA 2023}}}
\begin{enumerate}
\begin{multicols}{4}
\item $x\brak{x-3}$
\item $x\brak{x-3}^{3}$
\item $x^{3}\brak{x-3}$
\item $x^{2}\brak{x-3}^{2}$
\end{multicols}
\end{enumerate}
\item Consider $\mathbb{R}^{4}$ with the inner product $\langle x,y\rangle=\sum_{i=1}^{4}x_{i}y_{i}$, for $x=\brak{x_{1},x_{2},x_{3},x_{4}}$ and $y=\brak{y_{1},y_{2},y_{3},y_{4}}$.  
Let $\vec{M}=\{\, \brak{x_{1},x_{2},x_{3},x_{4}}\in \mathbb{R}^{4} : x_{1}=x_{3}\,\}$ and let $\vec{M}^{\perp}$ denote the orthogonal complement of $\vec{M}$. The dimension of $\vec{M}^{\perp}$ is equal to \underline{\hspace{2cm}}.

\hfill{\brak{\text{MA 2023}}}
\item Let
\[
\vec{M}=\myvec{3 & -1 & -2\\ 0 & 2 & 4\\ 0 & 0 & 1}
\quad \text{and} \quad
\vec{I}=\myvec{1 & 0 & 0\\ 0 & 1 & 0\\ 0 & 0 & 1}.
\]
If $6\vec{M}^{-1}=\vec{M}^{2}-6\vec{M}+\alpha \vec{I}$ for some $\alpha \in \mathbb{R}$, then the value of $\alpha$ is equal to \underline{\hspace{2cm}}.

\hfill{\brak{\text{MA 2023}}}
\item Consider the linear system $\vec{M}\vec{x}=\vec{b}$, where 
\[
\vec{M}=\myvec{2 & -1\\ -4 & 3} \quad \text{and} \quad \vec{b}=\myvec{-2\\ 5}.
\]
Suppose $\vec{M}=\vec{L}\vec{U}$, where $\vec{L}$ and $\vec{U}$ are lower triangular and upper triangular square matrices, respectively. Consider the following statements
\begin{enumerate}[label=\Alph*:,start=16]
	\item  If each element of the main diagonal of $\vec{L}$ is $1$, then $\text{trace}\brak{\vec{U}}=3$.
	\item  For any choice of the initial vector $\vec{x}^{\brak{0}}$, the Jacobi iterates $\vec{x}^{\brak{k}},\ k=1,2,3,\ldots$ converge to the unique solution of the linear system $\vec{M}\vec{x}=\vec{b}$.
\end{enumerate}
Then
\hfill{\brak{\text{MA 2023}}}
\begin{multicols}{2}
\begin{enumerate}
\item both P and Q are TRUE
\item P is FALSE and Q is TRUE
\item P is TRUE and Q is FALSE
\item both P and Q are FALSE
\end{enumerate}
\end{multicols}
\item Let $\vec{A}$ be a $3\times 3$ real matrix with $\det\brak{\vec{A}+i\vec{I}}=0$, where $i=\sqrt{-1}$ and $\vec{I}$ is the $3\times 3$ identity matrix. If $\det\brak{\vec{A}}=3$, then the trace of $\vec{A}^{2}$ is \underline{\hspace{2cm}}.
\hfill{\brak{\text{MA 2023}}}
\item Let $\vec{A}=\brak{a_{ij}}$ be a $3\times 3$ real matrix such that
\[
\vec{A}\myvec{1\\2\\1}=2\myvec{1\\2\\1},\qquad
\vec{A}\myvec{0\\1\\1}=2\myvec{0\\1\\1}\ \text{ and }\ 
\vec{A}\myvec{-1\\1\\0}=4\myvec{-1\\1\\0}.
\]
If $m$ is the degree of the minimal polynomial of $\vec{A}$, then $a_{11}+a_{21}+a_{31}+m$ equals \underline{\hspace{2cm}}.
\hfill{\brak{\text{MA 2023}}}
