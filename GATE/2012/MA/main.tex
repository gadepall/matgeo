\item
The straight lines $L_1: x = 0$, $L_2: y = 0$, and $L_3: x + y = 1$ are mapped by the transformation $w = z^2$ into the curves $C_1$, $C_2$, and $C_3$ respectively. The angle of intersection between the curves at $w = 0$ is
\hfill\brak{\text{MA 2012}}
\begin{multicols}{4}
\begin{enumerate}
  \item $0$
  \item $\dfrac{\pi}{4}$
  \item $\dfrac{\pi}{2}$
  \item $\pi$
\end{enumerate}
\end{multicols}

\item
Let $H$ be a complex Hilbert space, $T: H \to H$ a bounded linear operator and $T^*$ its adjoint. Which of the following statements are true?
\hfill{\brak{\text{MA 2012}}}
\begin{enumerate}[label=\Alph*., start=16]
\item $\brak{ Tx, y } = \brak{ x, T^*y }$ for all $x, y \in H$
\item $\brak{ x, Ty } = T \brak{ x, y }$ for all $x, y \in H$
\item $\brak{ x, Ty } = \brak{ x, T^*y }$ for all $x, y \in H$
\item $\brak{ Tx, Ty } = T^* \brak{ x, Ty }$ for all $x, y \in H$
\end{enumerate}
\begin{multicols}{4}
\begin{enumerate}
  \item P and Q
  \item P and R
  \item Q and S
  \item P and S
\end{enumerate}
\end{multicols}
\item
Let $\alpha = e^{2 \pi i/5}$ and the matrix
\begin{align*}
\vec{M} = \myvec{
\alpha & \alpha^2 & \alpha^3 & 4 \\
\alpha^2 & \alpha^3 & \alpha^4 & 0 \\
0 & 0 & 0 & 0 \\
0 & 0 & 0 & 0
}
\end{align*}
Then the trace of the matrix $\vec{I} + \vec{M} + \vec{M}^2$ is
\hfill\brak{\text{MA 2012}}
\begin{multicols}{4}
\begin{enumerate}
  \item $5$
  \item $0$
  \item $3$
  \item $-5$
\end{enumerate}
\end{multicols}
\item
Let $V = \mathbb{C}^2$ be the vector space over the complex numbers and $B = \{(1, i), (i, 1)\}$ be a given ordered basis of $V$. Then which of the following pairs $\{f_1, f_2\}$ is a dual basis of $B$ over $\mathbb{C}$?
\hfill\brak{\text{MA 2012}}
\begin{multicols}{2}
\begin{enumerate}
  \item $f_1(z_1, z_2) = \frac{z_1 - z_2}{2}, \quad f_2(z_1, z_2) = \frac{z_1 + z_2}{2}$
  \item $f_1(z_1, z_2) = \frac{z_1 + z_2}{2}, \quad f_2(z_1, z_2) = \frac{z_1 + z_2 i}{2}$
  \item $f_1(z_1, z_2) = \frac{z_1 - z_2}{2}, \quad f_2(z_1, z_2) = \frac{z_1 - z_2 i}{2}$
  \item $f_1(z_1, z_2) = \frac{z_1 + z_2}{2}, \quad f_2(z_1, z_2) = \frac{z_1 - z_2}{2}$
\end{enumerate}
\end{multicols}
\item
If 
\begin{align*}
\vec{A} = \myvec{
1 & 0 & 0 \\
1 & 0 & 1 \\
0 & 1 & 0
},
\end{align*}
then $\vec{A}^{50}$ is
\hfill\brak{\text{MA 2012}}
\begin{multicols}{4}
\begin{enumerate}
  \item $\myvec{
1 & 0 & 0 \\
50 & 1 & 0 \\
50 & 0 & 1
}$
  \item $\myvec{
1 & 0 & 0 \\
48 & 1 & 0 \\
48 & 0 & 1
}$
  \item $\myvec{
1 & 0 & 0 \\
25 & 1 & 0 \\
25 & 0 & 1
}$
  \item $\myvec{
1 & 0 & 0 \\
24 & 1 & 0 \\
24 & 0 & 1
}$
\end{enumerate}
\end{multicols}
\item
Let the linear transformation 
\begin{align*}
T : F^2 \to F^3
\end{align*}
be defined by
\begin{align*}
T(x_1, x_2) = (x_1 + x_2, x_1, x_2).
\end{align*}
Then the nullity of $T$ is
\hfill\brak{\text{MA 2012}}
\begin{multicols}{4}
\begin{enumerate}
  \item $0$
  \item $1$
  \item $2$
  \item $3$
\end{enumerate}
\end{multicols}
\item
The approximate eigenvalue of the matrix
\begin{align*}
\vec{A} = \myvec{
15 & 4 & 3 \\
10 & 12 & 6 \\
20 & 4 & 2
}
\end{align*}
obtained after two iterations of the Power method, with the initial vector $(1,1,1)^{\top}$, is
\rule{1cm}{0.01pt}.
\hfill\brak{\text{MA 2012}}
\iffalse
\begin{multicols}{4}
\begin{enumerate}
  \item $7.768$
  \item $9.468$
  \item $10.548$
  \item $19.468$
\end{enumerate}
\end{multicols}
\fi
\item
For the matrix
\begin{align*}
    \vec{M} = \myvec{
	    2 &3+2i & -4 \\ 3- 2i & 5 & 6i \\ -4 & -6i & 3
}
\end{align*}
which of the following statements are correct?
\hfill\brak{\text{MA 2012}}
\begin{enumerate}[label=\Alph*., start=16]
\item $\vec{M}$ is skew-Hermitian and $i\vec{M}$ is Hermitian
\item $\vec{M}$ is Hermitian and $i\vec{M}$ is skew-Hermitian
\item Eigenvalues of $\vec{M}$ are real
\item Eigenvalues of $i\vec{M}$ are real
\end{enumerate}
\begin{multicols}{4}
\begin{enumerate}
\item P and R only
\item Q and R only
\item P and S only
\item Q and S only
\end{enumerate}
\end{multicols}
\item
Let $T: P_3 \to P_3$ be the map defined by
\begin{align*}
    T(p)(x) = \int_0^x p(t) \, dt.
\end{align*}
If the matrix of $T$ relative to the standard basis $\cbrak{1, x, x^2, x^3}$ is $\vec{M}$ and $\vec{M}^{\top}$ denotes transpose of $\vec{M}$, then $\vec{M} + \vec{M}^{\top}$ is
\hfill\brak{\text{MA 2012}}
\begin{multicols}{4}
\begin{enumerate}
\item $\myvec{
0 & 1 & 1 & 1 \\
1 & 2 & 0 & 0 \\
0 & 0 & 2 & 0 \\
0 & 0 & 0 & 2
}$
\item $\myvec{
1 & 0 & 0 & 2 \\
0 & 1 & 1 & 0 \\
0 & 1 & 2 & 0 \\
2 & 0 & 0 & 1
}$
\item $\myvec{
2 & 0 & 0 & 1 \\
0 & 2 & 1 & 0 \\
1 & 1 & 0 & 1 \\
0 & 1 & -1 & 0
}$
\item $\myvec{
0 & 2 & 2 & 2 \\
2 & 1 & 0 & 0 \\
2 & 0 & 1 & 0 \\
2 & 0 & 0 & 1
}$
\end{enumerate}
\end{multicols}
\item
System of equations
\begin{align*}
    \myvec{
5 & 1 & 1 \\
10 & 2 & 4 \\
0 & 12 & 1
}
\myvec{x \\ y \\ z} =
\myvec{1 \\ 1 \\ 5}.
\end{align*}
Using Jacobis method with initial guess $\myvec{2.0 \\ 3.0 \\ 0.0}$, the approximate solution after two iterations is
\hfill\brak{\text{MA 2012}}
\begin{multicols}{4}
\begin{enumerate}
\item $\myvec{2.64 \\ -1.70 \\ -1.12}$
\item $\myvec{2.64 \\ 1.70 \\ -1.12}$
\item $\myvec{2.64 \\ 1.70 \\ 1.12}$
\item $\myvec{2.64 \\ -1.70 \\ 1.12}$
\end{enumerate}
\end{multicols}

