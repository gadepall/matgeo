\item The solution to the purely under-determined problem \(\vec{G}\vec{m}=\vec{d}\) is given by
\hfill{\brak{\text{GG 2012}}}

\begin{multicols}{1}
\begin{enumerate}
\item \((\vec{G}^{\top}\vec{G})^{-1}\vec{G}^{\top}\vec{d}\)
\item \((\vec{G}^{\top}\vec{G})^{-1}\vec{G}\,\vec{d}^{\top}\)
\item \(\vec{G}^{\top}(\vec{G}\vec{G}^{\top})^{-1}\vec{d}\)
\item \(\vec{G}^{\top}\vec{d}\,(\vec{G}\vec{G}^{\top})^{-1}\)
\end{enumerate}
\end{multicols}
\item Given the following matrix equation:
\(\vec{A}_{m\times n}\,\vec{X}_{x\times 1}=\vec{b}_{m\times 1}\), the nature of this system of equations is
\hfill{\brak{\text{GG 2012}}}
\begin{multicols}{2}
\begin{enumerate}
\item over-determined if \(m>n\)
\item under-determined if \(m<n\)
\item even-determined if \(m=n\)
\item determined by the rank of the matrix \(\vec{A}\)
\end{enumerate}
\end{multicols}
