\item Let $P_1$ and $P_2$ be two projection operators on a vector space. 
Then 

\hfill{(MA 2007)}
\begin{enumerate}    
    \item $P_1 + P_2$ is a projection if $P_1 P_2 = P_2 P_1 = 0$
    \item $P_1 - P_2$ is a projection if $P_1 P_2 = P_2 P_1 = 0$
    \item $P_1 + P_2$ is a projection
    \item $P_1 - P_2$ is a projection
\end{enumerate}
\item Consider the system of linear equations         
\begin{align*}
x + y + z &= 3 ,\\ 
x - y - z &= 4 , \\
-5y + kz &= 6
\end{align*}
Then the value of $k$ for which this system has an infinite number of solutions is

\hfill{(MA 2007)}
\begin{multicols}{4}
\begin{enumerate}    
    \item $k = -5$
    \item $k = 0$
    \item $k = 1$
    \item $k = 3$
\end{enumerate}
\end{multicols}
%
\item Let     
$$\vec{A} = \myvec{
1 & 1 & 1 \\
2 & 2 & 3 \\
x & y & z}$$
and let $V = \cbrak{\brak{x, y, z} \in \mathbb{R}^3 : \det\brak{\vec{A}} = 0}$. Then the dimension of $V$ equals

        \hfill{(MA 2007)}
\begin{multicols}{4}
\begin{enumerate}    
    \item $0$
    \item $1$
    \item $2$
    \item $3$
\end{enumerate}
\end{multicols}
\item Let
\begin{align*}
S = \cbrak{\brak{0,1,1}, \brak{1,0,1}, \brak{-1,2,1}} \subseteq \mathbb{R}^3.
\end{align*}
Suppose $\mathbb{R}^3$ is endowed with the standard inner product. Define 
\begin{align*}
M = \cbrak{\vec{x} \in \mathbb{R}^3 :  \vec{x}^\top \vec{y}  = 0 \forall  \vec{y} \in S}
\end{align*}
Then the dimension of $M$ equals \hfill{(MA 2007)}
\begin{multicols}{4}
\begin{enumerate}    
    \item $0$
    \item $1$
    \item $2$
    \item $3$ 
\end{enumerate}
\end{multicols}
%
\item A basis of the vector space 
\begin{align*}
W = \cbrak{\brak{x,y,z,w} \in \mathbb{R}^4 : x + y + z = 0, y + z + w = 0, 2x + y - z + w = 0}
\end{align*}
is  \hfill{(MA 2007)}
\begin{multicols}{2}
\begin{enumerate}    
    \item $\cbrak{\brak{1,1,1,1}, \brak{2,1,1,1}}$
    \item $\cbrak{\brak{1,-1,0,1}, \brak{0,1,-1,0}}$
    \item $\cbrak{\brak{1,0,-1,0}, \brak{2,1,1,1}}$
    \item $\cbrak{\brak{1,0,-1,0}, \brak{0,1,-1,0}}$
\end{enumerate}
\end{multicols}
\item Consider $\mathbb{R}^3$ with the standard inner product. Let
\begin{align*}
 S = \cbrak{\brak{1,1,1}, \brak{2,-1,2}, \brak{-1,2,1}}   
\end{align*}
%
For a subset $W$ of $\mathbb{R}^3$, let $L\brak{W}$ denote the linear span of $W$ in $\mathbb{R}^3$. Then an orthonormal set $T$ with $L\brak{S} = L\brak{T}$ is  \hfill{(MA 2007)}
    \begin{multicols}{2}
\begin{enumerate}    
        \item $\left\{ \frac{1}{\sqrt{3}}\brak{1,1,1}, \frac{1}{\sqrt{6}}\brak{1,0,-2}, \frac{1}{\sqrt{2}}\brak{1,-1,0} \right\}$ 
        \item $\{\brak{0,0,0}, \brak{0,1,0}, \brak{0,0,1}\}$
        \item $\left\{ \frac{1}{\sqrt{3}}\brak{1,1,1}, \frac{1}{\sqrt{2}}\brak{1,0,-1} \right\}$        \item $\left\{ \frac{1}{\sqrt{3}}\brak{1,1,1}, \frac{1}{\sqrt{2}}\brak{1,-1,0} \right\}$
    \end{enumerate}
    \end{multicols}
\item Let $\vec{A}$ be a $3 \times 3$ matrix. Suppose that the eigenvalues of $\vec{A}$ are $-1, 0, 1$ with respective eigenvectors $\brak{1, -1, 0}^\top$, $\brak{1, 1, -2}^\top$ and $\brak{1, 1, 1}^\top$. Then $6\vec{A}$ equals

\hfill{(MA 2007)}
\begin{multicols}{4}
\begin{enumerate}    
        \item $\myvec{ -1 & 5 & 2 \\ 5 & -1 & 2 \\ 2 & 2 & -1 }$
        \item $\myvec{ 1 & 0 & 0 \\ 0 & -1 & 0 \\ 0 & 0 & 0 }$
        \item $\myvec{ 1 & 5 & 3 \\ 5 & 1 & 3 \\ 3 & 3 & 3 }$
        \item $\myvec{ -3 & 9 & 0 \\ 9 & -3 & 0 \\ 0 & 0 & 6}$ 
    \end{enumerate}
    \end{multicols}
\item Let $T:\mathbb{R}^3 \rightarrow \mathbb{R}^3$ be a linear transformation defined by
\begin{align*}
T\brak{\brak{x, y, z}} = \brak{x + y - z, x + y + z, y - z}.
\end{align*}
Then the matrix of the linear transformation $T$ with respect to the ordered basis
\begin{align*}
    B = \{\brak{0,1,0}, \brak{0,0,1}, \brak{1,0,0}\} \in \mathbb{R}^3
\end{align*}
is \hfill{(MA 2007)}
\begin{multicols}{2}
\begin{enumerate}    
        \item $\myvec{ 1 & 1 & 0 \\ 0 & -1 & 1 \\ 1 & 1 & 1 }$
        \item $\myvec{ 1 & 0 & 1 \\ 1 & 1 & 1 \\ -1 & 0 & 1 }$
        \item $\myvec{ 1 & -1 & 0 \\ 1 & -1 & 1 \\ 1 & 1 & 1  }$
        \item $\myvec{ 1 & 1 & 1 \\ -1 & 0 & 1 \\ 1 & -1 & 0 }$ 
    \end{enumerate}
\end{multicols}
\item Let $Y\brak{x} = \brak{y_1\brak{x}, y_2\brak{x}}^\top$ and let
\begin{align*}
\vec{A} = \myvec{ -3 & 1 \\ k & -1 }.
\end{align*}
Further, let $S$ be the set of values of $k$ for which all the solutions of the system of equations $Y'\brak{x} = A Y\brak{x}$ tend to zero as $x \rightarrow \infty$. 
Then $S$ is given by      

\hfill{(MA 2007)}
\begin{multicols}{2}
\begin{enumerate}    
        \item $\{k : k \leq -1\}$
        \item $\{k : k \leq 3\}$
        \item $\{k : k < -1\}$
        \item  $\{k : k < 3\}$
    \end{enumerate}
\end{multicols}
\item Let
\begin{align*}
\vec{A} = \myvec{
3 & 0 & 0 \\
0 & 6 & 2 \\
0 & 2 & 6
}
\end{align*}
and let $\lambda_1 \geq \lambda_2 \geq \lambda_3$ be the eigenvalues of $A$.
    \begin{enumerate}
\item The triple $\brak{\lambda_1, \lambda_2, \lambda_3}$ equals    \hfill{(MA 2007)}
\begin{multicols}{4}
\begin{enumerate}    
        \item $\brak{9, 4, 2}$
        \item $\brak{8, 4, 3}$
        \item $\brak{9, 3, 3}$
        \item $\brak{7, 5, 3}$
    \end{enumerate}
\end{multicols}
%
\item The matrix $\vec{P}$ such that   \hfill{(MA 2007)}
	\begin{align*}
	\vec{P}^{-1} \vec{A} \vec{P} = \myvec{
\lambda_1 & 0 & 0 \\
0 & \lambda_2 & 0 \\
0 & 0 & \lambda_3
}
\end{align*}
is 
\begin{multicols}{2}
\begin{enumerate}    
        \item  $\myvec{
    \dfrac{1}{\sqrt{3}} & 0 & \dfrac{-2}{\sqrt{6}} \\
    \dfrac{1}{\sqrt{3}} & \dfrac{1}{\sqrt{2}} & \dfrac{1}{\sqrt{6}} \\
    \dfrac{1}{\sqrt{3}} & \dfrac{-1}{\sqrt{2}} & \dfrac{1}{\sqrt{6}}
    }$
        \item $\myvec{
    \dfrac{1}{\sqrt{3}} & \dfrac{-2}{\sqrt{6}} & 0 \\
    \dfrac{1}{\sqrt{3}} & \dfrac{1}{\sqrt{6}} & \dfrac{1}{\sqrt{2}} \\
    \dfrac{1}{\sqrt{3}} & \dfrac{1}{\sqrt{6}} & \dfrac{-1}{\sqrt{2}}
    }$
        \item $\myvec{
    0 & 0 & 1 \\
    \dfrac{1}{\sqrt{2}} & \dfrac{1}{\sqrt{2}} & 0 \\
    \dfrac{1}{\sqrt{2}} & \dfrac{-1}{\sqrt{2}} & 0
   }$
        \item $\myvec{
    0 & 1 & 0 \\
    \dfrac{1}{\sqrt{2}} & 0 & \dfrac{1}{\sqrt{2}} \\
    \dfrac{1}{\sqrt{2}} & 0 & \dfrac{-1}{\sqrt{2}}
    }$
    \end{enumerate}
\end{multicols}
\end{enumerate}
