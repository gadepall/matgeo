\documentclass[12pt]{article}
\usepackage{amsmath, amssymb}
\usepackage{enumitem}
\usepackage{geometry}
\geometry{margin=1in}

\newcommand{\myvec}[1]{\begin{bmatrix}#1\end{bmatrix}}
\newcommand{\brak}[1]{\left( #1 \right)}






\begin{document}

\begin{center}
    {\LARGE \textbf{GATE -2007 MT}} \\
    \vspace{1em}
    ai25btech11009 \qquad Dasu Harshith Kumar
\end{center}

\section*{Q.1 -- Q.20: One mark each}

\begin{enumerate}

\item The number of boundary conditions required to solve a steady-state two-dimensional diffusion equation ($\nabla^2 C = 0$) is (GATE 2007 MT):
    \begin{enumerate}
        \item 1
        \item 2
        \item 3
        \item 4
    \end{enumerate}

\item The determinant of the matrix
\[
\myvec{
1 & 3 & 2 \\
2 & 6 & 4 \\
-5 & 3 & 1
}
\]

is (GATE 2007 MT):
\begin{enumerate}
    \item -10
    \item -5
    \item 0
    \item 5
\end{enumerate}


\item With $\varepsilon$ (true plastic strain) and $n$ (strain-hardening coefficient), necking in a cylindrical tensile specimen of a work-hardened metal occurs when (GATE 2007 MT):
    \begin{enumerate}
        \item $\varepsilon = n$
        \item $\varepsilon = 2n$
        \item $\varepsilon = n^{0.5}$
        \item $\varepsilon = n^2$
    \end{enumerate}

\item A perfectly plastic metal piece with cross-section $4\,\mathrm{mm} \times 4\,\mathrm{mm}$ and length $25\,\mathrm{mm}$ is stretched to $100\,\mathrm{mm}$. The deformed cross-section is (GATE 2007 MT):
    \begin{enumerate}
        \item $1\,\mathrm{mm} \times 1\,\mathrm{mm}$
        \item $2\,\mathrm{mm} \times 2\,\mathrm{mm}$
        \item $3\,\mathrm{mm} \times 3\,\mathrm{mm}$
        \item $4\,\mathrm{mm} \times 4\,\mathrm{mm}$
    \end{enumerate}

\item Loading in Mode I fracture refers to (GATE 2007 MT):
    \begin{enumerate}
        \item Opening mode
        \item Sliding mode
        \item Tearing mode
        \item Twisting mode
    \end{enumerate}

\item Cyclones are primarily used for (GATE 2007 MT):
    \begin{enumerate}
        \item Comminution
        \item Concentration
        \item Dewatering
        \item Classification
    \end{enumerate}

\item A typical collector used in sulphide flotation is (GATE 2007 MT):
    \begin{enumerate}
        \item Pine oil
        \item Potassium ethyl xanthate
        \item Oleic acid
        \item Polyacrylamide
    \end{enumerate}

\item In a three component system at constant pressure, the maximum number of phases that can coexist at equilibrium is (GATE 2007 MT):
    \begin{enumerate}
        \item 2
        \item 3
        \item 4
        \item 5
    \end{enumerate}

\item Metal extracted by leaching is (GATE 2007 MT):
    \begin{enumerate}
        \item Iron
        \item Aluminum
        \item Lead
        \item Gold
    \end{enumerate}

\item In niobium micro-alloyed steel joined by fusion welding, loss of strength in heat affected zone (HAZ) is due to (GATE 2007 MT):
    \begin{enumerate}
        \item Precipitate coarsening and grain growth
        \item Coarse pearlite and grain boundary precipitation
        \item Tempered martensite and grain boundary carbide
        \item Bainite formation
    \end{enumerate}

\item Primary heat source in cupola melting is (GATE 2007 MT):
    \begin{enumerate}
        \item $C + O_2 \rightarrow CO_2$
        \item $C + H_2O \rightarrow CO + H_2$
        \item $C + CO_2 \rightarrow 2CO$
        \item $CaCO_3 \rightarrow CaO + CO_2$
    \end{enumerate}

\item Solder wire does not work harden at room temperature because (GATE 2007 MT):
    \begin{enumerate}
        \item Dislocations become immobilized
        \item Grains grow preferentially
        \item Recrystallization temperature below room temperature
        \item Grains have preferred orientation
    \end{enumerate}

\item Typical cooling rate during atomization is (GATE 2007 MT):
    \begin{enumerate}
        \item $10^{4}$ K/s
        \item 1 K/s
        \item $10^{4}$ K/s
        \item $10^{0}$ K/s
    \end{enumerate}

\item Fluidity of alloy does not increase with (GATE 2007 MT):
    \begin{enumerate}
        \item Superheat
        \item Channel size
        \item Flow velocity
        \item Heat transfer coefficient
    \end{enumerate}

\item In polymers, mass averaged molecular weight is (GATE 2007 MT):
    \begin{enumerate}
        \item Greater than number averaged molecular weight
        \item Smaller than number averaged molecular weight
        \item Equal to number averaged molecular weight
        \item Unrelated to number averaged molecular weight
    \end{enumerate}

\item Alloy system with complete solid solubility is (GATE 2007 MT):
    \begin{enumerate}
        \item Cu-Ni
        \item Fe-Cu
        \item Pb-Sn
        \item Cu-Zn
    \end{enumerate}

\item Small addition of thoria in tungsten filaments (GATE 2007 MT):
    \begin{enumerate}
        \item Decreases diffusivity
        \item Enhances boundary mobility
        \item Increases segregation
        \item Limits grain growth
    \end{enumerate}

\item Activity of carbon with respect to graphite in iron is (GATE 2007 MT):
    \begin{enumerate}
        \item 0.5
        \item 0.85
        \item 1.0
        \item 1.5
    \end{enumerate}

\item Number of interstitial sites in FCC unit cell are (GATE 2007 MT):
    \begin{enumerate}
        \item 4 tetrahedral and 8 octahedral
        \item 8 tetrahedral and 4 octahedral
        \item 12 tetrahedral and 4 octahedral
        \item 4 tetrahedral and 4 octahedral
    \end{enumerate}

\item Dimension of thermal conductivity is (GATE 2007 MT):
    \begin{enumerate}
        \item $ML^2T^{-3}\Theta^{-1}$
        \item $MT^{-3}\Theta^{-1}$
        \item $L^2T^{-1}$
        \item $MLT^{-3}\Theta$
    \end{enumerate}

\end{enumerate}

\begin{enumerate}

\setcounter{enumi}{20} % Continue numbering from 21

\item The configurational entropy \(S_c\) in an ideal solid solution is given by
\begin{align*}
S = -R \brak{ x \ln x + (1-x) \ln (1-x) },
\end{align*}

where \(x\) is the mole fraction of solute. The limit \(x \to 0\) of \(S_c\) is (GATE 2007 MT):
\begin{enumerate}
  \item \(\infty\)
  \item \(R \ln 2\)
  \item \(R\)
  \item 0
\end{enumerate}

\item The directions \([100]\) and \(\) in a cubic crystal are coplanar with (GATE 2007 MT):
\begin{enumerate}
  \item \(\)
  \item \([1]\)
  \item \(\)
  \item \(\)
\end{enumerate}

\item In a RH degasser, the hydrogen mass balance is given by
\[
- W \frac{dC_H}{dt} = R (C_H - C_{H,eq}),
\]
where \(W=150\) tons, \(C_H\) is hydrogen concentration at time \(t\), \(C_{H,eq}=0.5\) ppm, \(R\) is circulation rate in tons/min. To reduce hydrogen from 5 ppm to 1 ppm in 20 min, the required \(R\) is (GATE 2007 MT):
\begin{enumerate}
  \item 10.05
  \item 12.51
  \item 14.73
  \item 16.48
\end{enumerate}

\item Given \(\mathbf{V} = (4xy - 3z^3)\mathbf{i} + 2x^2 \mathbf{j} - 9xz^2 \mathbf{k}\), the divergence of \(\mathbf{V}\) is (GATE 2007 MT):
\begin{enumerate}
  \item \(4xy - 18z\)
  \item \(4y^2 - 9xz\)
  \item \(4y - 18xz\)
  \item \(2xy + 18 z^2\)
\end{enumerate}

\item The carbon concentration profile in decarburization is given by:
\[
C(x,t) = L + M \operatorname{erf}\left(\frac{x}{2\sqrt{Dt}}\right),
\]
where \(x=0.5\,mm\), \(C_{initial} = 1.2\%\), \(C_{final} = 0.8\%\), \(D = 1.28 \times 10^{-11} m^2/s\). The approximate time \(t\) is (GATE 2007 MT):
\begin{enumerate}
  \item 30 hours
  \item 3 hours
  \item 3 minutes
  \item 30 seconds
\end{enumerate}

\item The probability distribution function is \(p(x) = \frac{1}{\sqrt{\pi}} e^{-x^2}\). Using the trapezoidal rule, the probability that \(x\) lies between 0.6 and 0.8 is (GATE 2007 MT):
\begin{enumerate}
  \item 0
  \item 0.069
  \item 0.138
  \item 0.56
\end{enumerate}

\item In the TTT diagram for eutectoid steel, the delay at temperatures above 550°C arises due to (GATE 2007 MT):
\begin{enumerate}
  \item low driving force
  \item low mobility of dislocations
  \item low vacancy concentration
  \item low diffusivity
\end{enumerate}

\item A cylindrical specimen is elastically deformed; initial length \(100\,mm\), diameter \(10\,mm\), final length \(100.1\,mm\), diameter \(9.996\,mm\). Using \(E=200\,GPa\), calculate the shear modulus \(G\) (\(E=2G(1+\nu)\)) (GATE 2007 MT):
\begin{enumerate}
  \item 71.43 GPa
  \item 76.92 GPa
  \item 83.33 GPa
  \item 100.00 GPa
\end{enumerate}

\item Surface energy of a brittle material is doubled without change in modulus. The approximate increase in fracture strength is (GATE 2007 MT):
\begin{enumerate}
  \item 100%
  \item 73%
  \item 50%
  \item 41%
\end{enumerate}

\item Matching exercise: Match fracture mechanisms with fracture surface morphologies (GATE 2007 MT):

Groups: (P) ductile fracture, (Q) brittle fracture, (R) fatigue fracture

Morphologies: (1) cleavage, (2) dimples, (3) striations, (4) veins

\begin{enumerate}
  \item P-4, Q-2, R-3
  \item P-2, Q-1, R-3
  \item P-2, Q-3, R-1
  \item P-4, Q-3, R-2
\end{enumerate}

\item Settling velocity of a \(0.5 \mu m\) diameter particle (density \(4900\,kg/m^3\)) in water (\(1000\,kg/m^3\)), viscosity \(1\,cP\) is (GATE 2007 MT):
\begin{enumerate}
  \item \(53.08 \times 10^{-6}\, m/s\)
  \item \(40 \times 10^{-6}\, m/s\)
  \item \(53.08 \times 10^0\, m/s\)
  \item \(106.16 \times 10^{-6}\, m/s\)
\end{enumerate}

\item Calculation: The recovery of gold in given flotation data (GATE 2007 MT):
\begin{enumerate}
  \item 8.25%
  \item 22.26%
  \item 85.80%
  \item 91.80%
\end{enumerate}

\item For a pulp with 65% wt solids and particle specific gravity 2.7, volume % solids is (GATE 2007 MT):
\begin{enumerate}
  \item 72.9%
  \item 65%
  \item 59.3%
  \item 40.7%
\end{enumerate}

\item Superplastic deformation depends on which factors (GATE 2007 MT):

P. Extremely fine and uniform grain size  
Q. High homologous temperature  
R. High strain rate  
S. Coarse non-uniform grain size

\begin{enumerate}
  \item P, Q
  \item Q, R, S
  \item P, Q, R
  \item Q, R
\end{enumerate}

\item For hot rolling: Roll grooves are made along the roll axis to achieve (GATE 2007 MT):

P. Increase bite angle  
Q. Decrease rolling load  
R. Achieve larger reduction  
S. Decrease roll flattening

\begin{enumerate}
  \item P, Q
  \item Q, R
  \item P, R
  \item Q, S
\end{enumerate}
\setcounter{enumi}{40} % If not already set after Q40—this makes next question 41

\item The atomic packing factor for the diamond cubic structure is (GATE 2007 MT):
  \begin{enumerate}[label=(\Alph*)]
    \item 0.74
    \item 0.68
    \item 0.34
    \item 0.25
  \end{enumerate}

\item The maximum amount of proeutectoid austenite that can form in 3.5\% C steel (max solubility of C in $\gamma = 2.11$\%) is (GATE 2007 MT):
  \begin{enumerate}[label=(\Alph*)]
    \item 24.80\%
    \item 36.53\%
    \item 67.87\%
    \item 72.52\%
  \end{enumerate}

\item Identify the incorrect statement with reference to LD steel making (GATE 2007 MT):
  \begin{enumerate}[label=(\Alph*)]
    \item The temperature of the LD furnace is maintained at around 1600$^\circ$C
    \item The basicity of slag is maintained at around unity
    \item Dephosphorization and decarburization proceed simultaneously
    \item High silicon hot metal may lead to slopping
  \end{enumerate}

% --- 44 Table Matching ---

\item Match the processes in Group I with products in Group II (GATE 2007 MT)\\[1ex]
\begin{table}[h!]
\centering
\caption{Process-Product Matching}
\label{tab:q44_matching}
\begin{tabular}{|c|l||c|l|}
\hline
\multicolumn{2}{|c||}{\textbf{Group I (Processes)}} & \multicolumn{2}{c|}{\textbf{Group II (Products)}} \\
\hline
P & Czochralski process & 1 & Single crystal of GaAs \\
Q & Calendaring         & 2 & Hypoeutectic Al-Si alloy \\
R & Pultrusion          & 3 & Vinyl floor tile \\
S & Thixocasting        & 4 & Polymer matrix composite \\
\hline
\end{tabular}
\end{table}

  \begin{enumerate}[label=(\Alph*)]
    \item P-1, Q-2, R-3, S-4
    \item P-1, Q-3, R-4, S-2
    \item P-4, Q-1, R-3, S-2
    \item P-4, Q-3, R-1, S-2
  \end{enumerate}

% --- 45 Table Matching ---

\item Match the applications in Group I with materials in Group II (GATE 2007 MT)\\[1ex]
\begin{table}[h!]
\centering
\caption{Application-Material Matching}
\label{tab:q45_matching}
\begin{tabular}{|c|l||c|l|}
\hline
\multicolumn{2}{|c||}{\textbf{Group I (Applications)}} & \multicolumn{2}{c|}{\textbf{Group II (Materials)}} \\
\hline
P & Electric motor cores     & 1 & $\gamma$-Fe$_2$O$_3$ particles \\
Q & Credit card stripe       & 2 & Barium titanate \\
R & Permanent magnet         & 3 & Co$_5$Sm intermetallic \\
S & Multilayer capacitors    & 4 & Grain-oriented silicon steel \\
\hline
\end{tabular}
\end{table}

  \begin{enumerate}[label=(\Alph*)]
    \item P-3, Q-1, R-2, S-4
    \item P-4, Q-1, R-3, S-2
    \item P-4, Q-1, R-2, S-3
    \item P-2, Q-1, R-3, S-4
  \end{enumerate}

% --- 46 Table Matching ---

\item Match the phases in Group I with the descriptions in Group II (GATE 2007 MT)\\[1ex]
\begin{table}[h!]
\centering
\caption{Phase-Description Matching}
\label{tab:q46_matching}
\begin{tabular}{|c|l||c|l|}
\hline
\multicolumn{2}{|c||}{\textbf{Group I (Phases)}} & \multicolumn{2}{c|}{\textbf{Group II (Descriptions)}} \\
\hline
P & $\varepsilon$-Carbide         & 1 & Eutectic in cast iron \\
Q & Sigma phase                   & 2 & Embrittling in stainless steel \\
R & $\delta$-Ferrite              & 3 & Tempered carbide \\
S & Steadite                      & 4 & Magnetic weld deposit \\
\hline
\end{tabular}
\end{table}

  \begin{enumerate}[label=(\Alph*)]
    \item P-1, Q-3, R-4, S-2
    \item P-1, Q-2, R-4, S-3
    \item P-3, Q-2, R-1, S-4
    \item P-3, Q-2, R-4, S-1
  \end{enumerate}

% --- 47 Table Matching ---

\item Match the materials in Group I with the bond types in Group II (GATE 2007 MT)\\[1ex]
\begin{table}[h!]
\centering
\caption{Material-Bond Type Matching}
\label{tab:q47_matching}
\begin{tabular}{|c|l||c|l|}
\hline
\multicolumn{2}{|c||}{\textbf{Group I (Materials)}} & \multicolumn{2}{c|}{\textbf{Group II (Bond Types)}} \\
\hline
P & Silicon          & 1 & Metallic \\
Q & Copper           & 2 & Covalent \\
R & Sodium chloride  & 3 & Ionic \\
  &                  & 4 & van der Waals \\
\hline
\end{tabular}
\end{table}

  \begin{enumerate}[label=(\Alph*)]
    \item P-2, Q-1, R-3
    \item P-1, Q-3, R-4
    \item P-4, Q-1, R-3
    \item P-2, Q-1, R-4
  \end{enumerate}

% Continue numbering from the last question
\setcounter{enumi}{47}

\item The activation energy for a reaction is 100 kJ/mol. The approximate increase in temperature required to double the reaction rate (from 25$^\circ$C) is (GATE 2007 MT):  
  \begin{enumerate}[label=(\Alph*)]
    \item 5$^\circ$C
    \item 10$^\circ$C
    \item 15$^\circ$C
    \item 20$^\circ$C
  \end{enumerate}

\item For the reaction $2\mathrm{Fe} + \frac{3}{2}\mathrm{O}_2 = \mathrm{Fe}_2\mathrm{O}_3$ with standard free energy $\Delta G^\circ$, the approximate pressure for dissociation of Fe$_2$O$_3$ at 1100$^\circ$C is (GATE 2007 MT):  
  \begin{enumerate}[label=(\Alph*)]
    \item $1.0 \times 10^{-20}$ atm
    \item $1.46 \times 10^{-12}$ atm
    \item $2.3 \times 10^{7}$ atm
    \item $3.55 \times 10^{-15}$ atm
  \end{enumerate}

\item Identify the incorrect statement related to unit processes in extractive metallurgy (GATE 2007 MT):  
  \begin{enumerate}[label=(\Alph*)]
    \item Selective distillation is a purification technique used in extractive metallurgy
    \item Coking of coal is carried out in a shaft furnace
    \item Precipitation is a hydrometallurgy route of purification
    \item Predominance area diagram is used to select operating conditions of roasting
  \end{enumerate}

\item Identify the correct statement with reference to blast furnace iron making (GATE 2007 MT):  
  \begin{enumerate}[label=(\Alph*)]
    \item Hematite is reduced to magnetite in the lower part of the shaft
    \item Coke rate cannot be improved by oil injection through tuyere
    \item High exit gas temperature may indicate "channeling"
    \item Pressure drop in blast furnace cannot be improved by proper burden distribution
  \end{enumerate}

% --- Q52: Matching type (table) ---

\item Match the processes in Group I with their descriptions in Group II (GATE 2007 MT):\\[1ex]
\begin{table}[h!]
\centering
\caption{Process-Description Matching}
\label{tab:q52_matching}
\begin{tabular}{|c|l||c|l|}
\hline
\multicolumn{2}{|c||}{\textbf{Group I (Processes)}} & \multicolumn{2}{c|}{\textbf{Group II (Descriptions)}} \\
\hline
P & COREX process        & 1 & Decarburization of liquid steel \\
Q & OBM process          & 2 & Steelmaking using oxygen \\
R & Carbonyl process     & 3 & Nickel refining \\
S & AOD process          & 4 & Alternative route of liquid iron \\
\hline
\end{tabular}
\end{table}
  \begin{enumerate}[label=(\Alph*)]
    \item P-4, Q-2, R-3, S-1
    \item P-3, Q-1, R-2, S-4
    \item P-1, Q-4, R-3, S-2
    \item P-2, Q-1, R-3, S-4
  \end{enumerate}

\item The process of cementation involves (GATE 2007 MT):  
  \begin{enumerate}[label=(\Alph*)]
    \item Separation of the desired metal by adding a more reactive metal
    \item Elimination of a more reactive metal by preferential oxidation
    \item Refining by preferential dissolution in organic solvent
    \item Extraction by selective dissolution in inorganic solvent
  \end{enumerate}

\item Identify the incorrect statement (GATE 2007 MT):  
  \begin{enumerate}[label=(\Alph*)]
    \item A concentration gradient in the electrolyte may lead to galvanic cell formation
    \item Chromium is added to improve oxidation resistance of stainless steels
    \item Cathodic protection can be provided by applying a coating
    \item A steel bolt or nut is permissible on a large copper vessel
  \end{enumerate}

\item Nanoparticles have a higher surface atom fraction $f_s$ compared to bulk $f_0$. The ratio $(f_s/f_0)$ varies with the particle size $r$ as (GATE 2007 MT):  
  \begin{enumerate}[label=(\Alph*)]
    \item $r^{-3}$
    \item $r^{-2}$
    \item $r^{-1}$
    \item $r^{2}$
  \end{enumerate}

\item The theoretical shear strength of dislocation-free single crystal aluminium ($G = 28$ GPa) is approximately (GATE 2007 MT):  
  \begin{enumerate}[label=(\Alph*)]
    \item 28.0 GPa
    \item 4.5 GPa
    \item 0.56 GPa
    \item 0.07 GPa
  \end{enumerate}

\item The equilibrium vacancy concentration in copper is 588 ppm at $1000^\circ$C and 134 ppm at $800^\circ$C. The molar enthalpy of vacancy formation is (GATE 2007 MT):  
  \begin{enumerate}[label=(\Alph*)]
    \item 49 kJ/mol
    \item 84 kJ/mol
    \item 168 kJ/mol
    \item 243 kJ/mol
  \end{enumerate}

\item In a cubic crystal, which dislocation reaction is vectorially correct and energetically feasible (GATE 2007 MT)?  
  \begin{enumerate}[label=(\Alph*)]
    \item $[\bar{1}11] + [111] \to a$
    \item $ +  \to $
    \item $ +  \to $
    \item $ \to  + $
  \end{enumerate}

\item The mechanical response of an elastomer is characterized by (GATE 2007 MT):  
  \begin{enumerate}[label=(\Alph*)]
    \item Increase in elastic modulus with increasing temperature
    \item Large recoverable strains
    \item Decrease in elastic modulus with increasing temperature
    \item Adiabatic decrease in temperature on stretching
  \end{enumerate}

\item Which of the following statements are true about edge dislocations (GATE 2007 MT)?  
  \begin{enumerate}[label=(\Alph*)]
    \item Do not have an extra half plane
    \item Burgers vector is perpendicular to the line direction
    \item Can avoid obstacles by cross-slip
    \item Parallel edge dislocations of opposite sign can attract or repel depending on geometry
  \end{enumerate}

\item A structural component in the form of a very wide 10 mm thick plate is to be fabricated from 4340 steel. If the design stress level is 50\% of the yield strength, the critical flaw size is (Yield strength = 1515 MPa, $K_{IC}=60.4$ MPa$\sqrt{\text{m}}$, Geometry factor $Y=1$) (GATE 2007 MT):  
  \begin{enumerate}[label=(\Alph*)]
    \item 1.0 mm
    \item 2.0 mm
    \item 3.0 mm
    \item 4.0 mm
  \end{enumerate}

\item The tensile yield strength of a ductile metal is 100 MPa. If the material is subjected to tensile stresses $\sigma_2 = \sigma_3 = 50$ MPa along the other principal directions, the material yields when (GATE 2007 MT):  
  \begin{enumerate}[label=(\Alph*)]
    \item $\sigma_1 = 50$ MPa in compression or 150 MPa in tension
    \item $\sigma_1 = 50$ MPa in compression or 50 MPa in tension
    \item $\sigma_1 = 100$ MPa in tension
    \item $\sigma_1 = 0$
  \end{enumerate}

% ---- Q64: Matching (table) ----

\item Match the energy gaps in Group I with the materials in Group II (GATE 2007 MT):\\[1ex]
\begin{table}[h!]
\centering
\caption{Energy Gap and Material Matching}
\label{tab:q64_matching}
\begin{tabular}{|c|l||c|l|}
\hline
\multicolumn{2}{|c||}{\textbf{Group I (Energy Gaps)}} & \multicolumn{2}{c|}{\textbf{Group II (Materials)}} \\
\hline
P & Diamond       & 1 & 0.1 eV \\
Q & Silicon       & 2 & 0.7 eV \\
R & Gray Tin      & 3 & 1.1 eV \\
                    &   & 4 & 6.0 eV \\
\hline
\end{tabular}
\end{table}
  \begin{enumerate}[label=(\Alph*)]
    \item P-1, Q-3, R-4
    \item P-2, Q-4, R-1
    \item P-3, Q-1, R-2
    \item P-4, Q-3, R-1
  \end{enumerate}

% ---- Q65: Matching (table) ---

\item Match the terms from Group I to their descriptions in Group II (GATE 2007 MT):\\[1ex]
\begin{table}[h!]
\centering
\caption{Term-Description Matching}
\label{tab:q65_matching}
\begin{tabular}{|c|l||c|l|}
\hline
\multicolumn{2}{|c||}{\textbf{Group I (Terms)}} & \multicolumn{2}{c|}{\textbf{Group II (Descriptions)}} \\
\hline
P & Hall-Petch Effect     & 1 & Solute-dislocation interaction \\
Q & Bauschinger Effect    & 2 & Dislocation multiplication \\
R & Cottrell atmosphere   & 3 & Grain boundary strengthening \\
                          &   & 4 & Barrelling under compression \\
                          &   & 5 & Mechanical hysteresis during plasticity \\
\hline
\end{tabular}
\end{table}
  \begin{enumerate}[label=(\Alph*)]
    \item P-3, Q-5, R-1
    \item P-1, Q-4, R-3
    \item P-5, Q-1, R-2
    \item P-3, Q-4, R-2
  \end{enumerate}

% ---- Q66–Q70: Normal MCQ ----

\item Enthalpy of formation at 298 K, $\Delta H$, of CO$_2$ and PbO are $-393$ kJ mol$^{-1}$ and $-220$ kJ mol$^{-1}$ respectively. The enthalpy change for $2$PbO $+$ C $\rightarrow$ $2$Pb $+$ CO$_2$ is (GATE 2007 MT):  
  \begin{enumerate}[label=(\Alph*)]
    \item $-173$ kJ
    \item $15$ kJ
    \item $47$ kJ
    \item $440$ kJ
  \end{enumerate}

\item In normalized hypoeutectoid plain carbon steels, how do the fraction of proeutectoid ferrite ($f$) and yield strength ($\sigma_y$) change with increasing carbon (GATE 2007 MT):  
  \begin{enumerate}[label=(\Alph*)]
    \item $f$ increases and $\sigma_y$ decreases
    \item both $f$ and $\sigma_y$ increase
    \item both $f$ and $\sigma_y$ decrease
    \item $f$ decreases and $\sigma_y$ increases
  \end{enumerate}

\item Identify the correct statement about manganese in steels (GATE 2007 MT):  
  \begin{enumerate}[label=(\Alph*)]
    \item it decreases hardenability
    \item it makes the steel susceptible to hot-shortness
    \item it is a strong austenite stabilizer
    \item it decreases hardness of martensite
  \end{enumerate}

\item When one mole of copper is quenched from 1000 K to 300 K, the amount of heat released is (specific heat $C_p = 22.68 + 6.3 \times 10^{-3} T$ J/mol·K):  
  \begin{enumerate}[label=(\Alph*)]
    \item 9.37 kJ
    \item 15.87 kJ
    \item 18.74 kJ
    \item 22.68 kJ
  \end{enumerate}

\item A suitable technique for monitoring a growing crack in an alloy is (GATE 2007 MT):  
  \begin{enumerate}[label=(\Alph*)]
    \item Acoustic emission
    \item Radiography
    \item Magnetic particle technique
    \item Liquid penetrant test
  \end{enumerate}

\end{enumerate}
\end{document}

