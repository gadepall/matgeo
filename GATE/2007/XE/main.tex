    \item Let $$\vec{M} =\myvec{ 1 & 1 & 1 \\ 0 & 1 &1 \\ 0 & 0 &1}.$$ Then the maximum number of linearly independent eigenvectors of $\vec{M}$ is
    \hfill{(2007 XE)}
    \begin{multicols}{4}
    \begin{enumerate}
        \item 0
        \item 1
        \item 2
        \item 3
     \end{enumerate}
     \end{multicols}
\item Let $\vec{A}\vec{x} = \vec{b}$ be a system of $m$ linear equations in $n$ unknowns with $m < n$ and $\vec{b} \neq \vec{0}$. Then the system has
    \hfill{(2007 XE)}
    \begin{multicols}{2}
    \begin{enumerate}
        \item $n - m$  solutions
        \item either zero or infinitely many solutions
        \item exactly one solution
        \item $n$  solutions
    \end{enumerate}
\end{multicols}
    \item Let $\vec{R}$ be an $n \times n $ nonsingular matrix. Let $\vec{P}$ and $\vec{Q}$ be two  $n \times n $ matrices such that $ \vec{Q} = \vec{R}^{-1}\vec{P}\vec{R} $. If $\vec{x}$  is an eigenvector of  $\vec{P}$ corresponding to a nonzero eigenvalue $\lambda$ of $\vec{P}$, then
    \hfill{(2007 XE)}
    \begin{enumerate}
        \item $\vec{Rx}$  is an eigenvector of $\vec{Q}$  corresponding to eigenvalue $\lambda$ of  $\vec{Q}$ 
        \item  $\vec{Rx}$  is an eigenvector of $\vec{Q}$  corresponding to eigenvalue $ \frac{1}{\lambda} $ of $\vec{Q}$ 
        \item $\vec{R}^{-1}\vec{x}$  is an eigenvector of $\vec{Q}$  corresponding to eigenvalue $\lambda$ of $\vec{Q}$ 
        \item $ \vec{R}^{-1}\vec{x}$ is an eigenvector of $\vec{Q}$  corresponding to eigenvalue $ \frac{1}{\lambda} $ of $\vec{Q}$ 
    \end{enumerate}
    \item Let $\vec{M}$  be a $2 \times 2$ matrix with eigenvalues 1 and 2. Then  $ \vec{M}^{-1}$ is
    \hfill{(2007 XE)}
    \begin{multicols}{4}
    \begin{enumerate}
        \item $\vec{M} - 3\vec{I}$
        \item $3\vec{I} - \vec{M}$
        \item $2\vec{I} - \vec{M}$
        \item $\vec{M}^{-1} - 3\vec{I}$
    \end{enumerate}
\end{multicols}
    \item The number of $n \times n$ matrices that are simultaneously Hermitian, unitary and diagonal is
    \hfill{(2007 XE)}
    \begin{multicols}{4}
    \begin{enumerate}
        \item $2^n$
        \item $n^2$
        \item $2n$
        \item 2
    \end{enumerate}
    \end{multicols}
    \item Let $$\vec{M} = \myvec{ 1 & b & a \\ 0 & 2 & c \\ 0 & 0 & 1},$$ where  $a, b, c$ are real numbers. Then $\vec{M}$ is diagonalizable
    \hfill{(2007 XE)}
    \begin{multicols}{2}
    \begin{enumerate}
        \item for all values of $a, b, c$
        \item only when $bc \neq a$
        \item only when $b + c = a$ 
        \item only when $bc = a$
    \end{enumerate}
\end{multicols}
    \item Let $S_1$ be the sum of the eigenvalues of a $2 \times 2$ matrix $\vec{P}$ and $S_2$ the sum of eigenvalues of another $2 \times 2$ matrix $\vec{Q}$. If $S_1 = S_2$, then $\vec{P}$ and $\vec{Q}$  are
    \hfill{(2007 XE)}
    \begin{multicols}{2}
    \begin{enumerate}
        \item $\myvec{4 & 1 \\ 3 & 5}$ and $\myvec{ 3 & 2 \\ 3 & 5}$
        \item $\myvec{3 & 4 \\ 5 & 1 }$ and $\myvec{3 & 4 \\ 5 & 1 }$
        \item $\myvec{4 & 1 \\ 3 & 5 }$ and $\myvec{3 & 1 \\ 5 & 1 }$
        \item $\myvec{4 & 3 \\ 5 & 5 }$ and $\myvec{4 & 3 \\ 5 & 5 }$
    \end{enumerate}
\end{multicols}
    \item If the diagonal elements of a lower triangular square matrix $\vec{A}$ are all $\neq 0$, then $\vec{A}$ will always be
    \hfill{(2007 XE)}
    \begin{multicols}{4}
    \begin{enumerate}
        \item symmetric
        \item non-symmetric
        \item singular
        \item non-singular
    \end{enumerate}
\end{multicols}
    \item If two eigenvalues of the matrix $$\vec{M} = \myvec{2 & 6 & 0 \\ 1 & p & 0 \\ 0 & 0 & 3}$$ are -1 and 4, then $p$ is
    \hfill{(2007 XE)}
    \begin{multicols}{4}
    \begin{enumerate}
        \item 4
        \item 2
        \item 1
        \item -1
    \end{enumerate}
\end{multicols}
    \item Consider the system
	    \begin{align*}
	     x + 10y &= 5 \\ y + 5z &= 1 \\ 10x - y + z &= 0
     \end{align*}
	     On applying Gauss–Seidel method, $x$ correct up to 4 decimal places is
    \hfill{(2007 XE)}
    \begin{multicols}{4}
    \begin{enumerate}
        \item 0.0385
        \item 0.0395
        \item 0.0405
        \item 0.0410
    \end{enumerate}
\end{multicols}
    \item The equation of the best fit line (least squares) for 
	    $$x: 1\ 2\ 3\ 4\ 5, y: 14\ 13\ 9\ 5\ 2 $$ is
    \hfill{(2007 XE)}
    \begin{multicols}{4}
    \begin{enumerate}
        \item $y = 18 - 3x    $ 
        \item $y = 18.1 - 3.1x$
        \item $y = 18.2 - 3.2x$
        \item $y = 18.3 - 3.3x$
    \end{enumerate}
\end{multicols}
