\item  $\vec{x} = \myvec{x_1 & x_2 & \ldots & x_n}^{\top}$ is an $n$-tuple nonzero vector. The $n \times n$ matrix $\vec{V} = \vec{x} \vec{x}^{\top}$ 

	\hfill{(EE 2007)} 
\begin{multicols}{2}
\begin{enumerate}
\item has rank zero
\item has rank 1
\item is orthogonal
\item has rank $n$
\end{enumerate}
\end{multicols}
\item 
A three phase balanced star connected voltage source with frequency $\omega$ rad/s is connected to a star connected balanced load which is purely inductive. The instantaneous line currents and phase to neutral voltages are denoted by $(i_a, i_b, i_c)$ and $(v_{an}, v_{bn}, v_{cn})$ respectively, and their rms values are denoted by $V$ and $I$.
If 
$$R = 
\myvec{
v_{an} & v_{bn} & v_{cn}
}
\myvec{
0 & \frac{1}{\sqrt{3}} & -\frac{1}{\sqrt{3}} \\
-\frac{1}{\sqrt{3}} & 0 & \frac{1}{\sqrt{3}} \\
\frac{1}{\sqrt{3}} & -\frac{1}{\sqrt{3}} & 0
}
\myvec{
i_a \\ i_b \\ i_c
}, 
$$
then the magnitude of $R$ is
\hfill{(EE 2007)} 
\begin{multicols}{4}
\begin{enumerate}
\item  $3VI$
\item  $VI$
\item  $0.7VI$
\item  $0$
\end{enumerate}
\end{multicols}

\item  
Suppose we define a sequence transformation between $a-b-c$ and $p-n-o$ variables as follows
$$
\myvec{
f_a \\ f_b \\ f_c
}
=
k
\myvec{
1 & 1 & 1 \\ 
\alpha^2 & \alpha & 1 \\ 
\alpha & \alpha^2 & 1
}
\myvec{
f_p \\ f_n \\ f_o
}
$$
where  $\alpha = e^{j\frac{2\pi}{3}}$  and $k$ is a constant.
Now, if it is given that
$$
\myvec{ V_p \\ V_n \\ V_o } = 
\myvec{ 
0.5 & 0 & 0 \\
0 & 0.5 & 0 \\
0 & 0 & 2.0
}
\myvec{ i_p \\ i_n \\ i_o }
\text{ and }
\myvec{ V_a \\ V_b \\ V_c } = 
\myvec{ i_a \\ i_b \\ i_c }
$$
then, \hfill{(EE 2007)} 
\begin{multicols}{2}
\begin{enumerate}
\item  $Z = 
\myvec{ 
1.0 & 0.5 & 0.75 \\
0.75 & 1.0 & 0.5 \\
0.5 & 0.75 & 1.0 
}$ 

\item  $Z = 
\myvec{ 
1.0 & 0.5 & 0.5 \\
0.5 & 1.0 & 0.5 \\
0.5 & 0.5 & 1.0 
}$ 

\item  $Z = 3k^2
\myvec{ 
1.0 & 0.75 & 0.5 \\
0.5 & 1.0 & 0.75 \\
0.75 & 0.5 & 1.0 
}$ 

\item  $Z = \dfrac{k^2}{3}
\myvec{ 
1.0 & -0.5 & -0.5 \\
-0.5 & 1.0 & -0.5 \\
-0.5 & -0.5 & 1.0 
}$ 
\end{enumerate}
\end{multicols}
%
\item  Let $\vec{x}$ and $\vec{y}$ be two vectors in a 3 dimensional space. 
Then the determinant

\hfill{(EE 2007)} 
$$
\det\myvec{
\vec{x}^{\top} \vec{x} & \vec{x}^{\top} \vec{y} \\
\vec{y}^{\top} \vec{x} & \vec{y}^{\top} \vec{y} 
}
$$
\begin{enumerate}
\item is zero when $\vec{x}$ and $\vec{y}$ are linearly independent
\item is positive when $\vec{x}$ and $\vec{y}$ are linearly independent
\item is non-zero for all non-zero $\vec{x}$ and $\vec{y}$
\item is zero only when either $\vec{x}$ or $\vec{y}$ is zero
\end{enumerate}
%
\item  The linear operation $\mathrm{L}(\vec{x})$ is defined by the cross product $\mathrm{L}(\vec{x}) = \vec{b} \times \vec{x}$, where $\vec{b} = \myvec{0 &1&0}^{\top}$ and $\vec{x} = \myvec{x_1&x_2&x_3}^{\top}$ are three dimensional vectors. The $3 \times 3$ matrix $\vec{M}$ of this operation satisfies
$
    \mathrm{L}(\vec{x}) = \vec{M}\vec{x}.     
$
Then the eigenvalues of $\vec{M}$ are 

\hfill{(EE 2007)} 
\begin{enumerate}
\begin{multicols}{4}
    \item $0$, $+1$, $-1$
    \item $1$, $-1$, $1$
    \item $i$, $-i$, $1$
    \item $i$, $-i$, $0$
    \end{multicols}
\end{enumerate}
%
\item  
The matrix $A$ given below is the node incidence matrix of a network. The columns correspond to branches of the network while the rows correspond to nodes. Let $\vec{V} = \brak{v_1\, v_2\, \ldots\, v_6}^{\top}$ denote the vector of branch voltages while $\vec{I} = \brak{i_1\, i_2\, \ldots\, i_6}^{\top}$ that of branch currents. The vector $\vec{E} = \brak{e_1\, e_2\, e_3\, e_4}^{\top}$ denotes the vector of node voltages relative to a common ground.
$$ A =
\myvec{
1 & 1 & 1 & 0 & 0 & 0 \\
0 & -1 & 0 & -1 & 1 & 0 \\
-1 & 0 & 0 & 0 & -1 & -1 \\
0 & 0 & -1 & 1 & 0 & 1 \\
}.
$$
Which of the following statements is true?\hfill{(EE 2007)}
\begin{enumerate}
    \item The equations
	    \begin{align*}
        v_1 - v_2 + v_3 &= 0, \\ v_3 + v_4 - v_5 &= 0
\end{align*}
        are KVL equations for the network for some loops.
    \item The equations
	    \begin{align*}
        v_1 - v_3 - v_6 &= 0, \\ v_4 + v_5 - v_6 &= 0
\end{align*}
        are KVL equations for the network for some loops.
    \item $\vec{E} = \vec{A}\vec{V}$
    \item $\vec{A}\vec{V} = 0$ are KVL equations for the network
\end{enumerate}
%
\item 
Consider a matrix
\hfill{(EE 2007)}
$$
\vec{A} =
\myvec{
-3 & 2 \\
-1 & 0
}
$$
\begin{enumerate}
\item  $\vec{A}$ satisfies the relation
\begin{multicols}{2}
\begin{enumerate}
    \item  $\vec{A} + 3\vec{I} + 2\vec{A}^{-1} = 0$ 
 \item  $\vec{A}^2 + 2\vec{A} + 3\vec{I} = 0$
 \item  $\brak{\vec{A} + \vec{I}}\brak{\vec{A} + 2\vec{I}} = 0$ 
 \item  $\exp\brak{\vec{A}} = 0$ 
\end{enumerate}
\end{multicols}
%
\item  $\vec{A}^9$ equals
\begin{multicols}{4}
\begin{enumerate}
\item $511 \vec{A} + 510 \vec{I}$
\item $309 \vec{A} + 104 \vec{I}$
\item $154 \vec{A} + 155 \vec{I}$
\item $\exp\brak{9\vec{A}}$ 
\end{enumerate}
\end{multicols}
\end{enumerate}

