    \item If 
	    \begin{align*}
	    f\brak{\theta} = \myvec{\cos\theta & \sin\theta \\
        -\sin\theta & \cos\theta}, 
\end{align*}
then $f\brak{\alpha} f\brak{\beta} = $

\hfill (AE 2007)
    \begin{multicols}{4}
    \begin{enumerate}
        \item $f\brak{\alpha/\beta}$ 
        \item $f\brak{\alpha + \beta}$ 
        \item $f\brak{\alpha - \beta}$ 
        \item  zero matrix
    \end{enumerate}
    \end{multicols}
%
    \item Let $\vec{P}$ and $\vec{Q}$ be two square matrices of same size. Consider the following statements

\hfill (AE 2007)
\begin{enumerate}[label=(\roman*)]
	    \item $\vec{P}\vec{Q} = 0$ implies $\vec{P} = 0$ or $\vec{Q} = 0$ or both
        \item $\vec{P}\vec{Q} = \vec{I}$ implies $\vec{Q}\vec{P} = \vec{I}$
        \item $\brak{\vec{P} + \vec{Q}}^2 = \vec{P}^2 + 2\vec{P}\vec{Q} + \vec{Q}^2$
	\item $\brak{\vec{P} - \vec{Q}}^2 = \vec{P}^2 - 2\vec{P}\vec{Q} + \vec{Q}^2$
    \end{enumerate}
    where $\vec{I}$ is the identity matrix. Which of the following statement is correct? 

\hfill (AE 2007)
    \begin{enumerate}
        \item Both \brak{i} and \brak{ii} are true
        \item \brak{i} is true but \brak{ii} is false
        \item \brak{i} is false but \brak{ii} is true
        \item Both \brak{i} and \brak{ii} are false
    \end{enumerate}
%
    \item The eigenvalues of the matrix, 
$	    \vec{A} =
    \myvec{2 & 1 \\
        0 & 3}
	$
    are 
\hfill (AE 2007)
    \begin{multicols}{4}
    \begin{enumerate}
        \item 1 and 2 
        \item 1 and 2
        \item 2 and 3 
        \item 2 and 4
    \end{enumerate}
    \end{multicols}
%
    \item The eigenvalues of the matrix, $\vec{A}^{-1}$, where 
	    $\vec{A} = \myvec{2 & 1 \\
    0 & 3},$
    are 
\hfill (AE 2007)
    \begin{multicols}{4}
    \begin{enumerate}
        \item 1 and 1/2 
        \item 1 and 1/3
        \item 2 and 3 
        \item 1/2 and 1/3
    \end{enumerate}
    \end{multicols}
    \item Let a system of linear equations be as follows
    \begin{align*}
        x - y + 2z &= 0 \\
        2x + 3y - z &= 0 \\
        2x - 2y + 4z &= 0
    \end{align*}
    This system of equations has
\hfill (AE 2007)
    \begin{enumerate}
        \item No non-trivial solution
        \item Infinite number of non-trivial solutions
        \item An unique non-trivial solution
        \item Two non-trivial solutions
    \end{enumerate}
