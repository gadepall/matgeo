\item Consider the following equations of straight lines:
\begin{align*}
	\text{Line L1} &\colon 2x - 3y = 5 \\
	\text{Line L2} &\colon 3x + 2y = 8 \\
	\text{Line L3} &\colon 4x - 6y = 5 \\
	\text{Line L4} &\colon 6x - 9y = 6
\end{align*}
Which one among the following is the correct statement?
\hfill{\brak{\text{XE 2022}}}
\begin{enumerate}
	\item L1 is parallel to L2 and L1 is perpendicular to L3
	\item L2 is parallel to L4 and L2 is perpendicular to L1
	\item L3 is perpendicular to L4 and L3 is parallel to L2
	\item L4 is perpendicular to L2 and L4 is parallel to L3
\end{enumerate}
\item Let $\vec{A} = \myvec{2 & 0 & 1 & 1 \\ 1 & 2 & 5 & -5 \\ 0 & 0 & 3 & 0 \\ 0 & 0 & 1 & 3}$. Then the sum of the geometric multiplicities of the distinct eigenvalues of $\vec{A}$ is equal to \underline{\hspace{2cm}}.
\hfill{\brak{\text{XE 2022}}}
\item Let $\vec{A}$ and $\vec{B}$ be $n \times n$ matrices with real entries.
Consider the following statements:
\begin{description}
    \item[P:] If $\vec{A}$ is symmetric then $\text{rank}(\vec{A}) =$ Number of nonzero eigenvalues (counting multiplicity) of  $\vec{A}$.
    \item[Q:] If $\vec{A}\vec{B} = 0$ then $\text{rank}(\vec{A}) + \text{rank}(\vec{B}) \leq n$.
\end{description}
Then
\hfill{\brak{\text{XE 2022}}}
\begin{multicols}{2}
\begin{enumerate}
    \item both P and Q are TRUE
    \item P is TRUE and Q is FALSE
    \item P is FALSE and Q is TRUE
    \item both P and Q are FALSE
\end{enumerate}
\end{multicols}
