\item Four points $\vec{P}(0, 1), \vec{Q}(0, -3), \vec{R}(-2, -1)$ and $\vec{S}(2, -1)$ represent the vertices of a quadrilateral.
	 What is the area enclosed by the quadrilateral?
	 	\hfill (ST 2022)
	 \begin{enumerate}
		 \begin{multicols}{4}
	 	\item $ 4 $
	 	\item $ 4\sqrt{2} $
	 	\item $ 8 $
	 	\item $ 8\sqrt{2} $
		\end{multicols}
	 \end{enumerate}
 \item Let $\vec{M}$ be a $2 \times 2$ real matrix such that 
$(\vec{I} + \vec{M})^{-1} = \vec{I} - \alpha \vec{M}$, 
where $\alpha$ is a non-zero real number and $\vec{I}$ is the $2 \times 2$ identity matrix. 
If the trace of the matrix $\vec{M}$ is $3$, then the value of $\alpha$ is
	\hfill (ST 2022)
\begin{enumerate} 
		 \begin{multicols}{4}
	\item $\frac{3}{4}$
	\item $\frac{1}{3}$
	\item $\frac{1}{2}$
	\item $\frac{1}{4}$
	\end{multicols}
\end{enumerate}
\item Consider the following transition matrices $\vec{P}_1$ and $\vec{P}_2$ of two Markov chains
	\begin{align*}
P_1 = 
\myvec{
	1 & 0 & 0 \\
	\frac{1}{3} & \frac{1}{2} & \frac{1}{6} \\
	0 & 0 & 1
}
\quad \text{and} \quad
P_2 = 
\myvec{
	\frac{1}{6} & \frac{1}{3} & \frac{1}{2} \\
	\frac{1}{4} & 0 & \frac{3}{4} \\
	0 & 1 & 0
}.
\end{align*}
Then which one of the following statements is true? 
\hfill
(ST 2022)
\begin{enumerate}
	\item Both $\vec{P}_1$ and $\vec{P}_2$ have unique stationary distributions. 
	\item  $\vec{P}_1$ has a unique stationary distribution, but $\vec{P}_2$ has infinitely many stationary distributions. 
		\item $\vec{P}_1$ has infinitely many stationary distributions, but $\vec{P}_2$ has a unique stationary distribution. 
		\item Neither $\vec{P}_1$ nor $\vec{P}_2$ has a unique stationary distribution.
	\end{enumerate}
\item Let $\vec{M}$ be any square matrix of arbitrary order $n$ such that $\vec{M}^2  = \vec{0}$ and the NULLITY of $\vec{M}$ is 6. Then the maximum possible value of $n$ is \underline{\hspace{2cm}}.\hfill (ST 2022)
\item Consider the usual inner product in $\mathbb{R}^4$. 
Let $\vec{u} \in \mathbb{R}^4$ be a unit vector orthogonal to the subspace
\begin{align*}
S = \{ (x_1, x_2, x_3, x_4)^{\top} \in \mathbb{R}^4 
\mid x_1 + x_2 + x_3 + x_4 = 0 \}.
\end{align*}
If $\vec{v} = (1, -2, 1, 1)^{\top}$, and the vectors $\vec{u}$ and 
$\vec{v} - \alpha \vec{u}$, $\alpha \in \mathbb{R}$, are orthogonal, 
then the value of $\alpha^2$  is equal to 
\underline{\hspace{2cm}}. \hfill (ST 2022)
\item Let $\vec{M}$ be a $3 \times 3$ real symmetric matrix with eigenvalues -1, 1, 2 and the corresponding unit eigenvectors $\vec{u, v, w}$, respectively. Let $\vec{x}$ and $\vec{y}$ be two vectors in $\mathbb{R}^3$ such that
\begin{align*}
\vec{M} \vec{x} = \vec{u} + 2(\vec{v} + \vec{w})  \text{ and }  \vec{M}^2\vec{y} = \vec{u} -(\vec{v} + 2\vec{w})
\end{align*}
Considering the usual inner product in $\mathbb{R}^3$, the value of ${|\vec{x} + \vec{y}|}^2$, where $|\vec{x} + \vec{y}|$ is the length of the vector $\vec{x + y}$, is
	\hfill (ST 2022)
\begin{enumerate} 
		 \begin{multicols}{4}
	\item 1.25
	\item 0.25
	\item 0.75
	\item 1
	\end{multicols}
\end{enumerate}
\item Let $\vec{M}$ be any $3 \times 3$ symmetric matrix with eigenvalues 1, 2 and 3. Let $\vec{N}$ be any $3 \times 3$ matrix with real eigenvalues such that $\vec{MN} + \vec{NM} = 3\vec{I}$, where $\vec{I}$ is the $3 \times 3$ identity matrix. Then which of the following cannot be eigenvalue(s) of the matrix $\vec{N}$?
	\hfill (ST 2022)
\begin{enumerate} 
		 \begin{multicols}{4}
	\item $\frac{1}{4}$ 
	\item $\frac{3}{4}$
	\item $\frac{1}{2}$
	\item $\frac{7}{4}$
	\end{multicols}
\end{enumerate}
\item Let $\vec{M}$ be a $3 \times 2$ real matrix having a singular value decomposition as $\vec{M} = \vec{USV}^{\top},$ where the matrix $$
	\vec{S} = \myvec{
	\sqrt{3} & 0 \\
	0 & 1 \\
	0 & 0
},$$
$\vec{U}$ is a $3 \times 3$ orthogonal matrix, and $\vec{V}$ is a $2 \times 2$ orthogonal matrix. Then which of the following statements is/are true?
	\hfill (ST 2022)
\begin{enumerate} 
	\item The rank of the matrix $\vec{M}$ is 1.
	\item The trace of the matrix $\vec{M}^{\top}\vec{M}$ is 4.
	\item The largest singular value of the matrix $(\vec{M}^{\top}\vec{M})^{-1}\vec{M}^{\top}$ is 1.
	\item The nullity of the matrix $\vec{M}$ is 1.
\end{enumerate}
