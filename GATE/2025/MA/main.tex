\item Let $p_{\vec{A}}(x)$ denote the characteristic polynomial of a square matrix $\vec{A}$. Then, for which of the following invertible matrices $\vec{M}$, the polynomial $p_{\vec{M}}(x) — p_{\vec{M}^{-1}}(x)$ is constant?  
\hfill{\brak{\text{MA 2025}}}
\begin{enumerate}
\begin{multicols}{2}
\item $\vec{M} = \myvec{3 & 2 \\ 1 & 1}$
\item $\vec{M} = \myvec{3 & 4 \\ 2 & 2}$
\item $\vec{M} = \myvec{3 & 1 \\ 0 & 1}$
\item $\vec{M} = \myvec{3 & 0 \\ 0 & 1}$
\end{multicols}
\end{enumerate}  
\item Let $\vec{M}$ be a $7 \times 7$ matrix with entries in $\mathbb{R}$ and having the characteristic polynomial  
$$c_\vec{M}(x) = (x-1)(x-2)(x-3)^2.$$ 
Let $\text{rank}(\vec{M}-\vec{I}_7) = \text{rank}(\vec{M}-2\vec{I}_7) = \text{rank}(\vec{M}-3\vec{I}_7) = 5$, where $\vec{I}_7$ is the $7 \times 7$ identity matrix.  
If $m_\vec{M}(x)$ is the minimal polynomial of $\vec{M}$, then $m_{\vec{M}}(5)$ is equal to \underline{\hspace{2cm}}.  
\hfill{\brak{\text{MA 2025}}}
\item Let $\vec{a}$ be a unit vector parallel to the tangent at the point $\vec{P}(1,1,\sqrt{2})$ to the curve of intersection of the surfaces $2x^2+3y^2-z^2=3$ and $x^2+y^2=z^2$.  
Then, the absolute value of the directional derivative of $f(x,y,z)=x^2+2y^2-2\sqrt{11}z$ at $\vec{P}$ in the direction of $\vec{a}$ is equal to \underline{\hspace{2cm}}.  
\hfill{\brak{\text{MA 2025}}}

\item Consider the linear system $\vec{A}\vec{x}=\vec{b}$, where $\vec{A}=[a_{ij}], i,j=1,2,3,$ and $a_{ii}\neq 0$ for $i=1,2,3$, is a matrix with entries in $\mathbb{R}$.  
For 
$$
\vec{D}=\myvec{A_{11} & 0 & 0 \\ 0 & A_{22} & 0 \\ 0 & 0 & A_{33}},
$$  
let  
$$
\vec{D}^{-1}\vec{A} = \myvec{3 & 1 & 2 \\ 1 & 1 & 1 \\ 4 & 0 & 0}, \quad \vec{D}^{-1}\vec{b} = \myvec{4 \\ 1 \\ 1}.
$$  
\begin{enumerate}[label=S\arabic*:]
	\item The approximation of $\vec{x}$ after one iteration of the Jacobi scheme with initial vector $\vec{x}^{(0)}=\myvec{1 \\ 1 \\ 1}$ is $\vec{x}_1=\myvec{5 \\ -1 \\ -1}$.  
	\item There exists an initial vector $\vec{x}^{(0)}$ for which Jacobi iterative scheme diverges.  
\end{enumerate}
Then, which one of the following is correct?  
\hfill{\brak{\text{MA 2025}}}
\begin{enumerate}
\begin{multicols}{2}
\item S1 is TRUE and S2 is FALSE
\item S2 is TRUE and S1 is FALSE
\item both S1 and S2 are TRUE
\item neither S1 nor S2 is TRUE
\end{multicols}
\end{enumerate}
\item Let $C$ be the curve of intersection of the surfaces $z^2=x^2+y^2$ and $4x+z=7$.  
If $\vec{P}$ is a point on $C$ at minimum distance from the $xy$–plane, then the distance of $\vec{P}$ from the origin is  
\hfill{\brak{\text{MA 2025}}}
\begin{enumerate}
\begin{multicols}{2}
\item $\tfrac{7}{5}$
\item $\tfrac{7\sqrt{2}}{5}$
\item $\tfrac{14}{5}$
\item $\tfrac{14\sqrt{2}}{5}$
\end{multicols}
\end{enumerate}
\item Let $M_2(\mathbb{R})$ be the space of $2\times 2$ reals. Define $T(\vec{X})=\vec{A}\vec{X}\vec{B}$ with matrices  

$\vec{A}=\myvec{1 & -4 \\ 6 & -2}, \; \vec{B}=\myvec{5 & 0 \\ -1 & -1}.$  

Let $\vec{P}$ be the representation matrix of $T$. Which of the following are TRUE?  
\hfill{\brak{\text{MA 2025}}}
\begin{multicols}{2}
\begin{enumerate}
\item $\vec{P}$ is invertible
\item The trace of $\vec{P}$ is 25
\item The rank of $(\vec{P}^2-4\vec{I}_4)$ is 4
\item The nullity of $(\vec{P}-2\vec{I}_4)$ is 0
\end{enumerate}
\end{multicols}
\item Let $\{x_k\}$ orthonormal in Hilbert space X. For fixed n, let $Y=\text{span}\{x_1,\dots,x_n\}$. Define $S_n(x)=\sum_{k=1}^n \langle x,x_k\rangle x_k$. Then
\hfill{\brak{\text{MA 2025}}}
\begin{enumerate}
\item $S_n(x)$ is orthogonal projection on Y
\item $S_n(x)$ is projection on $Y^\perp$
\item $x-S_n(x)\perp S_n(x)$
\item $\sum_{k=1}^n\abs{\langle x,x_k\rangle}^2=\abs{x}^2$ for all x
\end{enumerate}
\item For $\vec{X}=\myvec{x_1 \\ x_2 \\ x_3}$, quadratic form  
$$
Q(\vec{X})=2x_1^2+2x_2^2+3x_3^2+4x_1x_2+2x_1x_3+2x_2x_3.
$$  
Let $\vec{M}$ be symmetric matrix of Q. For $\vec{Y}\in\mathbb{R}^3$ non-zero define  
$$
a_n=\frac{\vec{Y}^{\top}(\vec{M}+\vec{I}_3)^{n+1}\vec{Y}}{\vec{Y}^{\top}(\vec{M}+\vec{I}_3)^n\vec{Y}}, \, n=1,2,\dots
$$  
Then $\lim_{n\to\infty}a_n=$ \underline{\hspace{2cm}}.  
\hfill{\brak{\text{MA 2025}}}

