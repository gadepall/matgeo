\item Let $$S = \brak{\brak{x, y, z} \in \mathbb{R}^3 \setminus \cbrak{\brak{0,0,0}} : z = -\brak{x + y} }.$$ Denote
$$
S^+ =\brak{\brak{p,q,r} \in \mathbb{R}^3 : px + qy + rz = 0 \text{ for all } \brak{x, y, z} \in S}.
$$
Then which one of the following options is correct?
\hfill{(ST 2025)}
\begin{multicols}{2}
\begin{enumerate}
\item $S^+$ is not a subspace of $\mathbb{R}^3$
\item $S^+ = \cbrak{\brak{0,0,0}}$
\item $\dim\brak{S^+} = 1$
\item $\dim\brak{S^+} = 2$
\end{enumerate}
\end{multicols}
\item Let 
$\vec{P}=\myvec{ 2 & -1 \\ -1 & 4 }$, 
$\vec{Q}=\vec{P}^3 - 2\vec{P}^2 -4\vec{P} + 13\vec{I}_2$.  
Then $\det\brak{\vec{Q}}=$ \rule{1cm}{0.01pt}.
\hfill{(ST 2025)}
\item Let $\vec{T}:\mathbb{R}^3\to\mathbb{R}^3$ be defined by
$
\vec{T}\brak{x_1,x_2,x_3} = \brak{3x_1+5x_2+x_3,\; x_3,\; 2x_1+2x_3}.
$
The rank of $\vec{T}$ is \rule{1cm}{0.01pt}.
\hfill{(ST 2025)}
\item Let $\vec{P}=\brak{a_{ij}}_{10\times 10}$ with 
$a_{ij}=\tfrac{1}{10}$ if $i\neq j$, and $a_{ii}=\tfrac{9}{10}$. Then $\mathrm{rank}\brak{\vec{P}}=$
\hfill{(ST 2025)}
\begin{multicols}{4}
\begin{enumerate}
\item 10 \item 9 \item 1 \item 8
\end{enumerate}
\end{multicols}
\item Let $\mathcal{O} = \cbrak{\vec{P}: \vec{P} \text{ is a } 3\times 3 \text{ real matrix with } \vec{P}^T\vec{P}=\vec{I}_3, \det\brak{\vec{P}}=1}$.  
Which of the following options is/are correct?
\hfill{(ST 2025)}
\begin{enumerate}
\item There exists $\vec{P} \in \mathcal{O}$ with $\lambda=\tfrac{1}{2}$ as an eigenvalue
\item There exists $\vec{P} \in \mathcal{O}$ with $\lambda=2$ as an eigenvalue
\item If $\lambda$ is the only real eigenvalue of $\vec{P} \in \mathcal{O}$, then $\lambda=1$
\item There exists $\vec{P} \in \mathcal{O}$ with $\lambda=-1$ as an eigenvalue
\end{enumerate}
\item Let  
$
\vec{P}=\myvec{
0 & 1 & 1 & 1 & 1 \\
-1 & 0 & 1 & 1 & 1 \\
-1 & -1 & 0 & 1 & 1 \\
-1 & -1 & -1 & 0 & 1 \\
-1 & -1 & -1 & -1 & 0
}.
$
If $\lambda_1,\lambda_2,\lambda_3,\lambda_4,\lambda_5$ are eigenvalues of $\vec{P}$, then $\prod_{i=1}^5 \lambda_i =$ \rule{1cm}{0.01pt}.
\hfill{(ST 2025)}
\item Let  
$
\vec{P}=\myvec{2 & 1 \\ 1 & 2}, 
\quad 
\vec{Q}=\myvec{1 & 1 \\ -2 & 4}.
$
Then $\text{trace}\brak{\vec{P}^5+\vec{Q}^4}=$ \underline{\phantom{imagine}}.
\hfill{(ST 2025)}
