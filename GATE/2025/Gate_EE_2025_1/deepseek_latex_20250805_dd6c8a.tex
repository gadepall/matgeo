\documentclass{article}
\usepackage{amsmath}
\usepackage{graphicx}
\usepackage{array}
\usepackage{enumitem}
\usepackage{multicol}

\begin{document}

\section*{GATE 2019 Petroleum Engineering Questions}

\subsection*{General Aptitude (GA)}

\begin{enumerate}[leftmargin=*,label=\textbf{Q.\arabic*},start=1]
    \item (Q.1) The fishermen, \_\_\_\_\_\_ the flood victims owed their lives, were rewarded by the government.
    \begin{multicols}{4}
    \begin{enumerate}[label=(\Alph*)]
        \item whom
        \item to which
        \item to whom
        \item that
    \end{enumerate}
    \end{multicols}
    \textbf{GATE EE 2025}

    \item (Q.2) Some students were not involved in the strike. If the above statement is true, which of the following conclusions is/are logically necessary?
    \begin{enumerate}[label=\arabic*.]
        \item Some who were involved in the strike were students.
        \item No student was involved in the strike.
        \item At least one student was involved in the strike.
        \item Some who were not involved in the strike were students.
    \end{enumerate}
    \begin{multicols}{4}
    \begin{enumerate}[label=(\Alph*)]
        \item 1 and 2
        \item 3
        \item 4
        \item 2 and 3
    \end{enumerate}
    \end{multicols}
    \textbf{GATE EE 2025}

    \item (Q.3) The radius as well as the height of a circular cone increases by 10\%. The percentage increase in its volume is \_\_\_\_\_.
    \begin{multicols}{4}
    \begin{enumerate}[label=(\Alph*)]
        \item 17.1
        \item 21.0
        \item 33.1
        \item 72.8
    \end{enumerate}
    \end{multicols}
    \textbf{GATE EE 2025}

    \item (Q.4) Five numbers 10, 7, 5, 4 and 2 are to be arranged in a sequence from left to right following the directions given below:
    \begin{enumerate}[label=\arabic*.]
        \item No two odd or even numbers are next to each other.
        \item The second number from the left is exactly half of the left-most number.
        \item The middle number is exactly twice the right-most number.
    \end{enumerate}
    Which is the second number from the right?
    \begin{multicols}{4}
    \begin{enumerate}[label=(\Alph*)]
        \item 2
        \item 4
        \item 7
        \item 10
    \end{enumerate}
    \end{multicols}
    \textbf{GATE EE 2025}

    \item (Q.5) Until Iran came along, India had never been \_\_\_\_\_ in kabaddi.
    \begin{multicols}{4}
    \begin{enumerate}[label=(\Alph*)]
        \item defeated
        \item defeating
        \item defeat
        \item defeatist
    \end{enumerate}
    \end{multicols}
    \textbf{GATE EE 2025}
\end{enumerate}

\subsection*{Petroleum Engineering (PE)}

\begin{enumerate}[leftmargin=*,label=\textbf{Q.\arabic*},start=1]
    \item (Q.1) Let $r$ and $\theta$ be the modulus and argument of the complex number $z = 1 + i$, respectively. Then $(r, \theta)$ equals
    \begin{multicols}{4}
    \begin{enumerate}[label=(\Alph*)]
        \item $(\sqrt{2}, \frac{\pi}{4})$
        \item $(2, \frac{\pi}{2})$
        \item $(2, \frac{\pi}{3})$
        \item $(\sqrt{2}, \pi)$
    \end{enumerate}
    \end{multicols}
    \textbf{GATE EE 2025}

    \item (Q.2) Let $\lambda_1$ and $\lambda_2$ be the two eigenvalues of the matrix $A = \begin{pmatrix} 0 & -1 \\ 1 & 1 \end{pmatrix}$. Then, $\lambda_1 + \lambda_2$ and $\lambda_1 \lambda_2$, are respectively
    \begin{multicols}{4}
    \begin{enumerate}[label=(\Alph*)]
        \item 1 and 1
        \item 1 and -1
        \item -1 and 1
        \item -1 and -1
    \end{enumerate}
    \end{multicols}
    \textbf{GATE EE 2025}

    \item (Q.3) The Laplace transform of the function $f(t) = e^{-t}$ is given by
    \begin{multicols}{4}
    \begin{enumerate}[label=(\Alph*)]
        \item $\frac{1}{(s+1)^2}$
        \item $\frac{1}{s-1}$
        \item $\frac{1}{s+1}$
        \item $\frac{1}{(s-1)^2}$
    \end{enumerate}
    \end{multicols}
    \textbf{GATE EE 2025}

    \item (Q.4) The relative decline rate of oil is given by $\frac{1}{q} \frac{dq}{dt} = -aq^b$, where $q$ is the oil production rate, $a > 0$ is the decline rate and $b$ is a constant. The equation gives harmonic decline curve when $b$ is
    \begin{multicols}{4}
    \begin{enumerate}[label=(\Alph*)]
        \item 1.5
        \item 1
        \item 0.5
        \item 0
    \end{enumerate}
    \end{multicols}
    \textbf{GATE EE 2025}

    \item (Q.22) The value of $\lim_{x \to 0} \frac{(x+1)\sin x}{x^2 + 2x}$ is \_\_\_\_\_ (round off to 2 decimal places).
    \textbf{GATE EE 2025}

    \item (Q.26) The general solution of the differential equation $\frac{d^2 y}{dx^2} - 2 \frac{dy}{dx} + y = 0$ is (here $C_1$ and $C_2$ are arbitrary constants)
    \begin{multicols}{2}
    \begin{enumerate}[label=(\Alph*)]
        \item $y = C_1 e^x + C_2 e^{-x}$
        \item $y = C_1 xe^x + C_2 xe^{2x}$
        \item $y = C_1 e^x + C_2 xe^{-x}$
        \item $y = C_1 e^x + C_2 xe^x$
    \end{enumerate}
    \end{multicols}
    \textbf{GATE EE 2025}

    \item (Q.55) In a pressure draw-down testing, the well bore flowing pressure $P_{wf}$ is given by
    \[ P_{wf} = P_i - \frac{162.6 q \mu B}{kh} \left[ \log \left( \frac{k t}{\phi \mu c r_w^2} \right) - 3.23 + 0.87 S \right] \]
    The following data is given in the oil field units:
    \begin{itemize}
        \item Initial reservoir pressure $(P_i) = 5000$ psia
        \item Pressure after 1 hr of production $(P_{1hr}) = 4000$ psia
        \item Oil flow rate $(q) = 500$ STB/day
        \item Porosity $(\phi) = 0.25$
        \item Viscosity of oil $(\mu) = 2$ cP
        \item Formation volume factor of oil $(\beta) = 1.2$ bbl/STB
        \item Formation thickness $(h) = 20$ ft
        \item Total compressibility $(c) = 30 \times 10^{-6}$ psi$^{-1}$
        \item Well bore radius $(r_w) = 0.3$ ft
    \end{itemize}
    The slope of $P_{wf}$ versus $\log t$ is $-100$ psi/cycle. Then, the skin factor $(S)$ for this well is \_\_\_\_\_ (round off to 1 decimal place).
    \textbf{GATE EE 2025}
\end{enumerate}

\end{document}