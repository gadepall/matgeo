\item A circle with center at \brak{x,y} = \brak{0.5, 0} and radius = $0.5$ intersects with another circle with center at \brak{x, y} = \brak{1, 1} and radius = $1$ at two points. One of the points of intersection \brak{x, y} is
\hfill{\brak{\text{CE 2025}}}
\begin{enumerate}
    \begin{multicols}{4}
    \item \brak{0,0}
    \item \brak{0.2, 0.4}
    \item \brak{0.5, 0.5}
    \item \brak{1, 2}
    \end{multicols}
\end{enumerate}
\item Suppose $\lambda$ is an eigenvalue of matrix $\vec{A}$ and $\vec{x}$ is the corresponding eigenvector. Let $\vec{x}$ also be an eigenvector of the matrix $\vec{B} = \vec{A} - 2\vec{I}$, where $\vec{I}$ is the identity matrix. Then, the eigenvalue of $\vec{B}$ corresponding to the eigenvector $\vec{x}$ is equal to
\hfill{\brak{\text{CE 2025}}}
\begin{enumerate}
    \begin{multicols}{4}
    \item $\lambda$
    \item $\lambda + 2$
    \item $2\lambda$
    \item $\lambda - 2$
    \end{multicols}
\end{enumerate}
\item Let $\vec{A} = \myvec{1 & 1 \\ 1 & 3 \\ -2 & -3}$ and $\vec{b} = \myvec{b_1 \\ b_2 \\ b_3}$. For $\vec{A}\vec{x}=\vec{b}$ to be solvable, which one of the following options is the correct condition on $b_1$, $b_2$, and $b_3$
\hfill{\brak{\text{CE 2025}}}
\begin{enumerate}
    \begin{multicols}{2}
    \item $b_1 + b_2 + b_3 = 1$
    \item $3b_1 + b_2 + 2b_3 = 0$
    \item $b_1 + 3b_2 + b_3 = 2$
    \item $b_1 + b_2 + b_3 = 2$
    \end{multicols}
\end{enumerate}
