\item The dimension of the vector space of \(7 \times 7\) real symmetric matrices with trace zero and the sum of the off-diagonal elements zero is \hfill(ST 2019)
\begin{multicols}{4}
\begin{enumerate}
    \item 47
    \item 28
    \item 27
    \item 26
\end{enumerate}
\end{multicols}
%
\item Let \(\vec{A}\) be a \(6 \times 6\) complex matrix with \(\vec{A}^3 \neq 0\) and \(\vec{A}^4 = 0\). Then the number of Jordan blocks of \(\vec{A}\) is \hfill(ST 2019)
\begin{multicols}{4}
\begin{enumerate}
     \item 1 or 6
     \item 2 or 3
     \item 4
     \item 5
\end{enumerate}
\end{multicols}
\item Consider a discrete time Markov chain on the state space \(\{1,2,3\}\) with one-step transition probability matrix
$\vec{P} = \myvec{
0.7 & 0.3 & 0 \\
0 & 0.6 & 0.4 \\
0 & 0 & 1}.$
Which of the following statements is true?\hfill(ST 2019)
\begin{enumerate}
\item States 1, 3 are recurrent and state 2 is transient.
\item State 3 is recurrent and states 1, 2 are transient.
\item States 1, 2, 3 are recurrent.
\item States 1, 2 are recurrent and state 3 is transient.
\end{enumerate}

\item The minimal polynomial of the matrix
\myvec{1 & 1 & 2 & 0 \\
0 & 2 & 1 & 0 \\
0 & 0 & 1 & 0 \\
0 & 0 & 0 & 2}
is \hfill(ST 2019)
\begin{multicols}{4}
\begin{enumerate}
\item \((x - 1)(x - 2)\)
\item \((x - 1)^2 (x - 2)\)
\item \((x - 1)(x - 2)^2\)
\item \((x - 1)^2 (x - 2)^2\)
\end{enumerate}
\end{multicols}

\item The matrix
\myvec{1 & x & z \\
0 & 2 & y \\
0 & 0 & 1}
is diagonalizable when \((x,y,z)\) equals \hfill(ST 2019)
\begin{multicols}{4}
\begin{enumerate}
\item \((0,0,1)\)
\item \((1,1,0)\)
\item \((\sqrt{2}, \sqrt{2}, 2)\)
\item \((\sqrt{2}, \sqrt{2}, \sqrt{2})\)
\end{enumerate}
\end{multicols}
\item Consider a discrete time Markov chain on the state space \(\{1, 2\}\) with one-step transition probability matrix
\begin{align*}
\vec{P} = \myvec{0.2 & 0.8 \\
0.3 & 0.7}.
\end{align*}
Then
\begin{align*}
\lim_{n \to \infty} \vec{P}^n
\end{align*} is \hfill(ST 2019)
\begin{multicols}{4}
\begin{enumerate}
\item \(\myvec{\frac{3}{11} & \frac{8}{11} \\ \frac{3}{11} & \frac{8}{11}}\)
\item \(\myvec{ 1 & 0 \\ 0 & 1}\)
\item \(\myvec{ 0 & 1 \\ 1 & 0 }\)
\item \(\myvec{ \frac{8}{11} & \frac{3}{11} \\ \frac{8}{11} & \frac{3}{11}}\)
\end{enumerate}
\end{multicols}
\item Let \(\vec{A}\) be an \(n \times n\) positive semi-definite matrix with eigenvalues \(\lambda_1 \geq \cdots \geq \lambda_n\), and with \(\alpha\) as the maximum diagonal entry. We can find a vector \(\vec{x}\) such that \(\vec{x}^{\top} \vec{x} = 1\), where \(t\) denotes transpose, and \hfill(ST 2019)
\begin{enumerate}
\item \(\vec{x}^{\top} \vec{A} \vec{x} > \lambda_1\)
\item \(\vec{x}^{\top} \vec{A} \vec{x} < \lambda_n\)
\item \(\lambda_n \leq \vec{x}^{\top} \vec{A} \vec{x} \leq \lambda_1\)
\item \(\vec{x}^{\top} \vec{A} \vec{x} > n \alpha\)
\end{enumerate}
\item Consider a discrete time Markov chain on the state space \(\{1,2,3\}\) with one-step transition probability matrix
\myvec{0 & 0.2 & 0.8 \\
0.5 & 0 & 0.5 \\
0.6 & 0.4 & 0 }.
Then the period of the Markov chain is \rule{1cm}{0.01pt} \hfill(ST 2019)
\item Let \(\vec{I}\) be the \(4 \times 4\) identity matrix and \(\vec{v} = (1, 2, 3, 4)^{\top}\), where \(t\) denotes transpose. Then the determinant of $\vec{I} + \vec{v} \vec{v}^{\top}$
is equal to \rule{1cm}{0.01pt} \hfill(ST 2019)


\end{enumerate}






\end{document}
