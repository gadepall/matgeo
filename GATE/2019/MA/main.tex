%iffalse
\let\negmedspace\undefined
\let\negthickspace\undefined
\documentclass[journal,12pt,onecolumn]{IEEEtran}
\usepackage{cite}
\usepackage{amsmath,amssymb,amsfonts,amsthm}
\usepackage{algorithmic}
\usepackage{graphicx}
\usepackage{textcomp}
\usepackage{xcolor}
\usepackage{txfonts}
\usepackage{listings}
\usepackage{enumitem}
\usepackage{mathtools}
\usepackage{gensymb}
\usepackage{comment}
\usepackage[breaklinks=true]{hyperref}
\usepackage{tkz-euclide} 
\usepackage{listings}
\usepackage{gvv}
\def\inputGnumericTable{}                                 
\usepackage[utf8]{inputenc}                              
\usepackage{color}                                         
\usepackage{array}                                        
\usepackage{longtable}                                     
\usepackage{calc}                                          
\usepackage{multirow}                                      
\usepackage{hhline}                                        
\usepackage{ifthen}                                        
\usepackage{lscape}
\newtheorem{theorem}{Theorem}[section]
\newtheorem{problem}{Problem}
\newtheorem{proposition}{Proposition}[section]
\newtheorem{lemma}{Lemma}[section]
\newtheorem{corollary}[theorem]{Corollary}
\newtheorem{example}{Example}[section]
\newtheorem{definition}[problem]{Definition}
\newcommand{\BEQA}{\begin{eqnarray}}
\newcommand{\EEQA}{\end{eqnarray}}
\newcommand{\define}{\stackrel{\triangle}{=}}
\theoremstyle{remark}
\newtheorem{rem}{Remark}
\graphicspath{ {./Figures/} }
\usepackage{float} % For the [H] float option
\usepackage{textcomp}
\usepackage{multicol}

\begin{document}
\begin{enumerate}[start=1, label=Q.\arabic*]


\item The fishermen, the flood victims owed their lives, were rewarded by the government.  
\begin{enumerate}
\begin{multicols}{4}
\item whom  
\item to which  
\item to whom  
\item that  
\end{multicols}
\end{enumerate}

\hfill{\brak{\text{GATE MA 2019}}}


\item Some students were not involved in the strike.  

If the above statement is true, which of the following conclusions is/are logically necessary?  

1. Some who were involved in the strike were students.  
2. No student was involved in the strike.  
3. At least one student was involved in the strike.  
4. Some who were not involved in the strike were students.  

\begin{enumerate}
\begin{multicols}{4}
\item 1 and 2  
\item 3  
\item 4  
\item 2 and 3  
\end{multicols}
\end{enumerate}

\hfill{\brak{\text{GATE MA 2019}}}


\item The radius as well as the height of a circular cone increases by $10\%$.  
The percentage increase in its volume is  
\begin{enumerate}
\begin{multicols}{4}
\item $17.1$  
\item $21.0$  
\item $33.1$  
\item $72.8$  
\end{multicols}
\end{enumerate}

\hfill{\brak{\text{GATE MA 2019}}}


\item Five numbers $10, 7, 5, 4$ and $2$ are to be arranged in a sequence from left to right following the directions given below:  

1. No two odd or even numbers are next to each other.  
2. The second number from the left is exactly half of the left-most number.  
3. The middle number is exactly twice the right-most number.  

Which is the second number from the right?  
\begin{enumerate}
\begin{multicols}{4}
\item 2  
\item 4  
\item 7  
\item 10  
\end{multicols}
\end{enumerate}

\hfill{\brak{\text{GATE MA 2019}}}


\item Until Iran came along, India had never been in kabaddi.  
\begin{enumerate}
\begin{multicols}{4}
\item defeated  
\item defeating  
\item defeat  
\item defeatist  
\end{multicols}
\end{enumerate}

\hfill{\brak{\text{GATE MA 2019}}}


\item Since the last one year, after a 125 basis point reduction in repo rate by the Reserve Bank of India, banking institutions have been making a demand to reduce interest rates on small saving schemes. Finally, the government announced yesterday a reduction in interest rates on small saving schemes to bring them on par with fixed deposit interest rates.  

Which one of the following statements can be inferred from the given passage?  
\begin{enumerate}
\item Whenever the Reserve Bank of India reduces the repo rate, the interest rates on small saving schemes are also reduced  
\item Interest rates on small saving schemes are always maintained on par with fixed deposit interest rates  
\item The government sometimes takes into consideration the demands of banking institutions before reducing the interest rates on small saving schemes  
\item A reduction in interest rates on small saving schemes follow only after a reduction in repo rate by the Reserve Bank of India  
\end{enumerate}

\hfill{\brak{\text{GATE MA 2019}}}


\item In a country of $1400$ million population, $70\%$ own mobile phones. Among the mobile phone owners, only $294$ million access the Internet. Among these Internet users, only half buy goods from e-commerce portals. What is the percentage of these buyers in the country?  
\begin{enumerate}
\begin{multicols}{4}
\item 10.50  
\item 14.70  
\item 15.00  
\item 50.00  
\end{multicols}
\end{enumerate}

\hfill{\brak{\text{GATE MA 2019}}}


\item The nomenclature of Hindustani music has changed over the centuries. Since the medieval period dhrupad styles were identified as baanis. Terms like gayaki and baaj were used to refer to vocal and instrumental styles, respectively. With the institutionalization of music education the term gharana became acceptable. Gharana originally referred to hereditary musicians from a particular lineage, including disciples and grand disciples.  

Which one of the following pairings is NOT correct?  
\begin{enumerate}
\begin{multicols}{2}
\item dhrupad, baani  
\item gayaki, vocal  
\item baaj, institution  
\item gharana, lineage  
\end{multicols}
\end{enumerate}

\hfill{\brak{\text{GATE MA 2019}}}


\item Two trains started at 7 AM from the same point. The first train travelled north at a speed of $80 \, \text{km/h}$ and the second train travelled south at a speed of $100 \, \text{km/h}$. The time at which they were $540 \, \text{km}$ apart is AM.  
\begin{enumerate}
\begin{multicols}{4}
\item 9  
\item 10  
\item 11  
\item 11.30  
\end{multicols}
\end{enumerate}

\hfill{\brak{\text{GATE MA 2019}}}


\item “I read somewhere that in ancient times the prestige of a kingdom depended upon the number of taxes that it was able to levy on its people. It was very much like the prestige of a head-hunter in his own community.”  

Based on the paragraph above, the prestige of a head-hunter depended upon  
\begin{enumerate}
\begin{multicols}{2}
\item the prestige of the kingdom  
\item the prestige of the heads  
\item the number of taxes he could levy  
\item the number of heads he could gather  
\end{multicols}
\end{enumerate}

\hfill{\brak{\text{GATE MA 2019}}}


\item For a balanced transportation problem with three sources and three destinations where costs, availabilities and demands are all finite and positive, which one of the following statements is FALSE?  
\begin{enumerate}
\begin{multicols}{2}
\item The transportation problem does not have unbounded solution  
\item The number of non-basic variables of the transportation problem is 4  
\item The dual variables of the transportation problem are unrestricted in sign  
\item The transportation problem has at most 5 basic feasible solutions  
\end{multicols}
\end{enumerate}

\hfill{\brak{\text{GATE MA 2019}}}


\item Let $f : [a,b] \to \mathbb{R}$ be any function which is twice differentiable in $\brak{a,b}$ with only one root $\alpha \in \brak{a,b}$. Let $f'(x)$ and $f''(x)$ denote the first and second order derivatives of $f(x)$ with respect to $x$. If $\alpha$ is a simple root and is computed by the Newton-Raphson method, then the method converges if  
\begin{enumerate}
\item $\abs{f(x) f''(x)} < \brak{f'(x)}^2, \; \forall x \in \brak{a,b}$  
\item $\abs{f(x)} < \brak{f'(x)}^2, \; \forall x \in \brak{a,b}$  
\item $\abs{f(x) f'(x)} < f''(x), \; \forall x \in \brak{a,b}$  
\item $\abs{f(x) f'(x)} < \abs{f''(x)}, \; \forall x \in \brak{a,b}$  
\end{enumerate}

\hfill{\brak{\text{GATE MA 2019}}}


\item Let $f : \mathbb{C} \to \mathbb{C}$ be defined by  
\[
f(x+iy) = x^3 + 3xy^2 + i \brak{y^3 + 3x^2 y}, \quad i = \sqrt{-1}.
\]  
Let $f'(z)$ denote the derivative of $f$ with respect to $z$. Then which one of the following statements is TRUE?  
\begin{enumerate}
\item $f'(1+i)$ exists and $\abs{f'(1+i)} = 3\sqrt{5}$  
\item $f$ is analytic at the origin  
\item $f$ is not differentiable at $i$  
\item $f$ is differentiable at $1$  
\end{enumerate}

\hfill{\brak{\text{GATE MA 2019}}}


\item The partial differential equation
\[
\brak{x^2 + y^2 - 1}\,\frac{\partial^2 u}{\partial x^2}
\;+\; 2\,\frac{\partial^2 u}{\partial x \partial y}
\;+\; \brak{x^2 + y^2 - 1}\,\frac{\partial^2 u}{\partial y^2}
= 0
\]
is
\begin{enumerate}
\begin{multicols}{2}
\item parabolic in the region $x^2 + y^2 > 2$
\item hyperbolic in the region $x^2 + y^2 > 2$
\item elliptic in the region $0 < x^2 + y^2 < 2$
\item hyperbolic in the region $0 < x^2 + y^2 < 2$
\end{multicols}
\end{enumerate}

\hfill{\brak{\text{GATE MA 2019}}}

\hfill{\brak{\text{GATE MA 2019}}}


\item If  
\[
u_n = \int_0^1 t^n e^t \, dt, \quad n = 1,2,3,\dots,
\]  
then which one of the following statements is TRUE?  
\begin{enumerate}
\item Both the sequence $\{u_n\}$ and the series $\sum u_n$ are convergent  
\item Both the sequence $\{u_n\}$ and the series $\sum u_n$ are divergent  
\item The sequence $\{u_n\}$ is convergent but the series $\sum u_n$ is divergent  
\item $\lim_{n \to \infty} u_n = 2$  
\end{enumerate}

\hfill{\brak{\text{GATE MA 2019}}}


\item Let $T = \{(x,y,z) \in \mathbb{R}^3 : -1<x<1, -1<y<1, -1<z<1\}$ and $\phi : T \to \mathbb{R}$ be a function whose all second order partial derivatives exist and are continuous. If $\phi$ satisfies the Laplace equation $\nabla^2 \phi = 0$ for all $(x,y,z) \in T$, then which one of the following statements is TRUE in $T$?  
\begin{enumerate}
\item $\nabla \phi$ is solenoidal but not irrotational  
\item $\nabla \phi$ is irrotational but not solenoidal  
\item $\nabla \phi$ is both solenoidal and irrotational  
\item $\nabla \phi$ is neither solenoidal nor irrotational  
\end{enumerate}

\hfill{\brak{\text{GATE MA 2019}}}


\item Let $X = \{(x_1,x_2,\dots) : x_i \in \mathbb{R} \; \text{and only finitely many } x_i \neq 0\}$ and $d : X \times X \to \mathbb{R}$ be a metric on $X$ defined by  
\[
d(x,y) = \sup \abs{x_i - y_i}, \quad x=(x_1,x_2,\dots), \; y=(y_1,y_2,\dots).
\]  
Consider the following statements:  

P: $(X,d)$ is a complete metric space.  
Q: The set $\{x \in X : d(0,x) < 1\}$ is compact, where $0$ is the zero element of $X$.  

Which of the above statements is/are TRUE?  
\begin{enumerate}
\begin{multicols}{2}
\item Both P and Q  
\item P only  
\item Q only  
\item Neither P nor Q  
\end{multicols}
\end{enumerate}

\hfill{\brak{\text{GATE MA 2019}}}


\item Consider the following statements:

I. The set $\mathbb{Q} \times \mathbb{Z}$ is uncountable. \\
II. The set $\{\, f : f \text{ is a function from } \mathbb{N} \to \{0,1\} \,\}$ is uncountable. \\
III. The set $\{\sqrt{p} : p \text{ is a prime number}\}$ is uncountable. \\
IV. For any infinite set, there exists a bijection from the set to one of its proper subsets. \\

\brak{\text{$\mathbb{Q}$ is the set of all rational numbers, $\mathbb{Z}$ is the set of all integers and $\mathbb{N}$ is the set of all natural numbers}} \\
Which of the above statements are TRUE?
\begin{enumerate}
\begin{multicols}{4}
\item I and IV only
\item II and IV only
\item II and III only
\item I, II and IV only
\end{multicols}
\end{enumerate}

\hfill{\brak{\text{GATE MA 2019}}}


\item Let $f : \mathbb{R}^2 \to \mathbb{R}$ be defined by
\[
f\brak{x,y} = x^6 - 2x^2 y - x^4 y + 2y^2.
\]
\brak{\text{$\mathbb{R}$ is the set of all real numbers and $\mathbb{R}^2 = \{\brak{x,y} : x,y \in \mathbb{R}\}$}}
Which one of the following statements is TRUE?
\begin{enumerate}
\item $f$ has a local maximum at origin
\item $f$ has a local minimum at origin
\item $f$ has a saddle point at origin
\item The origin is not a critical point of $f$
\end{enumerate}

\hfill{\brak{\text{GATE MA 2019}}}

\item Let \(\{a_n\}_{n=0}^{\infty}\) be any sequence of real numbers such that \(\sum_{n=0}^{\infty} \abs{a_n}^2 < \infty\).
If the radius of convergence of \(\sum_{n=0}^{\infty} a_n x^n\) is \(r\), then which one of the following statements is necessarily TRUE?
\begin{enumerate}
\item $r = 1$ or $r$ is infinite  
\item $r < 1$  
\item $r = \left( \sum_{n=0}^{\infty} |a_n|^2 \right)^{1/2}$ 
\item $r = \sum_{n=0}^{\infty} |a_n|^2$
\end{enumerate}

\hfill{\brak{\text{GATE MA 2019}}}

\item Let $\mathcal{T}_1$ be the co-countable topology on $\mathbb{R}$ and $\mathcal{T}_2$ be the co-finite topology on $\mathbb{R}$. Consider the following statements\brak{:}

I : In $\brak{\mathbb{R},\mathcal{T}_1}$, the sequence $\{1/n\}_{n=1}^{\infty}$ converges to $0$. \\
II : In $\brak{\mathbb{R},\mathcal{T}_2}$, the sequence $\{1/n\}_{n=1}^{\infty}$ converges to $0$. \\
III : In $\brak{\mathbb{R},\mathcal{T}_1}$, there is no sequence of rational numbers which converges to $\sqrt{3}$. \\
IV : In $\brak{\mathbb{R},\mathcal{T}_2}$, there is no sequence of rational numbers which converges to $\sqrt{3}$. \\

Which of the above statements are TRUE?
\begin{enumerate}
\begin{multicols}{2}
\item I and II only
\item II and III only
\item III and IV only
\item I and IV only
\end{multicols}
\end{enumerate}

\hfill{\brak{\text{GATE MA 2019}}}


\item Let $X$ and $Y$ be normed linear spaces, and let $T : X \to Y$ be any bijective linear map with closed graph. Then which one of the following statements is TRUE?
\begin{enumerate}
\begin{multicols}{2}
\item The graph of $T$ is equal to $X \times Y$
\item $T$ is continuous
\item The graph of $T^{-1}$ is closed
\item $T^{-1}$ is continuous
\end{multicols}
\end{enumerate}

\hfill{\brak{\text{GATE MA 2019}}}


\item Let $g : \mathbb{R}^2 \to \mathbb{R}^2$ be defined by $g\brak{x,y}=\brak{e^x\cos y,\; e^x\sin y}$ and let $\brak{a,b}=g\brak{1,\tfrac{\pi}{3}}$. Which one of the following statements is TRUE?
\begin{enumerate}
\item $g$ is injective
\item If $h$ is the continuous inverse of $g$, defined in some neighbourhood of $\brak{a,b}\in \mathbb{R}^2$, such that $h\brak{a,b}=\brak{1,\tfrac{\pi}{3}}$, then the Jacobian of $h$ at $\brak{a,b}$ is $e^{-2}$
\item If $h$ is the continuous inverse of $g$, defined in some neighbourhood of $\brak{a,b}\in \mathbb{R}^2$, such that $h\brak{a,b}=\brak{1,\tfrac{\pi}{3}}$, then the Jacobian of $h$ at $\brak{a,b}$ is $e^{2}$
\item $g$ is surjective
\end{enumerate}

\hfill{\brak{\text{GATE MA 2019}}}


\item Let 
\[
u_n=\frac{n!}{1\cdot 3 \cdot 5 \cdots \brak{2n-1}}, \quad n\in \mathbb{N}.
\]
Then $\lim_{n\to\infty} u_n$ is equal to \underline{\hspace{2cm}}

\hfill{\brak{\text{GATE MA 2019}}}

\item If the differential equation
\[
\frac{dy}{dx} = \sqrt{x^2 + y^2}, \quad y(1)=2
\]
is solved using the Euler’s method with step-size $h=0.1$, then $y(1.2)$ is equal to \underline{\hspace{2cm}} (round off to 2 places of decimal).

\hfill{\brak{\text{GATE MA 2019}}}


\item Let $f$ be any polynomial function of degree at most $2$ over $\mathbb{R}$ (the set of all real numbers).  
If the constants $a$ and $b$ are such that
\[
\frac{df}{dx} = a f(x) + 2f(x+1) + b f(x+2), \quad \text{for all } x \in \mathbb{R},
\]
then $4a+3b$ is equal to \underline{\hspace{2cm}} (round off to 2 places of decimal).

\hfill{\brak{\text{GATE MA 2019}}}


\item Let $L$ denote the value of the line integral
\[
\oint_C \brak{3x - 4x^2 y}\,dx + \brak{4xy^2 + 2y}\,dy,
\]
where $C$, a circle of radius $2$ with centre at origin of the $xy$-plane, is traversed once in the anti-clockwise direction. Then $\dfrac{L}{\pi}$ is equal to \underline{\hspace{2cm}}.

\hfill{\brak{\text{GATE MA 2019}}}
\item The temperature $T : \mathbb{R}^3 \setminus \{\brak{0,0,0}\} \to \mathbb{R}$ at any point $P\brak{x,y,z}$ is inversely proportional to the square of the distance of $P$ from the origin. If the value of the temperature $T$ at the point $R\brak{0,0,1}$ is $\sqrt{3}$, then the rate of change of $T$ at the point $Q\brak{1,1,2}$ in the direction of $\overrightarrow{QR}$ is equal to \underline{\hspace{2cm}} (round off to 2 places of decimal).  

\brak{\text{$\mathbb{R}$ is the set of all real numbers, $\mathbb{R}^3=\{\brak{x,y,z} : x,y,z\in\mathbb{R}\}$ and $\mathbb{R}^3\setminus\{\brak{0,0,0}\}$ denotes $\mathbb{R}^3$ excluding the origin}}

\hfill{\brak{\text{GATE MA 2019}}}


\item Let $f$ be a continuous function defined on $\brak{0,2}$ such that $f(x) \ge 0$ for all $x \in \brak{0,2}$. If the area bounded by $y=f(x), x=0, y=0$ and $x=b$ is $\sqrt{3+b^2} - \sqrt{3}$, where $b \in \brak{0,2}$, then $f(1)$ is equal to \underline{\hspace{2cm}} (round off to 1 place of decimal).  

\hfill{\brak{\text{GATE MA 2019}}}


\item If the characteristic polynomial and minimal polynomial of a square matrix $A$ are  
\[
(\lambda - 1)(\lambda + 1)^4(\lambda - 2)^5
\quad \text{and} \quad
(\lambda - 1)(\lambda + 1)(\lambda - 2),
\]
respectively, then the rank of the matrix $A+I$ is \underline{\hspace{2cm}}, where $I$ is the identity matrix of appropriate order.  

\hfill{\brak{\text{GATE MA 2019}}}


\item Let $\omega$ be a primitive complex cube root of unity and $i=\sqrt{-1}$. Then the degree of the field extension $\mathbb{Q}\brak{i, \sqrt{3}, \omega}$ over $\mathbb{Q}$ (the field of rational numbers) is \underline{\hspace{2cm}}.  

\hfill{\brak{\text{GATE MA 2019}}}

\item Let
\[
\alpha = \oint_{C} \frac{e^{i\pi z}}{2z^2 - 5z + 2}\,dz, 
\qquad
C : \{\cos t + i \sin t : 0 \le t \le 2\pi\}, \; i=\sqrt{-1}.
\]
Then the greatest integer less than or equal to $\abs{\alpha}$ is \underline{\hspace{2cm}}.

\hfill{\brak{\text{GATE MA 2019}}}


\item Consider the system:
\[
\begin{aligned}
3x_1 + x_2 + 2x_3 - x_4 &= a,\\
x_1 + x_2 + x_3 - 2x_4 &= 3,\\
x_1, x_2, x_3, x_4 &\ge 0.
\end{aligned}
\]
If $x_1=1$, $x_2=b$, $x_3=0$, $x_4=c$ is a basic feasible solution of the above system \brak{\text{where $a,b,c$ are real constants}}, then $a+b+c$ is equal to \underline{\hspace{2cm}}.

\hfill{\brak{\text{GATE MA 2019}}}


\item Let $f : \mathbb{C} \to \mathbb{C}$ be a function defined by $f(z)=z^6-5z^4+10$.  
Then the number of zeros of $f$ in $\{z \in \mathbb{C} : \abs{z} < 2\}$ is \underline{\hspace{2cm}}.  
\brak{\text{$\mathbb{C}$ is the set of all complex numbers}}

\hfill{\brak{\text{GATE MA 2019}}}
\item Let 
\[
\ell^2=\{x=(x_1,x_2,\dots) : x_i \in \mathbb{C},\; \sum_{i=1}^{\infty} \abs{x_i}^2 < \infty\}
\]
be a normed linear space with the norm
\[
\|x\|_2=\left(\sum_{i=1}^{\infty} \abs{x_i}^2\right)^{\tfrac{1}{2}}.
\]
Let $g : \ell^2 \to \mathbb{C}$ be the bounded linear functional defined by
\[
g(x)=\sum_{n=1}^{\infty} \frac{x_n}{3^n} \quad \text{for all } x=(x_1,x_2,\dots)\in \ell^2.
\]
Then $\left(\sup \{ \abs{g(x)} : \|x\|_2 \le 1 \}\right)^2$ is equal to \underline{\hspace{2cm}} \; \brak{\text{round off to 3 places of decimal}}.  

\hfill{\brak{\text{GATE MA 2019}}}
\item For the linear programming problem \brak{\text{LPP}}:
\[
\text{Maximize } Z = 2x_1 + 4x_2,
\]
subject to
\[
- x_1 + 2x_2 \le 4,\qquad 3x_1 + \beta x_2 \le 6,\qquad x_1,x_2 \ge 0,\ \beta \in \mathbb{R},
\]
\brak{\text{$\mathbb{R}$ is the set of all real numbers}}
consider the following statements\brak{:}

I :The LPP always has a finite optimal value for any $\beta \ge 0$.\\
II : The dual of the LPP may be infeasible for some $\beta \ge 0$.\\
III : If for some $\beta$, the point $\brak{1,2}$ is feasible to the dual of the LPP, then $Z \le 16$, for any feasible solution $\brak{x_1,x_2}$ of the LPP.\\
IV : If for some $\beta$, $x_1$ and $x_2$ are the basic variables in the optimal table of the LPP with $x_1 = \tfrac{1}{2}$, then the optimal value of dual of the LPP is $10$.

Then which of the above statements are TRUE?
\begin{enumerate}
\item I and III only
\item I, III and IV only
\item III and IV only
\item II and IV only
\end{enumerate}

\hfill{\brak{\text{GATE MA 2019}}}

\item Let $f : \mathbb{R}^2 \to \mathbb{R}$ be defined by
\[
f\brak{x,y} =
\begin{cases}
\brak{x^2+y^2}\,\sin\!\brak{\dfrac{1}{x^2+y^2}}, & \text{if } \brak{x,y} \ne \brak{0,0},\\[6pt]
0, & \text{if } \brak{x,y}=\brak{0,0}.
\end{cases}
\]
Consider the following statements\brak{:}

I  The partial derivatives $\dfrac{\partial f}{\partial x}, \dfrac{\partial f}{\partial y}$ exist at $\brak{0,0}$, but are unbounded in any neighbourhood of $\brak{0,0}$.\\
II  $f$ is continuous but not differentiable at $\brak{0,0}$.\\
III  $f$ is not continuous at $\brak{0,0}$.\\
IV  $f$ is differentiable at $\brak{0,0}$.\\
\brak{\text{$\mathbb{R}$ is the set of all real numbers and $\mathbb{R}^2=\{\brak{x,y}: x,y \in \mathbb{R}\}$}}


Which of the above statements is/are TRUE?
\begin{enumerate}
\item I and II only
\item I and IV only
\item IV only
\item I only
\end{enumerate}

\hfill{\brak{\text{GATE MA 2019}}}
\item Let $K=\brak{k_{ij}}_{j,i=1}^{\infty}$ be an infinite matrix over $\mathbb{C}$ \brak{\text{the set of all complex numbers}} such that  
\brak{i} for each $i \in \mathbb{N}$ \brak{\text{the set of all natural numbers}}, the $i^{\text{th}}$ row $\brak{k_{i1},k_{i2},\dots}$ of $K$ is in $\ell^{\infty}$, and  
\brak{ii} for every $x=\brak{x_1,x_2,\dots} \in \ell^1$, $\sum_{j=1}^{\infty} k_{ij} x_j$ is summable for all $i \in \mathbb{N}$, and $y=\brak{y_1,y_2,\dots} \in \ell^1$, where $y_i=\sum_{j=1}^{\infty} k_{ij} x_j$.  

Let the set of all rows of $K$ be denoted by $E$. Consider the following statements\brak{:}

P : $E$ is a bounded set in $\ell^{\infty}$.\\
Q : $E$ is a dense set in $\ell^{\infty}$.

\[
\ell^{1}=\left\{ \brak{x_1,x_2,\dots} : x_i \in \mathbb{C},\; \sum_{i=1}^{\infty} \abs{x_i} < \infty \right\}, \qquad
\ell^{\infty}=\left\{ \brak{x_1,x_2,\dots} : x_i \in \mathbb{C},\; \sup_{i \in \mathbb{N}} \abs{x_i} < \infty \right\}
\]

Which of the above statements is/are TRUE?
\begin{enumerate}
\begin{multicols}{2}
\item Both P and Q
\item P only
\item Q only
\item Neither P nor Q
\end{multicols}
\end{enumerate}

\hfill{\brak{\text{GATE MA 2019}}}


\item Consider the following heat conduction problem for a finite rod
\[
\frac{\partial u}{\partial t}-\frac{\partial^2 u}{\partial x^2}= x e^{t}-2t,\quad t>0,\; 0<x<\pi,
\]
with the boundary conditions $u\brak{0,t}=-t^2$, $u\brak{\pi,t}=-\pi e^{t}-t^2$, $t>0$ and the initial condition
$u\brak{x,0}=\sin x - \sin^{3} x - x$, $0 \le x \le \pi$. If $v\brak{x,t}=u\brak{x,t}+x e^{t}+t^{2}$, then which one of the following is CORRECT?
\begin{enumerate}
\begin{multicols}{2}
\item $v\brak{x,t}=\dfrac{1}{4}\brak{e^{-t}\sin x + e^{-9t}\sin 3x}$
\item $v\brak{x,t}=\dfrac{1}{4}\brak{7e^{-t}\sin x - e^{-9t}\sin 3x}$
\item $v\brak{x,t}=\dfrac{1}{4}\brak{e^{-t}\sin x + e^{-3t}\sin 3x}$
\item $v\brak{x,t}=\dfrac{1}{4}\brak{3e^{-t}\sin x - e^{-3t}\sin 3x}$
\end{multicols}
\end{enumerate}

\hfill{\brak{\text{GATE MA 2019}}}
\item Let $f \mathbb{C} \to \mathbb{C}$ be non-zero and analytic at all points in $\mathbb{Z}$.  
If $F(z)=\pi f(z)\cot(\pi z)$ for $z \in \mathbb{C}\setminus \mathbb{Z}$, then the residue of $F$ at $n \in \mathbb{Z}$ is \underline{\hspace{2cm}}.  

\brak{\text{$\mathbb{C}$ is the set of all complex numbers, $\mathbb{Z}$ is the set of all integers and $\mathbb{C}\setminus\mathbb{Z}$ denotes the set of all complex numbers excluding integers}}

\begin{enumerate}
\begin{multicols}{2}
\item $\pi f(n)$  
\item $f(n)$  
\item $\dfrac{f(n)}{\pi}$  
\item $\left.\dfrac{df}{dz}\right|_{z=n}$  
\end{multicols}
\end{enumerate}

\hfill{\brak{\text{GATE MA 2019}}}


\item Let the general integral of the partial differential equation
\[
(2xy-1)\frac{\partial z}{\partial x} + \brak{z-2x^2}\frac{\partial z}{\partial y} = 2(x-yz)
\]
be given by $F(u,v)=0$, where $F : \mathbb{R}^2 \to \mathbb{R}$ is a continuously differentiable function.  
\brak{\text{$\mathbb{R}$ is the set of all real numbers and $\mathbb{R}^2=\{\brak{x,y}:x,y \in \mathbb{R}\}$}}  
Then which one of the following is TRUE?
\begin{enumerate}
\begin{multicols}{2}
\item $u=x^2+y^2+z, \; v=x+ y$  
\item $u=x^2+y^2-z, \; v=x-y$  
\item $u=x^2-y^2+z, \; v=y+ x$  
\item $u=x^2+y^2-z, \; v=y-x$  
\end{multicols}
\end{enumerate}

\hfill{\brak{\text{GATE MA 2019}}}


\item Consider the following statements:  

I : If $\mathbb{Q}$ denotes the additive group of rational numbers and $f :\mathbb{Q}\to \mathbb{Q}$ is a non-trivial homomorphism, then $f$ is an isomorphism.\\
II : Any quotient group of a cyclic group is cyclic.\\
III : If every subgroup of a group $G$ is a normal subgroup, then $G$ is abelian.\\
IV : Every group of order $33$ is cyclic.  

Which of the above statements are TRUE?
\begin{enumerate}
\begin{multicols}{2}
\item I and IV only  
\item II and III only  
\item I, II and IV only  
\item I, III and IV only  
\end{multicols}
\end{enumerate}

\hfill{\brak{\text{GATE MA 2019}}}

\item A solution of the Dirichlet problem
\[
\nabla^2 u\brak{r,\theta}=0,\; 0<r<1,\; -\pi \le \theta \le \pi,\qquad
u\brak{1,\theta}=\abs{\theta},\; -\pi \le \theta \le \pi,
\]
is given by
\begin{enumerate}
\item $u\brak{r,\theta}=\dfrac{\pi}{2}+\dfrac{2}{\pi}\displaystyle\sum_{n=1}^{\infty}
\brak{\dfrac{(-1)^n-1}{n^2}}\, r^{n}\cos\brak{n\theta}$
\item $u\brak{r,\theta}=\dfrac{2}{\pi}\displaystyle\sum_{n=1}^{\infty}
\brak{\dfrac{(-1)^{n+1}}{n^2}}\, r^{n}\cos\brak{n\theta}$
\item $u\brak{r,\theta}=\dfrac{\pi}{2}+\dfrac{2}{\pi}\displaystyle\sum_{n=1}^{\infty}
\brak{\dfrac{(-1)^n}{n^2}}\, r^{n}\cos\brak{n\theta}$
\item $u\brak{r,\theta}=\dfrac{\pi}{2}-\dfrac{2}{\pi}\displaystyle\sum_{n=1}^{\infty}
\brak{\dfrac{(-1)^n+1}{n^2}}\, r^{n}\cos\brak{n\theta}$
\end{enumerate}

\hfill{\brak{\text{GATE MA 2019}}}


\item Consider the subspace $Y=\{\brak{x,x}: x\in \mathbb{C}\}$ of the normed linear space $\brak{\mathbb{C}^2,\|\cdot\|_{\infty}}$.  
If $\phi$ is a bounded linear functional on $Y$, defined by $\phi\brak{x,x}=x$, then which one of the following sets is equal to
\[
\left\{\psi\brak{1,0}: \psi \text{ is a norm preserving extension of }\phi \text{ to } \brak{\mathbb{C}^2,\|\cdot\|_{\infty}}\right\}.
\]
\brak{\text{$\mathbb{C}$ is the set of all complex numbers, $\mathbb{C}^2=\{\brak{x,y}: x,y\in\mathbb{C}\}$ and }
\ \ \ \ \ \ \ \ \ \ \text{$\|(x_1,x_2)\|_{\infty}=\sup\{\abs{x_1},\abs{x_2}\}$}}
\begin{enumerate}
\begin{multicols}{2}
\item $\{1\}$
\item $\left[\dfrac{1}{2},\, \dfrac{3}{2}\right]$
\item $\brak{[1,\infty)}$
\item $\brak{[0,1]}$
\end{multicols}
\end{enumerate}

\hfill{\brak{\text{GATE MA 2019}}}
\item Consider the following statements\brak{:}

I : The ring $\mathbb{Z}\brak{\sqrt{-1}}$ is a unique factorization domain.\\
II : The ring $\mathbb{Z}\brak{\sqrt{-5}}$ is a principal ideal domain.\\
III : In the polynomial ring $\mathbb{Z}_n[x]$, the ideal generated by $x^3 + x + 1$ is a maximal ideal.\\
IV : In the polynomial ring $\mathbb{Z}_n[x]$, the ideal generated by $x^6 + 1$ is a prime ideal.\\

\brak{\text{$\mathbb{Z}$ denotes the set of all integers, $\mathbb{Z}_n$ denotes the set of all integers modulo $n$, for any positive integer $n$}}

Which of the above statements are TRUE?
\begin{enumerate}
\begin{multicols}{2}
\item I, II and III only
\item I and III only
\item I, II and IV only
\item II and III only
\end{multicols}
\end{enumerate}

\hfill{\brak{\text{GATE MA 2019}}}


\item Let $M$ be a $3 \times 3$ real symmetric matrix with eigenvalues $0, 2$ and $a$ with the respective eigenvectors $u=\myvec{4 \\ b \\ c}$, $v=\myvec{-1 \\ 2 \\ 0}$ and $w=\myvec{1 \\ 1 \\ 1}$.  

Consider the following statements\brak{:}

I : $a + b - c = 10$.\\
II : The vector $x=\myvec{0 \\ \tfrac{3}{2} \\ \tfrac{1}{2}}$ satisfies $Mx = v + w$.\\
III : For any $d \in \text{span}\{u,v,w\}$, $Mx = d$ has a solution.\\
IV : The trace of the matrix $M^2 + 2M$ is $8$.\\
\brak{\text{$y^T$ denotes the transpose of the vector $y$}}

Which of the above statements are TRUE?
\begin{enumerate}
\begin{multicols}{2}
\item I, II and III only
\item I and II only
\item II and IV only
\item III and IV only
\end{multicols}
\end{enumerate}

\hfill{\brak{\text{GATE MA 2019}}}
\item Consider the region
\[
\Omega = \{x+iy : -1 \leq x \leq 2, \ -\tfrac{\pi}{3} \leq y \leq \tfrac{\pi}{3} \}, \quad i = \sqrt{-1}
\]
in the complex plane. The transformation $x+iy \mapsto e^{x+iy}$ maps the region $\Omega$ onto the region $S \subset \mathbb{C}$ \brak{\text{the set of all complex numbers}}. Then the area of the region $S$ is equal to
\begin{enumerate}
\begin{multicols}{2}
\item $\tfrac{\pi}{3}\brak{e^{4}-e^{-2}}$
\item $\tfrac{\pi}{4}\brak{e^{4}+e^{-2}}$
\item $\tfrac{2\pi}{3}\brak{e^{4}-e^{-2}}$
\item $\tfrac{\pi}{6}\brak{e^{4}-e^{-2}}$
\end{multicols}
\end{enumerate}

\hfill{\brak{\text{GATE MA 2019}}}


\item Consider the sequence $\{g_n\}_{n=1}^{\infty}$ of functions, where $g_n(x)=\tfrac{x}{1+nx^{2}}, \ x \in \mathbb{R}, \ n \in \mathbb{N}$ and $g_n'(x)$ is the derivative of $g_n(x)$ with respect to $x$.  

\brak{\mathbb{R} \text{ is the set of all real numbers, } \mathbb{N} \text{ is the set of all natural numbers}}


Then which one of the following statements is TRUE?
\begin{enumerate}
\item $\{g_n\}_{n=1}^{\infty}$ does NOT converge uniformly on $\mathbb{R}$
\item $\{g_n'\}_{n=1}^{\infty}$ converges uniformly on any closed interval which does NOT contain $1$
\item $\{g_n'\}_{n=1}^{\infty}$ converges pointwise to a continuous function on $\mathbb{R}$
\item $\{g_n'\}_{n=1}^{\infty}$ converges uniformly on any closed interval which does NOT contain $0$
\end{enumerate}

\hfill{\brak{\text{GATE MA 2019}}}
\item Consider the boundary value problem \brak{\text{BVP}}
\[
\frac{d^{2}y}{dx^{2}} + \alpha y(x)=0,\quad \alpha \in \mathbb{R} \ \brak{\text{the set of all real numbers}},
\]
\brak{k \text{ is a non-zero real number}}
 

Then which one of the following statements is TRUE?
\begin{enumerate}
\item For $\alpha=1$, the BVP has infinitely many solutions
\item For $\alpha=1$, the BVP has a unique solution
\item For $\alpha=-1, \ k<0$, the BVP has a solution $y(x)$ such that $y(x)>0$ for all $x\in(0,\pi)$
\item For $\alpha=-1, \ k>0$, the BVP has a solution $y(x)$ such that $y(x)>0$ for all $x\in(0,\pi)$
\end{enumerate}

\hfill{\brak{\text{GATE MA 2019}}}


\item Consider the ordered square $I_{0}^{2}$, the set $[0,1]\times[0,1]$ with the dictionary order topology. Let the general element of $I_{0}^{2}$ be denoted by $x\times y$, where $x,y\in[0,1]$. Then the closure of the subset
\[
S=\left\{x\times \tfrac{3}{4} : 0<a<x<b<1 \right\}\quad \text{in } I_{0}^{2}
\]
is
\begin{enumerate}
\item $S \cup \brak{(a,b)\times\{0\}} \cup \brak{[a,b)\times\{1\}}$
\item $S \cup \brak{[a,b)\times\{0\}} \cup \brak{(a,b]\times\{1\}}$
\item $S \cup \brak{(a,b)\times\{0\}} \cup \brak{(a,b)\times\{1\}}$
\item $S \cup \brak{(a,b]\times\{0\}}$
\end{enumerate}

\hfill{\brak{\text{GATE MA 2019}}}
\item Let $P_{2}$ be the vector space of all polynomials of degree at most $2$ over $\mathbb{R}$ \brak{\text{the set of real numbers}}. Let a linear transformation $T: P_{2}\to P_{2}$ be defined by
\[
T\brak{a+bx+cx^{2}}=(a+b)+(b-c)x+(a+c)x^{2}.
\]

Consider the following statements:  

I. The null space of $T$ is $\left\{\alpha \brak{-1+x+x^{2}} : \alpha \in \mathbb{R}\right\}$.  

II. The range space of $T$ is spanned by the set $\{1+x^{2},1+x\}$.  

III. $T(T(1+x))=1+x^{2}$.  

IV. If $M$ is the matrix representation of $T$ with respect to the standard basis $\{1,x,x^{2}\}$ of $P_{2}$, then the trace of the matrix $M$ is $3$.  

Which of the above statements are TRUE?
\begin{enumerate}
\item I and II only
\item I, III and IV only
\item I, II and IV only
\item II and IV only
\end{enumerate}

\hfill{\brak{\text{GATE MA 2019}}}
\item Let $T_{1}$ and $T_{2}$ be two topologies defined on $\mathbb{N}$ \brak{\text{the set of all natural numbers}}, where $T_{1}$ is the topology generated by $\mathcal{B}=\left\{\{2n-1,2n\}: n\in\mathbb{N}\right\}$ and $T_{2}$ is the discrete topology on $\mathbb{N}$.  

Consider the following statements:  

I. In $\brak{\mathbb{N},T_{1}}$, every infinite subset has a limit point.  

II. The function $f: \brak{\mathbb{N},T_{1}}\to \brak{\mathbb{N},T_{2}}$ defined by
\[
f\brak{n}=\begin{cases}
\dfrac{n}{2}, & \text{if $n$ is even},\\[6pt]
\dfrac{n+1}{2}, & \text{if $n$ is odd}
\end{cases}
\]
is a continuous function.  

Which of the above statements is/are TRUE?
\begin{enumerate}
\item Both I and II
\item I only
\item II only
\item Neither I nor II
\end{enumerate}

\hfill{\brak{\text{GATE MA 2019}}}


\item Let $1\le p<q<\infty$. Consider the following statements:  

I. $\ell^{p} \subset \ell^{q}$.  

II. $L^{p}[0,1]\subset L^{q}[0,1]$,  

where $\ell^{p}=\left\{ \brak{x_{1},x_{2},\ldots}: x_{i}\in\mathbb{R},\ \sum_{i=1}^{\infty} \abs{x_{i}}^{p}<\infty \right\}$ and
\[
L^{p}[0,1]=\left\{ f: [0,1]\to\mathbb{R}: f \text{ is $\mu$-measurable},\ \int_{[0,1]} \abs{f}^{p}\, d\mu < \infty,\ \mu \text{ is the Lebesgue measure} \right\}.
\]
\brak{\mathbb{R} \text{ is the set of all real numbers}}  

Which of the above statements is/are TRUE?
\begin{enumerate}
\item Both I and II
\item I only
\item II only
\item Neither I nor II
\end{enumerate}

\hfill{\brak{\text{GATE MA 2019}}}
\item Consider the differential equation
\[
t\frac{d^{2}y}{dt^{2}} + 2\frac{dy}{dt} + t y = 0,\quad t>0,\quad y\brak{0+}=1,\quad \left.\frac{dy}{dt}\right|_{t=0+}=0.
\]
If $Y\brak{s}$ is the Laplace transform of $y\brak{t}$, then the value of $Y\brak{1}$ is \underline{\hspace{2cm}} \brak{\text{round off to $2$ places of decimal}}.  
\brak{\text{Here, the inverse trigonometric functions assume principal values only.}}

\hfill{\brak{\text{GATE MA 2019}}}


\item Let $R$ be the region in the $xy$-plane bounded by the curves $y=x^{2},\ y=4x^{2},\ xy=1$ and $xy=5$. Then the value of the integral
\[
\iint_{R}\frac{y^{2}}{x}\, dy\, dx
\]
is \underline{\hspace{2cm}}.

\hfill{\brak{\text{GATE MA 2019}}}


\item Let $V$ be the vector space of all $3\times 3$ matrices with complex entries over the real field. If
\[
W_{1}=\{A\in V : A=\bar{A}^{\,t}\}\quad \text{and}\quad W_{2}=\{A\in V : \text{trace}\brak{A}=0\},
\]
then the dimension of $W_{1}+W_{2}$ is equal to \underline{\hspace{2cm}}.  
\brak{\bar{A}^{\,t} \text{ denotes the conjugate transpose of } A}

\hfill{\brak{\text{GATE MA 2019}}}


\item The number of elements of order $15$ in the additive group $\mathbb{Z}_{60}\times \mathbb{Z}_{50}$ is \underline{\hspace{2cm}}.  
\brak{\mathbb{Z}_{n} \text{ denotes the group of integers modulo } n, \text{ under the operation of addition modulo } n, \text{ for any positive integer } n}

\hfill{\brak{\text{GATE MA 2019}}}
\item Consider the following cost matrix of assigning four jobs to four persons:

\begin{table}[H]
\centering
\begin{tabular}{|c|c|c|c|c|}
\hline
 & $J_1$ & $J_2$ & $J_3$ & $J_4$ \\
\hline
$P_1$ & $5$ & $8$ & $6$ & $10$ \\
\hline
$P_2$ & $2$ & $5$ & $4$ & $8$ \\
\hline
$P_3$ & $6$ & $7$ & $6$ & $9$ \\
\hline
$P_4$ & $6$ & $9$ & $8$ & $10$ \\
\hline
\end{tabular}
\caption*{}
\label{tab:assign4x4}
\end{table}

Then the minimum cost of the assignment problem subject to the constraint that job $J_4$ is assigned to person $P_2$, is \underline{\hspace{2cm}}.

\hfill{\brak{\text{GATE MA 2019}}}


\item Let $y: [-1,1]\to \mathbb{R}$ with $y\brak{1}=1$ satisfy the Legendre differential equation
\[
\brak{1-x^{2}}\frac{d^{2}y}{dx^{2}}-2x\frac{dy}{dx}+6y=0 \quad \text{for } \abs{x}<1.
\]
Then the value of $\displaystyle \int_{-1}^{1} y\brak{x}\brak{x+x^{2}}\,dx$ is equal to \underline{\hspace{2cm}} \ \brak{\text{round off to $2$ places of decimal}}.

\hfill{\brak{\text{GATE MA 2019}}}

\item Let $\mathbb{Z}_{125}$ be the ring of integers modulo $125$ under the operations of addition modulo $125$ and multiplication modulo $125$. If $m$ is the number of maximal ideals of $\mathbb{Z}_{125}$ and $n$ is the number of non-units of $\mathbb{Z}_{125}$, then $m+n$ is equal to \underline{\hspace{2cm}}.

\hfill{\brak{\text{GATE MA 2019}}}


\item The maximum value of the error term of the composite Trapezoidal rule when it is used to evaluate the definite integral
\[
\int_{0.2}^{1.4} \brak{\sin x - \log_{e} x}\, dx
\]
with $12$ sub-intervals of equal length, is equal to \underline{\hspace{2cm}} \ \brak{\text{round off to $3$ places of decimal}}.

\hfill{\brak{\text{GATE MA 2019}}}


\item By the Simplex method, the optimal table of the linear programming problem:  

\[
\text{Maximize } Z = \alpha x_{1} + 3x_{2}
\]
subject to
\begin{align*}
\beta x_{1} + x_{2} + x_{3} &= 8,\\
2x_{1} + x_{2} + x_{4} &= \gamma,\\
x_{1},x_{2},x_{3},x_{4} &\ge 0,
\end{align*}
where $\alpha, \beta, \gamma$ are real constants, is

\begin{table}[H]
\centering
\begin{tabular}{|c|c|c|c|c|c|}
\hline
$C_j \to$ & $\alpha$ & $3$ & $0$ & $0$ & \\
\hline
Basic variable & $x_1$ & $x_2$ & $x_3$ & $x_4$ & Solution \\
\hline
$x_2$ & $1$ & $0$ & $2$ & $-1$ & $6$ \\
\hline
$x_1$ & $0$ & $1$ & $-1$ & $1$ & $2$ \\
\hline
$Z_j - C_j$ & $0$ & $0$ & $2$ & $1$ & - \\
\hline
\end{tabular}
\caption*{}
\label{tab:simplex}
\end{table}

Then the value of $\alpha+\beta+\gamma$ is \underline{\hspace{2cm}}.

\hfill{\brak{\text{GATE MA 2019}}}



\item Consider the inner product space $P_{2}$ of all polynomials of degree at most $2$ over the field of real numbers with the inner product 
\[
\langle f,g\rangle = \int_{0}^{1} f(t)g(t)\, dt \quad \text{for } f,g \in P_{2}.
\]
Let $\{f_{0},f_{1},f_{2}\}$ be an orthogonal set in $P_{2}$, where $f_{0}=1,\ f_{1}=t+c_{1},\ f_{2}=t^{2}+c_{2}f_{1}+c_{3}$ and $c_{1},c_{2},c_{3}$ are real constants. Then the value of $2c_{1}+c_{2}+3c_{3}$ is equal to \underline{\hspace{2cm}}.

\hfill{\brak{\text{GATE MA 2019}}}


\item Consider the system of linear differential equations
\[
\frac{dx_{1}}{dt}=5x_{1}-2x_{2}, \quad \frac{dx_{2}}{dt}=4x_{1}-x_{2},
\]
with the initial conditions $x_{1}(0)=0,\ x_{2}(0)=1$.  

Then $\log_{e}\brak{x_{2}(2)-x_{1}(2)}$ is equal to \underline{\hspace{2cm}}.

\hfill{\brak{\text{GATE MA 2019}}}


\item Consider the differential equation
\[
x\brak{1+x^{2}}\frac{d^{2}y}{dx^{2}} - 9\frac{dy}{dx}+7y=0.
\]
The sum of the roots of the indicial equation of the Frobenius series solution for the above differential equation in a neighborhood of $x=0$ is equal to \underline{\hspace{2cm}}.

\hfill{\brak{\text{GATE MA 2019}}}












\end{enumerate}
\end{document}
