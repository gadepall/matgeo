\documentclass[journal,12pt,onecolumn]{IEEEtran}
\usepackage{graphicx, float}
\graphicspath{{Figs/}}
\usepackage{multicol}
\usepackage{parskip}
\usepackage{titlesec}
\usepackage{color}
\usepackage{enumitem}
\usepackage{amsmath,amssymb,amsfonts,amsthm}
\usepackage{array}
\usepackage{booktabs}
\usepackage[table]{xcolor}
\usepackage{longtable}
\usepackage{gensymb}
\usepackage{cite}
\usepackage{algorithmic}
\usepackage{textcomp}
\usepackage{txfonts}
\usepackage{listings}
\usepackage{mathtools}
\usepackage{comment}
\usepackage{tkz-euclide}
\usepackage[breaklinks=true]{hyperref}
\usepackage{gvv}
\usepackage[utf8]{inputenc}
\usetikzlibrary{arrows.meta, positioning}
\usepackage{xparse}
\usepackage{calc}
\usepackage{multirow}
\usepackage{hhline}
\usepackage{ifthen}
\usepackage{lscape}
\usepackage{tabularx}

\begin{document}

\title{
ASSIGNMENT 1: GATE 2019 \\
BT: BIOTECHNOLOGY ENGINEERING}
\author{AI25BTECH11025-R Nikhil}
\maketitle
\renewcommand{\thefigure}{\theenumi}
\renewcommand{\thetable}{\theenumi}


\section*{ - General Aptitude (GA) Questions}

\begin{enumerate}

    \item The fishermen,------------- the flood victims owed their lives, were rewarded by the government.
    \begin{multicols}{4}
    \begin{enumerate}
        \item whom  
        \item to which  
        \item to whom  
        \item that  
    \end{enumerate}
    \end{multicols}\hfill(GATE BT 2019)

    \item Some students were not involved in the strike.

    If the above statement is true, which of the following conclusions is/are logically necessary?
    \begin{enumerate}
        \item Some who were involved in the strike were students.  
        \item No student was involved in the strike.  
        \item At least one student was involved in the strike.  
        \item Some who were not involved in the strike were students.  
    \end{enumerate}
    \begin{multicols}{4}
    \begin{enumerate}
        \item $1$ and $2$  
        \item $3$  
        \item $4$  
        \item $2$ and $3$  
    \end{enumerate}
    \end{multicols}\hfill(GATE BT 2019)

    \item The radius as well as the height of a circular cone increases by $10\%$. The percentage increase in its volume is:
    \begin{multicols}{4}
    \begin{enumerate}
        \item $17.1$  
        \item $21.0$  
        \item $33.1$  
        \item $72.8$  
    \end{enumerate}
    \end{multicols}\hfill(GATE BT 2019)

    \item Five numbers $10$, $7$, $5$, $4$ and $2$ are to be arranged in a sequence from left to right following the directions given below:
    \begin{itemize}
        \item No two odd or even numbers are next to each other.  
        \item The second number from the left is exactly half of the left-most number.  
        \item The middle number is exactly twice the right-most number.  
    \end{itemize}
    Which is the second number from the right?
    \begin{multicols}{4}
    \begin{enumerate}
        \item $2$  
        \item $4$  
        \item $7$  
        \item $10$  
    \end{enumerate}
    \end{multicols}\hfill(GATE BT 2019)

    \item Until Iran came along, India had never been ---------- in kabaddi.
    \begin{multicols}{4}
    \begin{enumerate}
        \item defeated  
        \item defeating  
        \item defeat  
        \item defeatist  
    \end{enumerate}
    \end{multicols}\hfill(GATE BT 2019)


    \item Since the last one year, after a 125 basis point reduction in repo rate by the Reserve Bank of India, banking institutions have been making a demand to reduce interest rates on small saving schemes. Finally, the government announced yesterday a reduction in interest rates on small saving schemes to bring them on par with fixed deposit interest rates. \\ 
    Which one of the following statements can be inferred from the given passage?
    \begin{enumerate}
        \item Whenever the Reserve Bank of India reduces the repo rate, the interest rates on small saving schemes are also reduced  
        \item Interest rates on small saving schemes are always maintained on par with fixed deposit interest rates  
        \item The government sometimes takes into consideration the demands of banking institutions before reducing the interest rates on small saving schemes  
        \item A reduction in interest rates on small saving schemes follow only after a reduction in repo rate by the Reserve Bank of India  
    \end{enumerate}
    \hfill(GATE BT 2019)

    \item In a country of $1400$ million population, $70\%$ own mobile phones. Among the mobile phone owners, only $294$ million access the Internet. Among these Internet users, only half buy goods from e-commerce portals. What is the percentage of these buyers in the country?
    \begin{multicols}{4}
    \begin{enumerate}
        \item $10.50$  
        \item $14.70$  
        \item $15.00$  
        \item $50.00$  
    \end{enumerate}
    \end{multicols}\hfill(GATE BT 2019)

    \item The nomenclature of Hindustani music has changed over the centuries. Since the medieval period, \textit{dhrupad} styles were identified as \textit{baanis}. Terms like \textit{gayaki} and \textit{baaj} were used to refer to vocal and instrumental styles, respectively. With the institutionalization of music education, the term \textit{gharana} became acceptable. \textit{Gharana} originally referred to hereditary musicians from a particular lineage, including disciples and grand disciples.  
    Which one of the following pairings is NOT correct?
    \begin{enumerate}
        \item dhrupad, baani  
        \item gayaki, vocal  
        \item baaj, institution  
        \item gharana, lineage  
    \end{enumerate}
    \hfill(GATE BT 2019)

    \item Two trains started at $7$ AM from the same point. The first train travelled north at a speed of $80$ km/h and the second train travelled south at a speed of $100$ km/h. The time at which they were $540$ km apart is AM.
    \begin{multicols}{4}
    \begin{enumerate}
        \item $9$  
        \item $10$  
        \item $11$  
        \item $11.30$  
    \end{enumerate}
    \end{multicols}\hfill(GATE BT 2019)

    \item “I read somewhere that in ancient times the prestige of a kingdom depended upon the number of taxes that it was able to levy on its people. It was very much like the prestige of a head-hunter in his own community.”  
    Based on the paragraph above, the prestige of a head-hunter depended upon:
    \begin{enumerate}
        \item the prestige of the kingdom  
        \item the prestige of the heads  
        \item the number of taxes he could levy  
        \item the number of heads he could gather  
    \end{enumerate}
    \hfill(GATE BT 2019)


% Q1
\item The Bt toxin gene from \textit{Bacillus thuringiensis} used to generate genetically modified crops is
\begin{multicols}{4}
\begin{enumerate}
\item cry
\item cro
\item cdc
\item cre
\end{enumerate}
\end{multicols}\hfill(GATE BT 2019)

% Q2
\item Which one of the following is used as a pH indicator in animal cell culture medium?
\begin{multicols}{2}
\begin{enumerate}
\item Acridine orange
\item Phenol red
\item Bromophenol blue
\item Coomassie blue
\end{enumerate}
\end{multicols}\hfill(GATE BT 2019)

% Q3
\item Tetracycline inhibits the
\begin{enumerate}
\item interaction between tRNA and mRNA
\item translocation of mRNA through ribosome
\item peptidyl transferase activity
\item binding of amino-acyl tRNA to ribosome
\end{enumerate}
\hfill(GATE BT 2019)

% Q4
\item Which one of the following is a database of protein sequence motifs?
\begin{multicols}{4}
\begin{enumerate}
\item PROSITE
\item TrEMBL
\item SWISSPROT
\item PDB
\end{enumerate}
\end{multicols}\hfill(GATE BT 2019)

% Q5
\item Which one of the following enzymes is encoded by human immunodeficiency virus (HIV) genome?
\begin{multicols}{2}
\begin{enumerate}
\item Reverse transcriptase
\item Phospholipase
\item Phosphatase
\item ATP synthase
\end{enumerate}
\end{multicols}\hfill(GATE BT 2019)

% Q6
\item DNA synthesis in eukaryotes occurs during which phase of the mitotic cell cycle?
\begin{multicols}{4}
\begin{enumerate}
\item M
\item $G_1$
\item S
\item $G_0$
\end{enumerate}
\end{multicols}\hfill(GATE BT 2019)

% --------- Question 7 (Matching type MCQ) ---------
\item Match the human diseases in \textbf{Group I} with the causative agents in \textbf{Group II}.

\begin{table}[H]
\begin{tabular}{cc}
\textbf{Group I} & \textbf{Group II} \\[6pt]
P) Amoebiasis & 1) \textit{Leishmania donovani} \\ 
Q) African sleeping sickness & 2) \textit{Trypanosoma cruzi} \\ 
R) Kala azar & 3) \textit{Entamoeba histolytica} \\ 
S) Chagas’ disease & 4) \textit{Trypanosoma gambiense} \\ 
\end{tabular}
\end{table}

\begin{multicols}{2}
\begin{enumerate}
\item P-3, Q-4, R-2, S-1
\item P-3, Q-2, R-1, S-4
\item P-3, Q-4, R-1, S-2
\item P-4, Q-3, R-1, S-2
\end{enumerate}
\end{multicols}\hfill(GATE BT 2019)


% Q8
\item Which one of the following techniques can be used to compare the expression of a large number of genes in two biological samples in a single experiment?

\begin{multicols}{2}
\begin{enumerate}
\item Polymerase chain reaction
\item DNA microarray
\item Northern hybridization
\item Southern hybridization
\end{enumerate}
\end{multicols}\hfill(GATE BT 2019)

% Q9
\item Which of the following processes can increase genetic diversity of bacteria in nature? \\[4pt]
P. Conjugation 

Q. Transformation 

R. Transduction 

S. Transfection

\begin{multicols}{4}
\begin{enumerate}
\item P only
\item P and Q only
\item P, Q and R only
\item P, Q, R and S
\end{enumerate}
\end{multicols}\hfill(GATE BT 2019)

% Q10
\item Which one of the following is NOT a part of the human nonspecific defense system?

\begin{multicols}{4}
\begin{enumerate}
\item Interferon
\item Mucous
\item Saliva
\item Antibody
\end{enumerate}
\end{multicols}\hfill(GATE BT 2019)


    \item A mutation in a gene that codes for a polypeptide results in a variant polypeptide that lacks the last three amino acids. What type of mutation is this?
    \begin{multicols}{2}
    \begin{enumerate}[label=(\Alph*)]
        \item Synonymous mutation  
        \item Nonsense mutation  
        \item Missense mutation  
        \item Silent mutation  
    \end{enumerate}
    \end{multicols}\hfill(GATE BT 2019)

   \item Which one of the following equations represents a one-dimensional wave equation?

\begin{multicols}{2}
\begin{enumerate}
\item $\frac{\partial u}{\partial t} = c^2 \frac{\partial^2 u}{\partial x^2}$
\item $\frac{\partial^2 u}{\partial t^2} = c^2 \frac{\partial^2 u}{\partial x^2}$
\item $\frac{\partial^2 u}{\partial x^2} = c^2 \frac{\partial u}{\partial x}$
\item $\frac{\partial^2 u}{\partial t^2} + \frac{\partial^2 u}{\partial x^2} = 0$
\end{enumerate}
\end{multicols}\hfill(GATE BT 2019)

    \item Which of the following are geometric series?
    \begin{itemize}
        \item[P.] $1,\ 6,\ 11,\ 16,\ 21,\ 26,\ \ldots$  
        \item[Q.] $9,\ 6,\ 3,\ 0,\ -3,\ -6,\ \ldots$  
        \item[R.] $1,\ 3,\ 9,\ 27,\ 81,\ \ldots$  
        \item[S.] $4,\ -8,\ 16,\ -32,\ 64,\ \ldots$  
    \end{itemize}
    \begin{multicols}{4}
    \begin{enumerate}[label=(\Alph*)]
        \item P and Q only  
        \item R and S only  
        \item Q and S only  
        \item P, Q and R only  
    \end{enumerate}
    \end{multicols}\hfill(GATE BT 2019)

\item Which one of the following statements is CORRECT for enzyme catalyzed reactions? 
($\Delta G$ is Gibbs free energy change, $K_{eq}$ is equilibrium constant)

\begin{enumerate}
\item Enzymes affect $\Delta G$, but not $K_{eq}$
\item Enzymes affect $K_{eq}$, but not $\Delta G$
\item Enzymes affect both $\Delta G$ and $K_{eq}$
\item Enzymes do not affect $\Delta G$ or $K_{eq}$
\end{enumerate}
\hfill(GATE BT 2019)

\item Which one of the following can NOT be a limiting substrate if Monod’s growth kinetics is applicable?

\begin{enumerate}
\item Extracellular carbon source
\item Extracellular nitrogen source
\item Dissolved oxygen
\item Intracellular carbon source
\end{enumerate}
\hfill(GATE BT 2019)

    \item Which one of the following is the unit of heat transfer coefficient?
    \begin{multicols}{4}
    \begin{enumerate}[label=(\Alph*)]
        \item W m$^2$ K$^{-1}$  
        \item W m$^{-2}$K  
        \item W m$^{-2}$K$^{-1}$  
        \item W m$^2$K  
    \end{enumerate}
    \end{multicols}\hfill(GATE BT 2019)

    \item Which one of the following is catabolized during endogenous metabolism in a batch bacterial cultivation?
    \begin{multicols}{2}
    \begin{enumerate}[label=(\Alph*)]
        \item internal reserves  
        \item extracellular substrates  
        \item extracellular products  
        \item toxic substrates  
    \end{enumerate}
    \end{multicols}\hfill(GATE BT 2019)

    \item Which one of the following need NOT be conserved in a biochemical reaction?
    \begin{multicols}{2}
    \begin{enumerate}[label=(\Alph*)]
        \item Total mass  
        \item Total moles  
        \item Number of atoms of each element  
        \item Total energy  
    \end{enumerate}
    \end{multicols}\hfill(GATE BT 2019)

    \item The number of possible rooted trees in a phylogeny of three species is ----------

\hfill(GATE BT 2019)    

  % Q20
\item Matrix $A = \begin{bmatrix} 0 & 6 \\ p & 0 \end{bmatrix}$ will be skew-symmetric when $p = \underline{\hspace{1cm}}$.

\hfill(GATE BT 2019)

% Q21
\item The solution of $\lim\limits_{x \to 8} \dfrac{x^2 - 64}{x - 8}$ is $\underline{\hspace{1cm}}$.

\hfill(GATE BT 2019)

% Q22
\item The median value for the dataset $(12, 10, 16, 8, 90, 50, 30, 24)$ is $\underline{\hspace{1cm}}$.

\hfill(GATE BT 2019)

% Q23
\item The degree of reduction for acetic acid $(C_2H_4O_2)$ is $\underline{\hspace{1cm}}$.

\hfill(GATE BT 2019)
% Q24
\item The mass of 1 kmol of oxygen molecules is $\underline{\hspace{1cm}}$ g (rounded off to the nearest integer).

\hfill(GATE BT 2019)

% Q25
\item Protein concentration of a crude enzyme preparation was $10 \, \text{mg mL}^{-1}$. 
$10 \, \mu L$ of this sample gave an activity of $5 \, \mu\text{mol min}^{-1}$ under standard assay conditions. 
The specific activity of this crude enzyme preparation is $\underline{\hspace{1cm}}$ units mg$^{-1}$.

\hfill(GATE BT 2019)

% Q26
\item In general, which one of the following statements is NOT CORRECT?

\begin{enumerate}
\item Hydrogen bonds result from electrostatic interactions
\item Hydrogen bonds contribute to the folding energy of proteins
\item Hydrogen bonds are weaker than van der Waals interactions
\item Hydrogen bonds are directional
\end{enumerate}
\hfill(GATE BT 2019)

% Q27
\item For site-directed mutagenesis, which one of the following restriction enzymes is used to digest methylated DNA?

\begin{multicols}{4}
\begin{enumerate}
\item KpnI
\item DpnI
\item XhoI
\item MluI
\end{enumerate}
\end{multicols}\hfill(GATE BT 2019)

% Q28 (Match the following)
\item Match the organelles in Group I with their functions in Group II. \\[6pt]

\begin{tabular}{c c}
\textbf{Group I} & \textbf{Group II} \\
P. Lysosome & 1. Digestion of foreign substances \\
Q. Smooth ER & 2. Protein targeting \\
R. Golgi apparatus & 3. Lipid synthesis \\
S. Nucleolus & 4. Protein synthesis \\
 & 5. rRNA synthesis \\
\end{tabular}

\begin{multicols}{2}
\begin{enumerate}
\item P-1, Q-3, R-2, S-5
\item P-1, Q-4, R-5, S-3
\item P-2, Q-5, R-3, S-4
\item P-1, Q-3, R-4, S-5
\end{enumerate}
\end{multicols}\hfill(GATE BT 2019)

% Q29
\item Which of the following statements are CORRECT when a protein sequence database is searched using the BLAST algorithm? \\[6pt]

P. A larger E-value indicates higher sequence similarity \\
Q. E-value $< 10^{-10}$ indicates sequence homology \\
R. A higher BLAST score indicates higher sequence similarity \\
S. E-value $> 10^{10}$ indicates sequence homology

\begin{multicols}{2}
\begin{enumerate}
\item P, Q and R only
\item Q and R only
\item P, R and S only
\item P and S only
\end{enumerate}
\end{multicols}\hfill(GATE BT 2019)

% Q30
\item Which one of the following is coded by the ABO blood group locus in the human genome?

\begin{multicols}{2}
\begin{enumerate}
\item Acyl transferase
\item Galactosyltransferase
\item Transposase
\item $\beta$-Galactosidase
\end{enumerate}
\end{multicols}\hfill(GATE BT 2019)


% Q31
\item Which of the following factors affect the fidelity of DNA polymerase in polymerase chain reaction? 

P. $Mg^{2+}$ concentration 

Q. pH 

R. Annealing temperature

\begin{multicols}{2}
\begin{enumerate}
\item P and Q only
\item P and R only
\item Q and R only
\item P, Q and R
\end{enumerate}
\end{multicols}\hfill(GATE BT 2019)

% Q32 Match the following
\item Group I lists spectroscopic methods and Group II lists biomolecular applications of those methods. Match the methods in Group I with the applications in Group II. \\[6pt]

\begin{tabular}{c c}
\textbf{Group I} & \textbf{Group II} \\
P. Infrared & 1. Identification of functional groups \\
Q. Circular Dichroism & 2. Determination of secondary structure \\
R. Nuclear Magnetic Resonance & 3. Estimation of molecular weight \\
 & 4. Determination of 3-D structure \\
\end{tabular}

\begin{multicols}{2}
\begin{enumerate}
\item P-4, Q-3, R-1
\item P-2, Q-1, R-3
\item P-1, Q-2, R-4
\item P-3, Q-2, R-4
\end{enumerate}
\end{multicols}\hfill(GATE BT 2019)

% Q33
\item The hexapeptide P has an isoelectric point (pI) of 6.9. Hexapeptide Q is a variant of P that contains valine instead of glutamate at position 3. The two peptides are analyzed by polyacrylamide gel electrophoresis at pH 8.0. Which one of the following statements is CORRECT?
\begin{enumerate}
\item P will migrate faster than Q towards the anode
\item P will migrate faster than Q towards the cathode
\item Both P and Q will migrate together
\item Q will migrate faster than P towards the anode
\end{enumerate}\hfill(GATE BT 2019)
\hfill(GATE BT 2019)

% Q34
\item Antibody-producing hybridoma cells are generated by the fusion of a

\begin{multicols}{2}
\begin{enumerate}
\item T cell with a myeloma cell
\item B cell with a myeloma cell
\item Macrophage with a myeloma cell
\item T cell and a B cell
\end{enumerate}
\end{multicols}\hfill(GATE BT 2019)

% Q35
\item Which of the following statements are CORRECT about the function of fetal bovine serum in animal cell culture? \\[6pt]

P. It stimulates cell growth 

Q. It enhances cell attachment 

R. It provides hormones and minerals 

S. It maintains pH at 7.4

\begin{multicols}{4}
\begin{enumerate}
\item P and Q only
\item P and S only
\item P, Q and R only
\item P, Q, R and S
\end{enumerate}
\end{multicols}\hfill(GATE BT 2019)



% Q36
\item Which one of the following covalent linkages exists between 7-Methyl guanosine (m$^7$G) and mRNAs?

\begin{multicols}{2}
\begin{enumerate}
\item 2'-3' triphosphate
\item 3'-5' triphosphate
\item 5'-5' triphosphate
\item 2'-5' triphosphate
\end{enumerate}
\end{multicols}\hfill(GATE BT 2019)

% Q37
\item Which one of the following amino acid residues will destabilize an $\alpha$-helix when inserted in the middle of the helix?

\begin{multicols}{4}
\begin{enumerate}
\item Pro
\item Val
\item Ile
\item Leu
\end{enumerate}
\end{multicols}\hfill(GATE BT 2019)

% Q38
\item What is the solution of the differential equation $\dfrac{dy}{dx} = \dfrac{x}{y}$, with the initial condition, at $x = 0, y = 1$?

\begin{multicols}{4}
\begin{enumerate}
\item $x^2 = y^2 + 1$
\item $y^2 = x^2 + 1$
\item $y^2 = 2x^2 + 1$
\item $x^2 - y^2 = 0$
\end{enumerate}
\end{multicols}\hfill(GATE BT 2019)

% Q39
\item The Laplace transform of the function $f(t) = t^2 + 2t + 1$ is

\begin{multicols}{4}
\begin{enumerate}
\item $\dfrac{1}{s^3} + \dfrac{3}{s^2} + \dfrac{1}{s}$
\item $\dfrac{4}{s^3} + \dfrac{4}{s^2} + \dfrac{1}{s}$
\item $\dfrac{2}{s^3} + \dfrac{2}{s^2} + \dfrac{1}{s}$
\item $\dfrac{2}{s^3} + \dfrac{3}{s^2} + \dfrac{1}{s}$
\end{enumerate}
\end{multicols}\hfill(GATE BT 2019)

% Q40
\item Which of the following factors can influence the lag phase of a microbial culture in a batch fermentor? \\[6pt]

P. Inoculum size 

Q. Inoculum age 

R. Medium composition

\begin{multicols}{4}
\begin{enumerate}
\item P and Q only
\item Q and R only
\item P and R only
\item P, Q and R
\end{enumerate}
\end{multicols}\hfill(GATE BT 2019)

% Q41
\item Which one of the following statements is CORRECT about proportional controllers?


\begin{enumerate}
\item The initial change in control output signal is relatively slow
\item The initial corrective action is greater for larger error
\item They have no offset
\item There is no corrective action if the error is a constant
\end{enumerate}
\hfill(GATE BT 2019)

% Q42
\item Match the instruments in Group I with their corresponding measurements in Group II. \\[6pt]

\begin{tabular}{c c}
\textbf{Group I} & \textbf{Group II} \\
P. Manometer & 1. Agitator speed \\
Q. Rotameter & 2. Pressure difference \\
R. Tachometer & 3. Cell number \\
S. Haemocytometer & 4. Air flow rate \\
\end{tabular}

\begin{multicols}{2}
\begin{enumerate}
\item P-4, Q-1, R-2, S-3
\item P-3, Q-4, R-1, S-2
\item P-2, Q-4, R-1, S-3
\item P-2, Q-1, R-4, S-3
\end{enumerate}
\end{multicols}\hfill(GATE BT 2019)

% Q43
\item Which of the following statements is ALWAYS CORRECT about an ideal chemostat?  

P. Substrate concentration inside the chemostat is equal to that in the exit stream  

Q. Optimal dilution rate is lower than critical dilution rate  

R. Biomass concentration increases with increase in dilution rate  

S. Cell recirculation facilitates operation beyond critical dilution rate  

\begin{multicols}{4}
\begin{enumerate}
\item P and Q only  
\item P, R and S only  
\item P and S only  
\item P, Q and S only  
\end{enumerate}
\end{multicols}\hfill(GATE BT 2019)

% Q44
\item Determine the correctness or otherwise of the following Assertion [a] and the Reason [r]  

\textbf{Assertion} [a]: It is possible to regenerate a whole plant from a single plant cell.  

\textbf{Reason} [r]: It is easier to introduce transgenes into plants than animals.  

\begin{enumerate}
\item Both [a] and [r] are true and [r] is the correct reason for [a]  
\item Both [a] and [r] are true but [r] is not the correct reason for [a]  
\item Both [a] and [r] are false  
\item only [a] is true but [r] is false  
\end{enumerate}
\hfill(GATE BT 2019)


% Q45
\item A UV-visible spectrophotometer has a minimum detectable absorbance of 0.02. The minimum concentration of a protein sample that can be measured reliably in this instrument with a cuvette of $1 \, \text{cm}$ path length is $\\\\$ $\mu M$. The molar extinction coefficient of the protein is $10,000 \, L \, mol^{-1} \, cm^{-1}$.

\hfill(GATE BT 2019)


% Q46
\item The difference in concentrations of an uncharged solute between two compartments is 1.6-fold. The energy required for active transport of the solute across the membrane separating the two compartments is $\\\\$ cal mol$^{-1}$ (rounded off to the nearest integer). (R = $1.987 \, \text{cal mol}^{-1}\text{K}^{-1}$, T = $37^\circ C$)

\hfill(GATE BT 2019)


% Q47
\item In pea plants, purple color of flowers is determined by the dominant allele while white color is determined by the recessive allele. A genetic cross between two purple flower-bearing plants results in an offspring with white flowers. The probability that the third offspring from these parents will have purple flowers is $\\\\$ (rounded off to 2 decimal places).

\hfill(GATE BT 2019)


% Q48
\item The molecular mass of a protein is 22 kDa. The size of the cDNA (excluding the untranslated regions) that codes for this protein is $\\\\$ kb (rounded off to 1 decimal place).

\hfill(GATE BT 2019)




% Q49
\item A new game is being introduced in a casino. A player can lose Rs. 100, break even, win Rs. 100, or win Rs. 500. The probabilities $P(X)$ of each of these outcomes ($X$) are given in the following table:

\begin{table}[H]
\centering
\begin{tabular}{|c|c|c|c|c|}
\hline
$X$ (in Rs.) & -100 & 0 & 100 & 500 \\
\hline
$P(X)$ & 0.25 & 0.5 & 0.2 & 0.05 \\
\hline
\end{tabular}
\end{table}

The standard deviation $(\sigma)$ for the casino payout is $\\\\$ (rounded off to the nearest integer).

\hfill(GATE BT 2019)

% Q50
\item $\int_{-1}^{1} f(x) dx$ calculated using trapezoidal rule for the values given in the table is $\\\\$ (rounded off to 2 decimal places).  

\begin{table}[H]
\centering
\begin{tabular}{|c|c|c|c|c|c|c|c|}
\hline
$x$ & -1 & $-\tfrac{2}{3}$ & $-\tfrac{1}{3}$ & 0 & $\tfrac{1}{3}$ & $\tfrac{2}{3}$ & 1 \\
\hline
$f(x)$ & 0.37 & 0.51 & 0.71 & 1.00 & 1.40 & 1.95 & 2.71 \\
\hline
\end{tabular}
\end{table}

\hfill(GATE BT 2019)


% Q51
\item Yeast biomass (C$6$H${10}$O$_3$N) grown on glucose is described by the stoichiometric equation given below:  

\[
C_6H_{12}O_6 + 0.48 \, NH_3 + 3 \, O_2 \;\; \rightarrow \;\; 0.48 \, C_6H_{10}O_3N + 3.12 \, CO_2 + 4.32 \, H_2O
\]

The amount of glucose needed for the production of 50 g L$^{-1}$ of yeast biomass in a batch reactor with a working volume of 100{,}000 L is $\\\\$ kg (rounded off to the nearest integer).

\hfill(GATE BT 2019)


% Q52
\item Phenolic wastewater discharged from an industry was treated with \textit{Pseudomonas} sp. in an aerobic bioreactor. The influent and effluent concentrations of phenol were 10{,}000 and 10 ppm, respectively. The inlet feed rate of wastewater was 80 L h$^{-1}$. The kinetic parameters of the organism are as follows:  

Maximum specific growth rate ($\mu_{m}$) = 1 h$^{-1}$  

Saturation constant ($K_{s}$) = 100 mg L$^{-1}$  

Cell death rate ($k_{d}$) = 0.01 h$^{-1}$  

Assuming that the bioreactor operates under ‘chemostat’ mode, the working volume required for this process is------------L (rounded off to the nearest integer).

\hfill(GATE BT 2019)


% Q53
\item In a cross-flow filtration process, the pressure drop $(\Delta P)$ driving the fluid flow is 2 atm, inlet feed pressure $(P_{i})$ is 3 atm and filtrate pressure $(P_{f})$ is equal to atmospheric pressure. The average transmembrane pressure drop $(\Delta P_{m})$ is ------- atm.

\hfill(GATE BT 2019)


% Q54
\item An industrial fermentor containing 10{,}000 L of medium needs to be sterilized. The initial spore concentration in the medium is $10^{6}$ spores mL$^{-1}$. The desired probability of contamination after sterilization is $10^{-3}$. The death rate of spores at 121$^{\circ}$C is $4 \, \text{min}^{-1}$. Assume that there is no cell death during heating and cooling phases. The holding time of the sterilization process is ------ min (rounded off to the nearest integer).

\hfill(GATE BT 2019)


% Q55
\item The dimensions and operating condition of a lab-scale fermentor are as follows:  

Volume = 1 L 

Diameter = 20 cm 

Agitator speed = 600 rpm 

Ratio of impeller diameter to fermentor diameter = 0.3  

This fermentor needs to be scaled up to 8,000 L for a large scale industrial application. If the scale-up is based on constant impeller tip speed, the speed of the agitator in the larger reactor is -------rpm. Assume that the scale-up factor is the cube root of the ratio of fermentor volumes.

\hfill(GATE BT 2019)



  
     
\end{enumerate}

\end{document}

   
   










