\begin{enumerate}[label=\thesubsection.\arabic*.,ref=\thesubsection.\theenumi]
%
%
\item 
	In \figref{fig:tri_alt_h},	
$AD \perp BC$ and $BE \perp AC$ are defined to be the altitudes of $\triangle ABC$.  Let $\vec{H}$ be the intersection of the altitudes $AD$ and $BE$ as shown in Fig. \ref{fig:tri_alt_h}.  $CH$ is extended to meet $AB$ at $\vec{F}$.  Show that $CF \perp AB$.
%
\begin{figure}[!ht]
	\begin{center}
		\resizebox{0.75\columnwidth}{!}{
		\input{figs/triangle/tri_alt_h.tex}
		}
	\end{center}
	\caption{Altitudes of a triangle meet at the orthocentre $H$}
	\label{fig:tri_alt_h}	
\end{figure}
%
\\
\solution $\because AD \perp BC, BE \perp AC$,
%
\begin{align}
\brak{\vec{B}-\vec{C}}^{\top}\brak{\vec{H}-\vec{A}} &= 0  
\\
\brak{\vec{C}-\vec{A}}^{\top}\brak{\vec{H}-\vec{B}} &= 0  
\end{align}
%
Adding both the above and simplifying, 
%
\begin{align}
\brak{\vec{B}-\vec{A}}^{\top}\brak{\vec{H}-\vec{C}} &= 0  
\end{align}
%
$\implies CH \perp AB$, or $CF \perp AB$.  
%
\item 
  In 
	\figref{fig:tri_med_isect}	
	If $\vec{G}$ divides $BE$ and $CF$ in the ratios $k_1$ and $k_2$ respectively, 
	show that
%	Using Fig. \ref{ch2_median_ratio_val}, 
	\begin{align}
\label{eq:tri_med_centroid_ratio}
k_1 = k_2 = 2 
	\end{align}
\begin{figure}[!ht]
	\begin{center}
		\resizebox{0.75\columnwidth}{!}{\input{figs/triangle/tri_med_isect.tex}}
%		\resizebox{\columnwidth}{!}{\input{./figs/coord/tri_med_meet.tex}}
	\end{center}
	\caption{$k_1=k_2$.}
	\label{fig:tri_med_isect}	
	%\label{fig:tri_med_meet}	
\end{figure}
%
\solution 
Using 
	  \eqref{eq:section_formula-alg}
\begin{align}
\vec{E}=\brak{\frac{\vec{A}+\vec{C}}{2}},
\vec{F}=\brak{\frac{\vec{A}+\vec{B}}{2}}
\end{align}
and
  \begin{align}
	  \label{eq:section_formula-G}
\vec{G} = 
	   \frac{k_1\vec{E}+ \vec{B}}{k_1+1}
	  &= \frac{k_2\vec{F}+ \vec{C}}{k_2+1}
	  \\
	  \implies 
	   \brak{k_2+1}\sbrak{k_1\brak{\frac{\vec{A}+\vec{C}}{2}}+ \vec{B}}
	  &= \brak{k_1+1}\sbrak{k_2\brak{\frac{\vec{A}+\vec{B}}{2}}+ \vec{C}}
	  \label{eq:section_formula-G-val}
  \end{align}
  yielding
\begin{multline}
\sbrak{k_2+1-\frac{k_2\brak{k_1+1}}{2}}\vec{B}-\sbrak{\frac{k_2\brak{k_1+1}}{2}-\frac{k_1\brak{k_2+1}}{2}}\vec{A}
\\
-\sbrak{k_1+1-\frac{k_1\brak{k_2+1}}{2}}\vec{C}
	  =0 
\end{multline}
Comparing the above with 
	  \eqref{eq:two-tri-indep},
	  we obtain the equations
\begin{align}
k_2+1-\frac{k_2\brak{k_1+1}}{2}=0
\\
\frac{k_2\brak{k_1+1}}{2}-\frac{k_1\brak{k_2+1}}{2}=0
\\
k_1+1-\frac{k_1\brak{k_2+1}}{2}
	  =0 
\end{align}
yielding
\begin{align}
k_1=k_2
,\
k_1^2-k_1-2=0
\\
\text{or, }\brak{k_1 - 2}\brak{k_1+1} = 0
\end{align}
  resulting in 
\eqref{eq:tri_med_centroid_ratio}.
 \item 
In	\figref{fig:tri_med_meet},	
$AG$ is extended to join $BC$ at $\vec{D}$.  Show that $AD$ is also a median.
\begin{figure}[!ht]
	\begin{center}
%		\resizebox{\columnwidth}{!}{\input{./figs/coord/tri_med_isect.tex}}
		\resizebox{0.75\columnwidth}{!}{\input{./figs/triangle/tri_med_meet.tex}}
	\end{center}
	\caption{$k_3 = 2, k_4 =1$}
%	\label{fig:tri_med_isect}	
	\label{fig:tri_med_meet}	
\end{figure}
	\\
	\solution 
Substituting $k_1=k_2 = 2$ in \eqref{eq:section_formula-G-val},
\begin{align}
	\vec{G} = \frac{\vec{A}+\vec{B}+\vec{C}}{3}
	  \label{eq:centroid-G}
\end{align}
	Considering the ratios in 
	\figref{fig:tri_med_meet},	
  \begin{align}
\vec{G} = 
	  \frac{k_3\vec{D}+\vec{A} }{k_3+1} 
	  ,\
	\vec{D}  =\frac{k_4\vec{C}+\vec{B} }{k_4+1} 
  \end{align}
  Substituting from 
	  \eqref{eq:centroid-G}
	  in the above, 
  \begin{align}
	  \brak{k_3+1}\brak{\frac{\vec{A}+\vec{B} + \vec{C}}{3}}
 = 
	  {k_3\brak{\frac{k_4\vec{C}+\vec{B} }{k_4+1}} +\vec{A} } 
  \end{align}
  which can be expressed as
\begin{align}
\brak{k_3+1}\brak{k_4+1}\brak{{\vec{A}+\vec{B} + \vec{C}}}
 = 
	  {3} \cbrak{ {k_3\brak{{k_4\vec{C}+\vec{B} }} +\brak{k_4+1}\vec{A} }} 
  \end{align}
  yielding
  \begin{multline}
	  \brak{k_3k_4+k_3-2k_4-2}\vec{A}
	-  \brak{-k_3k_4-k_4+2k_3-1}\vec{B}
	  \\
	  - \brak{-k_3-k_4 - 1 
+2k_3k_4} \vec{C} = \vec{0}
  \end{multline}
  Comparing the above with 
	  \eqref{eq:two-tri-indep},
  \begin{align}
	  p = {-k_3k_4-k_4+2k_3-1},\ q = {-k_3-k_4 - 1 
+2k_3k_4}
  \end{align}
  yielding 
  \begin{align}
	  \label{eq:centroid-G-meet-1}
	   {-k_3k_4-k_4+2k_3-1} = 0
	   \\ {-k_3-k_4 - 1 
+2k_3k_4} = 0
	  \label{eq:centroid-G-meet-2}
  \end{align}
  Subtracting 
	  \eqref{eq:centroid-G-meet-1}
	  from
	  \eqref{eq:centroid-G-meet-2},
  \begin{align}
	  3k_3\brak{k_4-1} = 0
	  \implies k_4=1
  \end{align}
  which upon substituting in 
	  \eqref{eq:centroid-G-meet-1}
	  yields
  \begin{align}
	  k_3 = 2
  \end{align}
\item 
		In \figref{fig:circ_tang},  $OC$ is the radius and $PC$ touches the circle at $\vec{C}$.  Show that	
  \begin{align}
	  OC \perp PC.
		\label{eq:circ_tang-line-orth}	
  \end{align}
	\begin{figure}[!ht]
		\begin{center}
			
			%\includegraphics[width=\columnwidth]{./figs/fig:circ_tang_icept}
			%\vspace*{-10cm}
			\resizebox{0.75\columnwidth}{!}{\input{figs/circle/circ_tang.tex}}
		\end{center}
		\caption{}
		\label{fig:circ_tang}	
	\end{figure}
	\\
		\solution
		The equation of $PC$ can be expressed as
  \begin{align}
		\label{erq:circ_tang-line}	
	  \vec{x} = 
	  \vec{C} + \mu\vec{m}
  \end{align}
  and the equation of the circle is 
  \begin{align}
		\label{eq:circ_tang-eq}	
	  \norm{\vec{x}-\vec{O}} = R
  \end{align}
  Substituting
		\eqref{erq:circ_tang-line}	
		in 
		\eqref{eq:circ_tang-eq},
  \begin{align}
	  \norm{\vec{C} + \mu\vec{m}- \vec{O}}^2 &= R^2
	  \\
	  \implies	  \mu^2 \norm{ \vec{m}}^2 
	  + 2\mu\vec{m}^{\top}\brak{\vec{C} -\vec{O}}
	  + \norm{\vec{C}-\vec{O}}^2 - R^2 &= 0
  \end{align}
  The above equation has only one root.  Hence the discriminant of the above quadratic should be zero. So, 
  \begin{align}
	  \cbrak{\vec{m}^{\top}\brak{\vec{C} -\vec{O}}}^2-\norm{ \vec{m}}^2 
	  \cbrak{ \norm{\vec{C}-\vec{O}}^2 - R^2} &= 0
		\label{erq:circ_tang-line-quad}	
  \end{align}
  Since $\vec{C}$ is a point on the circle, 
  \begin{align}
	   \norm{\vec{C}-\vec{O}}^2 - R^2 = 0
	   \\
	   \implies 
	  \vec{m}^{\top}\brak{\vec{C} -\vec{O}} = 0
		\label{eq:circ_tang-line-orth-dir}	
  \end{align}
upon substituting in \eqref{erq:circ_tang-line-quad}. Using the definition of the direction vector from 	
\eqref{eq:dir-vec}
  \begin{align}
	  \vec{m}=\vec{P}-\vec{C} 
	\\
	  \implies 
	  \brak{\vec{P}-\vec{C}}^{\top}\brak{\vec{C} -\vec{O}} = 0
		\label{eq:circ_tang-line-orth-pc}
  \end{align}
		which is equivalent to 
		\eqref{eq:circ_tang-line-orth}.
\item In $\triangle ABC, \vec{D}, \vec{E}$ and $\vec{F}$ are respectively the mid-points of sides $AB, BC$ and $CA $. Show that $\triangle ABC$ is divided into four congruent triangles by joining $\vec{D}, \vec{E}$ and $\vec{F}$.
\item  The line-segment joining the mid-points of any two sides of a triangle is parallel to the third side and is half of it.
\label{prob:tri_mid_similar}
%\label{prob:quad_similar}
%%
%\\
%\solution If $DE$ is the lie joining he mid points of $\triangle ABC$,  use cosine formula to find the lengths of $DE$ and $BC$. Then use cosine formula to show that all angles of $\triangle ADE$ are equal to the corresponding angles of $\triangle ABC$.
%
\item  A line through the mid-point of a side of a triangle parallel to another side 
bisects the third side.
\item ABC is a triangle right angled at $\vec{C}$. A line through the mid-point $M$ of hypotenuse $AB$ and parallel to $BC$ intersects $AC$ at $\vec{D}$. Show that (i) $\vec{D}$ is the mid-point of $AC$
(ii) $MD \perp AC$ (iii) $CM = MA = \frac{1}{2}AB$
\item $AB$ is a line-segment. $\vec{P}$ and $\vec{Q}$ are points on opposite sides of $AB$ such that each of them is equidistant from the points $\vec{A}$ and $\vec{B}$. Show that the line $PQ $ is the perpendicular bisector of $AB$.
%
\item $AB$ is a line segment and line $l$ is its perpendicular bisector. If a point $\vec{P}$ lies on $l$, show that $\vec{P}$ is equidistant from $\vec{A}$ and $\vec{B}$.
\item $ABCD$ is a trapezium with $AB  \parallel  DC$. $\vec{E}$ and $\vec{F}$ are points on non-parallel sides $AD$ and $BC$ respectively such that $EF$ is parallel to $AB$
. Show that
$\frac{AE}{ED}=\frac{ BF}{  FC}$ .
\item $\vec{O}$ is a point in the interior of $\triangle ABC$. $\vec{D}$ is a point on $OA$.  If $DE  \parallel  OB$ and $DF  \parallel  OC$. Show that $EF  \parallel  BC$.
\item $\vec{O}$ is a point in the interior of $\triangle PQR$.  $\vec{A}, \vec{B}$ and $\vec{C}$ are points on $OP, OQ$ and $OR$ respectively such that $AB  \parallel  PQ$ and $AC  \parallel  PR$. Show that $BC  \parallel  QR$.
\item The diagonals of a quadrilateral $ABCD$ intersect each other at the point $\vec{O}$ such that $\frac{AO}{ BO}=\frac{CO}{  DO}$.   Show that $ABCD$ is a trapezium.
\item In parallelogram $ABCD$, two points $\vec{P}$ and $\vec{Q}$ are taken on diagonal $BD$ such that $DP = BQ$. show that \begin{enumerate}
 \item  $\triangle  APD  \cong   \triangle  CQB$ 
\item $AP = CQ$ \item  $\triangle  AQB  \cong   \triangle  CPD$ 
\item $AQ = CP$ 
\item $APCQ$ is a parallelogram
\end{enumerate}
\item $ABCD$ is a parallelogram and $AP$ and $CQ$ are perpendiculars from vertices $\vec{A}$ and $\vec{C}$ on diagonal $BD$. Show that 
\begin{enumerate} 
\item  $\triangle  APB  \cong   \triangle  CQD $ 
\item $AP = CQ$
\end{enumerate}
\item $ABCD$ is a trapezium in which $AB  \parallel  DC, BD$ is a diagonal and $\vec{E}$ is the mid-point of $AD$. A line is drawn through $\vec{E}$ $\parallel$  $AB$ intersecting $BC$ at $\vec{F}$. Show that $\vec{F}$ is the mid-point of $BC$.
\item In a parallelogram $ABCD$, $\vec{E}$ and $\vec{F}$ are the mid-points of sides $AB$ and $CD$ respectively . Show that the line segments $AF$ and $EC$ trisect the diagonal $BD$.
\item Show that the line segments joining the mid-points of the opposite sides of a quadrilateral bisect each other.
\item $ABCD$ is a parallelogram in which $\vec{P}$ and $\vec{Q}$ are mid-points of opposite sides $AB$ and $CD$. If $AQ$ intersects $DP$ at $\vec{S}$ and $BQ$ intersects $CP$ at $\vec{R}$, show that: 
%
\begin{enumerate}
\item  $APCQ$ is a parallelogram. 
\item $DPBQ$ is a parallelogram. 
\item $PSQR$ is a parallelogram.
\end{enumerate}
%
\item Two circles intersect at two points $\vec{A}$ and $\vec{B}$. $AD$ and $AC$ are diameters to the two circles. Prove that $\vec{B}$ lies on the line segment $DC$.
\item  There is one and only one circle passing through three non-collinear points. 
\item If a line intersects two concentric circles (circles with the same centre) with centre $\vec{O}$ at $\vec{A}, \vec{B}, \vec{C}$ and $\vec{D}$, prove that $AB = CD$.
\item If diagonals of a cyclic quadrilateral are diameters of the circle through the vertices of
the quadrilateral, prove that it is a rectangle.
\item If circles are drawn taking two sides of a triangle as diameters, prove that the point of
intersection of these circles lie on the third side.
\end{enumerate}

