\begin{enumerate}[label=\thesubsection.\arabic*, ref=\thesubsection.\theenumi]
  \item For which values of $a$ and $b$ does the following pair of linear equations have an infinite number of solutions?
	\begin{align}
		2x+3y&=7\\
		(a-b)x+(a-b)y&=3a+b-2
	\end{align}
  \item For which value of $k$ will the following pair of linear equations have no solution?
	\begin{align}
		3x+y&=1\\
		(2k-1)x+(k-1)y&=2k+1
	\end{align}
\item Find the values of $k$ for which the line 
\begin{align}
(k-3)x-(4-k^2)y+k^2-7k+6=0 \label{eq:chapters/11/10/4/1/1}
\end{align}
is
\begin{enumerate}
\item Parallel to the $x$-axis
\item Parallel to the $Y$ axis
\item Passing through the origin
\end{enumerate}
    \solution 
		\begin{enumerate}[label=\thesection.\arabic*.,ref=\thesection.\theenumi]
\numberwithin{equation}{enumi}
\item The equation of a line is given by 
\begin{align}
			\label{eq:app-line-school}
	y &= mx + c
	\\
	\implies \myvec{x \\ y} &= \myvec{x \\ 
	 mx + c} =\myvec{0 \\ c} + x\myvec{1 \\ m}
\end{align}
			yielding \eqref{eq:geo-param}.
\item 			\eqref{eq:app-line-school} can also be expressed as
\begin{align}
	y - mx &= c 
	\\
	\implies \myvec{-m & 1}\myvec{x \\ y} &= c
\end{align}
			yielding \eqref{eq:geo-normal}.
		\item The direction vector is 
\begin{align}
			\label{eq:app-line-school-dir}
\vec{m} = \myvec{1 \\ m}
\end{align}
and the normal vector is
\begin{align}
\vec{n}=\myvec{-m \\ 1}
			\label{eq:app-line-school-normal}
\end{align}
  \item From \eqref{eq:geo-param}, 
	  if $\vec{A},\vec{D}$ and $\vec{C}$ are on the same line,
		\label{prop:app-lin-dep}
\begin{align}
			\vec{D}=\vec{A}+q\vec{m} 
			\\ 
			\vec{C}=\vec{D}+p\vec{m} \\
			\label{eq:app-collinear} 
			\implies 	p\brak{\vec{D}-\vec{A}} 
			+ q\brak{\vec{D}-\vec{C}} = 0, \quad p, q \ne 0 \\ 
			\implies \vec{D} = \frac{p\vec{A}+q\vec{C}}{p+q} 
			\end{align} 
			yielding \eqref{eq:section_formula} upon substituting \begin{align} k = \frac{p}{q}. \end{align} 
			$\brak{\vec{D}-\vec{A}}, \brak{\vec{D}-\vec{C}}$ 
		are then said to be {\em linearly dependent}.
	\item If $\vec{A}, \vec{B}, \vec{C}$ are collinear,  from \eqref{eq:geo-normal}, \begin{align}
	 \vec{n}^{\top}\vec{A} &=  c 
	 \\
	 \vec{n}^{\top}\vec{B} &=  c 
	 \\
	 \vec{n}^{\top}\vec{C} &=  c 
\end{align}
which can be expressed as 
\begin{align}
		\label{prop:app-lin-eq}
	\myvec{ \vec{A} & \vec{B} & \vec{C}}^{\top}\vec{n} = c\myvec{1 \\ 1 \\ 1}
	\\
	\equiv \myvec{ \vec{A} & \vec{B} & \vec{C}}^{\top}\vec{n} = \myvec{1 \\ 1 \\ 1}
		\label{prop:app-lin-eq-unit-mat},
	\\
	\implies 
	\myvec{ \vec{A} & \vec{B} & \vec{C}\\ 1 & 1 &1 }^{\top}\myvec{\vec{n} \\ -1} &= \vec{0}
		\label{prop:app-lin-dep-rank}
\end{align}
yielding
		\begin{align}
			\label{eq:app-line-rank-2}
			\rank{\myvec{1 & 1 & 1 \\ \vec{A}& \vec{B}&\vec{C}}} = 2
		\end{align}
			  Rank is defined to be the number of linearly indpendent rows or columns of a matrix.
		\item
The equation of a line can also be expressed as
\begin{align}
	 \vec{n}^{\top}\vec{x} &=   1
		\label{prop:app-lin-eq-unit}
\end{align}
	  \end{enumerate}

	\item Find the  equations of the lines,  which cutoff intercepts on the axes  whose sum and product are 1 and -6 respectively.
\\
\solution
		\begin{enumerate}[label=\thesection.\arabic*.,ref=\thesection.\theenumi]
\numberwithin{equation}{enumi}
\item The equation of a line is given by 
\begin{align}
			\label{eq:app-line-school}
	y &= mx + c
	\\
	\implies \myvec{x \\ y} &= \myvec{x \\ 
	 mx + c} =\myvec{0 \\ c} + x\myvec{1 \\ m}
\end{align}
			yielding \eqref{eq:geo-param}.
\item 			\eqref{eq:app-line-school} can also be expressed as
\begin{align}
	y - mx &= c 
	\\
	\implies \myvec{-m & 1}\myvec{x \\ y} &= c
\end{align}
			yielding \eqref{eq:geo-normal}.
		\item The direction vector is 
\begin{align}
			\label{eq:app-line-school-dir}
\vec{m} = \myvec{1 \\ m}
\end{align}
and the normal vector is
\begin{align}
\vec{n}=\myvec{-m \\ 1}
			\label{eq:app-line-school-normal}
\end{align}
  \item From \eqref{eq:geo-param}, 
	  if $\vec{A},\vec{D}$ and $\vec{C}$ are on the same line,
		\label{prop:app-lin-dep}
\begin{align}
			\vec{D}=\vec{A}+q\vec{m} 
			\\ 
			\vec{C}=\vec{D}+p\vec{m} \\
			\label{eq:app-collinear} 
			\implies 	p\brak{\vec{D}-\vec{A}} 
			+ q\brak{\vec{D}-\vec{C}} = 0, \quad p, q \ne 0 \\ 
			\implies \vec{D} = \frac{p\vec{A}+q\vec{C}}{p+q} 
			\end{align} 
			yielding \eqref{eq:section_formula} upon substituting \begin{align} k = \frac{p}{q}. \end{align} 
			$\brak{\vec{D}-\vec{A}}, \brak{\vec{D}-\vec{C}}$ 
		are then said to be {\em linearly dependent}.
	\item If $\vec{A}, \vec{B}, \vec{C}$ are collinear,  from \eqref{eq:geo-normal}, \begin{align}
	 \vec{n}^{\top}\vec{A} &=  c 
	 \\
	 \vec{n}^{\top}\vec{B} &=  c 
	 \\
	 \vec{n}^{\top}\vec{C} &=  c 
\end{align}
which can be expressed as 
\begin{align}
		\label{prop:app-lin-eq}
	\myvec{ \vec{A} & \vec{B} & \vec{C}}^{\top}\vec{n} = c\myvec{1 \\ 1 \\ 1}
	\\
	\equiv \myvec{ \vec{A} & \vec{B} & \vec{C}}^{\top}\vec{n} = \myvec{1 \\ 1 \\ 1}
		\label{prop:app-lin-eq-unit-mat},
	\\
	\implies 
	\myvec{ \vec{A} & \vec{B} & \vec{C}\\ 1 & 1 &1 }^{\top}\myvec{\vec{n} \\ -1} &= \vec{0}
		\label{prop:app-lin-dep-rank}
\end{align}
yielding
		\begin{align}
			\label{eq:app-line-rank-2}
			\rank{\myvec{1 & 1 & 1 \\ \vec{A}& \vec{B}&\vec{C}}} = 2
		\end{align}
			  Rank is defined to be the number of linearly indpendent rows or columns of a matrix.
		\item
The equation of a line can also be expressed as
\begin{align}
	 \vec{n}^{\top}\vec{x} &=   1
		\label{prop:app-lin-eq-unit}
\end{align}
	  \end{enumerate}

\item A ray of light passing through the point $\vec{P} = \brak{1,  2}$ reflects on the x-axis at point $\vec{A}$ and the reflected ray passes through the point $\vec{Q} =\brak{5,  3}$. Find the coordinates of $\vec{A}$.
\\
    \solution 
		\begin{enumerate}[label=\thesection.\arabic*.,ref=\thesection.\theenumi]
\numberwithin{equation}{enumi}
\item The equation of a line is given by 
\begin{align}
			\label{eq:app-line-school}
	y &= mx + c
	\\
	\implies \myvec{x \\ y} &= \myvec{x \\ 
	 mx + c} =\myvec{0 \\ c} + x\myvec{1 \\ m}
\end{align}
			yielding \eqref{eq:geo-param}.
\item 			\eqref{eq:app-line-school} can also be expressed as
\begin{align}
	y - mx &= c 
	\\
	\implies \myvec{-m & 1}\myvec{x \\ y} &= c
\end{align}
			yielding \eqref{eq:geo-normal}.
		\item The direction vector is 
\begin{align}
			\label{eq:app-line-school-dir}
\vec{m} = \myvec{1 \\ m}
\end{align}
and the normal vector is
\begin{align}
\vec{n}=\myvec{-m \\ 1}
			\label{eq:app-line-school-normal}
\end{align}
  \item From \eqref{eq:geo-param}, 
	  if $\vec{A},\vec{D}$ and $\vec{C}$ are on the same line,
		\label{prop:app-lin-dep}
\begin{align}
			\vec{D}=\vec{A}+q\vec{m} 
			\\ 
			\vec{C}=\vec{D}+p\vec{m} \\
			\label{eq:app-collinear} 
			\implies 	p\brak{\vec{D}-\vec{A}} 
			+ q\brak{\vec{D}-\vec{C}} = 0, \quad p, q \ne 0 \\ 
			\implies \vec{D} = \frac{p\vec{A}+q\vec{C}}{p+q} 
			\end{align} 
			yielding \eqref{eq:section_formula} upon substituting \begin{align} k = \frac{p}{q}. \end{align} 
			$\brak{\vec{D}-\vec{A}}, \brak{\vec{D}-\vec{C}}$ 
		are then said to be {\em linearly dependent}.
	\item If $\vec{A}, \vec{B}, \vec{C}$ are collinear,  from \eqref{eq:geo-normal}, \begin{align}
	 \vec{n}^{\top}\vec{A} &=  c 
	 \\
	 \vec{n}^{\top}\vec{B} &=  c 
	 \\
	 \vec{n}^{\top}\vec{C} &=  c 
\end{align}
which can be expressed as 
\begin{align}
		\label{prop:app-lin-eq}
	\myvec{ \vec{A} & \vec{B} & \vec{C}}^{\top}\vec{n} = c\myvec{1 \\ 1 \\ 1}
	\\
	\equiv \myvec{ \vec{A} & \vec{B} & \vec{C}}^{\top}\vec{n} = \myvec{1 \\ 1 \\ 1}
		\label{prop:app-lin-eq-unit-mat},
	\\
	\implies 
	\myvec{ \vec{A} & \vec{B} & \vec{C}\\ 1 & 1 &1 }^{\top}\myvec{\vec{n} \\ -1} &= \vec{0}
		\label{prop:app-lin-dep-rank}
\end{align}
yielding
		\begin{align}
			\label{eq:app-line-rank-2}
			\rank{\myvec{1 & 1 & 1 \\ \vec{A}& \vec{B}&\vec{C}}} = 2
		\end{align}
			  Rank is defined to be the number of linearly indpendent rows or columns of a matrix.
		\item
The equation of a line can also be expressed as
\begin{align}
	 \vec{n}^{\top}\vec{x} &=   1
		\label{prop:app-lin-eq-unit}
\end{align}
	  \end{enumerate}

\item The owner of a milk store finds that he can sell $980$ litres of milk each week at \rupee~14/litre and $1220$ litres of milk each week at \rupee~16/litre. Assuming a linear relationship between selling price and demand,  how many litres could he sell weekly at \rupee~17/ litre?
\item Prove that in any $\triangle{ABC}$,  cos A=$\frac{b^2+c^2-a^2}{2bc}$,  where a, b, c are the magnitudes of the sides opposite to the vertices $\vec{A}, \vec{B}, \vec{C}$ respectively.
\item Distance of the point $(\alpha,  \beta,  \gamma)$ from y-axis is
\begin{enumerate}
	\item $\beta$ 
	\item $\abs{\beta}$
	\item $\abs{\beta+\gamma}$
	\item $\sqrt{\alpha^2+\gamma^2}$
\end{enumerate}
\item The reflection of the point $(\alpha,  \beta,  \gamma )$ in the $XY$ plane is 
\begin{enumerate}
	\item $\alpha, \beta, 0)$
	\item $(0, 0, \gamma)$
	\item $(-\alpha, -\beta, \gamma)$
	\item $(\alpha, \beta, -\gamma)$
\end{enumerate}
\item The plane $ax+by=0$ is rotated about its line of intersection with the plane $z=0$ through an angle $\alpha.$ Prove that the equation of the plane in its new position is 
\begin{align*}
	ax+by \pm (\sqrt{a^2+b^2} \tan\alpha)z=0.
\end{align*}
\item The locus represented by $xy+yz=0$ is 
\begin{enumerate}
	\item A pair of perpendicular lines
	\item A pair of parallel lines
	\item A pair of parallel planes 
	\item A pair of perpendicular planes
\end{enumerate}
\item For what values of $a$ and $b$ the intercepts cut off on the coordinate axes by the line $ax+by+8=0$ are equal in length but opposite in signs to those cut off by the line $2x-3y=0$ on the axes.
\item If the equation of the base of an equilateral triangle is $x+y=2$ and the vertex is (2, -1),  then find the length of the side of the triangle. 
\item A variable line passes through a fixed point $\vec{P}$. The algebraic sum of the perpendiculars drawn from the points (2, 0),  (0, 2) and (1, 1) on the line is zero. Find the coordinates of the point $\vec{P}$.  
\item A straight line moves so that the sum of the reciprocals of its intercepts made on the axes is constant. Show that the line passes through a fixed point. 
\item If the sum of the distances of a moving point in a plane from the axes is $l$,  then finds the locus of the point.  
\item $\vec{P}_1, \vec{P}_2$ are points on either of the two lines $y-\sqrt{3}\abs{x}=2$ at a distance of 5 units from their point of intersection. Find the coordinates of the foot of the perpendiculars drawn from $\vec{P}_1,  \vec{P}_2$ on the bisector of the angle between the given lines.
\item If $p$ is the length of perpendicular from the origin on the line $\frac{x}{a}+\frac{y}{b}=1$ and $a^2, p^2, b^2$ are in A.P,  then show that $a^4+b^4=0$.
\item The point (4, 1) undergoes the following two successive transformations :
\begin{enumerate}
\item Reflection about the line $y=x$
\item Translation through a distance 2 units along the positive $x$-axis 
\end{enumerate}
Then the final coordinates of the point are
\begin{enumerate}
\item (4, 3)
\item (3, 4)
\item (1, 4)
\item $\frac{7}{2}$, $\frac{7}{2}$
\end{enumerate}
\item One vertex of the equilateral with centroid at the origin and one side as $x+y-2=0$ is
\begin{enumerate}
\item (-1, -1)
\item (2, 2)
\item (-2, -2)
\item (2, -2)
\end{enumerate}
\item If $a, b, c$ are is A.P.,  then the straight lines $ax+by+c=0$ will always pass through \rule{1cm}{0.15mm}.
\item The points (3, 4) and (2, -6) are situated on the \rule{1cm}{0.15mm} of the line $3x-4y-8=0$.
\item A point moves so that square of its distance from the point (3, -2) is numerically equal to its distance from the line $5x-12y=3$. The equation of its locus is %\rule{1cm}{0.15mm}.
\item Locus of the mid-points of the portion of the line $x\sin\theta+y\cos\theta=p$ intercepted between the axes is \rule{1cm}{0.15mm}.

State whether the following statements are true or false. Justify.
\item If the vertices of a triangle have integral coordinates,  then the triangle can not be equilateral.
\item The line $\frac{x}{a}+\frac{y}{b}=1$ moves in such a way that $\frac{1}{a^2}+\frac{1}{b^2}=\frac{1}{c^2}$,  where $c$ is a constant. The locus of the foot of the perpendicular from the origin on the given line is $x^2+y^2=c^2$.
\item 
Match the following
	\begin{table}[H]
\centering
	\resizebox{\columnwidth}{!}{
\begin{matchtabular}
  The coordinates of the points $\vec{P}$ and $\vec{Q}$ on the line $x + 5y = 13$ which are at a distance of 2 units from the line $12x – 5y + 26 = 0$ are & (3, 1), (-7, 11)\\
  The coordinates of the point on the line $x + y = 4$,  which are at a unit distance from the line $4x + 3y – 10 = 0$ are & $-\frac{1}{11}, \frac{11}{3}$ ,  $\frac{4}{3}, \frac{7}{3}$\\
  The coordinates of the point on the line joining $\vec{A} (–2,  5)$ and $\vec{B} (3,  1)$ such that $AP = PQ = QB$ are & 1, $\frac{12}{5}$ ,  $-3, \frac{16}{5}$\\
\end{matchtabular}
		}
		\caption{}
		\label{tab:lin-misc-1}
	\end{table}
\item The value of the $\lambda$,  if the lines\\$(2x+3y+4)+\lambda(6x-y+12)=0$ are
	\begin{table}[H]
\centering
	\resizebox{\columnwidth}{!}{
\begin{matchtabular}
parallel to $Y$ axis is & $\lambda =-\frac{3}{4}$\\
perpendicular to $7x+y-4=0$ is & $\lambda=-\frac{1}{3}$\\
passes through (1, 2) is & $\lambda=-\frac{17}{41}$\\
parallel to $X$ axis is & $\lambda=3$\\
\end{matchtabular}
		}
		\caption{}
		\label{tab:lin-misc-2}
	\end{table}
\item The equation of the line through the intersection of the lines $2x-3y=0$ and $4x-5y=2$ and
	\begin{table}[H]
\centering
	\resizebox{\columnwidth}{!}{
\begin{matchtabular}
through the point (2, 1) is & $2x-y=4$\\
perpendicular to the line & $x+y-5=0$\\
parallel to the line $3x-4y+5=0$ is & $x-y-1=0$\\
equally inclined to the axes is & $3x-4y-1=0$\\
\end{matchtabular}
		}
		\caption{}
		\label{tab:lin-misc-3}
	\end{table}
\item Point $\vec{R}\brak{h,  k}$ divides a line segment between the axes in the ratio 1: 2. Find the equation of the line.
\label{chapters/11/10/2/19}
	\\
	\solution 
\begin{enumerate}[label=\thesection.\arabic*.,ref=\thesection.\theenumi]
\numberwithin{equation}{enumi}
\item The equation of a line is given by 
\begin{align}
			\label{eq:app-line-school}
	y &= mx + c
	\\
	\implies \myvec{x \\ y} &= \myvec{x \\ 
	 mx + c} =\myvec{0 \\ c} + x\myvec{1 \\ m}
\end{align}
			yielding \eqref{eq:geo-param}.
\item 			\eqref{eq:app-line-school} can also be expressed as
\begin{align}
	y - mx &= c 
	\\
	\implies \myvec{-m & 1}\myvec{x \\ y} &= c
\end{align}
			yielding \eqref{eq:geo-normal}.
		\item The direction vector is 
\begin{align}
			\label{eq:app-line-school-dir}
\vec{m} = \myvec{1 \\ m}
\end{align}
and the normal vector is
\begin{align}
\vec{n}=\myvec{-m \\ 1}
			\label{eq:app-line-school-normal}
\end{align}
  \item From \eqref{eq:geo-param}, 
	  if $\vec{A},\vec{D}$ and $\vec{C}$ are on the same line,
		\label{prop:app-lin-dep}
\begin{align}
			\vec{D}=\vec{A}+q\vec{m} 
			\\ 
			\vec{C}=\vec{D}+p\vec{m} \\
			\label{eq:app-collinear} 
			\implies 	p\brak{\vec{D}-\vec{A}} 
			+ q\brak{\vec{D}-\vec{C}} = 0, \quad p, q \ne 0 \\ 
			\implies \vec{D} = \frac{p\vec{A}+q\vec{C}}{p+q} 
			\end{align} 
			yielding \eqref{eq:section_formula} upon substituting \begin{align} k = \frac{p}{q}. \end{align} 
			$\brak{\vec{D}-\vec{A}}, \brak{\vec{D}-\vec{C}}$ 
		are then said to be {\em linearly dependent}.
	\item If $\vec{A}, \vec{B}, \vec{C}$ are collinear,  from \eqref{eq:geo-normal}, \begin{align}
	 \vec{n}^{\top}\vec{A} &=  c 
	 \\
	 \vec{n}^{\top}\vec{B} &=  c 
	 \\
	 \vec{n}^{\top}\vec{C} &=  c 
\end{align}
which can be expressed as 
\begin{align}
		\label{prop:app-lin-eq}
	\myvec{ \vec{A} & \vec{B} & \vec{C}}^{\top}\vec{n} = c\myvec{1 \\ 1 \\ 1}
	\\
	\equiv \myvec{ \vec{A} & \vec{B} & \vec{C}}^{\top}\vec{n} = \myvec{1 \\ 1 \\ 1}
		\label{prop:app-lin-eq-unit-mat},
	\\
	\implies 
	\myvec{ \vec{A} & \vec{B} & \vec{C}\\ 1 & 1 &1 }^{\top}\myvec{\vec{n} \\ -1} &= \vec{0}
		\label{prop:app-lin-dep-rank}
\end{align}
yielding
		\begin{align}
			\label{eq:app-line-rank-2}
			\rank{\myvec{1 & 1 & 1 \\ \vec{A}& \vec{B}&\vec{C}}} = 2
		\end{align}
			  Rank is defined to be the number of linearly indpendent rows or columns of a matrix.
		\item
The equation of a line can also be expressed as
\begin{align}
	 \vec{n}^{\top}\vec{x} &=   1
		\label{prop:app-lin-eq-unit}
\end{align}
	  \end{enumerate}

\item The tangent of the angle between the lines whose intercepts on the axes are $a, -b$ and $b, -a$,  respectively,  is
	\begin{enumerate}[itemsep=1ex]
\item $\frac{a^2-b^2}{ab}$
\item $\frac{b^2-a^2}{2}$
\item $\frac{b^2-a^2}{2ab}$
\item none of these 
\end{enumerate}
\item Prove that the line through the point $(x_1, y_1)$ and parallel to the line $Ax+By+C=0$ is $A(x-x_1)+B(y-y_1)=0$.
\label{chapters/11/10/3/11}
\\
\solution
\begin{enumerate}[label=\thesection.\arabic*.,ref=\thesection.\theenumi]
\numberwithin{equation}{enumi}
\item The equation of a line is given by 
\begin{align}
			\label{eq:app-line-school}
	y &= mx + c
	\\
	\implies \myvec{x \\ y} &= \myvec{x \\ 
	 mx + c} =\myvec{0 \\ c} + x\myvec{1 \\ m}
\end{align}
			yielding \eqref{eq:geo-param}.
\item 			\eqref{eq:app-line-school} can also be expressed as
\begin{align}
	y - mx &= c 
	\\
	\implies \myvec{-m & 1}\myvec{x \\ y} &= c
\end{align}
			yielding \eqref{eq:geo-normal}.
		\item The direction vector is 
\begin{align}
			\label{eq:app-line-school-dir}
\vec{m} = \myvec{1 \\ m}
\end{align}
and the normal vector is
\begin{align}
\vec{n}=\myvec{-m \\ 1}
			\label{eq:app-line-school-normal}
\end{align}
  \item From \eqref{eq:geo-param}, 
	  if $\vec{A},\vec{D}$ and $\vec{C}$ are on the same line,
		\label{prop:app-lin-dep}
\begin{align}
			\vec{D}=\vec{A}+q\vec{m} 
			\\ 
			\vec{C}=\vec{D}+p\vec{m} \\
			\label{eq:app-collinear} 
			\implies 	p\brak{\vec{D}-\vec{A}} 
			+ q\brak{\vec{D}-\vec{C}} = 0, \quad p, q \ne 0 \\ 
			\implies \vec{D} = \frac{p\vec{A}+q\vec{C}}{p+q} 
			\end{align} 
			yielding \eqref{eq:section_formula} upon substituting \begin{align} k = \frac{p}{q}. \end{align} 
			$\brak{\vec{D}-\vec{A}}, \brak{\vec{D}-\vec{C}}$ 
		are then said to be {\em linearly dependent}.
	\item If $\vec{A}, \vec{B}, \vec{C}$ are collinear,  from \eqref{eq:geo-normal}, \begin{align}
	 \vec{n}^{\top}\vec{A} &=  c 
	 \\
	 \vec{n}^{\top}\vec{B} &=  c 
	 \\
	 \vec{n}^{\top}\vec{C} &=  c 
\end{align}
which can be expressed as 
\begin{align}
		\label{prop:app-lin-eq}
	\myvec{ \vec{A} & \vec{B} & \vec{C}}^{\top}\vec{n} = c\myvec{1 \\ 1 \\ 1}
	\\
	\equiv \myvec{ \vec{A} & \vec{B} & \vec{C}}^{\top}\vec{n} = \myvec{1 \\ 1 \\ 1}
		\label{prop:app-lin-eq-unit-mat},
	\\
	\implies 
	\myvec{ \vec{A} & \vec{B} & \vec{C}\\ 1 & 1 &1 }^{\top}\myvec{\vec{n} \\ -1} &= \vec{0}
		\label{prop:app-lin-dep-rank}
\end{align}
yielding
		\begin{align}
			\label{eq:app-line-rank-2}
			\rank{\myvec{1 & 1 & 1 \\ \vec{A}& \vec{B}&\vec{C}}} = 2
		\end{align}
			  Rank is defined to be the number of linearly indpendent rows or columns of a matrix.
		\item
The equation of a line can also be expressed as
\begin{align}
	 \vec{n}^{\top}\vec{x} &=   1
		\label{prop:app-lin-eq-unit}
\end{align}
	  \end{enumerate}

\item  If ${p}$ and ${q}$ are the lengths of perpendiculars from the origin to the lines ${x}\cos\theta - {y}\sin\theta =  {k}\cos2\theta$ and ${x}\sec\theta + {y}\cosec\theta = {k}$,  respectively,  prove that ${p}^2 + 4{q}^2 = {k}^2$
\label{chapters/11/10/3/16}
\\
\solution
\begin{enumerate}[label=\thesection.\arabic*.,ref=\thesection.\theenumi]
\numberwithin{equation}{enumi}
\item The equation of a line is given by 
\begin{align}
			\label{eq:app-line-school}
	y &= mx + c
	\\
	\implies \myvec{x \\ y} &= \myvec{x \\ 
	 mx + c} =\myvec{0 \\ c} + x\myvec{1 \\ m}
\end{align}
			yielding \eqref{eq:geo-param}.
\item 			\eqref{eq:app-line-school} can also be expressed as
\begin{align}
	y - mx &= c 
	\\
	\implies \myvec{-m & 1}\myvec{x \\ y} &= c
\end{align}
			yielding \eqref{eq:geo-normal}.
		\item The direction vector is 
\begin{align}
			\label{eq:app-line-school-dir}
\vec{m} = \myvec{1 \\ m}
\end{align}
and the normal vector is
\begin{align}
\vec{n}=\myvec{-m \\ 1}
			\label{eq:app-line-school-normal}
\end{align}
  \item From \eqref{eq:geo-param}, 
	  if $\vec{A},\vec{D}$ and $\vec{C}$ are on the same line,
		\label{prop:app-lin-dep}
\begin{align}
			\vec{D}=\vec{A}+q\vec{m} 
			\\ 
			\vec{C}=\vec{D}+p\vec{m} \\
			\label{eq:app-collinear} 
			\implies 	p\brak{\vec{D}-\vec{A}} 
			+ q\brak{\vec{D}-\vec{C}} = 0, \quad p, q \ne 0 \\ 
			\implies \vec{D} = \frac{p\vec{A}+q\vec{C}}{p+q} 
			\end{align} 
			yielding \eqref{eq:section_formula} upon substituting \begin{align} k = \frac{p}{q}. \end{align} 
			$\brak{\vec{D}-\vec{A}}, \brak{\vec{D}-\vec{C}}$ 
		are then said to be {\em linearly dependent}.
	\item If $\vec{A}, \vec{B}, \vec{C}$ are collinear,  from \eqref{eq:geo-normal}, \begin{align}
	 \vec{n}^{\top}\vec{A} &=  c 
	 \\
	 \vec{n}^{\top}\vec{B} &=  c 
	 \\
	 \vec{n}^{\top}\vec{C} &=  c 
\end{align}
which can be expressed as 
\begin{align}
		\label{prop:app-lin-eq}
	\myvec{ \vec{A} & \vec{B} & \vec{C}}^{\top}\vec{n} = c\myvec{1 \\ 1 \\ 1}
	\\
	\equiv \myvec{ \vec{A} & \vec{B} & \vec{C}}^{\top}\vec{n} = \myvec{1 \\ 1 \\ 1}
		\label{prop:app-lin-eq-unit-mat},
	\\
	\implies 
	\myvec{ \vec{A} & \vec{B} & \vec{C}\\ 1 & 1 &1 }^{\top}\myvec{\vec{n} \\ -1} &= \vec{0}
		\label{prop:app-lin-dep-rank}
\end{align}
yielding
		\begin{align}
			\label{eq:app-line-rank-2}
			\rank{\myvec{1 & 1 & 1 \\ \vec{A}& \vec{B}&\vec{C}}} = 2
		\end{align}
			  Rank is defined to be the number of linearly indpendent rows or columns of a matrix.
		\item
The equation of a line can also be expressed as
\begin{align}
	 \vec{n}^{\top}\vec{x} &=   1
		\label{prop:app-lin-eq-unit}
\end{align}
	  \end{enumerate}

\item If $p$ is the length of perpendicular from origin to the line whose intercepts on the axes are $a$ and $b$,  then show that 
\begin{align}
	\frac{1}{p^2} = \frac{1}{a^2}+ \frac{1}{b^2}
\label{eq:11/10/3/18}
\end{align}
\label{chapters/11/10/3/18}
\\
\solution
\input{chapters/11/10/3/18/dist.tex}
\item Find perpendicular distance from the origin to the line joining the points $(\cos\theta, \sin\theta)$ and $(\cos\phi, \sin\phi)$.
\\
\solution
		\input{chapters/11/10/4/5/dist.tex}
	\item Prove that the products of the lengths of the perpendiculars drawn from the points $\myvec{\sqrt{a^2-b^2}& 0}^{\top}$ and $\myvec{-\sqrt{a^2-b^2} &0}^{\top}$ to the line $\frac{x}{a} \cos{\theta} + \frac{y}{b}\sin{\theta} =1 $ is $ b^2 $.
\\
    \solution 
		\input{chapters/11/10/4/23/dist.tex}
\item $\vec{O}$ is the origin and $\vec{A}$ is $(a, b, c)$. Find the direction cosines of the line OA and the equation of the plane through $\vec{A}$ at right angle at OA.
\item Two systems of rectangular axis have the same origin. If a plane cuts them at distances $a, b, c$ and $a^{\prime}, b^{\prime}, c^{\prime}$,  respectively,  from the origin,  prove that $$\frac{1}{a^2}+\frac{1}{b^2}+\frac{1}{c^2}=\frac{1}{{a^{\prime}}^2}+\frac{1}{{b^{\prime}}^2}+\frac{1}{{c^{\prime}}^2}$$.
\item Equation of the line passing through the point $(a\cos^3\theta,  a\sin^3\theta)$ and perpendicular to the line $x\sec\theta+y\csc\theta=a$ is $x\cos\theta-y\sin\theta=\alpha\sin2\theta$.
\item The distance between the lines $y=mx+c$, $ and $$y=mx+c^2$ is
	\item Find the area of the triangle formed by the lines $y-x=0,  x+y=0,  $ and $ x-k=0$.
		\\
\solution
		\begin{enumerate}[label=\thesection.\arabic*.,ref=\thesection.\theenumi]
\numberwithin{equation}{enumi}
\item The equation of a line is given by 
\begin{align}
			\label{eq:app-line-school}
	y &= mx + c
	\\
	\implies \myvec{x \\ y} &= \myvec{x \\ 
	 mx + c} =\myvec{0 \\ c} + x\myvec{1 \\ m}
\end{align}
			yielding \eqref{eq:geo-param}.
\item 			\eqref{eq:app-line-school} can also be expressed as
\begin{align}
	y - mx &= c 
	\\
	\implies \myvec{-m & 1}\myvec{x \\ y} &= c
\end{align}
			yielding \eqref{eq:geo-normal}.
		\item The direction vector is 
\begin{align}
			\label{eq:app-line-school-dir}
\vec{m} = \myvec{1 \\ m}
\end{align}
and the normal vector is
\begin{align}
\vec{n}=\myvec{-m \\ 1}
			\label{eq:app-line-school-normal}
\end{align}
  \item From \eqref{eq:geo-param}, 
	  if $\vec{A},\vec{D}$ and $\vec{C}$ are on the same line,
		\label{prop:app-lin-dep}
\begin{align}
			\vec{D}=\vec{A}+q\vec{m} 
			\\ 
			\vec{C}=\vec{D}+p\vec{m} \\
			\label{eq:app-collinear} 
			\implies 	p\brak{\vec{D}-\vec{A}} 
			+ q\brak{\vec{D}-\vec{C}} = 0, \quad p, q \ne 0 \\ 
			\implies \vec{D} = \frac{p\vec{A}+q\vec{C}}{p+q} 
			\end{align} 
			yielding \eqref{eq:section_formula} upon substituting \begin{align} k = \frac{p}{q}. \end{align} 
			$\brak{\vec{D}-\vec{A}}, \brak{\vec{D}-\vec{C}}$ 
		are then said to be {\em linearly dependent}.
	\item If $\vec{A}, \vec{B}, \vec{C}$ are collinear,  from \eqref{eq:geo-normal}, \begin{align}
	 \vec{n}^{\top}\vec{A} &=  c 
	 \\
	 \vec{n}^{\top}\vec{B} &=  c 
	 \\
	 \vec{n}^{\top}\vec{C} &=  c 
\end{align}
which can be expressed as 
\begin{align}
		\label{prop:app-lin-eq}
	\myvec{ \vec{A} & \vec{B} & \vec{C}}^{\top}\vec{n} = c\myvec{1 \\ 1 \\ 1}
	\\
	\equiv \myvec{ \vec{A} & \vec{B} & \vec{C}}^{\top}\vec{n} = \myvec{1 \\ 1 \\ 1}
		\label{prop:app-lin-eq-unit-mat},
	\\
	\implies 
	\myvec{ \vec{A} & \vec{B} & \vec{C}\\ 1 & 1 &1 }^{\top}\myvec{\vec{n} \\ -1} &= \vec{0}
		\label{prop:app-lin-dep-rank}
\end{align}
yielding
		\begin{align}
			\label{eq:app-line-rank-2}
			\rank{\myvec{1 & 1 & 1 \\ \vec{A}& \vec{B}&\vec{C}}} = 2
		\end{align}
			  Rank is defined to be the number of linearly indpendent rows or columns of a matrix.
		\item
The equation of a line can also be expressed as
\begin{align}
	 \vec{n}^{\top}\vec{x} &=   1
		\label{prop:app-lin-eq-unit}
\end{align}
	  \end{enumerate}

\item The lines $ax+2y+1=0$,  $bx=3y+1=0$ and $cx+4y+1=0$ are concurrent if $a$,  $b$,  $c$ are in G.P.
\item 
$\vec{P}(a, b)$ is the mid-point of the line segment between axes. Show that the equation of the line is $\frac{x}{a}+\frac{y}{b}=2$
\label{chapters/11/10/2/18}
\\
\solution
\begin{enumerate}[label=\thesection.\arabic*.,ref=\thesection.\theenumi]
\numberwithin{equation}{enumi}
\item The equation of a line is given by 
\begin{align}
			\label{eq:app-line-school}
	y &= mx + c
	\\
	\implies \myvec{x \\ y} &= \myvec{x \\ 
	 mx + c} =\myvec{0 \\ c} + x\myvec{1 \\ m}
\end{align}
			yielding \eqref{eq:geo-param}.
\item 			\eqref{eq:app-line-school} can also be expressed as
\begin{align}
	y - mx &= c 
	\\
	\implies \myvec{-m & 1}\myvec{x \\ y} &= c
\end{align}
			yielding \eqref{eq:geo-normal}.
		\item The direction vector is 
\begin{align}
			\label{eq:app-line-school-dir}
\vec{m} = \myvec{1 \\ m}
\end{align}
and the normal vector is
\begin{align}
\vec{n}=\myvec{-m \\ 1}
			\label{eq:app-line-school-normal}
\end{align}
  \item From \eqref{eq:geo-param}, 
	  if $\vec{A},\vec{D}$ and $\vec{C}$ are on the same line,
		\label{prop:app-lin-dep}
\begin{align}
			\vec{D}=\vec{A}+q\vec{m} 
			\\ 
			\vec{C}=\vec{D}+p\vec{m} \\
			\label{eq:app-collinear} 
			\implies 	p\brak{\vec{D}-\vec{A}} 
			+ q\brak{\vec{D}-\vec{C}} = 0, \quad p, q \ne 0 \\ 
			\implies \vec{D} = \frac{p\vec{A}+q\vec{C}}{p+q} 
			\end{align} 
			yielding \eqref{eq:section_formula} upon substituting \begin{align} k = \frac{p}{q}. \end{align} 
			$\brak{\vec{D}-\vec{A}}, \brak{\vec{D}-\vec{C}}$ 
		are then said to be {\em linearly dependent}.
	\item If $\vec{A}, \vec{B}, \vec{C}$ are collinear,  from \eqref{eq:geo-normal}, \begin{align}
	 \vec{n}^{\top}\vec{A} &=  c 
	 \\
	 \vec{n}^{\top}\vec{B} &=  c 
	 \\
	 \vec{n}^{\top}\vec{C} &=  c 
\end{align}
which can be expressed as 
\begin{align}
		\label{prop:app-lin-eq}
	\myvec{ \vec{A} & \vec{B} & \vec{C}}^{\top}\vec{n} = c\myvec{1 \\ 1 \\ 1}
	\\
	\equiv \myvec{ \vec{A} & \vec{B} & \vec{C}}^{\top}\vec{n} = \myvec{1 \\ 1 \\ 1}
		\label{prop:app-lin-eq-unit-mat},
	\\
	\implies 
	\myvec{ \vec{A} & \vec{B} & \vec{C}\\ 1 & 1 &1 }^{\top}\myvec{\vec{n} \\ -1} &= \vec{0}
		\label{prop:app-lin-dep-rank}
\end{align}
yielding
		\begin{align}
			\label{eq:app-line-rank-2}
			\rank{\myvec{1 & 1 & 1 \\ \vec{A}& \vec{B}&\vec{C}}} = 2
		\end{align}
			  Rank is defined to be the number of linearly indpendent rows or columns of a matrix.
		\item
The equation of a line can also be expressed as
\begin{align}
	 \vec{n}^{\top}\vec{x} &=   1
		\label{prop:app-lin-eq-unit}
\end{align}
	  \end{enumerate}

\item Find the equation of the set of points which are equidistant from the points $(1, 2, 3)$ and $(3, 2, -1)$.
\item Find the equation of the set of points $\vec{P}$,  the sum of whose distances from $\vec{A}(4, 0, 0)$ and $\vec{B}(-4, 0, 0)$ is equal to $10$.
\item Find the values of $\theta$ and $p$,  if the equation $x \cos \theta + y \sin \theta = p$ is the normal form of the line $\sqrt 3x+y+2=0$.
\item Find the image of the point $(3, 8)$ with respect to the line $x+3y=7$ assuming the line to be a plane mirror.
\item Prove that if a plane has the intercept $a, b, c$ and is at a distance of $p$ units  from the origin,  then $\frac{1}{a^2}+\frac{1}{b^2}+\frac{1}{c^2}=\frac{1}{p^2}.$
\item The planes $2x-y+4z=5$ and $5x-2.5y+10z=6$ are 
\begin{enumerate}
\item Perpendicular
\item Parallel
\item Intersect $y$ axis
\item Pass through $\myvec{0, 0, \frac{5}{4}}$
\end{enumerate}
\end{enumerate}
