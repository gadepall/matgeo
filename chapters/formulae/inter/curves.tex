\begin{enumerate}[label=\thesubsection.\arabic*,ref=\thesubsection.\theenumi]
		\item
  The points of intersection of the line 
\begin{align}
L: \quad \vec{x} = \vec{h} + \kappa \vec{m} \quad \kappa \in \mathbb{R}
\label{eq:conic_tangent}
\end{align}
with the conic section in \eqref{eq:conic_quad_form} are given by
\begin{align}
\vec{x}_i = \vec{h} + \kappa_i \vec{m}
	\label{eq:chord-pts}
\end{align}
%
where
\begin{multline}
\kappa_i = \frac{1}
{
\vec{m}^{\top}\vec{V}\vec{m}
}
\lbrak{-\vec{m}^{\top}\brak{\vec{V}\vec{h}+\vec{u}}}
%\\
\pm
%{\small
\rbrak{\sqrt{
\sbrak{
\vec{m}^{\top}\brak{\vec{V}\vec{h}+\vec{u}}
}^2
	-\text{g}
\brak
{\vec{h}
%\vec{h}^{\top}\vec{V}\vec{h} + 2\vec{u}^{\top}\vec{h} +f
}
\brak{\vec{m}^{\top}\vec{V}\vec{m}}
}
}
%}
\label{eq:tangent_roots}
\end{multline}
See 
	 \ref{prop:chord}
	 for proof.
 \item 
	\eqref{eq:chord-pts}
	can be expressed as the solution matrix
\begin{align}
	\vec{X} = \myvec{\vec{h} &\vec{m} }\myvec{ \vec{1}&\bm{\kappa}}^\top
\end{align}
where 
\begin{align}
	\bm{\kappa} = \myvec{\kappa_1 \\ \kappa_2}.
\end{align}


  \item
\eqref{eq:conic_quad_form} represents a pair of straight lines if 
the matrix 
  \begin{align} 
	  \myvec{\vec{V} & \vec{u}\\ \vec{u}^{\top} & f}  
	  \label{eq:pair-mat-sing}
  \end{align} 
  is singular.
\item The intersection of two conics 
with parameters $\vec{V}_i, \vec{u}_i, f_i,\ i = 1,2$
	is defined
as
\begin{align}
	\vec{x}^{\top}\brak{\vec{V}_1 + \mu\vec{V}_2}\vec{x}+2 \brak{\vec{u}_1+\mu \vec{u}_2}^{\top} \vec{x} 
	+ \brak{f_1+\mu f_2}= 0
	  \label{eq:pair-mat-sing-conic}
    \end{align}
	  
	  
\item From \eqref{eq:pair-mat-sing}, \eqref{eq:pair-mat-sing-conic} represents a pair of straight lines if
\begin{align}
	  \label{eq:pair-mat-sing-conic-det}
\mydet{\vec{V}_1 + \mu\vec{V}_2 & \vec{u}_1+\mu \vec{u}_2\\ \brak{\vec{u}_1+\mu \vec{u}_2}^{\top} & f_1 + \mu f_2} &= 0
\end{align}
\end{enumerate}
