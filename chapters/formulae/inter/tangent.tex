\begin{enumerate}[label=\thesubsection.\arabic*,ref=\thesubsection.\theenumi]
\item
  If $L$ in \eqref{eq:conic_tangent} touches \eqref{eq:conic_quad_form} at exactly one point $\vec{q}$, 
  \begin{align}
  \vec{m}^{\top}\brak{\vec{V}\vec{q}+\vec{u}} = 0
  \label{eq:conic_tangent_mq}
  \end{align}
\item
  Given the point of contact $\vec{q}$, the equation of a tangent to \eqref{eq:conic_quad_form} is 
  \begin{align}
  \brak{\vec{V}\vec{q}+\vec{u}}^{\top}\vec{x}+\vec{u}^{\top}\vec{q}+f = 0
  \label{eq:conic_tangent_final}
  \end{align}
\item
  Given the point of contact $\vec{q}$, the equation of the normal to \eqref{eq:conic_quad_form} is 
  \begin{align}
    \brak{\vec{V}\vec{q}+\vec{u}}^{\top}\vec{R}\brak{\vec{x}-\vec{q}} =0
  \label{eq:conic_normal_final}
  \end{align}
\item
	\label{prop:conic-p-contact-nonparab}
  If $\vec{V}^{-1}$ exists, given the normal vector $\vec{n}$, the tangent points of contact to \eqref{eq:conic_quad_form} are given by
\begin{align}
  \begin{split}
\vec{q}_i &= \vec{V}^{-1}\brak{\kappa_i \vec{n}-\vec{u}}, i = 1,2
\\
\text{where }\kappa_i &= \pm \sqrt{
\frac{
f_0
%\vec{u}^{\top}\vec{V}^{-1}\vec{u}-f
}
{
\vec{n}^{\top}\vec{V}^{-1}\vec{n}
}
}
  \end{split}
\label{eq:conic_tangent_qk}
\end{align}
 \item 
\eqref{eq:conic_tangent_qk}
	can be expressed as the solution matrix
\begin{align}
	\vec{Q} =\vec{V}^{-1} \myvec{\vec{u} &\vec{n} }\myvec{ -\vec{1}&\bm{\kappa}}^\top
\label{eq:conic_tangent_Q}
\end{align}
\item
  If $\vec{V}$ is not invertible,  given the normal vector $\vec{n}$, the point of contact to \eqref{eq:conic_quad_form} is given by the matrix equation
\begin{align}
\label{eq:conic_tangent_q_eigen}
\myvec{
\vec{\brak{u+\kappa \vec{n}}}^{\top} \\ \vec{V}
}
\vec{q} = 
\myvec{
-f
\\
\kappa\vec{n}-\vec{u}
}
\quad 
\text{where }  \kappa = \frac{\vec{p}_1^{\top}\vec{u}}{\vec{p}_1^{\top}\vec{n}}, \quad \vec{V}\vec{p}_1 = 0
%\label{eq:conic_tangent_qk_eigen}
\end{align}

\item For a conic/hyperbola, a line with normal vector $\vec{n}$ cannot be a tangent if 
	\label{prop:conic-p-contact-nonparab-cond}
\begin{align}
\frac{
\vec{u}^{\top}\vec{V}^{-1}\vec{u}-f
}
{
\vec{n}^{\top}\vec{V}^{-1}\vec{n}
} < 0
	\label{eq:conic-p-contact-nonparab-cond}
\end{align}
\item For a circle, the points of contact are
	\begin{align}
	\vec{q}_{ij} &= \brak{\pm r \frac{\vec{n}_j}{\norm{\vec{n}_j}}-\vec{u}}, \quad i,j = 1,2
\label{eq:conic_tangent_qk-circ}
\end{align}
\item A point $\vec{h}$ lies on a normal to the conic in \eqref{eq:conic_quad_form} 
	if
\begin{multline}
	\label{eq:point_of_tangency-m}
	\brak{ {\vec{m}^\top(\vec{Vh}+\vec{u})}}^2\brak{\vec{n}^{\top}\vec{V}\vec{n}} 
	\\
	- 2\brak{\vec{m}^\top\vec{V}\vec{n}} \brak{ {\vec{m}^\top(\vec{Vh}+\vec{u})}\vec{n}^{\top}\brak{\vec{V}\vec{h}+\vec{u}}} 
	\\
+  \text{g}\brak{
  \vec{h}
	  }\brak{\vec{m}^\top\vec{V}\vec{n}}^2
%	\vec{h}^{\top}\vec{V}\vec{h} + 2\vec{u}^{\top}\vec{h} +f 
	= 0
\end{multline}
\item  A point $\vec{h}$ lies on a tangent to the conic in \eqref{eq:conic_quad_form} if 
\begin{align}
  \vec{m}^{\top}  \sbrak{\brak{\vec{V}\vec{h}+\vec{u}}
	  \brak{\vec{V}\vec{h}+\vec{u}}^{\top}
   -\vec{V}
	  \text{g}\brak{
  \vec{h}
	  }
	  }\vec{m} 
	  &= 0                                                                                             
	  \label{eq:h-tangents-cond}
\end{align}
\end{enumerate}
