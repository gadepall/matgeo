\begin{enumerate}[label=\thesubsection.\arabic*, ref=\thesubsection.\theenumi]
    \item If in $\triangle ABC$, $\overrightarrow{BA} = 2\vec{a}$ and $\overrightarrow{BC} = 3\vec{b}$, then $\overrightarrow{AC}$ is
\rule{1cm}{0.2pt}.
    \hfill (12, 2023)
	\item The coordinates of the three consecutive vertices of a parallelogram $ABCD$ are $\vec{\vec{A}}(1, 3)$, $\vec{B}(-1, 2)$, and $\vec{C}(2, 5)$. Find the coordinates of the fourth vertex $\vec{D}$. 

		\hfill (10, 2021)
\item Points $\vec{A}(3, 1)$, $\vec{B}(5, 1)$, $\vec{C}(a, b)$, and $\vec{D}(4, 3)$ are vertices of a parallelogram $ABCD$. Find the values of $a$ and $b$. \hfill (10, 2019)
\item If $\vec{A}\brak{1,3}$, $\vec{B}\brak{-1,2}$, $\vec{C}\brak{2,5}$ and $\vec{D}\brak{x,4}$ are the vertices of a parallelogram $ABCD$, then the value of $x$ is
\rule{1cm}{0.2pt}.
\hfill (10, 2012)
    \item If $(3,3),(6,y),(x,7)$ and $(5,6)$ are the vertices of a parallelogram taken in order, find the values of $x$ and $y$.
\hfill (10, 2011)
       % 
\item Show that the points $\vec{A}(6, 2), \vec{B}(2, 1), \vec{C}(1, 5)$ and $\vec{D}(5, 6)$ are the vertices of a square.
\hfill	(10, 2006)
\item  Find the coordinates of the vertex $\vec{A}$ of a parallelogram $ABCD$ whose three vertices are given as $\vec{B}\brak{0,0}$, $\vec{C}\brak{3,0}$, and $\vec{D}\brak{0,4}$. 
\hfill	(10, 2024)
%
\item $ABCD$ is a rectangle formed by the points $ \vec{A}\brak{-1,-1}, \vec{B}\brak{-1, 6}, \vec{C}\brak{3, 6}$ and $\vec{D}\brak{3
	, 1}, \vec{P}, \vec{Q}, \vec{R}$ and $\vec{S}$ are mid-points of sides $AB, BC, CD$ and $DA$ respectively. Show that diagonals of the
 quadrilateral $PQRS$ bisect each other.
\hfill	(10, 2024)
    \item The center of a circle is at $\brak{2, 0}$. If one end of a diameter is at $\brak{6, 0}$, then find the other end. 
\hfill	(10, 2024)
\item Find the ratio in which the point $\brak{8, y}$ divides the line segment joining the points $\brak{1, 2}$ and $\brak{2, 3}$. Also, find the value of  $y$.
\hfill	(10, 2024)
\end{enumerate}
